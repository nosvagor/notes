\documentclass[basic, header]{nosvagor-notes}
\usepackage{nosvagor-math}

\colorlet{title-color}{red}
\newcommand{\theTitle}{%
  \href{https://github.com/nosvagor/notes}%
  {Homework 3}%
}

\newcommand{\userName}{Cullyn Newman}
\newcommand{\class}{CS: 250}
\newcommand{\institution}{Portland State}

\begin{document}

\begin{enumerate}[itemsep=5em]

  \item A function \(f : A \to B \) is \textit{linear} if, \(\forall a,b \in
    \R, f(ax + b) = a f(x) + b\). \\
    Apply the definition of linear to:
    \begin{enumerate}[itemsep=3em]

      \item \(f(x) = 2x\)
        \[%%%%%%%%%%%%%%%%%%%%%%%%%%%%%%%%%%%%%%%%%%%%%%%%%%%%%%%%%%%%%%%%%%%%
          \then \forall a,b \in \R, \quad 2(ax + b) =  a 2x + b
        \]%%%%%%%%%%%%%%%%%%%%%%%%%%%%%%%%%%%%%%%%%%%%%%%%%%%%%%%%%%%%%%%%%%%%

      \item \(f(x) = x^2\)
        \[%%%%%%%%%%%%%%%%%%%%%%%%%%%%%%%%%%%%%%%%%%%%%%%%%%%%%%%%%%%%%%%%%%%%
          \then \forall a,b \in \R, \quad (ax + b)^2 = ax^2 + b
        \]%%%%%%%%%%%%%%%%%%%%%%%%%%%%%%%%%%%%%%%%%%%%%%%%%%%%%%%%%%%%%%%%%%%%

      \item \(\displaystyle f(x) = \sum_{i=0}^{\infty} a_i x^i \)
        \[%%%%%%%%%%%%%%%%%%%%%%%%%%%%%%%%%%%%%%%%%%%%%%%%%%%%%%%%%%%%%%%%%%%%
          \then \forall a,b \in \R, \quad \sum_{i=0}^{\infty} a_i (ax + b)^i =  a \left(\sum_{i=0}^{\infty} a_i x^i\right) + b
        \]%%%%%%%%%%%%%%%%%%%%%%%%%%%%%%%%%%%%%%%%%%%%%%%%%%%%%%%%%%%%%%%%%%%%

    \end{enumerate}

  \item A function \(f : \R \to \R \) is \textit{continuous} if, \(\forall
    \depsilon > 0, \exists \delta > 0 : f(x+\delta) - f(x) < \depsilon\).\\
    Apply the definition of continuous to:
    \begin{enumerate}[itemsep=3em]

      \item \(f(x) = |2x - 1|\)
        \[%%%%%%%%%%%%%%%%%%%%%%%%%%%%%%%%%%%%%%%%%%%%%%%%%%%%%%%%%%%%%%%%%%%%
          \forall \epsilon > 0, \exists \delta > 0 :
          |2(x+\delta) - 1 | - |2x -1 | < \epsilon
        \]%%%%%%%%%%%%%%%%%%%%%%%%%%%%%%%%%%%%%%%%%%%%%%%%%%%%%%%%%%%%%%%%%%%%

      \item \(f(x) = x^{-1}\)
        \[%%%%%%%%%%%%%%%%%%%%%%%%%%%%%%%%%%%%%%%%%%%%%%%%%%%%%%%%%%%%%%%%%%%%
          \forall \epsilon > 0, \exists \delta > 0 :
          (x + \delta)^{-1} - x^{-1} < \epsilon
        \]%%%%%%%%%%%%%%%%%%%%%%%%%%%%%%%%%%%%%%%%%%%%%%%%%%%%%%%%%%%%%%%%%%%%

      \item \(\displaystyle f(x) = \sum_{n=0}^{\infty} \cos(b^n \pi x)\)
        \[%%%%%%%%%%%%%%%%%%%%%%%%%%%%%%%%%%%%%%%%%%%%%%%%%%%%%%%%%%%%%%%%%%%%
          \forall \epsilon > 0, \exists \delta > 0 :
          \sum_{n=0}^{\infty} \cos (b^n \pi (x+\delta)) -
          \sum_{n=0}^{\infty} \cos (b^n \pi x) < \epsilon
        \]%%%%%%%%%%%%%%%%%%%%%%%%%%%%%%%%%%%%%%%%%%%%%%%%%%%%%%%%%%%%%%%%%%%%

    \end{enumerate}

%%%%%%%%%%%%%
  \newpage %%%%%%%%%%%%%%%%%%%%%%%%%%%%%%%%%%%%%%%%%%%%%%%%%%%%%%%%%%%%%%%%%%%%
%%%%%%%%%%%%%

  \item A relation \(\sim : A \times A \) is a \textit{chain} if, \(\forall x,y \in A,
    x\sim y \lor y \sim x\) \\
    Apply the definition of chain to:

    \begin{enumerate}[itemsep=2em]

      \item \(x \sim y, : x,y \in \R \land |x| \leq |y|\)
        \begin{align*}
          \forall x,y \in \R \times \R, &\quad |x| \leq |y|
          \lor
          |y| \leq |x|
        \end{align*}

      \item \(S\sim T \given S \in P(T)\), where \(S,T\) are sets and \(P()\)
        denotes power set.
        \begin{align*}
          \forall S \in P(T) &\then S \sim T \\
            &\lor \\
          \forall T \in P(S) &\then T \sim S
        \end{align*}

      \item \(\sigma_1 \sim \sigma_2 \given \sigma_1, \sigma_2, : A\to A\) are
        functions and \(\exists \tau : \sigma_1 = \tau \circ \sigma_2 \)
        \begin{align*}
          \forall \sigma_1, \sigma_2 \in A, \quad \exists \tau :
          \sigma_1 = \tau \circ \sigma_2
          \lor
          \exists \tau : \sigma_2 = \tau \circ \sigma_1
        \end{align*}

    \end{enumerate}

    \item
    \begin{enumerate}[itemsep=3em]

      \item Prove that there is no smallest positive rational number greater
        than 0.
        \begin{align*}
          \forall p,q \in \N, \exists q : 0 < \frac{p}{q+1} < \frac{p}{q}
        \end{align*}

      \item Prove that for every positive real number greater than 0 there is a
        smaller positive rational number.

        There is no smallest positive rational number by theorem (a), thus
        for any given positive real number there is always some rational number
        that could be smaller.


      \item Prove that there is no smallest positive real number greater than 0.
        \[\Q \subset \R\]
        Thus, for any given real number there is always a smaller positive real
        number by theorem (b).

    \end{enumerate}

%%%%%%%%%%%%%
  \newpage %%%%%%%%%%%%%%%%%%%%%%%%%%%%%%%%%%%%%%%%%%%%%%%%%%%%%%%%%%%%%%%%%%%%
%%%%%%%%%%%%%

  \item Fermat’s Last theorem is a famous theorem in Math that was unproven for
    200 years. The theorem says \(\forall n > 2,~ a,b,c \in \N \then  a^n +
    b^n \neq c^n\). Another way to state this is \(a^n + b^n = c^n\) has no integer
    solutions for $n$ larger than 2. Use this theorem to prove that \(\sqrt[n]{2}\) is
    irrational for $n$ larger than 2.
    \begin{align*}
      \sqrt[n]{2} \in \Q &\then \exists a,b \in \Z : \gcd{(a,b)}=1 \\
                         &\then a^n = 2b^n \then a^n = b^n + b^n
    \end{align*}
    Thus, this contradicts Fermat's Last theorem implying \(\sqrt[n]{2}\) is
    irrational for \(n > 2\).

    \begin{itemize}
      \item \textit{ Note: this is essentially
        \link{https://math.stackexchange.com/q/783392}{zscoder's proof}. No
        credit here, couldn't figure it out myself at first; it's pretty
        simple, so I couldn't formulate something else that was better
        without adding unnecessary steps}.

    \end{itemize}

\end{enumerate}

\end{document}
