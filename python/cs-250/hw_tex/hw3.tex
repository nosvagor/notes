\documentclass[basic, header]{nosvagor-notes}
\usepackage{nosvagor-math}

\colorlet{title-color}{red}
\newcommand{\theTitle}{%
  \href{https://github.com/nosvagor/notes}%
  {Homework 3}%
}

\newcommand{\userName}{Cullyn Newman}
\newcommand{\class}{CS: 250}
\newcommand{\institution}{Portland State}

\begin{document}

\begin{enumerate}[itemsep=5em]

  \item A function \(f : A \to B \) is \textit{linear} if, \(\forall a,b \in
    \R, f(ax + b) = a f(x) + b\). \\
    Apply the definition of linear to:
    \begin{enumerate}[itemsep=4em]

      \item \(f(x) = 2x\)

      \item \(f(x) = x^2\)

      \item \(\displaystyle f(x) = \sum_{i=0}^{\infty} a_i x^i \)

    \end{enumerate}

  \item A function \(f : \R \to \R \) is \textit{continuous} if, \(\forall
    \epsilon > 0, \exists \delta > 0 : f(x+\delta) - f(x) < \epsilon\).\\
    Apply the definition of continuous to:
    \begin{enumerate}[itemsep=4em]

      \item \(f(x) = |2x - 1|\)

      \item \(f(x) = x^{-1}\)

      \item \(\displaystyle f(x) = \sum_{n=0}^{\infty} \cos(b^n \pi x)\)

    \end{enumerate}

%%%%%%%%%%%%%
  \newpage %%%%%%%%%%%%%%%%%%%%%%%%%%%%%%%%%%%%%%%%%%%%%%%%%%%%%%%%%%%%%%%%%%%%
%%%%%%%%%%%%%

  \item A relation \(\sim : A \times A \) is a \textit{chain} if, \(\forall x,y \in A,
    x\sim y \lor y \sim x\) \\
    Apply the definition of chain to:

    \begin{enumerate}[itemsep=4em]

      \item \(x \sim y, : x,y \in \R \land |x| \leq |y|\)

      \item \(S\sim T \given S \in P(T)\), where \(S,T\) are sets and \(P()\)
        denotes power set.

      \item \(\sigma_1 \sim \sigma_2 \given \sigma_1, \sigma_2 : A\to A\) are
        functions and \(\sigma_1 = \tau \circ \sigma_2 \) for some function \(\tau\).

    \end{enumerate}

  \item
    \begin{enumerate}[itemsep=4em]

      \item Prove that there is no smallest positive rational number greater
        than 0.

      \item Prove that for every positive real number greater than 0 there is a
        smaller positive rational number.

      \item Prove that there is no smallest positive real number greater than 0.

    \end{enumerate}

%%%%%%%%%%%%%
  \newpage %%%%%%%%%%%%%%%%%%%%%%%%%%%%%%%%%%%%%%%%%%%%%%%%%%%%%%%%%%%%%%%%%%%%
%%%%%%%%%%%%%

  \item Fermat’s Last theorem is a famous theorem in Math that was unproven for
    200 years. The theorem says \(\forall n > 2,~ a,b,c \in \N \then  a^n +
    b^n \neq c^n\). Another way to state this is \(a^n + b^n = c^n\) has no integer
    solutions for n larger than 2. Use this theorem to prove that \(\sqrt[n]{2}\) is
    irrational for n larger than 2.

\end{enumerate}

\end{document}
