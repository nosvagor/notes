\documentclass[basic, header]{nosvagor-notes}
\usepackage{nosvagor-math}

\colorlet{title-color}{red}
\newcommand{\theTitle}{\href{https://github.com/nosvagor/notes}{Homework 1}}

\newcommand{\userName}{Cullyn Newman}
\newcommand{\class}{CS: 250}
\newcommand{\institution}{Portland State}

\usepackage{circuitikz}
\tikzstyle{branch}=[fill,shape=circle,minimum size=3pt,inner sep=0pt]

% blatantly stolen from MIT resources for teachers.
% lets me set up +*- equations easily
\newcommand\divi[2]{
    \texttt{#1} \: \begin{array}{|l}
        \hline \texttt{#2}
\end{array}
}
\newcommand\mult[2]{
$\begin{array}{rr}
   & \texttt{#1} \\
    \times & \texttt{#2} \\ \hline
 \end{array}$}

\newcommand\addi[2]{
  $\begin{array}{rr}
      &  \texttt{#1} \\
      + & \texttt{#2} \\ \hline
  \end{array}$}
\newcommand\subt[2]{
  $\begin{array}{rr}
      & \texttt{#1} \\
      - & \texttt{#2} \\ \hline
  \end{array}$}
\newcommand\answ[1]{\texttt{#1}\,\,}

\begin{document}

\begin{enumerate}[itemsep=2em]
  \item Fill in the following table of numbers in decimal, binary, octal, and
    hexadecimal.

    \begin{table}[h]
      \centering
      \begin{tabular}{r|c||cccc}
        \toprule
        base        & example &            \\
        \midrule
        decimal     &      10 & 256         &         512 & 32           &    \BB{512} \\
        binary      &   b1010 & b10000000   & b1000000000 & \BB{b100000} & b1000000000 \\
        octal       &     o12 & o400        &  \BB{o1000} & o40          &       o1000 \\
        hexadecimal &     0xA & \BB{0x100}  &       0x200 & 0x20         &       0x200 \\
        \bottomrule
      \end{tabular}
    \end{table}

    note: why are there duplicate columns?

    \begin{table}[h]
      \centering
      \begin{tabular}{r|c||cccc}
        \toprule
         base       & example &                  \\
        \midrule
        decimal     &      10 & 31582             &            153 & 196       &              \BB{65535} \\
        binary      &   b1010 & b111101101011110  & \BB{b10011001} & b11000100 & b1\ldots1 (\(2^{16} \)) \\
        octal       &     o12 & \BB{o75536}       &              o231 & o304         &                 o177777 \\
        hexadecimal &     0xA & 0x7B5E            &           0x99 & \BB{0xC4} &                  0xFFFF \\
        \bottomrule
      \end{tabular}
    \end{table}

    note: I used expansion steps in a calculator, e.g., o75536:
    \[%%%%%%%%%%
      7\cdot8^4 + 5\cdot8^3 + 5\cdot 8^2 + 3\cdot 8^1 + 6 = 31582
    \]%%%%%%%%%%
    but then it got tedious rather than insightful, so then I used python.


  \item Complete the following:  \prn{A:10,B:11,C:12,D:13,E:14,F:15}

  \begin{table}[h]
    \centering
    \begin{tabular}{rrr}
      \addi{0x189}{0x345}   &         \addi{b1010010100}{b0101101011} & \addi{o743}{o265} \\
      \answ{0x4CE}          &                      \answ{b1111111111} & \answ{o1230}      \\
                            &                                         &                   \\
      \addi{0xDEAD}{0xBEEF} &                  \addi{b1111111111}{b1} & \addi{o100}{o777} \\
       \answ{0x19D9C}       &                     \answ{b10000000000} & \answ{o1077}      \\
                            &                                         &                   \\
      \mult{0x89}{0xAB}     &         \mult{b0111111111}{b1000000001} & \mult{o74}{o26}   \\
        \answ{0x5B83}       & \addi{b1000000000}{b111111110000000000} & \answ{o2450}      \\
                            &              \answ{b111111111111111111} &                   \\
    \end{tabular}
  \end{table}

%%%%%%%%%%%%%
  \newpage %%%%%%%%%%%%%%%%%%%%%%%%%%%%%%%%%%%%%%%%%%%%%%%%%%%%%%%%%%%%%%%%%%%%%%
%%%%%%%%%%%%%

  \item Compute the following sets. The universe for all of these sets is \tbm{Z},
  \tbm{P} is the set of prime numbers, \tbm{E} is the set of even numbers, and \tbm{O} is the set of odd numbers.
  \begin{enumerate}
    \item \(\{1,2,3,5,8,13,21,35\} \cup \{2,3,5,7,11,13,17,19\}
      = \left\{ 1,2,3,5,7,8,11,13,17,19,21,35 \right\}
      \)
    \item \(\{1,2,3,5,8,13,21,35\} \cap \{2,3,5,7,11,13,17,19\}
      = \left\{ 2,3,5,13 \right\}
      \)
    \item \(\bm{E} \cap \bm{O} = \nil \)
    \item \(\bm{P} \cap \bm{E} = \left\{ x \in \bm{P} : x \in \bm{E} \right\} \)
    \item \(\bm{N}^{'} = \bm{N} \) \prn{\tbm{N} has not been defined?, assuming \(\N\)}
    \item \(\bm{E}^{'} \cap \bm{O}^{'} = \nil \)
    \item \(\{x : y \in \bm{N}, x = y^2\} \cap \{x : y \in \bm{N}, x = y^3\}
      = \left\{0, 1 \right\}
      \)
    \item \(P\left(\{1,2,3\}\right) \cap P\left(\{2,3,4\}\right) =
      \left\{ \nil, \left\{ 1 \right\}, \left\{ 2 \right\}, \left\{ 3 \right\},  \left\{ 2,3 \right\}    \right\} \)
  \end{enumerate}

  \item Give the truth tables for the following. Which are tautologies, and which
  are satisfiable?
  \begin{multicols}{2}
  \begin{enumerate}
    \item \(a \land (b \lor a)\) ; \textbf{satisfiable}
      \[%%%%%%%%%%
        \begin{array}{|ll|c|c|}
          \hline
          a & b & (b \lor a)  & (a)\\
          \hline
          1 & 1 & 1 & 1 \\
          1 & 0 & 1 & 1 \\
          0 & 1 & 1 & 0 \\
          0 & 0 & 0 & 0 \\
          \hline
        \end{array}
      \]%%%%%%%%%%


    \item \((a \land b) \lor (a \land a) \) ; \textbf{satisfiable}

      \[%%%%%%%%%%
        \begin{array}{|ll|c|c|}
          \hline
          a & b & (a \land b) & (b) \\
          \hline
          1 & 1 & 1 & 1 \\
          1 & 0 & 0 & 1 \\
          0 & 1 & 0 & 0 \\
          0 & 0 & 0 & 0 \\
          \hline
        \end{array}
      \]%%%%%%%%%%

    \item \(a \lor \lnot a\) ; \textbf{tautology}
      \[%%%%%%%%%%
        \begin{array}{|ll|c|}
          \hline
            a & \lnot a & (c) \\
          \hline
            1 & 0 & 1 \\
            1 & 0 & 1 \\
            0 & 1 & 1 \\
            0 & 1 & 1 \\
          \hline
        \end{array}
      \]%%%%%%%%%%

    \item \(\lnot (a \lor b)\) ; \textbf{satisfiable}
      \[%%%%%%%%%%
        \begin{array}{|ll|c|}
          \hline
            a & b & (d) \\
          \hline
            1 & 1 & 0 \\
            1 & 0 & 0 \\
            0 & 1 & 0 \\
            0 & 0 & 1 \\
          \hline
        \end{array}
      \]%%%%%%%%%%

    \item \((\lnot a) \land (\lnot b)\) ; \textbf{satisfiable}
      \[%%%%%%%%%%
        \begin{array}{|ll|c|}
          \hline
            \lnot a & \lnot b & (e) \\
          \hline
            0 & 0 & 0 \\
            0 & 1 & 0 \\
            1 & 0 & 0 \\
            1 & 1 & 1 \\
          \hline
        \end{array}
      \]%%%%%%%%%%

    \item \(\lnot a \land (\lnot b \lor c)\) ; \textbf{satisfiable}
      \[%%%%%%%%%%
        \begin{array}{|ccc|c|c|c|}
          \hline
          a & b & c & \lnot a & \lnot b \lor c & (f)\\
          \hline
          1 & 1 & 1 & 0 & 1 & 0\\
          1 & 1 & 0 & 0 & 0 & 0\\
          1 & 0 & 1 & 0 & 1 & 0\\
          1 & 0 & 0 & 0 & 1 & 0\\
          0 & 1 & 1 & 1 & 1 & 1\\
          0 & 1 & 0 & 1 & 0 & 0\\
          0 & 0 & 1 & 1 & 1 & 1\\
          0 & 0 & 0 & 1 & 1 & 1\\
          \hline
        \end{array}
      \]%%%%%%%%%%

  \end{enumerate}
  \end{multicols}
\item Now we're going to learn how to add with logic. Instead of thinking of
  using logic with true and false \(\left(\top ~\text{or}~ \bot\right)\), let's
  use 1 and 0. This yields the following truth table and circuit:

  \begin{center}
      \(
      \begin{array}{|ll|c|}
        \hline
        a & b & (a \lor b)\\
        \hline
        1 & 1 & 1 \\
        1 & 0 & 1 \\
        0 & 1 & 1 \\
        0 & 0 & 0 \\
        \hline
      \end{array}
      \)

      \begin{circuitikz}
        \node (x) at (0, 1) {$a$};
        \node (y) at (0, 0) {$b$};
        \node[or port, draw] at (2,.5) (xory) {};
        \draw (x) -- (xory.in 1);
        \draw (y) -- (xory.in 2);
        \node (y) at (2.65, .5) {$a \lor b$};
      \end{circuitikz}
  \end{center}

    Now we're going to add 2 1-bit numbers, but there's a problem: adding 2
    1-bit numbers might give us a two bit answer. Thus, we’ll break the into
    two problems the bit on the right will be the sum bit \(s\), and the bit on the
    left will be the carry bit \(c\), .e.g,
    \[%%%%%%%%%%
      b_1 + b_2 = cs
    \]%%%%%%%%%%

    \begin{enumerate}
      \item give the truth table for the \textit{sum} bit \(\rotatebox[origin=c]{270}{\Rsh}\)
      \item give the truth table for the \textit{carry} bit \(\rotatebox[origin=c]{270}{\Rsh}\)
        \[%%%%%%%%%%
          \begin{array}{|ll||c|c|}
            \hline
            b_1 & b_2 & c & s \\
            \hline
            1 & 1 & 1 & 0 \\
            1 & 0 & 0 & 1 \\
            0 & 1 & 0 & 1  \\
            0 & 0 & 0 & 0  \\
            \hline
          \end{array}
        \]%%%%%%%%%%
      \item give the logical formulas for the sum and carry bits.
        \[%%%%%%%%%%
            b_1 \land b_2 \then c \qquad (b_1 \lor b_2) \land \lnot(b_1 \land b_2) ~\text{~i.e.,}~ \(b_1 \oplus b_2 \) \then s
        \]%%%%%%%%%%
      \item give a circuit representing a \textit{sum} gate. This gate should have 2
        \(b_1, b_2\) inputs and 2 outputs \(s, c\)
    \end{enumerate}
    \begin{center}
      \begin{circuitikz}
        \node (b1) at (-2.5,0.3) {\(b_1\)};
        \node (b2) at (-2.5,-0.2) {\(b_2\)};
        \node (b1) at (-2.2,-0.7) {\(b_1\)};
        \node (b2) at (-2.2,-1.3) {\(b_2\)};
        \node (c) at (2.3,-1) {\(c\)};
        \node (s) at (0.3,0) {\(s\)};
        \draw (0,0) node[xor port](xor1) {}
        (2,-1) node[and port] (and2) {}
        (and2.in 2) -- +(-2.6,0)
        (and2.in 1) -- +(-2.6,0)
        (xor1.in 2) -- +(-0.8,0)
        (xor1.in 1) -- +(-0.8,0);
      \end{circuitikz}
    \end{center}

    note: I'm not familiar with circuitikz, I'm  not sure how to connect inputs.
\end{enumerate}

\end{document}
