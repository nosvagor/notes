\documentclass[basic, header]{nosvagor-notes}
\usepackage{nosvagor-math}

\colorlet{title-color}{red}
\newcommand{\theTitle}{\href{https://github.com/nosvagor/notes}{Homework 1}}

\newcommand{\userName}{Cullyn Newman}
\newcommand{\class}{CS-250}
\newcommand{\institution}{Portland State}

% blatantly stolen from MIT resources for teachers.
% lets me set up +*- equations easily
\newcommand\divi[2]{
    \texttt{#1} \: \begin{array}{|l}
        \hline \texttt{#2}
\end{array}
}
\newcommand\mult[2]{
$\begin{array}{rr}
   & \texttt{#1} \\
    \times & \texttt{#2} \\ \hline
 \end{array}$}

\newcommand\addi[2]{
  $\begin{array}{rr}
      &  \texttt{#1} \\
      + & \texttt{#2} \\ \hline
  \end{array}$}
\newcommand\subt[2]{
  $\begin{array}{rr}
      & \texttt{#1} \\
      - & \texttt{#2} \\ \hline
  \end{array}$}
\newcommand\answ[1]{\texttt{#1}\,\,}

\begin{document}

\begin{enumerate}[itemsep=2em]
  \item Fill in the following table of numbers in decimal, binary, octal, and
    hexadecimal.

    \begin{table}[h]
      \centering
      \begin{tabular}{r|c||cccc}
        \toprule
        base        & example &            \\
        \midrule
        decimal     &      10 & 256         &         512 & 32           &    \BB{512} \\
        binary      &   b1010 & b10000000   & b1000000000 & \BB{b100000} & b1000000000 \\
        octal       &     o12 & o400        &  \BB{o1000} & o40          &       o1000 \\
        hexadecimal &     0xA & \BB{0x100}  &       0x200 & 0x20         &       0x200 \\
        \bottomrule
      \end{tabular}
    \end{table}

    note: why are there duplicate columns?

    \begin{table}[h]
      \centering
      \begin{tabular}{r|c||cccc}
        \toprule
         base       & example &                  \\
        \midrule
        decimal     &      10 & 31582             &            153 & 196       &              \BB{65535} \\
        binary      &   b1010 & b111101101011110  & \BB{b10011001} & b11000100 & b1\ldots1 (\(2^{16} \)) \\
        octal       &     o12 & \BB{o75536}       &              o231 & o304         &                 o177777 \\
        hexadecimal &     0xA & 0x7B5E            &           0x99 & \BB{0xC4} &                  0xFFFF \\
        \bottomrule
      \end{tabular}
    \end{table}

    note: I used expansion steps in a calculator, e.g., o75536:
    \[%%%%%%%%%%
      7\cdot8^4 + 5\cdot8^3 + 5\cdot 8^2 + 3\cdot 8^1 + 6 = 31582
    \]%%%%%%%%%%
    but then it got tedious rather than insightful, so then I used python.


  \item Complete the following:  \prn{A:10,B:11,C:12,D:13,E:14,F:15}

  \begin{table}[h]
    \centering
    \begin{tabular}{rrr}
      \addi{0x189}{0x345}   &         \addi{b1010010100}{b0101101011} & \addi{o743}{o265} \\
      \answ{0x4CE}          &                      \answ{b1111111111} & \answ{o1230}      \\
                            &                                         &                   \\
      \addi{0xDEAD}{0xBEEF} &                  \addi{b1111111111}{b1} & \addi{o100}{o777} \\
       \answ{0x19D9C}       &                     \answ{b10000000000} & \answ{o1077}      \\
                            &                                         &                   \\
      \mult{0x89}{0xAB}     &         \mult{b0111111111}{b1000000001} & \mult{o74}{o26}   \\
        \answ{0x5B83}       & \addi{b1000000000}{b111111110000000000} & \answ{o2450}      \\
                            &              \answ{b111111111111111111} &                   \\
    \end{tabular}
  \end{table}

%%%%%%%%%%%%%
  \newpage %%%%%%%%%%%%%%%%%%%%%%%%%%%%%%%%%%%%%%%%%%%%%%%%%%%%%%%%%%%%%%%%%%%%%%
%%%%%%%%%%%%%

  \item Compute the following sets. The universe for all of these sets is \tbm{Z},
  \tbm{P} is the set of prime numbers, \tbm{E} is the set of even numbers, and \tbm{O} is the set of odd numbers.
  \begin{enumerate}
    \item \(\{1,2,3,5,8,13,21,35\} \cup \{2,3,5,7,11,13,17,19\} = \)
    \item \(\{1,2,3,5,8,13,21,35\} \cap \{2,3,5,7,11,13,17,19\} = \)
    \item \(\bm{E} \cap \bm{O} = \)
    \item \(\bm{P} \cap \bm{E} = \)
    \item \(\bm{N}^{'} = \)
    \item \(\bm{E}^{'} \cap \bm{O}^{'} = \)
    \item \(\{x : y \in \bm{N}, x = y^2\} \cap \{x : y \in \bm{N}, x = y^3\} = \)
    \item \(P\left(\{1,2,3\}\right) \cap P\left(\{2,3,4\}\right) = \)
  \end{enumerate}

  \item Give the truth tables for the following. Which are tautologies, and which
  are satisfiable?
  \begin{enumerate}
    \item \(a \land (b \lor a)\)
    \item \((a \land b) \lor (a \land b) \)

    \item \(a \lor \lnot a\)

    \item \(\lnot (a \lor b)\)

    \item \(\lnot a \land (\lnot b)\)

    \item \(\lnot a \land (\lnot b \lor c)\)

  \end{enumerate}

\end{enumerate}

\end{document}
