\documentclass[basic, header]{nosvagor-notes}
\usepackage{nosvagor-math}

\colorlet{title-color}{red}
\newcommand{\theTitle}{%
  \href{https://github.com/nosvagor/notes}%
  {Homework 7}%
}

\newcommand{\userName}{Cullyn Newman}
\newcommand{\class}{CS: 250}
\newcommand{\institution}{Portland State}

\begin{document}

\begin{enumerate}[itemsep=4em]

  \item Give the \(\Theta\) for the following and justify your answer:
    \begin{enumerate}[itemsep=3em]

      \item $5n^2 + 4n + 3$

        \vspace{1em}
        \(f(n) \in \Theta(n^2) \), since \(n^2\) is dominant term.

      \item $2^n + n!$

        \vspace{1em}
        \(f(n) \in \Theta(n!) \), since \(n!\) is dominant term.

      \item $n^2 + 2^n$

        \vspace{1em}
        \(f(n) \in \Theta(2^n) \), since \(2^n\) is dominant term.

      \item $\log(n) + n$

        \vspace{1em}
        \(f(n) \in \Theta(n) \), since \(n\) is dominant term.

      \item $\log(n!)$
        \begin{align*}
          \log(n!)
          = \log\left( \prod_{i=1}^{n} k_i \right)
          &= \sum_{k=1}^{n} \log(k) \in \Theta(n \log n)
        \end{align*}

    \end{enumerate}

%%%%%%%%%%%%%
  \newpage %%%%%%%%%%%%%%%%%%%%%%%%%%%%%%%%%%%%%%%%%%%%%%%%%%%%%%%%%%%%%%%%%%%%
%%%%%%%%%%%%%

  \item Give a closed form for the following, then give the $\Theta$
     \begin{enumerate}[itemsep=2em]

      \item $a_0 = 5$\\
        $a_n = 3a_{n-1}$
        \begin{align*}
          &= 5 + 3(5) + 3(3(5)5) + 3(3(3(15))) \\
          &= \sum_{k=0}^{n} 3^k(5) = \frac{5(3^{n+1} -1) }{2} \in \Theta(3^n)
        \end{align*}

      \item $a_4 = 2$\\
        $a_n = a_{n-1} + \log_2(n)$
        \begin{align*}
          &= 2 + \left( 2 + \log_2(5) \right) + \left( 2 + \left( 2 + \log_2(6) \right)  \right) \\
          &= \sum_{k=5}^{n} 2+ (2 + \log_2(k)) \\
          &= \sum_{k=5}^{n} 2 + \sum_{k=5}^{n} 2 + \log_2(k) \\
          &= \sum_{k=5}^{n} 2 + \sum_{k=5}^{n} 2 + \sum_{k=5}^{n} \log_2(k)\\
          &= 2\sum_{k=5}^{n} 2 + \sum_{k=5}^{n} \log_2(k)\\
          &= 4(n-4) + \log_2\left(\prod_{i=5}^{n} k_i\right) \\
          &= 4(n-4) + \log_2(n!) \in \Theta(n \log_2((n))\\
        \end{align*}

      \item $a_1 = 1$\\
        $a_n = 2a_{n-2} + 1$
        \begin{align*}
          &= 1 + (2(1) + 1) + (2(3) + 1) + (2(7) + 1) \\
          &= \sum_{k=0}^{n} 2k + 1 \given k\%2 = 1 &&\text{i.e., sum over odd indices}\\
          &= \sum_{i=1}^{n} 1 + \sum_{i=1}^{n} 2k+1 \\
          &= n + n(n+1) \\
          &= 2n + n^2 \in \Theta(n^2)
        \end{align*}

      \item $T(1) = 1$\\
        $T(n) = 3T(n/2) + 1$
        \begin{align*}
          a &= 3, b = 2, k = \log_2(n), f(n) = 1 \\
          \then T(n) &= 3^k + \sum_{i=0}^{k-1} 3^i (1) \\
               &= 3^k + \sum_{i=0}^{k-1} 3^i \\
               &= 3^k + \frac{3^k - 1}{2} \\
               &= 3^{\log_2(n)}  + \frac{3^{\log_2(n)} -1}{2} \in \Theta\left(n^{\log_2(n)} \right)
        \end{align*}

      \item $T(1) = 4$\\
        $T(n) = T(n/3) + 4$
        \begin{align*}
           a &= 1, b = 3, k = \log_3(n), f(n) = 4 \\
          \then T(n) &= (1)(4) + \sum_{i=0}^{k-1} (1)(4) \\
                     &= 4 + \sum_{i=0}^{k-1} 4 \\
                     &= 4 + 4(k-1) \\
                     &= 4\log_3(n) \in \Theta(\log_3(n))
        \end{align*}
    \end{enumerate}

%%%%%%%%%%%%%
  \newpage %%%%%%%%%%%%%%%%%%%%%%%%%%%%%%%%%%%%%%%%%%%%%%%%%%%%%%%%%%%%%%%%%%%%
%%%%%%%%%%%%%

  \item (Extra Credit):\\
    $T(0) = 1$\\
    $T(n) = 3T(n-2) + 4(n-2) + 2$
    \begin{align*}
      a &= 3, b = 2, k = \log_2(n), f(n) = 4(n-2) + 2 \\
      \then T(n) &= 3^k + \sum_{i=0}^{k-1} 3^i \frac{4n}{2^i} - 6\\
                 &= 3^k + 4n\sum_{i=0}^{k-1} \left(\frac{3}{2}\right)^i - \sum_{i=0}^{k-1} 6\\
                 &= 3^k + \frac{4n\left(\frac{3}{2}^{k}\right) - 1}{\frac{3}{2} - 1} - 6k-6\\
                 &= 3^{\log_2(n)} + \frac{4n\left(\frac{3}{2}^{\log_2(n)}\right) - 1}{\frac{3}{2} - 1} - 6\log_2(n)-6\\
                 &\in \Theta\left(n^\log_2(n)\right)
    \end{align*}

%%%%%%%%%%%%%
  \newpage %%%%%%%%%%%%%%%%%%%%%%%%%%%%%%%%%%%%%%%%%%%%%%%%%%%%%%%%%%%%%%%%%%%%
%%%%%%%%%%%%%

  \item Prove \textbf{theorem 2:} $x^k \in \O(x^{k + c})$

  \item Prove \textbf{theorem 3:} $x^k + c\cdot x^{k-r} \in \O(x^k)$

  \item Prove \textbf{theorem 5:} if $f(n) \in \O(g(n))$ and $g(n) \in
    \O(h(n))$, then $f(n) \in \O(h(n))$

%%%%%%%%%%%%%
  \newpage %%%%%%%%%%%%%%%%%%%%%%%%%%%%%%%%%%%%%%%%%%%%%%%%%%%%%%%%%%%%%%%%%%%%
%%%%%%%%%%%%%

  \item

    Give the $\Theta$ running time for the following selection sort algorithm
    \begin{verbatim}
    def selSort(l):
        for i in range(len(l)):
            min = l[i]
            minI = i
            for j in range(i,len(l)):
                if l[j] < min:
                    minI = j
                    min = l[j]
                #end if
            # end for
            (l[i], min) = (min, l[i])
        # end for
    \end{verbatim}

%%%%%%%%%%%%%
  \newpage %%%%%%%%%%%%%%%%%%%%%%%%%%%%%%%%%%%%%%%%%%%%%%%%%%%%%%%%%%%%%%%%%%%%
%%%%%%%%%%%%%

  \item Give the recurrence relation for badSort. Remember \texttt{l[a:b]}
    copies the elements from \texttt{l[a]} to \texttt{l[b]}, so even though
    it's an expression \texttt{l[a:b]} runs in $n-2$ time.

    \begin{verbatim}
    def badSort(l): n = len(l)

        if n == 1:
            return l

        first = badSort(l[0:n-2])
        middle = badSort(l[1:n-1])
        end = badSort(l[2:n])

        return [first[0]] + middle + [end[n-1]]
    \end{verbatim}

%%%%%%%%%%%%%
  \newpage %%%%%%%%%%%%%%%%%%%%%%%%%%%%%%%%%%%%%%%%%%%%%%%%%%%%%%%%%%%%%%%%%%%%
%%%%%%%%%%%%%

  \item The following algorithm is the merge sort we way in class

    \begin{multicols}{2}
      \begin{verbatim}
def merge(low, high):
    i = 0
    j = 0
    merged = []
    while i < len(low) and j < len(high):
        if low[i] < high[j]:
            merged += [low[i]]
            i += 1
        else:
            merged += [high[j]]
            j += 1
    return merged + low[i:] + high[j:]
      \end{verbatim}
      \vspace{5em}
      \begin{verbatim}
    def mergeSort(lst):
        n = len(lst)
        n2 = int(n/2)

        # base case:
        if n <= 1:
            return lst

        # recursive case:
        low = mergeSort(lst[0:n2])
        high = mergeSort(lst[n2:n])
        lst = merge(low,high)

        return lst
      \end{verbatim}
    \end{multicols}

  \begin{enumerate}[itemsep=4em]
    \item Give the $\Theta$ running time for merge.\\
      hint: what is the input size for merge?
    \item Use (a) to give a recurrence relation for the running time of mergeSort.
    \item Solve the recurrence to get a $\Theta$ running time for mergesort.
  \end{enumerate}

\end{enumerate}

\end{document}
