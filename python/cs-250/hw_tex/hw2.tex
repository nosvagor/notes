\documentclass[basic, header]{nosvagor-notes}
\usepackage{nosvagor-math}

\colorlet{title-color}{red}
\newcommand{\theTitle}{%
  \href{https://github.com/nosvagor/notes}%
  {Homework 2}%
}

\newcommand{\userName}{Cullyn Newman}
\newcommand{\class}{CS: 250}
\newcommand{\institution}{Portland State}

\begin{document}

\begin{enumerate}[itemsep=5em]
  \item Determine if the following are 1 to 1, onto, or total. Justify your answer.

    \begin{itemize}
      \item \dd{Total (\link{https://en.wikipedia.org/wiki/Partial_function}{total}
      )}: \(\forall x \in A \then f(x) \) is defined.

      \item \dd{1 to 1
        (\link{https://en.wikipedia.org/wiki/Injective_function}{injective})}:
        \(\forall x,y \in X, \quad f(x) = f(y) \then x = y\).

      \item \dd{Onto
         (\link{https://en.wikipedia.org/wiki/Surjective_function}{surjective})}:
         \(f : X \to Y, \quad \forall y \in Y, \quad \exists x
        \in X \then f(x) = y \)
    \end{itemize}

    \vspace{1em}

  \begin{enumerate}
    \item \(f : \R \to \R, \quad f(x) = \sin x\)
      \begin{itemize}
        \item Total: \true{true}: \(\forall x \in \R, \quad  f(x) \in \R\)
        \item Injective: \false{false}: \(\forall x, y \in \R : x \neq y, ~\text{some}~ f(x) = f(y)\)
        \item Surjective: \false{false}: e.g., \( 2 \in \R ~\text{but}\not\in [-1,1]\)
      \end{itemize}

    \item \(f : \R \to \R, \quad f(x) = \sqrt{x}\)
      \begin{itemize}
        \item Total: \false{false}: e.g., \(f(-1) = i, i \not\in \R\)
        \item Injective: \true{true (partial)}: \(\forall x,y \in \R, : x = y \iff f(x) = f(y)\)
        \item Surjective: \false{false}: \(\forall y \in \R^{-}, \quad \nexists x \in \R \then f(x) \neq y\)
      \end{itemize}

    \item \(f : \N \to \R^+, \quad f(x) = \sqrt{x}\)
      \begin{itemize}
        \item Total: \true{true}: \(\forall x \in \N, \quad  f(x) \in \R^+ \)
        \item Injective: \true{true}: \(\forall x, y \in \N : x = y \then f(x) = f(y)  \)
        \item Surjective: \true{true}: \(\forall y \in \R^+, \exists x \in \N : f(x) = y\)
      \end{itemize}

    \item \(f : \R^+ \to \N, \quad f(x) = \sqrt{x}\)
      \begin{itemize}
        \item Total: \false{false}: e.g., \(\sqrt{42} \not\in \N\)
        \item Injective: \true{true (partial)}: \(\forall x,y \in \R^+ : x = y \then f(x) = f(y) \iff f(x) \in \N \)
        \item Surjective: \true{true}: \(\forall y \in \N, \quad \exists x \in \R^+ : f(x) = y \)
      \end{itemize}

    \item \(f : \R \to \R^+, \quad f(x) = x^2\)
      \begin{itemize}
        \item Total: \true{true}: \(\forall x \in \R, \quad f(x) \in \R^+ \)
        \item Injective: \false{false}: e.g., \(f(-2) = 4, f(2) = 4, \text{i.e}, x\neq y, f(x) = f(y)\)
        \item Surjective: \true{true}: \(\forall \in \R^+, \quad \exists x \in \R : f(x) = y \)
      \end{itemize}

    \item \(f : \R \to \R, \quad f(x) = x^3\)
      \begin{itemize}
        \item Total: \true{true}: \(\forall x \in \R, \quad  f(x) \in \R\)
        \item Injective: \true{true}: \(\forall x, y \in \R, : x = y \then f(x) = f(y)\)
        \item Surjective: \true{true}: \(\forall y \in \R, \quad \exists x \in \R : f(x) = y\)
      \end{itemize}

  \end{enumerate}

  \item Determine if the following relations are reflexive, symmetric,
    antisymmetric, or  transitive. For this question
    \[%%%%%%%%%%
      a,b \in \N, \quad \frac{p}{q}, \frac{m}{n} \in \Q, \quad s, t\in \Sigma^*
    \]%%%%%%%%%
    Justify your answer.

    \begin{itemize}
      \item
        \dd{\link{https://en.wikipedia.org/wiki/Equivalence_relation}{Equivalence}}:
        \(\sim, \equiv \iff\) a relation is reflexive, symmetric, and transitive.

      \item
        \dd{\link{https://en.wikipedia.org/wiki/Reflexive_relation}{Reflexive}}:
        \(\forall a \in X, \quad a \sim a\)

      \item
        \dd{\link{https://en.wikipedia.org/wiki/Symmetric_relation}{Symmetric}}:
        \(\forall a,b \in X, \quad a \sim b \iff b \sim a \)

      \item
        \dd{\link{https://en.wikipedia.org/wiki/Antisymmetric_relation}{Antisymmetric}}:
        \(\forall a,b \in X, \quad a \sim b, a \neq b \then b \nsim a ~\text{\ldots equiv\ldots}~ a \sim b, b \sim a \then a = b\)

      \item
        \dd{\link{https://en.wikipedia.org/wiki/Transitive_relation}{Transitive}}:
        \(\forall a,b,c \in X, \quad : a \sim b, b \sim c \then a \sim c \)

    \end{itemize}
    \vspace{1em}

    \begin{enumerate}[itemsep=2.5em]
      \item $a \sim b$ if $a + b = 10$
        \begin{itemize}
          \item Reflexive: \false{false}: \(6 \nsim 6\)
          \item Symmetric: \true{true}: \(a = 10 - b, b = 10 - a\)
          \item Antisymmetric: \false{false}: \(6 \sim 4, 4\sim 6, \quad 6 \neq 4\)
          \item Transitive: \false{false}: \(4 \sim 6, 6 \sim 4, 4 \nsim 4\)
        \end{itemize}

      \item $a \sim b$ if $a$ and $b$ are both even (E) or both odd (O)
        \begin{itemize}
          \item Reflexive: \true{true}: \(a \sim a, b \sim b \iff a \land b \in (E \oplus O) \)
          \item Symmetric: \true{true}: \(\forall a,b \in (O \oplus E), \quad a \sim b \iff b \sim a\)
          \item Antisymmetric: \false{false}: \(2 \sim 4, 4\sim 2, \quad 2 \neq 4 \)
          \item Transitive: \true{true}: \(\forall a,b \in (O \oplus E ), \quad a \sim b, b \sim c \then a \sim c\)
        \end{itemize}

      \item $\frac p q  \sim \frac r s $ if $q \le s$
        \begin{itemize}
          \item let \(a = \frac{p}{q}, b = \frac{r}{s}, X = \left\{ a\sim b \right\}  \given q \leq s\)
          \item Reflexive: \true{true}: \(\forall x \in X, \quad x \sim x\)
          \item Symmetric: \true{true}: \(\forall x,y \in X, \quad x \sim y \iff y \sim x\)
          \item Antisymmetric: \false{false}: \(\frac{1}{2} \sim \frac{2}{2}, \frac{2}{2} \sim \frac{1}{2}, \quad \frac{1}{2} \neq \frac{2}{2}\)
          \item Transitive: \true{true}: \(\forall x,y,z \in X, \quad x \sim z \given x \sim y, y\sim z \)
        \end{itemize}

      \item $s \sim t$ if $s = reverse(t)$
        \begin{itemize}
          \item Reflexive: \true{true}: \(\forall s,t \in \Sigma^* s\sim s \land~ t \sim t\)
          \item Symmetric: \true{true}: \(\forall s,t \in \Sigma^*, \quad s \sim t \iff t \sim s\)
          \item Antisymmetric: \true{true}: \(\forall s,t \in \Sigma^*, \quad s\sim t, t\sim s \iff s = t\)
          \item Transitive: \true{true}: \(\forall s,t,w \in \Sigma^*, \quad  s \sim w \iff s = t = w\)
        \end{itemize}

      \item $a \sim b$ if $b = ca$ for some $c$
        \begin{itemize}
          \item Reflexive: \true{true}: \(\forall x \in \N, x \sim x \given c = 1\)
          \item Symmetric: \false{false}: \(2 \sim 4 \given c = 2, \quad 4 \nsim 2\)
            \begin{itemize}
              \item \true{true}: if \(c\) is allowed to vary between relations.
            \end{itemize}

          \item Antisymmetric: \true{true}: \(\forall a,b \in \N, a \sim b, b \sim a \iff c = 1 \land a = b\)
            \begin{itemize}
              \item \false{false}: if c is allowed to vary between relations.
            \end{itemize}

          \item Transitive: \false{false}: \(2 \sim 4, \quad 4 \sim 8, \quad 2 \nsim 8 \)
            \begin{itemize}
              \item \true{true}: if c is allowed to vary between relations.
            \end{itemize}

        \end{itemize}

      \item $a \sim b$ if $a^b = b^a$
        \begin{itemize}
          \item let \(X = \left\{ a\sim b \right\} \given a^b = b^a\)
          \item Reflexive: \true{true}: \(\forall x \in X, \quad x^x = x^x\)
          \item Symmetric: \true{true}: \(\forall a,b \in \N, : a^b = b^a \then b^a = a^b\)
          \item Antisymmetric: \false{false}: \(2 \sim 4, 4\sim 2, \quad 2\neq 4\)
          \item Transitive: \true{true}: \(\forall a,b,c \in \N, a \sim b \)
        \end{itemize}

    \end{enumerate}

%%%%%%%%%%%%%
  \newpage %%%%%%%%%%%%%%%%%%%%%%%%%%%%%%%%%%%%%%%%%%%%%%%%%%%%%%%%%%%%%%%%%%%%
%%%%%%%%%%%%%

    \item Prove that if $f : B \to C$ is 1 to 1, and $g : A \to B$ is 1 to 1,
      then $f \circ g$ is also 1 to 1.
        \begin{align*}
          & f \circ g = f(g(x)) = f: (g : A \to B ) \to C \\\\
          : ~   & \forall x, y \in A, \quad g(x) = g(y) \then x = y \\
          : ~   & \forall x, y \in B, \quad f(x) = f(y) \then x = y \\
          \thus & \forall x, y \in C, \quad f(g(x)) = f(g(y)) \then x = y
        \end{align*}

    \item Prove or disprove (where \(P\) is the power set):
      \begin{enumerate}
        \item for any sets $A$ and $B$, $P(A\cap B) = P(A) \cap P(B)$

          \true{true}:
          \begin{align*}
            P(A) &= \{ S : S \subseteq A\}, \quad P(B) = \{ S : S \subseteq B \} \\
                 & ~: A \cap B \subseteq A, \qquad  : B \cap A \subseteq B \\
                 & \then P(A \cap B) = \{ S : S \subseteq A \cap B \}
          \end{align*}


        \item for any sets $A$ and $B$, $P(A\cup B) = P(A) \cup P(B)$

          \false{false}:
          \begin{align*}
            A &= \{42\}, \quad B = \{69\}, \quad A \cup B = \{ 42, 69 \} \\
            P(A) \cup P(B) &= \{\nil, \{ 42 \}, \{ 69 \} \} \\
            P (A \cap B ) &= \{ \nil, \{ 42 \}, \{ 69 \}, \{ 42, 69 \} \} \\
                          &\then P(A) \cup P(B) \neq P(A \cup B)
          \end{align*}


      \end{enumerate}

%%%%%%%%%%%%%
  \newpage %%%%%%%%%%%%%%%%%%%%%%%%%%%%%%%%%%%%%%%%%%%%%%%%%%%%%%%%%%%%%%%%%%%%
%%%%%%%%%%%%%

    \item De Morgan's rule is a logical equivalence $(\lnot a) \lor (\lnot b) =
      \lnot (a \land b)$\\ You can verify this equivalence with a truth table.

      Set theory also has a version of De Morgan's rule. Let $A$ and $B$ be sets
      in universe $U$. Prove that $A^{'} \cup B^{'} = (A \cap B)^{'}$
      \begin{align*}
        A \cap B &= \{ x : x \in A \land x \in B \} \\
        (A \cap B)^{'}   & = \{ x : x \in U \land x \not\in (A \cap B) \}  \\
        A^{'} \cup B^{'} & = \{ x : x \not\in A \lor x \not\in B \} \\
                         & = \{ x : x \in U \land x \not\in (A \cap B) \} \\
                         & = (A \cap B)^{'}
      \end{align*}

      \vspace{-4em}

    \item Mathologer is a YouTube channel that does videos illustrating math
      concepts. He did a video on the fair division problem
      \begin{center}
        \linky{https://www.youtube.com/watch?v=7s-YM-kcKME}{NWT: Spanner's lemma defeats the rental harmony problem}
      \end{center}

      It's an interesting proof about coloring triangles. Unfortunately, He
      doesn't complete the proof. At about the 5 minute mark he says that
      ``there will always be an odd number of doors at the bottom of the
      triangle'',  but he never proves this. Show that there will always be an
      odd number of doors on the bottom of the triangle.

      \begin{itemize}
        \item Vertices of the main triangle must be of different colors, thus every
          time the triangle is increased in size, then at least one connection
          is split, increasing number of connections on the bottom side by two.
        \item Since the red cannot go on the bottom side, then that leaves only
          BG/BB/GG connections, which yields \(\{ BBB, BBG, BGB, BGG, GGG, GGB \}\)
        \item Starting with BG, then only options are BBG or GBB, thus number of odd doors is at least 1.
        \item Each new increase will do one of two options:
          \begin{enumerate}[leftmargin=2em]
            \item Split a door: yielding a miniature starting case (BBG, GBB) (doors += 0)

            \item Split a non-door:
              \begin{enumerate}
                \item into two doors (BGB, GBG) (doors += 2)
                \item into two non-doors (GGG, BBB) (doors += 0)
              \end{enumerate}

          \end{enumerate}
        \item Thus, max number of doors \(= 1 + 2n\) (always odd)

        \item note: this proof inspired by Fejfo's YouTube comment, but I think I
          made it more clear.
      \end{itemize}

%%%%%%%%%%%%%
  \newpage %%%%%%%%%%%%%%%%%%%%%%%%%%%%%%%%%%%%%%%%%%%%%%%%%%%%%%%%%%%%%%%%%%%%
%%%%%%%%%%%%%

    \item In class I said that two functions were equal if they give the same
      output for every input. So, if
      \[f : A \to B ~\text{and}~ g : A \to B, ~\text{and}~ \forall x\in A,
      f(x) = g(x)
      \then f= g
      \]
      Show that this is actually an equivalence relation.

      \begin{itemize}
        \item Reflexive: \(\forall x \in X, \quadc f(x) = g(x) \then x \sim x\)
        \item Symmetric: \(\forall x,y \in X, \quad f(x) = f(y) \then x \sim y \iff g(x) = g(y) \then y \sim x\)
        \item Transitive: \(\forall x,y,z \in X, \quad x \sim y, y \sim z, : f(x) = f(y) \land f(y) = f(z) \then x \sim z\)
        \item All three relations hold, thus the function can be said to be equivalent by it's equivalence relations.
      \end{itemize}

    \item In a dictionary all of the words are arranged in a specific order.
      ``a'', ``aardvark'', and so on.  This is called the dictionary order for
      words.

      Our ordering here is very simple.  Look at the first character of two
      words $w_1$ and $w_2$.  If the first character of $w_1$ comes before
      $w_2$ ($w_1[0] < w_2[0]$) then $w_1 < w_2$.

      If the characters are equal, then we move on to the second character. We
      continue this until we find a different character, or one of the words
      ends. Show that this dictionary order is a partial order (reflexive,
      antisymmetric, and transitive).

      \begin{itemize}
        \item Reflexive: \(\forall  s \in \Sigma^*, s \sim s \)
        \item Antisymmetric: \(\forall s, t \in \Sigma^*, s\sim t, t \sim s \iff s = t\) (not symmetric)
        \item Transitive: \(\forall s, t, w \in \Sigma^*, s \sim t, t \sim w \iff s = t = w \then s \sim w\)
      \end{itemize}

  \end{enumerate}

\end{document}
