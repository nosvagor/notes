\documentclass[basic, header]{nosvagor-notes}
\usepackage{nosvagor-math}

\colorlet{title-color}{red}
\newcommand{\theTitle}{%
  \href{https://github.com/nosvagor/notes}%
  {Homework 2}%
}

\newcommand{\userName}{Cullyn Newman}
\newcommand{\class}{CS: 250}
\newcommand{\institution}{Portland State}

\begin{document}

\begin{enumerate}[itemsep=5em]
  \item Determine if the following are 1 to 1, onto, or total. Justify your answer.
  \begin{enumerate}
    \item \(f : \R \to \R, \quad f(x) = \sin x\)
      \begin{itemize}
        \item
      \end{itemize}

    \item \(f : \R \to \R^+, \quad f(x) = \sqrt{x}\)
      \begin{itemize}
        \item
      \end{itemize}
    \item \(f : \R^+ \to \N, \quad f(x) = \sqrt{x}\)
      \begin{itemize}
        \item
      \end{itemize}
    \item \(f : \R \to \R^+, \quad f(x) = x^2\)
      \begin{itemize}
        \item
      \end{itemize}
    \item \(f : \R \to \R, \quad f(x) = x^3\)
      \begin{itemize}
        \item
      \end{itemize}

  \end{enumerate}

  \item Determine if the following relations are reflexive, symmetric,
    antisymmetric, or  transitive. For this question $a,b \in \N$, $\frac p
    q, \frac m n \in \Q$, and $s, t\in \Sigma^*$.
    \begin{enumerate}
      \item $a \sim b$ if $a + b = 10$
        \begin{itemize}
          \item
        \end{itemize}

      \item $a \sim b$ if $a$ and $b$ are both even, or both odd.
        \begin{itemize}
          \item
        \end{itemize}

      \item $\frac p q \sim \frac r s$ if $q \le s$
        \begin{itemize}
          \item
        \end{itemize}

      \item if $s$ and $t$ are strings, then $s \sim t$ if $s = reverse(t)$
        \begin{itemize}
          \item
        \end{itemize}

      \item $a \sim b$ if $b = c \cdot a$ for some $c$
        \begin{itemize}
          \item
        \end{itemize}

      \item $a \sim b$ if $a^b = b^a$
        \begin{itemize}
          \item
        \end{itemize}

    \end{enumerate}

%%%%%%%%%%%%%
  \newpage %%%%%%%%%%%%%%%%%%%%%%%%%%%%%%%%%%%%%%%%%%%%%%%%%%%%%%%%%%%%%%%%%%%%
%%%%%%%%%%%%%

    \item Prove that if $f : B \to C$ is 1 to 1, and $g : A \to B$ is 1 to 1,
      then $f \circ g$ is also 1 to 1.

    \item Prove or disprove:
      \begin{enumerate}
        \item for any sets $A$ and $B$, $P(A\cap B) = P(A) \cap P(B)$

        \item for any sets $A$ and $B$, $P(A\cup B) = P(A) \cup P(B)$

      \end{enumerate}

%%%%%%%%%%%%%
  \newpage %%%%%%%%%%%%%%%%%%%%%%%%%%%%%%%%%%%%%%%%%%%%%%%%%%%%%%%%%%%%%%%%%%%%
%%%%%%%%%%%%%

    \item DeMorgan's rule is a logical equivalence $(\lnot a) \lor (\lnot b) =
      \lnot (a \lor b)$\\ You can verify this equivalence with a truth table.

      Set theory also has a version of DeMorgan's rule. Let $A$ and $B$ be sets
      in universe $U$. Prove that $A^{'} \cup B^{'} = (A \cap B)^{'}$

%%%%%%%%%%%%%
  \newpage %%%%%%%%%%%%%%%%%%%%%%%%%%%%%%%%%%%%%%%%%%%%%%%%%%%%%%%%%%%%%%%%%%%%
%%%%%%%%%%%%%

    \item Mathologer is a YouTube channel that does videos illustrating math
      concepts. He did a video on the fair division problem
      \begin{center}
        \linky{https://www.youtube.com/watch?v=7s-YM-ACME}{NWT: Spanner's lemma defeats the rental harmony problem}
      \end{center}

      It's an interesting proof about coloring triangles. Unfortunately, He
      doesn't complete the proof. At about the 5 minute mark he says that
      ``there will always be an odd number of doors at the bottom of the
      triangle'',  but he never proves this. Show that there will always be an
      odd number of doors on the bottom of the triangle.

%%%%%%%%%%%%%
  \newpage %%%%%%%%%%%%%%%%%%%%%%%%%%%%%%%%%%%%%%%%%%%%%%%%%%%%%%%%%%%%%%%%%%%%
%%%%%%%%%%%%%

    \item In class I said that two functions were equal if they give the same
      output for every input. So, if
      \[f : A \to B ~\text{and}~ g : A \to B, ~\text{and}~ \forall x\in A,
      f(x) = g(x)
      \then f= g
      \]
      Show that this is actually an equivalence relation.

%%%%%%%%%%%%%
  \newpage %%%%%%%%%%%%%%%%%%%%%%%%%%%%%%%%%%%%%%%%%%%%%%%%%%%%%%%%%%%%%%%%%%%%
%%%%%%%%%%%%%

    \item In a dictionary all of the words are arranged in a specific order.
      ``a'', ``aardvark'', and so on.  This is called the dictionary order for
      words.

      Our ordering here is very simple.  Look at the first character of two
      words $w_1$ and $w_2$.  If the first character of $w_1$ comes before
      $w_2$ ($w_1[0] < w_2[0]$) then $w_1 < w_2$.

      If the characters are equal, then we move on to the second character. We
      continue this until we find a different character, or one of the words
      ends. Show that this dictionary order is a partial order (reflexive,
      antisymmetric, and transitive).

  \end{enumerate}

\end{document}
