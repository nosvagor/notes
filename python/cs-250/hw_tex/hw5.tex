\documentclass[basic, header]{nosvagor-notes}
\usepackage{nosvagor-math}

\colorlet{title-color}{red}
\newcommand{\theTitle}{%
  \href{https://github.com/nosvagor/notes}%
  {Homework 5}%
}

\newcommand{\userName}{Cullyn Newman}
\newcommand{\class}{CS: 250}
\newcommand{\institution}{Portland State}

\begin{document}

\begin{enumerate}[itemsep=4em]

  \item Use induction to prove the following summation formulas. Remember rules
    of logs. Specifically \(\log(ab) = \log(a) + \log(b)\).
    \begin{enumerate}[leftmargin=2em]
      \item \(\displaystyle \sum_{i=1}^{n} a_i + b_i = \sum_{i=1}^{n} a_i + \sum_{i=1}^{n} b_i  \)

        \dt{Proof.}\\
        Base case: \(n = 1 \then
        (a_1 + b_1) = (a_1) + (b_1)\) \\
        Inductive hypothesis: \(\displaystyle
        n = k \then
        \sum_{i=1}^{k} a_i + b_i = \sum_{i=1}^{k} a_i + \sum_{i=1}^{k} b_i
        \)

        Inductive case:
        \begin{align*}
          \sum_{i=1}^{k+1} a_i + b_i
          &= \left(\sum_{i=1}^{k} a_i + b_i\right) + (a_{k+1} + b_{k+1}) \\
          &= \BB{\left(\sum_{i=1}^{k} a_i + \sum_{i=1}^{k} b_i\right)} + (a_{k+1} + b_{k+1})
          \quad  \text{\BB{by I.H.}}\\
          &= \left(\sum_{i=1}^{k} a_i \right) + a_{k+1} + \left(\sum_{i=1}^{k} b_i\right) + b_{k+1}\\
          &= \sum_{i=1}^{k+1} a_i + \sum_{i=1}^{k+1} b_i
          &\qed
        \end{align*}

      \item \(\displaystyle \sum_{i=1}^{n}ca_i = c \sum_{i=1}^{n} a_i \)

        \dt{Proof.}\\
        Base case: \(n = 1 \then ca_1 = ca_1\)\\
        Inductive hypothesis: \(\displaystyle n = k
        \then
        \sum_{i=1}^{k} ca_i = c \sum_{i=1}^{k} a_i\)

        Inductive case:
        \begin{align*}
          \sum_{i=1}^{k+1} ca_i
           &= \left( \sum_{i=1}^{k} ca_i \right) + ca_{k+1}\\
           &= \BB{\left(c \sum_{i=1}^{k} a_i\right)} + ca_{k+1}
           & \text{\BB{by I.H.}} \\
           &= c \sum_{i=1}^{k+1} a_i & \qed
        \end{align*}

%%%%%%%%%%%%%
  \newpage %%%%%%%%%%%%%%%%%%%%%%%%%%%%%%%%%%%%%%%%%%%%%%%%%%%%%%%%%%%%%%%%%%%%
%%%%%%%%%%%%%

      \item \(\displaystyle \sum_{i=1}^{n} \log(a_i) = \log \left( \prod_{i=1}^{n} a \right) \)

        \dt{Proof.}\\
        Base case: \(n=1 \then \log(a_1) = \log(a_1)\) \\
        Inductive hypothesis: \(\displaystyle n= k
        \then
        \sum_{i=1}^{k} \log(a_i) = \log\left( \prod_{i=1}^{k} a_i  \right) \)

        Inductive case:
        \begin{align*}
         \sum_{i=1}^{k+1} \log(a_i)
         &= \sum_{i=1}^{k} \log(a_i) + \log(a_{k+1} )\\
         &= \BB{\log\left( \prod_{i=1}^{k} a_i \right)} + \log(a_{k+1} )
         & \text{\BB{by I.H.}}  \\
         &= \log\left(\prod_{i=1}^{k+1} a_i \right) & \qed
        \end{align*}


      \item \(\displaystyle \sum_{i=1}^{n} 2i - 1 = n^2 \)

        \dt{Proof.}\\
        Base case: \(n = 1 \then 2 - 1 = 1 = 1^2\) \\
        Inductive hypothesis: \(\displaystyle
        n = k \then
        \sum_{i=1}^{k} 2i - 1 = k^2\)

        Inductive case:
        \begin{align*}
          \sum_{i=1}^{k+1} 2i - 1
          &= \left( \sum_{i=1}^{k} 2i - 1 \right)  + 2(k + 1) - 1\\
          &= \BB{k^2} + 2(k+1) -1 & \text{\BB{by I.H.}} \\
          &= k^2 + 2k + 1 \\
          &= (k+1)^2 & \qed
        \end{align*}

%%%%%%%%%%%%%
  \newpage %%%%%%%%%%%%%%%%%%%%%%%%%%%%%%%%%%%%%%%%%%%%%%%%%%%%%%%%%%%%%%%%%%%%
%%%%%%%%%%%%%

      \item \(\displaystyle \sum_{i=1}^{n} i^3 = \left( \frac{n(n+1)}{2} \right)^2 \)

        \dt{Proof.}\\
        Base case: \(n = 1 \then \frac{1(1+1)}{2} = 1 = 1^3\) \\
        Inductive hypothesis: \(\displaystyle
        n = k \then
        \sum_{i=1}^{k} i^3 = \left( \frac{k(k+1)}{2} \right)^2
        \)

        Inductive case:
        \begin{align*}
          \sum_{i=1}^{k+1} i^3 &=
          \sum_{i=1}^{k} i^3
          + (k+1)^3 \\
          &= \BB{\left( \frac{k (k+1)}{2}\right)^2} + (k+1)^3
          & \text{\BB{by I.H.}} \\
          &= \left( \frac{k^2 (k+1)^2}{4}\right) + (k+1)^3 \\
          &=  \frac{k^4 + 2k^3+k^2}{4} +
          \frac{4k^3+12k^2+12k+4}{4} \\
          &=  \frac{k^4 + 6k^3+13k^2 + 4}{4} \\
          &= \frac{(k+1)^2 (k+2)^2}{4}\\
          &= \left(\frac{(k+1)(k+2)}{2}\right)^2
          & \qed
        \end{align*}

    \end{enumerate}

%%%%%%%%%%%%%
  \newpage %%%%%%%%%%%%%%%%%%%%%%%%%%%%%%%%%%%%%%%%%%%%%%%%%%%%%%%%%%%%%%%%%%%%
%%%%%%%%%%%%%

  \item Prove that \(\displaystyle n! > 2^n \quad \forall n > 4 \)

    \dt{Proof.}\\
    Base case: \(n = 5 \then 5! > 2^5 \equiv 120 > 32 \) \\
    Inductive hypothesis: \(\displaystyle
    k \geq n
    \then
    k! > 2^k
    \)

    Inductive case:
    \begin{align*}
      &(k+1)! > 2^{k+1} \\
      &\then \BB{k!}(k+1)
      \BB{> 2^{k}}(k+1)
      > 2^k(2)
      = 2^{k+1}
      \qquad \text{\BB{by I.H.}} \\
      &\thus k!(k+1) > 2^{k+1}
      & \qed
    \end{align*}

    \vspace{8em}

  \item A prime number \(p\) is a natural number greater than \(1\) where
    \(1|p\) and \(p|p\) and nothing else divides \(p\). In class we showed that
    if \(n^2\) is even, then \(n\) is even. Extend this to any prime number.

    Prove that if \(p\) is prime and \(p|n^2 \then p|n\).\\
    Equivalent Theorem: \(p \not| n \then p \not| n^2\).

    \dt{Proof.}
    \begin{align*}
      p \not| n &\then n = pq + r, \quad \exists r > 0 \\
                &\then n^2 = nn = (pq+r)(pq+r) = p(pq^2 + 2qr) + r^2\\
                &\then p \not| p(pq^2 + 2qr) + r^2 & \qed \\
    \end{align*}

%%%%%%%%%%%%%
  \newpage %%%%%%%%%%%%%%%%%%%%%%%%%%%%%%%%%%%%%%%%%%%%%%%%%%%%%%%%%%%%%%%%%%%%
%%%%%%%%%%%%%

  \item Prove:
    \[%%%%%%%%%%%%%%%%%%%%%%%%%%%%%%%%%%%%%%%%%%%%%%%%%%%%%%%%%%%%%%%%%%%%
      \sum_{i=1}^{n} ia^i = \frac{a-(n+1)a^{n+1} + na^{n+2}}{(a-1)^2}
    \]%%%%%%%%%%%%%%%%%%%%%%%%%%%%%%%%%%%%%%%%%%%%%%%%%%%%%%%%%%%%%%%%%%%%

    \dt{Proof.}\\
    Base case: \(n = 1 \then\)
    \begin{align*}
     a &= \frac{a - 2a^2 + a^3}{(a-1)^2} = \frac{a(a^2 - 2a + 1)}{a^2 - 2a + 1} = a
    \end{align*}
    Inductive hypothesis: \(\displaystyle
    n=k
    \then
      \sum_{i=1}^{k} ia^i = \frac{a-(k+1)a^{k+1} + ka^{k+2}}{(a-1)^2}
     \)

    Inductive case:
    \begin{align*}
      &p\sum_{i=1}^{k+1} ia^i = \frac{a-(k+1)a^{k+1} + ka^{k+2}}{(a-1)^2} + (k+1)a^{k+1}\\
      &\mathsmaller{\BB{\frac{a-(k+1+1)a^{k+1+1} + (k+1)a^{k+1+2}}{(a-1)^2}} =
      \frac{a-(k+1)a^{k+1} + ka^{k+2}}{(a-1)^2} + (k+1)a^{k+1}}
      ~\text{\BB{by I.H.}} \\
      &\mathsmaller{\frac{a-(k+1+1)a^{k+1+1} + (k+1)a^{k+1+2}}{(a-1)^2} -
      \frac{a-(k+1)a^{k+1} + ka^{k+2}}{(a-1)^2} - (k+1)a^{k+1} = 0}\\
      &(k+1)a^{k+1}  - (k+1)a^{k+1} = 0 \qquad~\text{by lots of algebra (used wolfram)}\\
      &0 = 0 & \qed
    \end{align*}

  \item Let’s prove some theorems about cardinality. For each of the following
    equations, give a bijection between the two sets.

    Hint: you can (should) reuse a bijection you've already defined.
    \begin{enumerate}[leftmargin=2em]

      \item \(\displaystyle |E| = |\N|, f: E \to \N, \quad f(e) = \frac{e}{2} \)

      \item \(\displaystyle |\N| = |\Z|, f: \N \to \Z, \quad f(n) = n+1\)

      \item \(\displaystyle |\N| = |\Q^+|, f: \N \to \Q^+, \quad f(n) = \frac{1}{n+1}\)

      \item \(\displaystyle |\Z| = |\Q|, f: \Z \to \Q, \quad f(z) = \frac{1}{z}\)

      \item \(\displaystyle |E| = |\Q|, f: E \to \Q, \quad f(e) = \frac{1}{e}\)

    \end{enumerate}

%%%%%%%%%%%%%
  \newpage %%%%%%%%%%%%%%%%%%%%%%%%%%%%%%%%%%%%%%%%%%%%%%%%%%%%%%%%%%%%%%%%%%%%
%%%%%%%%%%%%%

  \item Prove that \(\displaystyle \forall n, |\N| = |\N^n|\).

    You can assume that \(d : \N \to \N \times \N\) is the diagonal bijection
    from class.

    \dt{Proof.}\\
    Base case: \(|\N| = |\N \times \N|\) \prn{proved in class} \\
    Inductive hypothesis: \(\displaystyle
    n=k
    \then
    |\N| = |\N^k|
    \)

    Inductive case:
    \begin{align*}
      |\N^{k+1}   |
      &= |(\N_1 \times \ldots  \times \N_k) \times \N_{k+1}  |  \\
      &= |\BB{\N^k} \times \N_{k+1} | & \text{\BB{by I.H.}} \\
      &= |\N^{k+1} |
      & \qed
    \end{align*}

  \item Let's try to find a use for this infinity nonsense.

    In computer science it can be useful to look at problems as a language. A
    language is just a set of finite strings, so we can make a language
    describing $\pi$ as: \[L_\pi = \{"3", "3.1", "3.14", "3.141", \ldots\}\]

    So why do we care about languages?
    Well, we can phrase all of our problems in CS as different languages.
    For example $L_{factor} = \{"6=2\cdot3", "12=2\cdot2\cdot3", \ldots\}$
    is the language of numbers and their factors. We can make the language of
    graphs and their shortest paths, the languages of lists
    and their sorting. Really we can make a language for any problem.

    A language is decidable if there is a program that can (eventually)
    produce any string in that language. We want to prove that there is at
    least 1 undecidable language. That is, there is a problem that can’t be
    solved by a program.

    \begin{enumerate}[leftmargin=2em]

      \item Show that there are countably many programs we can write.

        Need to show: there exists an injective function from \(S \to \N\)

        Let \(f\) be a function that takes program and returns a binary
        sequence that represents the instructions of that program and vice
        versa.

        If two binary sequences are equal, then they must produce the same
        program, implying the function is injective \tqed


      \item Show that are uncountably many languages.

        Since \(\pi\) can be described above as \(L_\pi\), then two unique
        strings could easily map the same output. Thus, it cannot be an
        injective function and \(|L_\pi| > |N|\) \tqed


    \end{enumerate}

%%%%%%%%%%%%%
  \newpage %%%%%%%%%%%%%%%%%%%%%%%%%%%%%%%%%%%%%%%%%%%%%%%%%%%%%%%%%%%%%%%%%%%%
%%%%%%%%%%%%%

  \item Show why the following two lemmas are actually invalid.\\

    Note: the theorem is actually true despite this \prn{haha}

    Hint: part 1 is a problem with induction and part 2 is a problem with
    infinity.

    \dd{Lemma 0.1}. \textit{There are infinitely many green vegetables}.

    \dt{Proof.}

    Base case: There exists a head of lettuce.\\
    Inductive hypothesis: there is a set of \(k\) green vegetables.\\
    \RR{\(\uparrow\) ERROR: the inductive hypothesis must say something about
    the set of vegetables, not just that the set exists. This is \textbf{not a valid
    hypothesis}.}

    Inductive case: Suppose, by the inductive hypothesis, there are \(k\) green vegetables.

    Remember that head of lettuce from from the base case? Let’s add that into our set of k vegetables.
    Now we have a set of \(k + 1\) green vegetables.

    Therefore we must have a set of \(n\) vegetables for any \(n\), which means
    we have infinitely many green vegetables. \tqed

    \dd{Lemma 0.2}. \textit{Every green edible thing is a vegetable}.

    \dt{Proof.}

    Let’s consider the set of all possible green edible things \(G\). By the
    previous lemma \(G\) is clearly an infinite set. I’ll sort the set putting all
    of the green vegetables at the front. Now we want to show that there are
    ONLY vegetables in this set.\\
    \RR{\(\uparrow\) ERROR: (1) \(G\) is clearly not an infinite set. (2) You
    can't sort an infinite set, therefore, the following hypothesis is an
    impossible claim.}

    Base case: the first element is that dang head of lettuce from the last
    proof.\\ Inductive hypothesis: The first \(k\) elements of \(G\) are vegetables.\\
    Inductive case: Assume, by the inductive hypothesis, that the first \(k\)
    elements are vegetables. Then the next element must also be a vegetable. If
    it wasn't then there would only be a finite number of vegetables, but by
    the last theorem this is impossible.

    Therefore, the first \(k+1\) elements of \(G\) must be vegetables. Thus, for all
    n, the first n elements of \(G\) are vegetables. Now, since \(G\) is a countably
    infinite set, and we've shown that each position in the set is a vegetable,
    it must be the case that the set contains only vegetables.

    \dt{Theorem 0.2.1}: \textit{Mountain Dew is a vegetable}.

    \dt{Proof.}

    It’s green and edible. From the previous two lemmas, it’s clearly a vegetable. \tqed

\end{enumerate}

\end{document}
