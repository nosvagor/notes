\documentclass[basic, header]{nosvagor-notes}
\usepackage{nosvagor-math}

\colorlet{title-color}{red}
\newcommand{\theTitle}{%
  \href{https://github.com/nosvagor/notes}%
  {Homework 5}%
}

\newcommand{\userName}{Cullyn Newman}
\newcommand{\class}{CS: 250}
\newcommand{\institution}{Portland State}

\begin{document}

\begin{enumerate}[itemsep=4em]

  \item Use induction to prove the following summation formulas. Remember rules
    of logs. Specifically \(\log(ab) = \log(a) + \log(b)\).
    \begin{enumerate}[leftmargin=2em]
      \item \(\displaystyle \sum_{i=1}^{n} a_i + b_i = \sum_{i=1}^{n} a_i + \sum_{i=1}^{n} b_i  \)

      \item \(\displaystyle \sum_{i=1}^{n}ca_i = c \sum_{i=1}^{n} a_i \)

      \item \(\displaystyle \sum_{i=1}^{n} \log(a_i) = \log \left( \prod_{i=1}^{n} a \right) \)

      \item \(\displaystyle \sum_{i=1}^{n} 2i - 1 = n^2 \)

      \item \(\displaystyle \sum_{i=1}^{n} i^3 = \left( \frac{n(n+1)}{2} \right)^2 \)

    \end{enumerate}

%%%%%%%%%%%%%
  \newpage %%%%%%%%%%%%%%%%%%%%%%%%%%%%%%%%%%%%%%%%%%%%%%%%%%%%%%%%%%%%%%%%%%%%
%%%%%%%%%%%%%

  \item Prove that \(\displaystyle n! > 2^n \quad \forall n > 4 \)

  \item A prime number \(p\) is a natural number greater than \(1\) where
    \(1|p\) and \(p|p\) and nothing else divides \(p\). In class we showed that
    if \(n^2\) is even, then \(n\) is even. Extend this to any prime number.

    Prove that if \(p\) is prime and \(p|n^2\) then \(p|n\).

%%%%%%%%%%%%%
  \newpage %%%%%%%%%%%%%%%%%%%%%%%%%%%%%%%%%%%%%%%%%%%%%%%%%%%%%%%%%%%%%%%%%%%%
%%%%%%%%%%%%%

  \item Prove:
    \[%%%%%%%%%%%%%%%%%%%%%%%%%%%%%%%%%%%%%%%%%%%%%%%%%%%%%%%%%%%%%%%%%%%%
      \sum_{i=1}^{n} ia^i = \frac{a-(n+1)a^{n+1} + na^{n+2}}{(a-1)^2}
    \]%%%%%%%%%%%%%%%%%%%%%%%%%%%%%%%%%%%%%%%%%%%%%%%%%%%%%%%%%%%%%%%%%%%%

  \item Let’s prove some theorems about cardinality. For each of the following
    equations, give a bijection between the two sets.

    Hint: you can (should) reuse a bijection you've already defined.
    \begin{enumerate}[leftmargin=2em]

      \item \(\displaystyle |E| = |\N| \)

      \item \(\displaystyle |\N| = |\Z| \)

      \item \(\displaystyle |\N| = |\Q^+| \)

      \item \(\displaystyle |\Z| = |\Q| \)

      \item \(\displaystyle |E| = |\Q| \)

    \end{enumerate}

%%%%%%%%%%%%%
  \newpage %%%%%%%%%%%%%%%%%%%%%%%%%%%%%%%%%%%%%%%%%%%%%%%%%%%%%%%%%%%%%%%%%%%%
%%%%%%%%%%%%%

  \item Prove that \(\displaystyle \forall n, |\N| = |\N^n|\).

    You can assume that \(d : \N \to \N \times \N\) is the diagonal bijection
    from class.

  \item A language is decidable if there is a program that can (eventually)
    produce any string in that language. We want to prove that there is at
    least 1 undecidable language. That is, there is a problem that can’t be
    solved by a program.

    \begin{enumerate}[leftmargin=2em]

      \item Show that there are countably many programs we can write.

      \item Show that are uncountably many languages.

    \end{enumerate}

%%%%%%%%%%%%%
  \newpage %%%%%%%%%%%%%%%%%%%%%%%%%%%%%%%%%%%%%%%%%%%%%%%%%%%%%%%%%%%%%%%%%%%%
%%%%%%%%%%%%%

  \item Show why the following two lemmas are actually invalid.\\

    Note: the theorem is actually true despite this.

    Hint: part 1 is a problem with induction and part 2 is a problem with
    infinity.

    \dt{Lemma 0.1}. \textit{There are infinitely many green vegetables}.

    \dt{Proof.}

    Base case: There exists a head of lettuce.\\
    Inductive hypothesis: there is a set of \(k\) green vegetables.\\
    Inductive case: Suppose, by the inductive hypothesis, there are \(k\) green vegetables.

    Remember that head of lettuce from from the base case? Let’s add that into our set of k vegetables.
    Now we have a set of \(k + 1\) green vegetables.

    Therefore we must have a set of \(n\) vegetables for any \(n\), which means
    we have infinitely many green vegetables. \tqed

    \dt{Lemma 0.2}. \textit{Every green edible thing is a vegetable}.

    \dt{Proof.}

    Let’s consider the set of all possible green edible things G. By the
    previous lemma it’s clearly an infinite set. I’ll sort the set putting all
    of the green vegetables at the front. Now we want to show that there are
    ONLY vegetables in this set.

    Base case: the first element is that dang head of lettuce from the last
    proof.\\ Inductive hypothesis: The first k elements of G are vegetables.\\
    Inductive case: Assume, by the inductive hypothesis, that the first k
    elements are vegetables. Then the next element must also be a vegetable. If
    it wasn't then there would only be a finite number of vegetables, but by
    the last theorem this is impossible.

    Therefore, the first k + 1 elements of G must be vegetables. Thus, for all
    n, the first n elements of G are vegetables. Now, since G is a countably
    infinite set, and we've shown that each position in the set is a vegetable,
    it must be the case that the set contains only vegetables.

    \dt{Theorem 0.2.1}: \textit{Mountain Dew is a vegetable}.

    \dt{Proof.}

    It’s green and edible. From the previous two lemmas, it’s clearly a vegetable. \tqed

\end{enumerate}

\end{document}
