\documentclass[basic, header]{nosvagor-notes}
\usepackage{nosvagor-math}

\colorlet{title-color}{red}
\newcommand{\theTitle}{%
  \href{https://github.com/nosvagor/notes}%
  {Homework 4}%
}

\newcommand{\userName}{Cullyn Newman}
\newcommand{\class}{CS: 250}
\newcommand{\institution}{Portland State}

\begin{document}

\hrulefill
\vspace{0.5em}
  \begin{center}
  \dd{\large{Definitions}}
  \end{center}
  \vspace{-1em}
  \begin{align*}
    &\dd{(1)} \quad a|b \iff \exists c \in \Z \then b = ca \\
    &\dd{(2)} \quad a\%b = r \given \tfrac{a}{b} ~\text{has remainder}~ r \\
    &\dd{(3)} \quad a \equiv b \bmod n \iff n|b - a
  \end{align*}

  \dd{Theorem 4.1}: if \(n\) is even then \(n^2\) is even. \psrc{lecture}\\
  \dd{Theorem 4.2}: \(a|b \land a|c \then a|b+c\) \psrc{notes\_4.pdf}\\
  \dd{Theorem 4.3}: \(a \equiv a \% n \bmod n\) \psrc{notes\_4.pdf}\\

\hrulefill

\vspace{2em}

\begin{enumerate}[itemsep=8em]

  \item Prove that \(n^2 \neq 2 \bmod 3, \quad \forall n \in \Z \)


    \dt{Proof.}
    \begin{align*}
      &\forall n \in E, 2|n  \then 2|n^2 \quad \dt{by theorem 4.1} \\
      &2|n^2 = 2 \bmod 0 \neq 2 \bmod 3 \qed
    \end{align*}

  \item Fermat’s Last theorem is a famous theorem in Math that was unproven for
    200 years. The theorem says \(\forall n > 2,~ a,b,c \in \N \then  a^n +
    b^n \neq c^n\). Another way to state this is \(a^n + b^n = c^n\) has no integer
    solutions for $n$ larger than 2. Use this theorem to prove that \(\sqrt[n]{2}\) is
    irrational for $n$ larger than 2.

    \dt{Proof.}
    \begin{align*}
    \sqrt[n]{2} \in \Q &\then \exists a,b \in \Z : \gcd{(a,b)}=1 \\
                         &\then \sqrt[n]{2} = \frac{a}{b} \then a^n = 2b^n \\
                         &\then a^n = b^n + b^n \qed
    \end{align*}

    \textit{Note: this is essentially
      \link{https://math.stackexchange.com/q/783392}{zscoder's proof}. No real
      credit here; I couldn't figure it out myself at first. It's pretty simple
      though, so I couldn't formulate something else that was better without
      adding unnecessary steps (originally completed in hwy)}.

%%%%%%%%%%%%%
  \newpage %%%%%%%%%%%%%%%%%%%%%%%%%%%%%%%%%%%%%%%%%%%%%%%%%%%%%%%%%%%%%%%%%%%%
%%%%%%%%%%%%%

    \item Prove \(\forall a,b,c\in \Z : a|b \land a|c \then a|bx + cy \quad
      \forall x,y \in \Z\)

      \dt{Proof.}
      \begin{align*}
        b &= qa, \quad c = qa \quad \dt{by definition 1} \\
            &\then a|qax+qay = a|a(qx+qy) = a|qa \qed
      \end{align*}

    \item Prove \(\forall  n,a,b \in \Z, n|a-b \iff a\%n = b\%n\)

      \dt{Proof.}
      \begin{align*}
        a\%n = b\%n &\iff \exists q \in \Z : \frac{a}{n} = \frac{qb}{n} \\
                    &\then a=qb \\
                    &\then n|qb-b = n|b(q-1) \qed
      \end{align*}
      \dt{Proof by contradiction.}
      \begin{align*}
        a\%n \neq b\%n &\then \exists q \not\in \Z : \frac{a}{n} = \frac{qb}{n} \\
                       &\then a \neq qb \qed
      \end{align*}
      Thus, if \(a\%n = b\%n\) then one integer is guaranteed to be a
      multiple of the other, which must be true for \(a-b\) to be divisible by
      \(n\). Alternatively, a contradiction arises because every integer should
      be able to be represented as a multiple of some other integer.

%%%%%%%%%%%%%
  \newpage %%%%%%%%%%%%%%%%%%%%%%%%%%%%%%%%%%%%%%%%%%%%%%%%%%%%%%%%%%%%%%%%%%%%
%%%%%%%%%%%%%

    \item Let \(a, b \in \Z, n \in \N\). Prove that \[a \sim b \given a \equiv
      b \bmod n\]
      is an
      \link{https://en.wikipedia.org/wiki/Equivalence_relation}{equivalence~relation}
      for any \(n\).

      \dt{Proof.}
      \begin{align*}
        a\sim a \land b \sim b \quad \dt{by theorem 4.3} &\then \text{\true{reflexive}} \\\\
        n|b-a = n|a-b &\then
        a \sim b \iff b \sim a
        \quad \dt{by definition 3} \\
                         &\then \text{\true{symmetric}} \\\\
        n|b-a &\then a \sim b \land n|c-b \\
              \then b \sim c \then n|c-a &\then a \sim c \\
                         &\then \text{\true{transitive}} \qed
      \end{align*}

\vspace{-7em}

    \item The greatest common divisor of natural numbers \(a, b; \gcd(a,b) \),
      is the largest number \(\delta\) such that \(\delta|a \land \delta|b\)
      \begin{enumerate}

        \item Let \(\delta = \gcd(b, a\%b)\), prove that \(\delta|a \land
          \delta|b\)

          \dt{Proof.}
          \begin{align*}
            a\%b = 0 &\then a|b,~\gcd{(b,0)} = b \\
                     &\then b = \delta,~ b = ca \\
                     &\then \delta|b,~ \delta|ca \\
                     &\then \delta|b \land \delta|a
          \end{align*}
          \begin{align*}
            a\%b \neq 0 &\then a\%b = r \quad \dt{by definition 2} \\
                        &\then r|b-a \quad \dt{by definition 3} \\
                        &\then r|a \land r|b \quad \dt{by question 3} \\
                        \delta | r \quad \dt{by definition of gcd}
                        &\then \delta|a \land \delta|b \qed
          \end{align*}

        \item Use (a) to show that \(\gcd(a,b) = \gcd(b, a\%b)\)

          \dt{Proof.}
          \begin{align*}
            a\%b = 0 &\then a \leq b, \delta = \max(a,b) = b \quad \dt{by part (a)} \\
            a\%b \neq 0 &\then \delta|r, 0 < r < a \leq b \quad \dt{by part (a)} \\
            &\then \delta = \max(b, r) = b \qed
          \end{align*}

      \end{enumerate}

%%%%%%%%%%%%%
  \newpage %%%%%%%%%%%%%%%%%%%%%%%%%%%%%%%%%%%%%%%%%%%%%%%%%%%%%%%%%%%%%%%%%%%%
%%%%%%%%%%%%%

    \item We defined the identity function
      \[%%%%%%%%%%%%%%%%%%%%%%%%%%%%%%%%%%%%%%%%%%%%%%%%%%%%%%%%%%%%%%%%%%%%
       \id : A \to A, \quad \id(x) = x, \quad  \text{has property:}~\forall f :
       A\to A,~ \id \circ f = f \circ \id = f
      \]%%%%%%%%%%%%%%%%%%%%%%%%%%%%%%%%%%%%%%%%%%%%%%%%%%%%%%%%%%%%%%%%%%%%
      Prove that \(\id\) is the only function that can have this property.

      \dt{Proof by contradiction.}
      \begin{align*}
        \quad g \neq \id, \forall g : A \to A \then \forall a \in A : g(a)\not\in A \land \id(a) \in A \qed
      \end{align*}

      I.e., there is no other distinct function that can map an element to
      itself that isn't already mapped to itself by the identity function.

\end{enumerate}

\end{document}
