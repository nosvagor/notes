\documentclass[basic, header]{nosvagor-notes}
\usepackage{nosvagor-math}

\colorlet{title-color}{red}
\newcommand{\theTitle}{%
  \href{https://github.com/nosvagor/notes}%
  {Homework 4}%
}

\newcommand{\userName}{Cullyn Newman}
\newcommand{\class}{CS: 250}
\newcommand{\institution}{Portland State}

\begin{document}

\begin{enumerate}[itemsep=4em]

  \item Prove that \(n^2 \neq 2 \bmod 3, \quad \forall n \in \Z \)

    \dd{Theorem 1}: if \(n\) is even then \(n^2\) is even.

    \dt{Proof by}:
    \begin{align*}
      &\forall n \in E, 2|n  \then 2|n^2 \quad \dt{by theorem 1} \\
      &2|n^2 = 2 \bmod 0 \neq 2 \bmod 3 \qed
    \end{align*}

  \item Fermat’s Last theorem is a famous theorem in Math that was unproven for
    200 years. The theorem says \(\forall n > 2,~ a,b,c \in \N \then  a^n +
    b^n \neq c^n\). Another way to state this is \(a^n + b^n = c^n\) has no integer
    solutions for $n$ larger than 2. Use this theorem to prove that \(\sqrt[n]{2}\) is
    irrational for $n$ larger than 2.

    \dt{Proof by}:
    \begin{align*}
    \sqrt[n]{2} \in \Q &\then \exists a,b \in \Z : \gcd{(a,b)}=1 \\
                         &\then \sqrt[n]{2} = \frac{a}{b} \then a^n = 2b^n \\
                         &\then a^n = b^n + b^n \qed
    \end{align*}
    Thus, this contradicts Fermat's Last theorem implying \(\sqrt[n]{2}\) is
    irrational for \(n > 2\).

    \textit{Note: this is essentially
      \link{https://math.stackexchange.com/q/783392}{zscoder's proof}. No real
      credit here; I couldn't figure it out myself at first. It's pretty simple
      though, so I couldn't formulate something else that was better without
      adding unnecessary steps (originally completed in hw3)}.

    \item Prove that for any \(a,b,c \in \Z, \exists x,y \in \Z : a|bx+cy
      \given a|b \land a|c\)

    \item Prove that for any \( n,a,b \in \Z, n|a-b \iff a\%n = b\%n\)

%%%%%%%%%%%%%
  \newpage %%%%%%%%%%%%%%%%%%%%%%%%%%%%%%%%%%%%%%%%%%%%%%%%%%%%%%%%%%%%%%%%%%%%
%%%%%%%%%%%%%

    \item Let \(a, b \in \Z, n \in \N\). Prove that \[a \sim b \given a \equiv
      b \bmod n\]
      is a n
      \link{https://en.wikipedia.org/wiki/Equivalence_relation}{equivalence~relation}
      for any \(n\).

    \item The greatest common divisor of natural numbers \(a, b; \gcd(a,b) \),
      is the largest number \(\delta\) such that \(\delta|a \land \delta|b\)
      \begin{enumerate}

        \item Let \(\delta = \gcd(b, a\%b)\), prove that \(\delta|a \land
          \delta|b\)

        \item Use part (a) to show that \(\gcd(a,b) = \gcd(b, a\%b)\)

      \end{enumerate}

%%%%%%%%%%%%%
  \newpage %%%%%%%%%%%%%%%%%%%%%%%%%%%%%%%%%%%%%%%%%%%%%%%%%%%%%%%%%%%%%%%%%%%%
%%%%%%%%%%%%%

    \item We defined the identity function
      \[%%%%%%%%%%%%%%%%%%%%%%%%%%%%%%%%%%%%%%%%%%%%%%%%%%%%%%%%%%%%%%%%%%%%
       \id : A \to A, \quad \id(x) = x, \quad  \text{has property:}~\forall f :
       A\to A,~ \id \circ f = f \circ \id = f
      \]%%%%%%%%%%%%%%%%%%%%%%%%%%%%%%%%%%%%%%%%%%%%%%%%%%%%%%%%%%%%%%%%%%%%
      Prove that \(\id\) is the only function that can have this property.

\end{enumerate}

\end{document}
