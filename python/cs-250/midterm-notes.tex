\documentclass[basic]{nosvagor-notes}
\usepackage{nosvagor-math}

\colorlet{title-color}{red}
\newcommand{\theTitle}{%
  \href{https://github.com/nosvagor/notes}%
  {Midterm Notes}%
}

% \newcommand{\userName}{Cullyn Newman}
% \newcommand{\class}{CS: 250}
% \newcommand{\institution}{Portland State}

\topmargin = -5em
\hoffset = -4.5em
\headsep = 0pt
\headheight = 0pt
\textheight = 800pt
\textwidth = 575pt
\footskip = 3.5em

\begin{document}

\begin{multicols}{2}
  \dd{Union \(S \cup T\)}: \(\left\{ x : x \in S \lor x \in T \right\} \)\\
  \dd{Difference \(S \setminus T\)}: \(\left\{ x : x \in S \land x \not\in T \right\} \)\\
  \dd{Intersection \(S \cap T\)}: \(\left\{ x : x \in S \land x \in T \right\} \)\\
  \dd{Complement \(S'\)}: \(\left\{ x : x \not\in S \land x \not\in T \right\} \)
\end{multicols}

  \dd{Total (\link{https://en.wikipedia.org/wiki/Partial_function}{total}
      )}: \(\forall x \in A \then f(x) \) is defined\\
  \dd{1 to 1
        (\link{https://en.wikipedia.org/wiki/Injective_function}{injective})}:
        \(\forall x,y \in X, \quad f(x) = f(y) \then x = y\)\\
  \dd{Onto
         (\link{https://en.wikipedia.org/wiki/Surjective_function}{surjective})}:
         \(f : X \to Y, \quad \forall y \in Y, \quad \exists x
        \in X \then f(x) = y \)\\
  \dd{Composition}: \((g\circ f)(x) = g(f(x))\) \hspace{3.3em}
  \dd{Inverse}: \(f^{-1}(x) : f  \circ f^{-1} = f^{-1} \circ f = \id\)

  \dd{\link{https://en.wikipedia.org/wiki/Equivalence_relation}{Equivalence}}:
  \(\sim, \equiv \iff\) a relation is reflexive, symmetric (\dd{partial}: antisymmetric), and transitive.\\
  \dd{\link{https://en.wikipedia.org/wiki/Reflexive_relation}{Reflexive}}:
  \(\forall a \in X, \quad a \sim a\)\\
  \dd{\link{https://en.wikipedia.org/wiki/Symmetric_relation}{Symmetric}}:
  \(\forall a,b \in X, \quad a \sim b \iff b \sim a \)\\
  \dd{\link{https://en.wikipedia.org/wiki/Antisymmetric_relation}{Antisymmetric}}:
  \(\forall a,b \in X, \quad a \sim b, a \neq b \then b \nsim a ~\text{\ldots equiv\ldots}~ a \sim b, b \sim a \then a = b\)\\
  \dd{\link{https://en.wikipedia.org/wiki/Transitive_relation}{Transitive}}:
  \(\forall a,b,c \in X, \quad : a \sim b, b \sim c \then a \sim c \)

  \dd{Direct}: using previous theorem or definition \hspace{2.3em}
  \dd{Contradiction}: assume false, find contradiction\\
  \dd{Contrapositive}: \(A \then B = \lnot B \then \lnot A\)\qquad
  \dd{Cases}: prove all possible cases.\\
  \dd{Generalization}: prove \(\forall x\) pick arbitrary \(x\). WLOG = swap
  variables same difference.\\
  \dd{Prove set}: \(A = B \iff A \subseteq B \land B \subseteq A\)

  \begin{center}
  \hspace{-3em}\(a,b,c,q,r,x,y \in \Z \Downarrow\)
  \end{center}
  \vspace{-0.8em}
  \begin{multicols}{2}
  \(b = qa + r \given \exists q, r : 0 \leq r < b\)\\
  \(a|b \iff \exists q \then b = qa\)\\
  \(a\%b = r \given \tfrac{a}{b} ~\text{has remainder}~ r \)\\
  \(a \equiv b \bmod n \iff n|b - a \)\\
  \dd{Theorem}: \(a|b \land a|c \then a|bx+cy\\
  \dd{Theorem}: \(a \equiv a \% n \bmod n, \quad n|qn\)
  \end{multicols}
  \hrule

  \begin{multicols}{2}

  Prove \(\sqrt[n]{2}\) is irrational for \(n > 2\)\\
  \dt{Proof.}
  \begin{align*}
  \sqrt[n]{2} \in \Q &\then \exists a,b \in \Z : \gcd{(a,b)}=1 \\
                       &\then \sqrt[n]{2} = \frac{a}{b} \then a^n = 2b^n \\
                       &\then a^n = b^n + b^n \qed
  \end{align*}

  Prove \(\forall  n,a,b \in \Z, n|a-b \iff a\%n = b\%n\)
    \dt{Proof.}
    \begin{align*}
      a\%n = b\%n &\iff \exists q \in \Z : \frac{a}{n} = \frac{qb}{n} \\
                  &\then a=qb \\
                  &\then n|qb-b = n|b(q-1) \qed
    \end{align*}
    \dt{Proof by contradiction.}
    \begin{align*}
      a\%n \neq b\%n &\then \exists q \not\in \Z : \frac{a}{n} = \frac{qb}{n} \\
                     &\then a \neq qb \qed
    \end{align*}

  The greatest common divisor of natural numbers \(a, b; \gcd(a,b) \),
      is the largest number \(\delta\) such that \(\delta|a \land \delta|b\)
      \begin{enumerate}[label=(\alph*)]

        \item Let \(\delta = \gcd(b, a\%b)\), prove that \(\delta|a \land
          \delta|b\)

          \dt{Proof.}
          \begin{align*}
            a\%b = 0 &\then a|b,~\gcd{(b,0)} = b \\
                     &\then b = \delta,~ b = ca \\
                     &\then \delta|b,~ \delta|ca \\
                     &\then \delta|b \land \delta|a
          \end{align*}
          \begin{align*}
            a\%b \neq 0 &\then a\%b = r \quad \dt{by definition 2} \\
                        &\then r|b-a \quad \dt{by definition 3} \\
                        &\then r|a \land r|b \quad \dt{by question 3} \\
                        \delta | r
                        &\then \delta|a \land \delta|b \qed
          \end{align*}

        \item Use (a) to show that \(\gcd(a,b) = \gcd(b, a\%b)\)

          \dt{Proof.}
          \begin{align*}
           \hspace{-1em} a\%b = 0 &\then a \leq b, \delta = \max(a,b) = b \quad \dt{by (a)} \\
           \hspace{-1em} a\%b \neq 0 &\then \delta|r, 0 < r < a \leq b \quad \dt{by (a)} \\
            &\then \delta = \max(b, r) = b \qed
          \end{align*}

      \end{enumerate}

  \end{multicols}
\end{document}
