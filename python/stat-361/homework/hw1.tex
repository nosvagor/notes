\documentclass[basic, header]{nosvagor-notes}
\usepackage{nosvagor-math}

\colorlet{title-color}{red}
\newcommand{\theTitle}{%
  \href{https://github.com/nosvagor/notes}%
  {Homework 1}%
}

\newcommand{\userName}{Cullyn Newman}
\newcommand{\class}{STAT: 361}
\newcommand{\institution}{Portland State}

\begin{document}

\begin{enumerate}
  \item (2.3) Which of the following events are equal?
    \begin{enumerate}
      \item \(A = \left\{ 1,3 \right\} \)

      \item \(B = \left\{ x : x ~\text{is a number on a die}\right\} = \left\{ 1,2,3,4,5,6 \right\} \)

      \item \(C = \left\{ x : x^2 - 4x + 3 = 0 \right\} = \left\{ x : (x-1)(x-3) = 0\right\}  = \left\{ 1,3 \right\}\)

      \item \(D = \left\{ x : x ~\text{numbers of heads when six coins are tossed}~ \right\} = \left\{ 0,1,2,3,4,5,6 \right\} \)
      \[%%%%%%%%%%
        \boxed{A = C}
      \]%%%%%%%%%%

    \end{enumerate}

 \item (2.6) Two jurors are selected from 4 alternates to serve at a murder
   trial. Using the notation \(A_1A_3\), for example, to denote the simple
   event that alternates 1 and 3 are selected, list the 6 elements of the
   sample space \(\Omega\).
   \[%%%%%%%%%%
     \boxed{\Omega = \left\{ A_1A_2, A_1A_3, A_1A_4, A_2A_3, A_2A_4, A_3A_4 \right\}}
   \]%%%%%%%%%%

  \item (2.10)  An engineering firm is hired to determine if certain waterways
    in Virginia are safe for fishing. Samples are taken from three rivers.
    \begin{enumerate}
      \item  List the elements of a sample space S, using the
        letters F for safe to fish and N for not safe to fish.
        \[%%%%%%%%%%
         \boxed{S = \{FFF, FFN, FNF, NFF, FNN, NFN, NNF, NNN\}}
        \]%%%%%%%%%%

      \item List the elements of S corresponding to event E
        that at least two of the rivers are safe for fishing.
        \[%%%%%%%%%%
          \boxed{E = \left\{ FFF, FFN, FNF, NFF \right\}}
        \]%%%%%%%%%%

      \item  Define an event that has as its elements the points
        \[%%%%%%%%%%
          \left\{ FFF, NFF, FFN, NFN \right\} \to E = ~\text{River 2 is safe}~
        \]%%%%%%%%%%
    \end{enumerate}

  \item (2.14) If \[S = \{0, 1, 2, 3, 4, 5, 6, 7, 8, 9\}\] \[A = \{0, 2, 4, 6,
    8\}, B = \{1, 3, 5, 7, 9\}\] \[C = \{2, 3, 4, 5\}, \[D = \{1, 6, 7\},\]
    list the elements of the sets corresponding to the following events:

    \begin{enumerate}
      \item \(A \cup C = \boxed{\left\{ 0,2,3,4,5,6,8 \right\} }\)
      \item \(A \cap B = \boxed{~\nil}\)
      \item \(C^{'} = \boxed{\left\{ 0,1,6,7,8,9 \right\}} \)

    \end{enumerate}

%%%%%%%%%%%%%
  \newpage %%%%%%%%%%%%%%%%%%%%%%%%%%%%%%%%%%%%%%%%%%%%%%%%%%%%%%%%%%%%%%%%%%%%
%%%%%%%%%%%%%

  \item (2.18) 2.18 Which of the following pairs of events are mutually
    exclusive?
    \begin{enumerate}
      \item[(c)]  A mother giving birth to a baby girl and a set of
        twin daughters on the same day.
        \begin{itemize}
          \item \textbf{Not mutually exclusive}.
        \end{itemize}

      \item[(d)]  A chess player losing the last game and winning the match.
        \begin{itemize}
          \item Ambiguous. Depends on how winning a match is defined. Assuming
            winning a match is \textit{first one} to say, 3 wins, then yes, it's \textbf{mutually
            exclusive}.
        \end{itemize}

    \end{enumerate}

  \item (2.22) In a medical study, patients are classified in 8 ways according to
    whether they have blood type AB+, AB-, A+, A-, B+, B-, O+, or O-, and also
    according to whether their blood pressure is low, normal, or high. Find
    the number of ways in which a patient can be classified.
    \[%%%%%%%%%%
      (8)(3) = \boxed{24} \qquad \text{using Rule 2.1}
    \]%%%%%%%%%%

  \item (2.27)  developer of a new subdivision offers a prospective home buyer
    a choice of 4 designs, 3 different heating systems, a garage or carport,
    and a patio or screened porch. How many different plans are available to
    this buyer?
    \[%%%%%%%%%%
      (4)(3)(2)(2) = \boxed{48} \qquad \text{using Rule 2.2}
    \]%%%%%%%%%%

\end{enumerate}

\end{document}
