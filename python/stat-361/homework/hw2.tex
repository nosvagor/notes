\documentclass[basic, header]{nosvagor-notes} \usepackage{nosvagor-math}

\colorlet{title-color}{red}
\newcommand{\theTitle}{%
  \href{https://github.com/nosvagor/notes}%
  {Homework 1}%
}

\newcommand{\userName}{Cullyn Newman}
\newcommand{\class}{STAT: 361}
\newcommand{\institution}{Portland State}

\begin{document}

\begin{enumerate}[itemsep=8em]

  \item (2.49) Find the errors in each of the following statements:
  \begin{enumerate}

    \item The probabilities that an automobile salesperson will sell 0, 1, 2,
      or 3 cars on any given day in February are, respectively, 0.19, 0.38,
      0.29, and 0.15.
      \begin{itemize}
        \item Error: \(P(\Omega) \nleq 1; 0.19 + 0.38 + 0.29 + 0.15 = 1.01\)

      \end{itemize}


    \item The probability that it will rain tomorrow is 0.40, and the
      probability that it will not rain tomorrow is 0.52.
      \begin{itemize}
        \item Error: \(P(A) + P(A^{'}) \neq 1; 0.40 + 0.52 = 0.92 \)

      \end{itemize}


    \item The probabilities that a printer will make 0, 1, 2, 3, or 4 or more
      mistakes in setting a document are, respectively, 0.19, 0.34, -0.25,
      0.43, and 0.29.
      \begin{itemize}
        \item Error: \(P(E) \ngeq 0; -0.25 < 0\)

      \end{itemize}

    \item On a single draw from a deck of playing cards, the probability of
      selecting a heart is 1/4, the probability of selecting a black card is
      1/2, and the probability of selecting both a heart and a black card is
      1/8.
      \begin{itemize}
        \item Error: \(P(\blacksquare) \cap P(\text{\RR{\faHeart}}) = \nil; P(E \in \F) = 0 \neq \frac{1}{8}\)

      \end{itemize}

  \end{enumerate}

  \item (2.53) The probability that an American industry will locate in
    Shanghai, China, is 0.7, the probability that it will locate in Beijing,
    China, is 0.4, and the probability that it will locate in either Shanghai
    or Beijing or both is 0.8. What is the probability that the industry will
    locate
    \[%%%%%%%%%%
    : P(S) = 0.7, \quad P(C) = 0.4, \quad P(S \cup C) = 0.8
    \]%%%%%%%%%%
    \begin{enumerate}
      \item in both cities?
        \begin{itemize}
          \item \(P(S \cap C) = 0.7 + 0.4 - 0.8 = 0.3\)

        \end{itemize}

      \item in neither city?
        \begin{itemize}
          \item \(P(S^{'} \cap C^{'}) = 1 - 0.8 = 0.2\)

        \end{itemize}

    \end{enumerate}

%%%%%%%%%%%%%
  \newpage %%%%%%%%%%%%%%%%%%%%%%%%%%%%%%%%%%%%%%%%%%%%%%%%%%%%%%%%%%%%%%%%%%%%
%%%%%%%%%%%%%

  \item (2.54) From past experience, a stockbroker believes that under
    present economic conditions a customer will invest in tax-free bonds with a
    probability of 0.6, will invest in mutual funds with a probability of 0.3,
    and will invest in both tax-free bonds and mutual funds with a probability
    of 0.15. At this time, find the probability that a customer will invest
    \[%%%%%%%%%%
      : P(T) = 0.6, \quad P(M) = 0.3, \quad P(T \cap M) = 0.15
    \]%%%%%%%%%%
    \begin{enumerate}
      \item in either tax-free bonds or mutual funds;
        \begin{itemize}
          \item \(P(T \cup M) = 0.6 + 0.3 - 0.15 = 0.75\)
        \end{itemize}

      \item in neither tax-free bonds nor mutual funds.
        \begin{itemize}
          \item \(P(T^{'} \cap M^{'}) = 1 - 0.75 = 0.25\)


        \end{itemize}

    \end{enumerate}

    \vspace{-5em}

  \item (2.55) If each coded item in a catalog begins with 3 distinct
    letters followed by 4 distinct nonzero digits, find the probability of
    randomly selecting one of these coded items with the first letter a vowel
    and the last digit even.
      \begin{align*}
          N  &= {}_{26}P_3 \cdot {}_9P_4
             = \frac{26!}{(26-3)!}\cdot \frac{9!}{(9-4)!}
             = \num{47174400}\\
          n &= 5 \cdot {}_{25}P_2 \cdot 4 \cdot {}_8P_3 = \num{4032000} \\
          P\left(\frac{n}{N}\right)  &= 0.085
      \end{align*}

    \vspace{-5em}

  \item (2.61) In a high school graduating class of 100 students, 54
    studied mathematics, 69 studied history, and 35 studied both mathematics
    and history. If one of these students is selected at random, find the
    probability that
    \[%%%%%%%%%%
      P(M) = 0.54, \quad P(H) = 0.69, \quad P(M\cap H) = 0.35
    \]%%%%%%%%%%
    \begin{enumerate}
      \item the student took mathematics or history;
        \begin{itemize}
          \item \(P(M \cup H) = 0.54 + 0.69 - 0.35 = 0.88\)

        \end{itemize}

      \item the student did not take either of these subjects;
        \begin{itemize}
          \item \(P(M' \cap H') = 1 - 0.88 = 0.12\)

        \end{itemize}

      \item the student took history but not mathematics.
        \begin{itemize}
          \item \(P(H \cap M') = P(H) - P(H \cap M) = 0.69 - 0.35 = 0.34 \)

        \end{itemize}

    \end{enumerate}

  \item (2.66) Factory workers are constantly encouraged to practice zero
    tolerance when it comes to accidents in factories. Accidents can occur
    because the working environment or conditions themselves are unsafe. On
    the other hand, accidents can occur due to carelessness or so-called human
    error. In addition, the worker’s shift, 7:00 A.M.--3:00 P.M. (day shift),
    3:00 P.M.--11:00 P.M. (evening shift), or 11:00 P.M.--7:00 A.M. (graveyard
    shift), may be a factor. During the last year, 300 accidents have
    occurred. The percentages of the accidents for the condition combination
    are as follows:
    \begin{table}[h]
      \centering
      \begin{tabular}{lcc}
        \toprule
         Shift      & Unsafe Conditions & Human Error \\
        \midrule
          Day       &               5\% & 32\%        \\
          Evening   &               6\% & 25\%        \\
          Graveyard &               2\% & 30\%        \\
        \bottomrule
      \end{tabular}
    \end{table}

    If an accident report is selected randomly from the 300 reports,
    \begin{enumerate}

     \item what is the probability that the accident occurred on the
     graveyard shift?
     \begin{itemize}
       \item \(P(G) = 0.3 + 0.02 = 0.32\)

     \end{itemize}

     \item what is the probability that the accident occurred
     due to human error?
     \begin{itemize}
       \item \(P(H) = 0.32 + 0.25 + 0.3 = 0.87\)

     \end{itemize}

     \item what is the probability that the accident occurred
     due to unsafe conditions?
     \begin{itemize}
       \item \(P(U) = 0.05 + 0.06 + 0.02 = 0.13\)

     \end{itemize}

     \item what is the probability that the accident
     occurred on either the evening or the graveyard shift?
     \begin{itemize}
       \item \(P(E \cup G) = P(D') = 1 - (0.05 + 0.32) = 0.63\)

     \end{itemize}

    \end{enumerate}

%%%%%%%%%%%%%
  \newpage %%%%%%%%%%%%%%%%%%%%%%%%%%%%%%%%%%%%%%%%%%%%%%%%%%%%%%%%%%%%%%%%%%%%
%%%%%%%%%%%%%

  \item (2.69) It is common in many industrial areas to use a filling
    machine to fill boxes full of product. This occurs in the food industry as
    well as other areas in which the product is used in the home, for example,
    detergent. These machines are not perfect, and indeed they may \(A\), fill to
    specification, \(B\), underfill, and \(C\), overfill. Generally, the practice of
    underfilling is that which one hopes to avoid. Let \(P(B) = 0.001\) while
    \(P(A) = 0.990\)
    \begin{enumerate}

      \item Give \(P(C)\).
        \begin{itemize}
          \item \(P(C) = P((A \cup B)') = 1 - (0.99 + 0.001) = 0.009\)

        \end{itemize}

      \item What is the probability that the machine does not underfill?
        \begin{itemize}
          \item \(P(B') = 1 - 0.001 = 0.999\)

        \end{itemize}

      \item  What is the probability that the machine either overfills or
        underfills?
        \begin{itemize}
          \item \(P(A') = 1 - 0.99 = 0.01 \)

        \end{itemize}

    \end{enumerate}



  \item (2.76) In an experiment to study the relationship of hypertension
    and smoking habits, the following data are collected for 180 individuals:
    \begin{table}[h]
      \centering
      \begin{tabular}{lccc}
        \toprule
           & Nonsmokers \(A\) & Moderate \(B\) & Heavy \(C\) \\
        \midrule
        H  &         21 & 36       &    30 \\
        H' &         48 & 26       &    19 \\
        \bottomrule
      \end{tabular}
    \end{table}

    where H and H' in the table stand for Hypertension and Non-hypertension,
    respectively. If one of these individuals is selected at random, find the
    probability that the person is
    \[%%%%%%%%%%
      P(H) = \frac{87}{180}, \quad P(H') = \frac{93}{180}
    \]%%%%%%%%%%

    \begin{enumerate}
      \item experiencing hypertension, given that the person is a heavy smoker;
        \begin{itemize}
          \item \(\displaystyle P(H \mid C) = \frac{P(H\cap C)}{P(C)} = \frac{30}{49} \)

        \end{itemize}
      \item a nonsmoker, given that the person is experiencing no hypertension.
        \begin{itemize}
          \item \(\displaystyle P(A \mid H' ) = \frac{P(A \cap H')}{P(H')} = \frac{48}{93} \)

        \end{itemize}
    \end{enumerate}


  \item (2.80) The probability that an automobile being filled with
    gasoline also needs an oil change is 0.25; the probability that it needs
    a new oil filter is 0.40; and the probability that both the oil and the
    filter need changing is 0.14.
    \[%%%%%%%%%%
      P(O) = 0.25, \quad P(F) = 0.40, \quad P(O \cap F) = 0.14
    \]%%%%%%%%%%
    \begin{enumerate}

      \item If the oil has to be changed, what is the probability that a new
        oil filter is needed?
        \begin{itemize}
          \item \(\displaystyle P(F \mid O) = \frac{0.14}{0.25} = 0.56 \)

        \end{itemize}

      \item If a new oil filter is needed, what is the probability that the
        oil has to be changed?
        \begin{itemize}
          \item \(\displaystyle P(O \mid F) = \frac{0.14}{0.40} = 0. \)

        \end{itemize}

    \end{enumerate}

    \vspace{-4em}
  \item (2.126) During bad economic times, industrial workers are
    dismissed and are often replaced by machines. The history of 100 workers
    whose loss of employment is attributable to technological advances is
    reviewed. For each of these individuals, it is determined if he or she was
    given an alternative job within the same company, found a job with another
    company in the same field, found a job in a new field, or has been
    unemployed for 1 year. In addition, the union status of each worker is
    recorded. The following table summarizes the results.
    \begin{table}[htpb]
      \centering
      \begin{tabular}{lcc}
        \toprule
                                  & \(U\) Union & \(U'\) Non-union \\
        \midrule
         \(S\) Same Company             &    40 & 15        \\
         \(C\) New Company (same field) &    13 & 10        \\
         \(F\) New Field                &     4 & 11        \\
         \(N\) Unemployed               &     2 & 5         \\
        \bottomrule
      \end{tabular}
    \end{table}
    \vspace{-1em}
      \[%%%%%%%%%%
        P(U) = 59
      \]%%%%%%%%%%
    \begin{enumerate}

      \item  If the selected worker found a job with a new company in the
        same field, what is the probability that the worker is a union member?
        \begin{itemize}
          \item \(\displaystyle P(U \mid C) = \frac{13}{13+10} = 0.5652 \)

        \end{itemize}
      \item  If the worker is a union member, what is the probability that
        the worker has been unemployed for a year?
        \begin{itemize}
          \item \(\displaystyle P(N \mid U) = \frac{2}{59} = 0.034\)

        \end{itemize}

    \end{enumerate}

\end{enumerate}

\end{document}
