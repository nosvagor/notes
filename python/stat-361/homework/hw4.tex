\documentclass[basic, header]{nosvagor-notes}
\usepackage{nosvagor-math}

\colorlet{title-color}{red}
\newcommand{\theTitle}{%
  \href{https://github.com/nosvagor/notes}%
  {Homework 4}%
}

\newcommand{\userName}{Cullyn Newman}
\newcommand{\class}{STAT: 361}
\newcommand{\institution}{Portland State}

\begin{document}

\begin{enumerate}[leftmargin=1.5em, itemsep=2em]

  \item \textbf{4.35} The random variable X, representing the number of
    errors per 100 lines of software code, has the following probability
    distribution:

    \begin{table}[h]
      \centering
      \begin{tabular}{c|ccccc}
        x &  2 & 3 & 4  & 5 & 6 \\
        \midrule
        \(f(x)\)  & 0.01 & 0.25 &0.4 & 0.3 & 0.04 \\
      \end{tabular}
    \end{table}

    Using Theorem 4.2 \(\BB{(\sigma^2 = E(X^2) - \mu^2) }\), find the
    variance of \(X\).
    \begin{align*}
      \mu &= \sum_{x=2}^{6} xf(x) = 4.11, \quad
      E(X^2) = \sum_{x=2}^{6} x^2 f(x) = 17.63 \\
      \then \sigma^2 &= 17.63 - 4.11^2 = \boxed{0.738}
    \end{align*}

  \item \textbf{4.36} Suppose that the probabilities are \(0.4, 0.3, 0.2,
    ~\text{and}~ 0.1\), respectively, that \(0, 1, 2, ~\text{or}~ 3\) power
    failures will strike a certain subdivision in any given year.

    Find the mean and variance of the random variable \(X\) representing the
    number of power failures striking this subdivision.
    \begin{align*}
      x &= [0,1,2,3], \quad f(x) = [0.4,0.3,0.2,0.1] \\
      \mu &= \sum_{x=0}^{3} xf(x) = \boxed{1}, \quad
      E(X^2) = \sum_{x=0}^{3} x^2 f(x) = 2 \\
      \then \sigma^2 &= 2 - 1 = \boxed{1}
    \end{align*}

  \item \textbf{4.37} A dealer’s profit, in units of \$5000, on a new
    automobile is a random variable \(X\) having the density function
    \[%%%%%%%%%%%%%%%%%%%%%%%%%%%%%%%%%%%%%%%%%%%%%%%%%%%%%%%%%%%%%%%%%%%%
      f(x) =
      \begin{cases}
        2(1-x),			& 0 < x < 1, \\
        0, & \text{else}
      \end{cases}
    \]%%%%%%%%%%%%%%%%%%%%%%%%%%%%%%%%%%%%%%%%%%%%%%%%%%%%%%%%%%%%%%%%%%%%

    Find the variance of \(X\).
    \begin{align*}
      \mu &= \int_{0}^{1} 2x(1-x)\, dx = \frac{1}{3} \\
      E(X^2) &= \int_{0}^{1} 2x^2(1-x)\, dx = \frac{1}{6} \\
      \sigma^2 &= \frac{1}{6} - \frac{1}{3^2} = \frac{1}{18}
      \then \var(X) = \boxed{\frac{5000}{18}^2}
    \end{align*}

%%%%%%%%%%%%%
  \newpage %%%%%%%%%%%%%%%%%%%%%%%%%%%%%%%%%%%%%%%%%%%%%%%%%%%%%%%%%%%%%%%%%%%%
%%%%%%%%%%%%%

  \item \textbf{4.38} The proportion of people who respond to a certain
    mail-order solicitation is a random variable \(X\) having the density
    function
    \[%%%%%%%%%%%%%%%%%%%%%%%%%%%%%%%%%%%%%%%%%%%%%%%%%%%%%%%%%%%%%%%%%%%%
      f(x) =
      \begin{cases}
        \frac{2}{5}(x+2),			& 0<x<1, \\
        0, & \text{else}
      \end{cases}
    \]%%%%%%%%%%%%%%%%%%%%%%%%%%%%%%%%%%%%%%%%%%%%%%%%%%%%%%%%%%%%%%%%%%%%

    Find the variance of \(X\).
    \begin{align*}
      \mu &= \int_{0}^{1} \frac{2x(x+2)}{5}\, dx = \frac{8}{15}\\
      E(X^2) &= \int_{0}^{1} \frac{2x^2(x+2)}{5}\, dx = \frac{11}{30}\\
      \sigma^2 &= \frac{11}{30} - \left(\frac{8}{15}\right)^2 = \boxed{0.082}
    \end{align*}


  \item \textbf{4.43 (Bonus)} The length of time, in minutes, for an airplane
    to obtain clearance for takeoff at a certain airport is a random variable
    \(Y = 3X - 2\), where \(X\) has the density function
    \[%%%%%%%%%%%%%%%%%%%%%%%%%%%%%%%%%%%%%%%%%%%%%%%%%%%%%%%%%%%%%%%%%%%%
      f(x) =
      \begin{cases}
        \frac{1}{4}e^{-\frac{x}{4}}, & x > 0 \\
        0, & ~\text{elsewhere}~
      \end{cases}
    \]%%%%%%%%%%%%%%%%%%%%%%%%%%%%%%%%%%%%%%%%%%%%%%%%%%%%%%%%%%%%%%%%%%%%

    Find the mean and variance of the random variable \(Y\).
    \begin{align*}
      \mu_Y &= E(3X - 2 ) =
      \int_{0}^{\infty} \frac{1}{4}(3x - 2)e^{-\frac{x}{4}}\, dx = 10 \\
      E(Y^2) &= \int_{0}^{\infty} \frac{1}{4}(3x - 2)^2e^{-\frac{x}{4}}\, dx = 244 \\
      \sigma_Y^2 &= 244 - 10^2 = \boxed{144}
    \end{align*}

  \item \textbf{4.50} For a laboratory assignment, if the equipment is working,
    the density function of the observed outcome \(X\) is
    \[%%%%%%%%%%%%%%%%%%%%%%%%%%%%%%%%%%%%%%%%%%%%%%%%%%%%%%%%%%%%%%%%%%%%
      f(x) =
      \begin{cases}
        2(1-x), & 0 < x < 1, \\
        0,      & \text{elsewhere}
      \end{cases}
    \]%%%%%%%%%%%%%%%%%%%%%%%%%%%%%%%%%%%%%%%%%%%%%%%%%%%%%%%%%%%%%%%%%%%%

    Find the variance and standard deviation of \(X\).
    \begin{align*}
      \sigma^2 &= \boxed{\frac{1}{18}} && \text{by question 3 (4.37)} \\
      \sigma &= \boxed{\sqrt{\frac{1}{18}}}
    \end{align*}

%%%%%%%%%%%%%
  \newpage %%%%%%%%%%%%%%%%%%%%%%%%%%%%%%%%%%%%%%%%%%%%%%%%%%%%%%%%%%%%%%%%%%%%
%%%%%%%%%%%%%

  \item \textbf{4.54} Using Theorem 4.5 and Corollary 4.6, i.e.,
    \[%%%%%%%%%%%%%%%%%%%%%%%%%%%%%%%%%%%%%%%%%%%%%%%%%%%%%%%%%%%%%%%%%%%%
      E(aX + b) = a E(X) + b, \quad b = 0 \then \sigma^2_{aX+c} = a^2 \sigma_X^2 = a^2 \sigma^2,
    \]%%%%%%%%%%%%%%%%%%%%%%%%%%%%%%%%%%%%%%%%%%%%%%%%%%%%%%%%%%%%%%%%%%%%
    find the mean and variance of the random variable \(Z = 5X +3\), where
    \(X\) has the probability distribution of Exercise 4.36 (Problem 2, \(\mu = 1, \sigma^2 = 1\)).
    \begin{align*}
      \sigma^2_{5X+3} = 5^2(1) = \boxed{25}
    \end{align*}

  \item \textbf{4.71 (Bonus)} The length of time \(Y\), in minutes, required
    to generate a human reflex to tear gas has the density function
    \[%%%%%%%%%%%%%%%%%%%%%%%%%%%%%%%%%%%%%%%%%%%%%%%%%%%%%%%%%%%%%%%%%%%%
      f(y) =
      \begin{cases}
        \frac{1}{4}e^{-\frac{y}{4}}, & 0 \leq y < \infty \\
         0, & \text{elsewhere}
      \end{cases}
    \]%%%%%%%%%%%%%%%%%%%%%%%%%%%%%%%%%%%%%%%%%%%%%%%%%%%%%%%%%%%%%%%%%%%%
    \begin{enumerate}[leftmargin=1.6em]

      \item What is the mean time to reflex?
        \begin{align*}
          \mu = \int_{0}^{\infty} \frac{1}{4}ye^{-\frac{y}{4}}\, dy = \boxed{4}
        \end{align*}

      \item Find \(E(Y^2)\) and \(\var(Y)\).
        \begin{align*}
          E(Y^2) &= \int_{0}^{\infty} \frac{1}{4}y^2e^{-\frac{y}{4}}\, dy = \boxed{32} \\
          \sigma^2 &= 32 - 4^2 = \boxed{16}
        \end{align*}

    \end{enumerate}

    \vspace{-2em}

  \item \textbf{4.101} Consider Review Exercise 3.73 on page 108. It
    involved \(Y\), the proportion of impurities in a batch, and the density
    function is given by
    \[%%%%%%%%%%%%%%%%%%%%%%%%%%%%%%%%%%%%%%%%%%%%%%%%%%%%%%%%%%%%%%%%%%%%
      f(y) =
      \begin{cases}
        10(1-y)^9,			& 0 \leq y < 1, \\
        0, & \text{elsewhere}
      \end{cases}
    \]%%%%%%%%%%%%%%%%%%%%%%%%%%%%%%%%%%%%%%%%%%%%%%%%%%%%%%%%%%%%%%%%%%%%
    \begin{enumerate}[leftmargin=1.6em]

      \item Find the expected percentage of impurities.
        \begin{align*}
          \mu &= \int_{0}^{1} 10y(1-y)^9 = \frac{1}{11} = \boxed{0.09}
        \end{align*}

      \item Find expected value of proportion of quality material, i.e.,
        \(E(1-Y)\).
      \[%%%%%%%%%%%%%%%%%%%%%%%%%%%%%%%%%%%%%%%%%%%%%%%%%%%%%%%%%%%%%%%%%%%%
        E(1-Y) = 1 - 0.09 = \boxed{0.91}
      \]%%%%%%%%%%%%%%%%%%%%%%%%%%%%%%%%%%%%%%%%%%%%%%%%%%%%%%%%%%%%%%%%%%%%

      \item Find the variance of the random variable \(Z = 1-Y\).
        \[%%%%%%%%%%%%%%%%%%%%%%%%%%%%%%%%%%%%%%%%%%%%%%%%%%%%%%%%%%%%%%%%%%%%
        \hspace{-3em}
          \sigma^2_Z = \sigma^2_{1-Y} = \sigma^2_Y
          = \int_{0}^{1} 10y^2(1-y)^9 - \frac{1}{11^2}
          = \frac{1}{66} - \frac{1}{11^2} = \frac{5}{726}= \boxed{0.0068}
        \]%%%%%%%%%%%%%%%%%%%%%%%%%%%%%%%%%%%%%%%%%%%%%%%%%%%%%%%%%%%%%%%%%%%%

    \end{enumerate}

%%%%%%%%%%%%%
  \newpage %%%%%%%%%%%%%%%%%%%%%%%%%%%%%%%%%%%%%%%%%%%%%%%%%%%%%%%%%%%%%%%%%%%%
%%%%%%%%%%%%%

  \item \textbf{4.62} If \(X\) and \(Y\) are independent random variables with
    variances \(\sigma^2_X = 5 ~\text{and}~ \sigma^2_Y = 3\), find the variance
    of the random variable \(Z = -2X + 4Y - 3\).
    \begin{align*}
      \sigma^2_Z = a_X\sigma^2_X + a_Y\sigma^2_Y = -2^2(5) + 4^2(3) = \boxed{68}, \quad \text{by corollary 4.11}
    \end{align*}

  \item \textbf{4.63} Repeat Exercise 4.62 if X and Y are not independent and
    \(\sigma_{XY} = 1\).
    \begin{align*}
      \sigma^2_Z = a_X\sigma^2_X + a_Y\sigma^2_Y + 2a_Xa_Y\sigma^2_{XY} = 68 + 2(-8)(1) = \boxed{52}
    \end{align*}

  \item Let \(X\) and \(Y\) be random variables with the following information:
    \[%%%%%%%%%%%%%%%%%%%%%%%%%%%%%%%%%%%%%%%%%%%%%%%%%%%%%%%%%%%%%%%%%%%%
     E(X) = 6, \quad E(Y) = -\frac{1}{2}, \quad \sigma^2_X = 4, \quad \sigma^2_Y = 6, \quad \sigma_{XY} = 2
    \]%%%%%%%%%%%%%%%%%%%%%%%%%%%%%%%%%%%%%%%%%%%%%%%%%%%%%%%%%%%%%%%%%%%%
    \begin{enumerate}[leftmargin=1.6em]

      \item Compute \(E(3X - 4Y)\)
        \begin{align*}
          &= E(3X) + E(4Y) = 18 - 2 = \boxed{16}
        \end{align*}
      \item Compute \(\var(3X - 4Y)\)
        \begin{align*}
          &= 3^2(4) + 4^2(6) + 2(-12)(2)= \boxed{84}\\
        \end{align*}
      \item Compute \(E(2X - Y^2)\)
        \begin{align*}
          &= E(2X) - E(YY) = 12 - \frac{1}{4} = \boxed{11.75}
        \end{align*}

    \end{enumerate}

  \item Let \(X\) and \(Y\) be independent random variables with the following
    information:
    \[%%%%%%%%%%%%%%%%%%%%%%%%%%%%%%%%%%%%%%%%%%%%%%%%%%%%%%%%%%%%%%%%%%%%
     E(X) = -1, \quad E(Y) = 4, \quad \sigma^2_X = 6, \quad \sigma^2_Y = 8
    \]%%%%%%%%%%%%%%%%%%%%%%%%%%%%%%%%%%%%%%%%%%%%%%%%%%%%%%%%%%%%%%%%%%%%
    \begin{enumerate}[leftmargin=1.6em]

      \item Compute \(E(9X + 2Y)\)
        \begin{align*}
          &= E(9X) + E(2Y) = -9 + 8 = \boxed{-1}
        \end{align*}

      \item Compute \(\var(9X + 2Y)\)
        \begin{align*}
          &= 9^2(6) + 2^2(8) = \boxed{518}
        \end{align*}

    \end{enumerate}

%%%%%%%%%%%%%
  \newpage %%%%%%%%%%%%%%%%%%%%%%%%%%%%%%%%%%%%%%%%%%%%%%%%%%%%%%%%%%%%%%%%%%%%
%%%%%%%%%%%%%

  \item \textbf{6.3} The daily amount of coffee, in liters, dispensed by a
    machine located in an airport lobby is a random variable \(X\) having a
    continuous uniform distribution with \(A = 7 ~\text{and}~ B = 10\).

    Find the probability that on a given day the amount of coffee dispensed by
    this machine will be
    \begin{enumerate}[leftmargin=1.6em]

      \item at most 8.8 liters;

      \item more than 7.4 liters but less than 9.5 liters;

      \item at least 8.5 liters.

    \end{enumerate}

  \item \textbf{6.4} A bus arrives every 10 minutes at a bus stop. It is
    assumed that the waiting time for a particular individual is a random
    variable with a continuous uniform
    \begin{enumerate}[leftmargin=1.6em]

      \item What is the probability that the individual waits more than 7
        minutes?

      \item  What is the probability that the individual waits between 2 and 7
        minutes?

    \end{enumerate}

\end{enumerate}

\end{document}
