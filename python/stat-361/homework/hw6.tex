\documentclass[basic, header]{nosvagor-notes}
\usepackage{nosvagor-math}

\colorlet{title-color}{red}
\newcommand{\theTitle}{%
  \href{https://github.com/nosvagor/notes}%
  {Homework 6}%
}

\newcommand{\userName}{Cullyn Newman}
\newcommand{\class}{STAT: 361}
\newcommand{\institution}{Portland State}

\begin{document}

\begin{enumerate}[itemsep=2em]

  \item \textbf{8.4} The number of tickets issued for traffic violations by 8 state troopers
    during the Memorial Day weekend are 5, 4, 7, 7, 6, 3, 8, and 6.
    \begin{enumerate}

      \item If these values represent the number of tickets issued by a
        random sample of 8 state troopers from Montgomery County in Virginia,
        define a suitable population.

        \aset{Tickets} given by \aset{all state troopers} in
        \aset{Montgomery County} over \aset{Memorial Day weekend}.

      \item  If the values represent the number of tickets issued by a random
        sample of 8 state troopers from South Carolina, define a suitable
        population.

        \aset{Tickets} given by \aset{all state troopers} in
        \aset{South Carolina} over \aset{Memorial Day weekend}.

    \end{enumerate}

  \item Let \(X_1, X_2,\ldots,X_n\) are a random sample of size \(n\). Classify
    each of the following as a statistics or not a statistic.
    \begin{enumerate}

      \item \(\displaystyle \frac{1}{n-1} \sum_{i=1}^{n} \left( \frac{X_i - \mu }{\sigma} \right)^2 \)

      \dd{Statistic}: this is a function of random variables that provides
      information about the random sample.

      \item \(\displaystyle \sum_{i=1}^{n} \left| X_i - \bar{X} \right|  \)

      \dd{Not a statistic}: this function sums the difference of random samples
      from the median, but the summation itself does not say about the sample;
      it would need to be divided by the sample size to become a statistic (mean absolute difference).

    \end{enumerate}

  \item \textbf{8.2} The lengths of time, in minutes, that 10 patients waited
    in a doctor’s office before receiving treatment were recorded as follows:
    5, 11, 9, 5, 10, 15, 6, 10, 5, and 10. Treating the data as a random
    sample, find
    \begin{enumerate}

      \item the mean;
        \[%%%%%%%%%%%%%%%%%%%%%%%%%%%%%%%%%%%%%%%%%%%%%%%%%%%%%%%%%%%%%%%%%%%%
        \bar{x} = \frac{1}{10} \sum_{i=1}^{n} [\text{data}]_i = \boxed{8.6 ~\text{min}}
        \]%%%%%%%%%%%%%%%%%%%%%%%%%%%%%%%%%%%%%%%%%%%%%%%%%%%%%%%%%%%%%%%%%%%%
      \item the median;
        \[%%%%%%%%%%%%%%%%%%%%%%%%%%%%%%%%%%%%%%%%%%%%%%%%%%%%%%%%%%%%%%%%%%%%
          \tilde{x} = [5 , 6,  \aset{9 , 10} , 11 , 15] = \boxed{9.5 ~\text{min}}
        \]%%%%%%%%%%%%%%%%%%%%%%%%%%%%%%%%%%%%%%%%%%%%%%%%%%%%%%%%%%%%%%%%%%%%
      \item the mode.
        \[%%%%%%%%%%%%%%%%%%%%%%%%%%%%%%%%%%%%%%%%%%%%%%%%%%%%%%%%%%%%%%%%%%%%
          \mode(\text{data}) = \boxed{5, 10 ~\text{min}}
        \]%%%%%%%%%%%%%%%%%%%%%%%%%%%%%%%%%%%%%%%%%%%%%%%%%%%%%%%%%%%%%%%%%%%%

    \end{enumerate}

%%%%%%%%%%%%%
  \newpage %%%%%%%%%%%%%%%%%%%%%%%%%%%%%%%%%%%%%%%%%%%%%%%%%%%%%%%%%%%%%%%%%%%%
%%%%%%%%%%%%%

  \item \textbf{8.3} The reaction times for a random sample of 9 subjects to a
    stimulant were recorded as 2.5, 3.6, 3.1, 4.3, 2.9, 2.3, 2.6, 4.1, and 3.4
    seconds. Calculate:
    \begin{enumerate}

      \item then;
        \[%%%%%%%%%%%%%%%%%%%%%%%%%%%%%%%%%%%%%%%%%%%%%%%%%%%%%%%%%%%%%%%%%%%%
        \bar{x} = \frac{1}{9} \sum_{i=1}^{9} [\text{data}]_i
        = \boxed{3.2 ~\text{sec}}
        \]%%%%%%%%%%%%%%%%%%%%%%%%%%%%%%%%%%%%%%%%%%%%%%%%%%%%%%%%%%%%%%%%%%%%
      \item the median;
        \[%%%%%%%%%%%%%%%%%%%%%%%%%%%%%%%%%%%%%%%%%%%%%%%%%%%%%%%%%%%%%%%%%%%%
          \tilde{x} = \boxed{3.1 ~\text{sec}}
        \]%%%%%%%%%%%%%%%%%%%%%%%%%%%%%%%%%%%%%%%%%%%%%%%%%%%%%%%%%%%%%%%%%%%%

    \end{enumerate}

  \item \textbf{8.10} For the sample of reaction times in Exercise 8.3,
    \begin{enumerate}

      \item the range;
        \[%%%%%%%%%%%%%%%%%%%%%%%%%%%%%%%%%%%%%%%%%%%%%%%%%%%%%%%%%%%%%%%%%%%%
          4.3 - 2.3 = \boxed{2~\text{sec}}
        \]%%%%%%%%%%%%%%%%%%%%%%%%%%%%%%%%%%%%%%%%%%%%%%%%%%%%%%%%%%%%%%%%%%%%
      \item the standard deviation.
        \[%%%%%%%%%%%%%%%%%%%%%%%%%%%%%%%%%%%%%%%%%%%%%%%%%%%%%%%%%%%%%%%%%%%%
          S^2 = \frac{1}{8} \sum_{i=1}^{8} \left( X_i - 3.2 \right)^2 = \boxed{0.4975}
        \]%%%%%%%%%%%%%%%%%%%%%%%%%%%%%%%%%%%%%%%%%%%%%%%%%%%%%%%%%%%%%%%%%%%%

    \end{enumerate}

  \item \textbf{8.11} For the data of Exercise 8.5, calculate the variance using the formula
    \[%%%%%%%%%%%%%%%%%%%%%%%%%%%%%%%%%%%%%%%%%%%%%%%%%%%%%%%%%%%%%%%%%%%%
      \text{data} = [2, 1, 3, 0, 1, 3, 6, 0, 3, 3, 5, 2, 1, 4, 2], n = 15, \bar{x} = 2.4
    \]%%%%%%%%%%%%%%%%%%%%%%%%%%%%%%%%%%%%%%%%%%%%%%%%%%%%%%%%%%%%%%%%%%%%
    \begin{enumerate}

      \item using form (8.2.1),
        \[%%%%%%%%%%%%%%%%%%%%%%%%%%%%%%%%%%%%%%%%%%%%%%%%%%%%%%%%%%%%%%%%%%%%
          S^2 = \frac{1}{14} \sum_{i=1}^{15} \left( \text{data}_i - 2.4 \right)^2 = \boxed{2.971}
        \]%%%%%%%%%%%%%%%%%%%%%%%%%%%%%%%%%%%%%%%%%%%%%%%%%%%%%%%%%%%%%%%%%%%%

      \item using theorem 8.1,
        \[%%%%%%%%%%%%%%%%%%%%%%%%%%%%%%%%%%%%%%%%%%%%%%%%%%%%%%%%%%%%%%%%%%%%
          S^2 = \frac{1}{15(14)} \left[ 15 \sum_{i=1}^{15} \text{data}_i^2 - \left( \sum_{i=1}^{15} \text{data}_i \right)^2  \right] = \boxed{2.971}
        \]%%%%%%%%%%%%%%%%%%%%%%%%%%%%%%%%%%%%%%%%%%%%%%%%%%%%%%%%%%%%%%%%%%%%

    \end{enumerate}

%%%%%%%%%%%%%
  \newpage %%%%%%%%%%%%%%%%%%%%%%%%%%%%%%%%%%%%%%%%%%%%%%%%%%%%%%%%%%%%%%%%%%%%
%%%%%%%%%%%%%

  \item \textbf{8.23} The random variable X, representing the number of cherries in a
    cherry puff, has the following probability distribution:
    \begin{table}[h]
      \centering
      \begin{tabular}{c|cccc}
        x & 4 & 5 & 6 & 7 \\
        \midrule
        P(X = x) & 0.2 & 0.4 & 0.3 & 0.1 \\
      \end{tabular}
    \end{table}
    \begin{enumerate}

      \item Find the mean \(\mu\) and the variance \(\sigma^2\) of \(X\).
          \begin{align*}
            \mu &= \sum xf(x) = \boxed{5.3} \\
            \sigma^2 &= \sum(x - \mu)^2f(x) = \boxed{0.81}
          \end{align*}
      \item Find the mean \(\mu_{\bar{X}} \) and the variance
        \(\sigma^2_{\bar{X}} \) of the mean \(\bar{X}\) for random samples of
        36 cherry puffs.
        \[%%%%%%%%%%%%%%%%%%%%%%%%%%%%%%%%%%%%%%%%%%%%%%%%%%%%%%%%%%%%%%%%%%%%
          \mu_{\bar{X}} = \mu = \boxed{5.3}, \quad \sigma^2_{\bar{X}} = \frac{\sigma^2}{36} = \boxed{0.0225}
        \]%%%%%%%%%%%%%%%%%%%%%%%%%%%%%%%%%%%%%%%%%%%%%%%%%%%%%%%%%%%%%%%%%%%%

      \item Find the probability that the average number of cherries in 36
        cherry puffs will be less than 5.5.
        \[%%%%%%%%%%%%%%%%%%%%%%%%%%%%%%%%%%%%%%%%%%%%%%%%%%%%%%%%%%%%%%%%%%%%
          \text{normalcdf}(-\infty, 5.5, 5.3, \sqrt{\tfrac{0.81}{36}}) = \boxed{0.9087}
        \]%%%%%%%%%%%%%%%%%%%%%%%%%%%%%%%%%%%%%%%%%%%%%%%%%%%%%%%%%%%%%%%%%%%%

    \end{enumerate}

  \item \textbf{8.25} The average life of a bread-making machine is 7 years,
    with a standard deviation of 1 year. Assuming that the lives of these
    machines follow approximately a normal distribution, find
    \begin{enumerate}

      \item the probability that the mean life of a random sample of 9 such
        machines falls between 6.4 and 7.2 years;
          \begin{align*}
            Z_l = \frac{6.4 - 7 }{\frac{1}{\sqrt{9}}} = -1.8, \quad
            Z_u = \frac{7.2 - 7 }{\frac{1}{\sqrt{9}}} = 0.6\\
            \text{normalcdf}(-1.8, 0.6, 0, 1) &= \boxed{0.9087}
          \end{align*}

      \item the value of \(x\) to the right of which 15\% of the means computed
        from random samples of size 9
        \begin{align*}
          Z = \text{invNorm}(1-0.15, 0, 1) = 1.04 \\
         \bar{x} = \frac{z}{\frac{\sigma}{\sqrt{n}}} + \mu = 1.04(0.3) + 7 = \boxed{7.31}
        \end{align*}

    \end{enumerate}

%%%%%%%%%%%%%
  \newpage %%%%%%%%%%%%%%%%%%%%%%%%%%%%%%%%%%%%%%%%%%%%%%%%%%%%%%%%%%%%%%%%%%%%
%%%%%%%%%%%%%

    \item \textbf{8.27} In a chemical process, the amount of a certain type of
      impurity in the output is difficult to control and is thus a random
      variable. Speculation is that the population mean amount of the impurity
      is 0.20 gram per gram of output. It is known that the standard deviation
      is 0.1 gram per gram. An experiment is conducted to gain more insight
      regarding the speculation that \(\mu = 0.2\)

      The process is run on a lab scale 50 times and the sample average \(\bar{x}\)
      turns out to be 0.23 gram per gram. Comment on the speculation that the
      mean amount of impurity is 0.20 gram per gram. Make use of the Central
      Limit Theorem in your work.
      \[%%%%%%%%%%%%%%%%%%%%%%%%%%%%%%%%%%%%%%%%%%%%%%%%%%%%%%%%%%%%%%%%%%%%
        \mu = 0.2, \quad \sigma = 0.1, n = 50, \bar{x} = 0.23
      \]%%%%%%%%%%%%%%%%%%%%%%%%%%%%%%%%%%%%%%%%%%%%%%%%%%%%%%%%%%%%%%%%%%%%
      \begin{align*}
        Z = \frac{0.23 - 0.2}{\tfrac{0.1}{\sqrt{50}}} = 2.12 \\
        \text{normalcdf}(2.12, \infty,0,1) = 0.017 \to \boxed{\text{very unlikely, 1.7\% chance}}
      \end{align*}

    \item \textbf{8.33} The chemical benzene is highly toxic to humans.
      However, it is used in the manufacture of many medicine dyes, leather,
      and coverings. Government regulations dictate that for any production
      process involving benzene, the water in the output of the process must
      not exceed 7950 parts per million (ppm) of benzene. For a particular
      process of concern, the water sample was collected by a manufacturer 25
      times randomly and the sample average \(\bar{x}\) was 7960 ppm. It is
      known from historical data that the standard deviation \(\sigma\) is 100
      ppm.
    \begin{enumerate}

      \item What is the probability that the sample average in this experiment
        would exceed the government limit if the population mean is equal to
        the limit? Use the Central Limit Theorem.

      If \(\mu \approx \bar{X}\), then \(Z \approx 0\), implying \(P(Z\geq0) = \boxed{\dfrac{1}{2}}\)


      \item Is an observed \(\bar{x}\) = 7960 in this experiment firm evidence
        that the population mean for the process exceeds the government limit?
        Answer your question by computing
        \[%%%%%%%%%%%%%%%%%%%%%%%%%%%%%%%%%%%%%%%%%%%%%%%%%%%%%%%%%%%%%%%%%%%%
        P(\bar{X} \geq 7960 : \mu = 7950)
      \]%%%%%%%%%%%%%%%%%%%%%%%%%%%%%%%%%%%%%%%%%%%%%%%%%%%%%%%%%%%%%%%%%%%%
      Assume that the distribution of benzene concentration is normal.
      \begin{align*}
        Z = \frac{7960 - 7950}{\frac{100}{\sqrt{25}}} = 0.5 \\
        \text{normalcdf}(0.5, \infty,0,1) = 0.3085 \to \boxed{\text{somewhat unlikely, 30.9\% chance}}
      \end{align*}

    \end{enumerate}

%%%%%%%%%%%%%
  \newpage %%%%%%%%%%%%%%%%%%%%%%%%%%%%%%%%%%%%%%%%%%%%%%%%%%%%%%%%%%%%%%%%%%%%
%%%%%%%%%%%%%

  \item \textbf{8.49} A normal population with unknown variance has a mean of
    20. Is one likely to obtain a random sample of size 9 from this population
    with a mean of 24 and a standard deviation of 4.1? If not, what conclusion
    would you draw?
    \begin{align*}
      T = \frac{24 - 20}{\frac{4.1}{3}} = 2.926 \\
      t_{0.01} = \text{invT}(.99, 9-1) = 2.896 \\
      T > t_{0.01} \then \boxed{\text{not likely, under strong confidence interval}}
    \end{align*}


  \item \textbf{8.50} A maker of a certain brand of low-fat cereal bars claims that the
    average saturated fat content is 0.5 gram. In a random sample of 8 cereal
    bars of this brand, the saturated fat content was 0.6, 0.7, 0.7, 0.3, 0.4,
    0.5, 0.4, and 0.2. Would you agree with the claim? Assume a normal
    distribution.
    \begin{align*}
      \mu &= 0.5 \\
      \bar{x} = \frac{1}{8} \sum_{i=1}^{8} [\text{data}]_i
      &= 0.475 \\
      S^2 = \frac{1}{7} \sum_{i=1}^{8} \left( \text{data}_i - 0.475 \right)^2
      &= 0.0336 \\
      Z = \frac{4.75 - 0.5 }{\sqrt{\frac{0.0336}{8}}}
      &= -0.39 \\
      \text{normalcdf}(-\infty, -0.39,0,1) = 0.3482
      &\to \boxed{\text{somewhat unlikely, 34.8\% chance}}
    \end{align*}

\end{enumerate}

\end{document}
