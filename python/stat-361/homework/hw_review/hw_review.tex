\documentclass[basic, header]{nosvagor-notes}
\usepackage{nosvagor-math}

\colorlet{title-color}{red}
\newcommand{\theTitle}{%
  \href{https://github.com/nosvagor/notes}%
  {Review Homework}%
}

\newcommand{\userName}{Cullyn Newman}
\newcommand{\class}{STAT: 361}
\newcommand{\institution}{Portland State}

\begin{document}

\subsection{Part 1: Evaluate}
\begin{enumerate}[label=(\alph*), itemsep=1.5em, leftmargin=2em]
  \item \(7! = 7 \cdot 6 \cdot 5 \ldots \cdot 1 = \boxed{5040} \)
  \item \(\displaystyle \sum_{x=1}^{20} x = 1 + 2 + 3 + \ldots + 20 = \boxed{210} \)
  \item \(\displaystyle \sum_{i=1}^{20} w = \boxed{20w} \)
  \item \(\displaystyle \sum_{x=1}^{3} \left[cx^3 + 1\right] = (c+1) + (c8 + 1) + (c 27 + 1) = \boxed{36c +3}\)
  \item Expand \((x+4)^2 \to \boxed{(x^2+8x + 16)}\)
  \item Expand \((x-4)^2 \to \boxed{(x^2-8x + 16)}\)
  \item If \(\displaystyle f(x) =
    \begin{cases}
      \frac{1}{8} &: x = 0,3 \\
      \frac{3}{8} &: x = 1,2 \\
      0 &: ~\text{otherwise}~
    \end{cases}
    \), then compute the following:
    \begin{enumerate}[label=(\roman*)]
      \item \(\displaystyle \sum_{\forall x} \left[ x f(x) \right] = 3\cdot\frac{1}{8} + 2\cdot \frac{3}{8} + \frac{3}{8} = \frac{12}{8} = \boxed{\frac{3}{2}}\)
      \item \(\displaystyle \sum_{\forall x} \left[ (x-1.5)^2 f(x) \right] =
        (-1.5)^2\frac{1}{8} + (-0.5)^2\frac{3}{8} + (0.5)^2\frac{3}{8} + (1.5)^2\frac{1}{8} =
        \boxed{\frac{3}{4}}\)
    \end{enumerate}
  \item \(\displaystyle \int_{0}^{1} x~dx = \frac{x^2}{2} \bigg|_{0}^{1} = \frac{1}{2} - 0 = \boxed{\frac{1}{2}} \)
  \item \(\displaystyle \int_{1}^{3} x^2 ~dx = \frac{x^3}{3} \bigg|_{1}^{3}
        = 9 - \frac{1}{3} = \boxed{\frac{26}{3}}
    \)
  \item \(\displaystyle \int_{0}^{1} (x^3 + 1)dx = \frac{x^4}{4} \bigg|_{0}^{1} + x \bigg|_{0}^{1}   = \frac{1}{4} + 1 = \boxed{\frac{5}{4}} \)
  \item \(\displaystyle \int_{0}^{\infty} \left[ ke^{-\frac{x}{3}}  \right] dx
      = k \int_{0}^{\infty} e^{-\frac{x}{3}} dx
      = -3ke^{-\frac{x}{3}} \bigg|_{0}^{\infty} = 0 - (-3k) = \boxed{3k}
      \)
    \item If \(\displaystyle f(x) =
    \begin{cases}
      \frac{x^2}{3} &: -1 < x < 2 \\
      0 &: ~\text{otherwise}~
    \end{cases}
    \) , then compute the following:
    \begin{enumerate}[label=(\roman*)]
      \item \(\displaystyle \int_{-\infty}^{\infty} \left[ xf(x) \right] dx
          = \int_{-1}^{2} \frac{x^3}{3} dx = \frac{1}{3} \int_{-1}^{2} x^3 dx= \frac{1}{3} \cdot \frac{x^4}{4}\bigg|_{-1}^{2}
          = \frac{1}{3} \cdot \frac{15}{4} = \boxed{\frac{5}{4}}
          \)
      \item \(\displaystyle \int_{-\infty}^{\infty} \left[ x^2f(x) \right] dx
          = \int_{-1}^{2} \frac{x^4}{3} dx = \frac{1}{3} \int_{-1}^{2} x^4 dx= \frac{1}{3} \cdot \frac{x^5}{5}\bigg|_{-1}^{2}
          = \frac{1}{3} \cdot \frac{33}{5} = \boxed{\frac{11}{5}}
          \)
    \end{enumerate}
\end{enumerate}

\vspace{2em}
\subsection{Part 2: Sketch}
\begin{enumerate}[label=(\alph*), itemsep=1.5em, leftmargin=2em]
  \item \(\displaystyle f(x) =
    \begin{cases}
    2x &: -0.5 < x < 0 \\
    0 &: ~\text{otherwise}
    \end{cases}
    \)

   \begin{center}
    \begin{tikzpicture}
      \begin{axis}[
        height=7cm,
        width=12cm,
        xmin=-1,ymin=-1.05,
        xmax=1,ymax=1,
        samples=100,
        axis lines = center,
        xlabel = {\(x\)},
        ylabel = {\(f(x)\)},
        ticklabel style = {font=\scriptsize},
        ]
        \addplot[blue+1, thick, domain=-0.5:0] (x,2*x);
        \addplot[red+1,  thick, domain=-10:-0.5] (x, 0*x);
        \addplot[red+1,  thick, domain=0:10] (x, 0*x);
        \draw[blue+1,fill=white](axis cs:0,0)circle(0.7mm);
        \draw[blue+1,fill=white](axis cs:-0.5,-1)circle(0.7mm);
      \end{axis}
    \end{tikzpicture}
    \end{center}

  \item \(\displaystyle f(x) =
    \begin{cases}
    5e^{-5x}  &: x > 0 \\
    0 &: ~\text{otherwise}
    \end{cases}
    \)

    \begin{center}
    \begin{tikzpicture}
      \begin{axis}[
        height=7cm,
        width=12cm,
        xmin=-3,ymin=-1,
        xmax=3,ymax=3,
        samples=200,
        axis lines = center,
        axis line style=<->,
        xlabel = {\(x\)},
        ylabel = {\(~ ~f(x)\)},
        ticklabel style = {font=\scriptsize},
        ]
        \addplot[blue+1, thick, domain=0:10] (x,5*(e^(-5*x)));
        \addplot[red+1,  thick, domain=-10:0] (x, 0*x);
      \end{axis}
    \end{tikzpicture}
    \end{center}

  \item \(\displaystyle f(x) =
    \begin{cases}
    \frac{1}{2\pi} &: -\pi < x < \pi \\
    0 &: ~\text{otherwise}
    \end{cases}
    \)

    \begin{center}
    \begin{tikzpicture}
      \begin{axis}[
        height=7cm,
        width=12cm,
        xmin=-6,ymin=-1,
        xmax=6,ymax=4,
        samples=100,
        axis lines = center,
        axis line style=<->,
        xlabel = {\(x\)},
        ylabel = {\(f(x)\)},
        ticklabel style = {font=\scriptsize},
        ]
        \addplot[blue+1, thick, domain=-pi:pi] (x,1/2*pi);
        \addplot[red+1,  thick, domain=-10:-3.14] (x, 0*x);
        \addplot[red+1,  thick, domain=3.14:10] (x, 0*x);
        \draw[blue+1,fill=white](axis cs:-pi,1/2*pi)circle(0.7mm);
        \draw[blue+1,fill=white](axis cs:pi,1/2*pi)circle(0.7mm);
      \end{axis}
    \end{tikzpicture}
    \end{center}

  \item \(\displaystyle f(x) =
    \begin{cases}
    0 &: x < 0 \\
    \frac{x^2}{4} &: 0 \leq x < 1 \\
    \frac{x+1}{4} &: 1 \leq x < 2 \\
    1 &: x \geq 2 \\
    \end{cases}
    \)

    \begin{center}
    \begin{tikzpicture}
      \begin{axis}[
        height=7cm,
        width=12cm,
        xmin=-2,ymin=-2,
        xmax=3,ymax=4,
        samples=100,
        axis lines = center,
        axis line style=<->,
        xlabel = {\(x\)},
        every axis x label/.style={
          at={(ticklabel* cs:1.02)},
          anchor=west,
        },
        every axis y label/.style={
          at={(ticklabel* cs:1.02)},
          anchor=south,
        },
        ylabel = {\(f(x)\)},
        ticklabel style = {font=\scriptsize},
        ]
        \addplot[blue+1, thick, domain=-0:1] (x,(x^2)/4);
        \addplot[blue+1, thick, domain=1:2] (x,(x+1)/4);
        \draw[blue+1,fill=white](axis cs:1.97,2.98)circle(0.7mm);
        \draw[blue+1,fill=white](axis cs:1,1)circle(0.7mm);
        \addplot[red+1,  thick, domain=-10:0] (x, 0*x);
        \addplot[red+1,  thick, domain=2:10] (x, 1);
      \end{axis}
    \end{tikzpicture}
    \end{center}

\end{enumerate}

\end{document}
