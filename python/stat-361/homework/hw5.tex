\documentclass[basic, header]{nosvagor-notes}
\usepackage{nosvagor-math}

\colorlet{title-color}{red}
\newcommand{\theTitle}{%
  \href{https://github.com/nosvagor/notes}%
  {Homework 5}%
}

\newcommand{\userName}{Cullyn Newman}
\newcommand{\class}{STAT: 361}
\newcommand{\institution}{Portland State}

\begin{document}

\begin{enumerate}[itemsep=4em]

  \item \textbf{6.10} According to Chebyshev’s theorem, the probability that any
    random variable assumes a value within 3 standard deviations of the mean is
    at least \(\frac{8}{9}\). If it is known that the probability distribution of a random
    variable \(X\) is normal with mean \(\mu\) and variance \(\sigma^2\), what is the exact value
    of \(P(\mu - 3\sigma < X < \mu + 3\sigma)\)?
    \begin{align*}
      Z &= \frac{X - \mu}{\sigma} \\
        &\then P(-3 < Z < 3) = 0.9987 - 0.0013 & \text{by table 3A}\\
        &= \boxed{0.9974}
    \end{align*}

  \item \textbf{6.12} The loaves of rye bread distributed to local stores by a certain
    bakery have an average length of 30 centimeters and a standard deviation of
    2 centimeters. Assuming that the lengths are normally distributed, what
    percentage of the loaves are
    \begin{enumerate}

      \item longer than 31.7 centimeters?
        \[%%%%%%%%%%%%%%%%%%%%%%%%%%%%%%%%%%%%%%%%%%%%%%%%%%%%%%%%%%%%%%%%%%%%
          \text{normalcdf($31.7,\infty,30,2$)} = \boxed{19.77\%}
        \]%%%%%%%%%%%%%%%%%%%%%%%%%%%%%%%%%%%%%%%%%%%%%%%%%%%%%%%%%%%%%%%%%%%%

      \item between 29.3 and 33.5 centimeters in length?
        \begin{align*}
              \text{normalcdf($-29.3,33.5,30,2$)} = \boxed{59.67\%}
        \end{align*}

      \item shorter than 25.5 centimeters?
        \[%%%%%%%%%%%%%%%%%%%%%%%%%%%%%%%%%%%%%%%%%%%%%%%%%%%%%%%%%%%%%%%%%%%%
          \text{normalcdf($-\infty,25.5,30,2$)} = \boxed{1.22\%}
        \]%%%%%%%%%%%%%%%%%%%%%%%%%%%%%%%%%%%%%%%%%%%%%%%%%%%%%%%%%%%%%%%%%%%%
    \end{enumerate}

  \item \textbf{6.14} The finished inside diameter of a piston ring is normally
    distributed with a mean of 10 centimeters and a standard deviation of 0.03
    centimeter.
    \begin{enumerate}

      \item What proportion of rings will have inside diameters exceeding
        10.075 centimeters?
        \[%%%%%%%%%%%%%%%%%%%%%%%%%%%%%%%%%%%%%%%%%%%%%%%%%%%%%%%%%%%%%%%%%%%%
          \text{normalcdf($10.075,\infty,10,0.03$)} = \boxed{0.62\%}
        \]%%%%%%%%%%%%%%%%%%%%%%%%%%%%%%%%%%%%%%%%%%%%%%%%%%%%%%%%%%%%%%%%%%%%

      \item What is the probability that a piston ring will have an inside
        diameter between 9.97 and 10.03 centimeters?
        \[%%%%%%%%%%%%%%%%%%%%%%%%%%%%%%%%%%%%%%%%%%%%%%%%%%%%%%%%%%%%%%%%%%%%
          \text{normalcdf($9.97,10.03,10,0.03$)} = \boxed{0.6826}
        \]%%%%%%%%%%%%%%%%%%%%%%%%%%%%%%%%%%%%%%%%%%%%%%%%%%%%%%%%%%%%%%%%%%%%

      \item Below what value of inside diameter will 15\% of the piston rings
        fall?
        \[%%%%%%%%%%%%%%%%%%%%%%%%%%%%%%%%%%%%%%%%%%%%%%%%%%%%%%%%%%%%%%%%%%%%
          \text{invNorm($0.15,10,0.03$)} = \boxed{9.969 \text{ cm}}
        \]%%%%%%%%%%%%%%%%%%%%%%%%%%%%%%%%%%%%%%%%%%%%%%%%%%%%%%%%%%%%%%%%%%%%

    \end{enumerate}

%%%%%%%%%%%%%
  \newpage %%%%%%%%%%%%%%%%%%%%%%%%%%%%%%%%%%%%%%%%%%%%%%%%%%%%%%%%%%%%%%%%%%%%
%%%%%%%%%%%%%

  \item \textbf{6.17} The average life of a certain type of small motor is 10 years with
    a standard deviation of 2 years. The manufacturer replaces free all motors
    that fail while under guarantee. If she is willing to replace only 3\% of
    the motors that fail, how long a guarantee should be offered? Assume that
    the lifetime of a motor follows a normal distribution.
    \[%%%%%%%%%%%%%%%%%%%%%%%%%%%%%%%%%%%%%%%%%%%%%%%%%%%%%%%%%%%%%%%%%%%%
          \text{invNorm($0.03,10,2$)} = \boxed{6.24 \text{ yrs}}
    \]%%%%%%%%%%%%%%%%%%%%%%%%%%%%%%%%%%%%%%%%%%%%%%%%%%%%%%%%%%%%%%%%%%%%

  \item \textbf{6.19} A company pays its employees an average wage of \$15.90 an hour
    with a standard deviation of \$1.50. If the wages are approximately normally
    distributed and paid to the nearest cent,
    \begin{enumerate}

      \item what percentage of the workers receive wages between \$13.75 and \$16.22 an hour inclusive?
        \[%%%%%%%%%%%%%%%%%%%%%%%%%%%%%%%%%%%%%%%%%%%%%%%%%%%%%%%%%%%%%%%%%%%%
          \text{normalcdf($13.75,16.22,15.9,1.5$)} = \boxed{50.86\%}
        \]%%%%%%%%%%%%%%%%%%%%%%%%%%%%%%%%%%%%%%%%%%%%%%%%%%%%%%%%%%%%%%%%%%%%

      \item the highest 5\% of the employee hourly wages is greater than what amount?
        \begin{align*}
          \text{invNorm($0.95,15.9,1.5$)} &= \boxed{\$18.37} \\
        \end{align*}

    \end{enumerate}

  \item \textbf{6.22} If a set of observations is normally distributed, what
    percent of these differ from the mean by
    \begin{enumerate}

      \item more than 1.3\(\sigma\)?
        \[%%%%%%%%%%%%%%%%%%%%%%%%%%%%%%%%%%%%%%%%%%%%%%%%%%%%%%%%%%%%%%%%%%%%
        2P(Z > 1.3) \then
        2 \cdot \text{normalcdf($1.3,\infty,0,1$)} = \boxed{19.36\%}
        \]%%%%%%%%%%%%%%%%%%%%%%%%%%%%%%%%%%%%%%%%%%%%%%%%%%%%%%%%%%%%%%%%%%%%

      \item less than 0.52\(\sigma\)?
        \[%%%%%%%%%%%%%%%%%%%%%%%%%%%%%%%%%%%%%%%%%%%%%%%%%%%%%%%%%%%%%%%%%%%%
        P(-0.52 < Z < 0.52) \then
        \text{normalcdf($-0.52,0.52,0,1$)} = \boxed{39.69\%}
        \]%%%%%%%%%%%%%%%%%%%%%%%%%%%%%%%%%%%%%%%%%%%%%%%%%%%%%%%%%%%%%%%%%%%%

    \end{enumerate}

%%%%%%%%%%%%%
  \newpage %%%%%%%%%%%%%%%%%%%%%%%%%%%%%%%%%%%%%%%%%%%%%%%%%%%%%%%%%%%%%%%%%%%%
%%%%%%%%%%%%%

  \item \textbf{6.80} In a human factor experimental project, it has been
    determined that the reaction time of a pilot to a visual stimulus is
    normally distributed with a mean of \(0.5\) seconds and standard
    deviation of \(0.4\) seconds.
    \begin{enumerate}

      \item  What is the probability that a reaction from the pilot takes more
        than 0.3 second?
        \[%%%%%%%%%%%%%%%%%%%%%%%%%%%%%%%%%%%%%%%%%%%%%%%%%%%%%%%%%%%%%%%%%%%%
          \text{normalcdf($0.3,\infty,0.5,0.4$)} = \boxed{0.6915}
        \]%%%%%%%%%%%%%%%%%%%%%%%%%%%%%%%%%%%%%%%%%%%%%%%%%%%%%%%%%%%%%%%%%%%%

      \item What reaction time is that which is exceeded 95\% of the time?
        \[%%%%%%%%%%%%%%%%%%%%%%%%%%%%%%%%%%%%%%%%%%%%%%%%%%%%%%%%%%%%%%%%%%%%
          \text{invNorm($1 - 0.95,0.5,0.4$)} = \boxed{-0.1579 \text{ sec}} \quad \text{\tiny{some pilots must be psychic, or model is wrong}}
        \]%%%%%%%%%%%%%%%%%%%%%%%%%%%%%%%%%%%%%%%%%%%%%%%%%%%%%%%%%%%%%%%%%%%%

    \end{enumerate}

  \item There are two machines available for cutting corks intended for use in
    wine bottles. The first machine produces corks with diameters that have a
    normal distribution with mean 3 cm and standard deviation 0.1 cm.\\ The
    second machine produces corks with diameters that have a normal
    distribution with mean 3.04 cm and standard deviation 0.02 cm.

    Acceptable corks have diameters between 2.9 cm and 3.1 cm. Which machine is more likely to
    produce an acceptable cork?

    \begin{align*}
      \text{machine}_1 &= \text{normalcdf($2.9,3.1,3,0.1$)} = 68.26\%\\
      \text{machine}_2 &= \text{normalcdf($2.9,3.1,3.04,0.02$)} = 99.86\%
    \end{align*}

    \boxed{\text{Machine 2, clearly}}
\end{enumerate}

\end{document}
