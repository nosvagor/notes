\chapter{1 Matrices and Gaussian Elimination}

% chktex-file 1
% chktex-file 3
% chktex-file 21
% chktex-file 36
% chktex-file 27

\section{1.2 The Geometry of Linear Equations}
\begin{itemize}
  \item []

  \subsection{Problems 1--12}
  \begin{enumerate}\color{foreground-2}
    \item For the equations \( x + y = 4, ~ 2x - 2y = 4 \), draw the row picture (two
      intersecting lines) and the column picture (combination of two columns
      equal to the column vector (4,4) on the right side).

      \begin{figure}[ht]
          \centering
          \incfig[1]{1.2.1}
          \caption{1.2.1}
      \end{figure}

    \item Solve to find a combination of the columns that equals \(b\):
      \begin{align*}
        u - v - w &= b_1 \\
        v + w &= b_2 \\
        w &= b_3
      \end{align*}
      \vspace{-35pt}
      \fg{\begin{align*}
        &\then w = b_3 \\
        &\then v = b_2 - b_3 \\
        &\then u = b_1 + v + w = b_1 + b_2
      \end{align*}}
      \vspace{-18pt}

    \item Describe the intersection of the three planes \(  u+v+w+z = 6 \) and
      \( u+w+z = 4 \) and \( u+w = 2 \) (all in four-dimensional space). Is it
      a line or a point or an empty set? What is the intersection if the fourth
      plane \( u = -1 \) is included? Find a fourth equation that leaves us
      with no solution.
      \begin{itemize}\color{foreground}
        \item \aset{A line}; \maybe{as \( u + w = 2 \) is only a line}. A fourth plane
          with \( u = -1 \) would produce a normally intersecting point. Any
          addition equation when \( u + w \neq 2 \) would produce an
          inconsistent equation.
      \end{itemize}

    \item Sketch these three lines and decide if the equations are solvable:
      \begin{align*}
        x + 2y &= 2 \\
        x - y &= 2 \\
        y &= 1
      \end{align*}
      \vspace{-30pt}
      \begin{figure}[h]
        \centering
        \incfig[1]{1.2.4}
        \caption{1.2.4}
      \end{figure}

      \aset{Inconsistent; multiple points of intersect}

      What happens if all right-hand sides are zero? Is there any nonzero
      choice of right-hand sides that allows the three lines to intersect at
      the same point?
      \begin{itemize}\color{foreground}
        \item If all the solutions were zero, then it would be a trivial
          solution.
        \item Yes, e.g., \( x - y = -1 \) would produce a single point of
          intersection.
      \end{itemize}

    \item Find two points on the line of intersection of the three planes \( t
      = 0 \) and \( z = 0 \) and \( x+y+z+t = 1 \) in four-dimensional space.

     \fg{\[
        \begin{bmatrix} 1 \\ 0 \\ 0 \\ 0 \end{bmatrix} \qquad
        \begin{bmatrix} 0 \\ 1 \\ 0 \\ 0  \end{bmatrix}
     \]}

    \item When \( b = (2,5,7) \), find a solution \( (u,v,w) \) to equation (4)
      different from the solution \( (1,0,1) \) mentioned in the text.
      \begin{itemize}\color{foreground}
        \item Since there are infinite solutions, and if \tbm{s} vector
          describing one solution and \( \lambda \) is any scalar, then \(
          \bm{s} \lambda \) is also a solution. E.g., \( \left( 1,0,1 \right)
          42 = \left( 42,0,42 \right)  \)
      \end{itemize}
      \vspace{30pt}
      \addtocounter{enumi}{1}
      \item Explain why the system
      \begin{align*}
        u + v + w &= 2 \\
        u + 2v + 3w &= 1 \\
        v + 2w &= 0
      \end{align*}
      is singular by finding a combination of the three equations that adds up
      to \(0 = 1\). What value should replace the last zero on the right side to
      allow the equations to have solutions---and what is one of the solutions?
      \fg{\begin{align*}
      \begin{bmatrix}[ccc|c]
        1 & 1 & 1 & 2 \\
        1 & 2 & 3 & 1 \\
        0 & 1 & 2 & 0
      \end{bmatrix}
      \xRightarrow{R_2 - R_1}
      \begin{bmatrix}[ccc|c]
        1 & 1 & 1 & 2 \\
        0 & 1 & 2 & -1 \\
        0 & 1 & 2 & 0
      \end{bmatrix}
      \xRightarrow{R_3-R_2}
      \begin{bmatrix}[ccc|c]
        1 & 1 & 1 & 2 \\
        0 & 1 & 2 & -1 \\
        0 & 0 & 0 & 1
      \end{bmatrix}
      \end{align*}}
      \begin{itemize}\color{foreground}
        \item Replacing the last zero with \aset{\( -1 \)} would yield infinite
          solutions. One solution would be \( \left[ 3,-1,0 \right]^T \)
      \end{itemize}

    \item The column picture for the previous exercise (singular system) is
      \begin{align*}
        u \begin{bmatrix} 1 \\ 1 \\ 0 \end{bmatrix} +
        v \begin{bmatrix} 1 \\ 2 \\ 1 \end{bmatrix} +
        w \begin{bmatrix} 1 \\ 3 \\ 2 \end{bmatrix} = b
      \end{align*}
      Show that the three columns on the left lie in the same plane by
      expressing the third as a combination of the first two. What are all the
      solutions \( (u,v,w) \) if \(b\) is the zero vector \( (0,0,0) \)?
      \fg{\begin{align*}
        -1 \begin{bmatrix} 1 \\ 1 \\ 0 \end{bmatrix} +
        2 \begin{bmatrix} 1 \\ 2 \\ 1 \end{bmatrix} =
        \begin{bmatrix} 1 \\ 3 \\ 2 \end{bmatrix}
      \end{align*}}
      \begin{itemize}\color{foreground}
        \item If is \tbm{b} equal to the zero vector \nil then the solutions
          \maybe{are equal to the kernel} i.e., \( -1x_1, 2x_2, 0x_3  = \nil\)
      \end{itemize}

    \item Under what condition on \( y_1, y_2, y_3 \) do the points \((0,y_1),
      (1,y_2), (2,y_3)\) lie on a straight line?
      \begin{itemize}\color{foreground}
        \item Question 9 describes the state at which they are collinear, i.e.,
          \( y_3 = 2y_2 - y_1 \)
      \end{itemize}


    \item These equations are certain to have the solution \( x = y = 0 \). For which
      values of \(a\) is there a whole line of solutions?
      \begin{align*}
        ax + 2y = 0 \\
        2x + ay = 0
      \end{align*}
      \begin{itemize}\color{foreground}
        \item Only the scalars that make the lines linearly dependent, i.e., \(
          \aset{a = 2, -2} \)
      \end{itemize}

  \end{enumerate}

  \subsection{Problems 17--23}
  \begin{enumerate}[resume]\color{foreground-2}
    \addtocounter{enumi}{5}
    \item The first of these equations plus the second equals the third:
      \begin{align*}
        x + y + z &= 2 \\
        x + 2y + z &= 3 \\
        2x + 3y + 2z &= 5
      \end{align*}

      The first two planes meet along a line. The third plane contains that
      line, because if \( x, y, z \) satisfy the first two equations then they
      also \aset{span all of \( \R^3 \)}. The equations have infinitely many solutions (the whole line \tbm{L}). Find
      three solutions.
      \begin{itemize}\color{foreground}
        \item \( \bm{v} = (4,4,0), \bm{w} = (6,3,2), \bm{u} = 2v + -1w\)
      \end{itemize}

    \item Move the third plane in Problem 17 to a parallel plane \( 2x + 3y + 2z =
      9 \). Now the three equations have no solution---\textit{why not}? The first two planes
      meet along the line \tbm{L}, but the third plane doesn't that
      \aset{cross} that line.

    \item In Problem 17 the columns are \( (1,1,2) \) and \( (1,2,3) \) and \(
      (1,1,2) \). This is a ``singular case'' because the third column is
      \aset{linearly dependent} Find two combinations of the columns that give
      \( b = (2,3,5) \). This is only possible for \( b = (4,6,c) \) if \(c = \aset{10}\)

    \item Normally 4 ``planes'' in four-dimensional space meet at a \aset{tensor}. Normally 4
      column vectors in four-dimensional space can combine to produce \(b\). What
      combination of \( (1,0,0,0), (1,1,0,0), (1,1,1,0), (1,1,1,1) \) produces \(b =
      (3,3,3,2)\)? \maybe{\((1,0,0,-2)\)} What 4 equations for \(x, y, z, t\)
      are you solving? \aset{A lower triangular matrix, i.e.,}
      \fg{\[
        \begin{bmatrix}[cccc|c]
          1 & 0 & 0 & 0 & 3 \\
          1 & 1 & 0 & 0 & 3 \\
          1 & 1 & 1 & 0 & 3 \\
          1 & 1 & 1 & 1 & 1
        \end{bmatrix}
      \]}

    \item When equation 1 is added to equation 2, which of these are changed:
      the planes in the row picture, the column picture, the coefficient
      matrix, the solution?
      \begin{itemize}\color{foreground}
        \item Row operations do not change the solution. Row 2 is changed, thus
          the second plane is changed. \maybe{All columns are changed.}
      \end{itemize}

      \newpage
    \item  If \((a,b)\) is a multiple of \((c,d)\) with \(abcd \neq 0\), show
      that \((a,c)\) is a multiple of \((b,d)\). This is surprisingly
      important: call it a challenge question. You could use numbers first to
      see how \(a, b, c\), and \(d\) are related. The question will lead to:

      If \(A = \begin{bsmallmatrix*} a && b \\ c && d\end{bsmallmatrix*}\)  has
      dependent rows then it has dependent columns.

      \vspace{25pt}
      \src{
        \link{https://www.math.colostate.edu/~clayton/teaching/m215s10/homework/hw1solutions.pdf}{Received
        help from, accessed 10/01/2021}
      }
      \fg{\begin{align*}
          \lambda \in \R, &\quad (a,b) = \lambda(c,d) = ( \lambda c, \lambda d) \\
        &\then a = \lambda c = \lambda c d^{-1} d = d^{-1} c\lambda d = d^{-1}c b \\
        &\then \left( a,c \right) = \left( d^{-1} c b, d^{-1}d c \right) = d^{-1}(b,d)
      \end{align*}}
      \fg{Thus, \( (b,d) \) is a multiple of  \( (a,c) \)}

      \item In these equations, the third column (multiplying w) is the same as
        the right side \(b\). The column form of the equations immediately
        gives what solution for \((u,v,w)\)?
        \begin{align*}
          6u + 7v + 8w &= 8 \\
          4u + 5v + 9w &= 9 \\
          2u - 2v + 7w &= 7
        \end{align*}
        \begin{itemize}\color{foreground}
          \item First two columns are irrelevant, \( u = 0, v = 0 \), only need \( \aset{w} \)
        \end{itemize}

  \end{enumerate}
\end{itemize}

\section{1.3 Gaussian Elimination}
\begin{itemize}
  \item []

  \subsection{Problems 1--9}
  \begin{enumerate}
    \item

  \end{enumerate}

 \subsection{Problems 10--19}
 \begin{enumerate}
   \item
 \end{enumerate}

 \subsection{Problems 20--22}
 \begin{itemize}
   \item
 \end{itemize}

 \subsection{Problems 23--31}
 \begin{enumerate}
   \item
 \end{enumerate}

\end{itemize}

\section{1.4 Matrix Notation and Matrix Multiplication}
\begin{itemize}
  \item []

\end{itemize}

\section{1.5 Triangular Factors and Row Exchanges}
\begin{itemize}
  \item []

\end{itemize}

\section{1.6 Inverses and Transposes}
\begin{itemize}
  \item []

\end{itemize}

\section{1.7 Special Matrices and Applications}
\begin{itemize}
  \item []


\end{itemize}
