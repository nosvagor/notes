\chapter{2 Vector Spaces}

\section{2.1 Vector Spaces and Subspaces}
\begin{itemize}
  \item[]

  \subsection{Problems 25, 26, 30, 31}
  \begin{itemize}
    \minor{\item[25.] If we add an extra column \tbm{b} to a matrix \tbm{A},
      then the column space gets larger unless \aset{<?>}.

      \item Give an example in which the column space gets larger and an example in
        which it doesn't.

      \item Why is Ax = b solvable exactly when the column space doesn't get larger
      by including b?}

      \minor{\item[26.] The columns of \tbm{AB} are combinations of the columns
      of \tbm{A}. This means: the column space of \tbm{AB} contained in
      (possibly equal to) the column space of \tbm{A}. Give an example where the
      column spaces of \tbm{A} and \tbm{AB} are not equal.}

      \minor{\item[30.] If the \(9 \times 12\) system \(\bm{Ax} = \bm{b}\) is
      solvable for every \tbm{b}, then \aset{\(C(A) =\)}}.

      \minor{\item[31.] Why isn't \(\R^2\) a subspace of \(\R^3\)?}
  \end{itemize}

\end{itemize}

\section{2.2 Solving Ax = 0 and Ax = b}
\begin{itemize}
  \item []

  \subsection{Problems 12, 24, 25, 70}
  \begin{enumerate}
    \minor{\item[12.] Which of these rules give a correct definition of the
    rank of \tbm{A}?
    \begin{enumerate}
      \item The number of nonzero rows in \tbm{R}.
      \item The number of columns minus the total number of rows.
      \item The number of columns minus the number of free columns.
      \item The number of 1s in \tbm{R}.
    \end{enumerate}}

    \minor{\item[24.] Every column of \tbm{AB} is a combination of the columns
    of \tbm{A}. Then the dimensions of the column spaces give \(\rank{\bm{AB}}
    \leq \rank{ \bm{A} }\).

    Problem: Prove also that \(\rank{\bm{AB}} \leq \rank{ \bm{B} }\).}

    \minor{\item[25.] (Important) Suppose \tbm{A} and \tbm{B} are \(n \times
    n\) matrices, and \tbm{AB} = \tbm{I}. Prove from \(\rank{ AB } \leq
    \rank{ A }\) that the rank of \tbm{A} is n. So A is invertible and \tbm{B}
    must be its two-sided inverse. Therefore \(\bm{BA} = \bm{I}\)}

    \minor{\item[70.] Explain why \tbm{A} and \(-\bm{A}\) always have the same
    reduced echelon form \tbm{R}.}
  \end{enumerate}



\end{itemize}

\section{2.3 Linear Independence, Basis, and Dimension}
\begin{itemize}
  \item []

  \subsection{Problems 9, 13, 28, 36}
  \begin{enumerate}
    \minor{\item[9.] Suppose \(\bm{v}_1, \bm{v}_2, \bm{v}_3, \bm{v}_4\) are
      vectors in \(\R^3\).
      \begin{enumerate}
        \item There four vectors are dependent because \aset{<?>}
        \item The two vectors \(\bm{v}_1\) and \(\bm{v}_2\) will be dependent
          if \aset{<?>}
        \item The vectors \(\bm{v}_1\) and \((0,0,0)\) are dependent because
          \aset{<?>}
      \end{enumerate}}

    \minor{\item[13.] Find the dimensions of:
      \begin{enumerate}
        \item the column space of \tbm{A}:
        \item the column space of \tbm{U}:
        \item the row space of \tbm{A}:
        \item the row space of \tbm{U}.
    \end{enumerate}
    \[%%%%%%%%%%
    \bm{A} =
    \begin{bmatrix}
      1 & 1 & 0 \\
      1 & 3 & 1 \\
      3 & 1 & -1
    \end{bmatrix} \quad ~\text{and}~ \quad
    \bm{U} =
    \begin{bmatrix}
      1 & 1 & 0  \\
      0 & 2 & 1  \\
      0 & 0 & 0
    \end{bmatrix}
    \]%%%%%%%%%%
    \begin{itemize}
      \item Which two of the spaces are the same?
    \end{itemize}}

    \minor{\item[28.] True or false (give a good reason)?
    \begin{enumerate}
      \item If the columns of a matrix are dependent, so are the rows.
      \item The column space of a \(2 \times 2\) matrix is the same as its row space.
      \item The column space of a \(2 \times 2\) matrix has the same dimension as its row space.
      \item The columns of a matrix are a basis for the column space.
    \end{enumerate}
  }

  \minor{\item[36.] If \tbm{A} is a \(64 \times 17\) matrix of rank 11, how
  many independent vectors satisfy \(\bm{Ax} = \nil\)?}

  \end{enumerate}
\end{itemize}

\section{2.4 The Four Fundamental Subspaces}
\begin{itemize}
  \item[]

  \subsection{Problems 6, 14, 15, 27}
  \begin{enumerate}
    \minor{\item[6.] Suppose \tbm{A} is an \(m \times n\) matrix of rank \(r\).
    Under what conditions on those numbers does
    \begin{enumerate}
      \item A have a two-sided inverse: \(\bm{A}\bm{A}^{-1} = \bm{A}^{-1}\bm{A} = \bm{I}\)?
      \item \(\bm{Ax} = \bm{b}\) have infinitely many solutions for every \tbm{b}?
    \end{enumerate}
    }

    \minor{\item[14.] 14. Find a left-inverse and/or a right-inverse (when they
      exist) for
      \[%%%%%%%%%%
      \bm{A} =
      \begin{bmatrix}
        1 & 1 & 0 \\
        0 & 1 & 1
      \end{bmatrix} \quad\text{and}\quad
      \bm{M} =
      \begin{bmatrix}
        1 & 0  \\
        1 & 1  \\
        0 & 1
      \end{bmatrix} \qand
      \bm{T} =
      \begin{bmatrix}
        a & b \\
        0 & a
      \end{bmatrix}
      \]%%%%%%%%%%
    }

    \minor{\item[15.] 15. If the columns of A are linearly independent, then
    the rank is \aset{<?>}, the nullspace is \aset{<?>}, the row space is
    \aset{<?>}, and there exists a \aset{<?>}-inverse.}

    \minor{\item[27.] (Important) \tbm{A} is an \(m \times n\) matrix of rank
    \(r\). Suppose there are right-hand sides \tbm{b} for which \(\bm{Ax} =
    \bm{b}\) has no solution.
    \begin{enumerate}
      \item What inequalities must be true between \(m, n, ~\text{and}~r\)?
      \item How do you know that \(\bm{A}^T \bm{y} = \nil\) has a nonzero
        solution?
    \end{enumerate}}

  \end{enumerate}
\end{itemize}

\section{2.6 Linear Transformations}
\begin{itemize}
  \item []

  \subsection{Problems 6, 14, 34, 39}
  \begin{enumerate}
    \minor{\item[6.] What 3 by 3 matrices represent the transformation that
      \begin{enumerate}
        \item project every vector onto the x-y plane?
        \item reflect every vector through the x-y plane?
        \item rotate the x-y plane through 90°, leaving the z-axis alone?
        \item rotate the x-y plane, then x-z, then y-z, through 90°?
        \item carry out the same three rotations, but each one through 180°?
      \end{enumerate}
    }

    \minor{\item[14.] Prove that \(T^2\) is a linear transformation if \(T\) is
    linear (from \(\R^3 \to \R^3) \)}

    \minor{\item[34.] The transformation T that transposes every matrix is
      definitely linear. Which of these extra properties are true?
      \begin{enumerate}
        \item \(T^2\) = identity transformation.
        \item The kernel of \(T\) is the zero matrix.
        \item Every matrix is in the range of \(T\).
        \item \(T(M) = -M\) is impossible.
      \end{enumerate}
    }

    \minor{\item[39.] If you keep the same basis vectors but put them in a
      different order, the change-of-basis matrix \tbm{M} is a \aset{<?>}
      matrix. If you keep the basis vectors in order but change their lengths,
      \tbm{M} is a \aset{<?>} matrix.
    }
  \end{enumerate}

\end{itemize}
