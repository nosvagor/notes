\chapter{2 Vector Spaces}

\section{2.1 Vector Spaces and Subspaces}
\begin{itemize}
  \item[]

  \subsection{Problems 25, 26, 30, 31}
  \begin{itemize}
    \minor{\item[25.] If we add an extra column \tbm{b} to a matrix \tbm{A},
      then the column space gets larger unless \aset{they are linearly
      dependent.}.

      \item Give an example in which the column space gets larger and an example in
        which it doesn't.
        \fg{\[%%%%%%%
        \text{Larger:}
        \begin{bmatrix}[cc|c]
          1 & 0 & 0 \\
          2 & 4 & 0 \\
          3 & 2 & 1
        \end{bmatrix}
        \qquad
        \text{No change:}
        \begin{bmatrix}[cc|c]
          1 & 0 & 3 \\
          2 & 4 & 6 \\
          3 & 2 & 9
        \end{bmatrix}
        \]}%%%%%%%%%%

      \item Why is \(\bm{Ax} = \bm{b}\) solvable exactly when the column space doesn't get larger
      by including \tbm{b}?}
      \begin{itemize}
        \item Because the solution would be in the image, leading to infinite
          solutions since it could be written as a linear combination of the
          vectors already in \tbm{A}.
      \end{itemize}

      \minor{\item[26.] The columns of \tbm{AB} are combinations of the columns
      of \tbm{A}. This means: the column space of \tbm{AB} contained in
      (possibly equal to) the column space of \tbm{A}. Give an example where the
      column spaces of \tbm{A} and \tbm{AB} are not equal.}

      \linkpy{vectorSpaces}
      \[%%%%%%%%%%
      \bm{A} =
      \begin{bmatrix}
        1 & 0 & 3 \\
        2 & 4 & 6 \\
        3 & 2 & 9
      \end{bmatrix} \qquad
      \bm{B} =
      \begin{bmatrix}
        2 & 0 & 0 \\
        0 & 0 & 0 \\
        0 & 0 & 0
      \end{bmatrix}
      \]%%%%%%%%%%
      \begin{itemize}
        \item The column space of \tbm{B} is clearly contained in \tbm{A}, but
          since the dimension of the null space of \tbm{B} is 2, then \tbm{AB}
          will not have the same column space of \tbm{A}.
      \end{itemize}


      \minor{\item[30.] If the \(9 \times 12\) system \(\bm{Ax} = \bm{b}\) is
      solvable for every \tbm{b}, then \aset{\(C(A) = 9\)}}.
      \begin{itemize}
        \item This follows from the Rank-nullity theorem, where the
          dimensionality of the column space and cokernel is equal to the rows
          of the original matrix.
        \item This also implies the dimensionality of the cokernel is 3.
      \end{itemize}


      \minor{\item[31.] Why isn't \(\R^2\) a subspace of \(\R^3\) ?}
      \begin{itemize}
        \item \(\R^2\) could be a \textit{subset}, but not a subspace; there
          are infinite 2-dimensional planes in \(\R^3\).

        \item If you included a third point of 0 in \(\R^2\), then it would
          indicate that includes the origin, which could make it a subspace or
          \(\R^3\). However, that extra coordinate would be make it be a vector
          in \(\R^3\).
      \end{itemize}

  \end{itemize}

\end{itemize}

\section{2.2 Solving Ax = 0 and Ax = b}
\begin{itemize}
  \item []

  \subsection{Problems 12, 24, 25, 70}
  \begin{enumerate}
    \minor{\item[12.] Which of these rules give a correct definition of the
    rank of \tbm{A}?
    \begin{enumerate}
      \item The number of nonzero rows in \tbm{R}. \true{true}
        \[%%%%%%%%%%
          \fg{\max{r} = r \in \N : 0 \leq r \leq \min{m, n}}
        \]%%%%%%%%%%
      \item The number of columns minus the total number of rows.
        \begin{itemize}\color{foreground}
          \item This would yield dimensionality of the null space.
        \end{itemize}

      \item The number of columns minus the number of free columns.
        \begin{itemize}\color{foreground}
          \item This would yield the dimensionality of the left null space.
        \end{itemize}

      \item The number of 1s in \tbm{R}.
        \begin{itemize}\color{foreground}
          \item This wouldn't tell you much of anything.
        \end{itemize}
    \end{enumerate}
    }

    \minor{\item[24.] Every column of \tbm{AB} is a combination of the columns
    of \tbm{A}. Then the dimensions of the column spaces give \(\rank{\bm{AB}}
    \leq \rank{ \bm{A} }\).

    Problem: Prove that \(\rank{\bm{AB}} \leq \rank{ \bm{B} }\).
    }
    \begin{align*}
      \rank{A} &= C(A) = C(A^T)\\
      \then \rank{A} &= \rank{A^T} \\
      \thus \rank{AB} &= \rank{(AB)^T} = \rank{B^TA^T} \leq \rank{B^T} = \rank{B}
    \end{align*}
    \begin{itemize}
      \item The rank of \tbm{AB} can only be decreased, if \tbm{B} is not full
        rank itself.
      \item If, both \tbm{A} and \tbm{B} are full rank, then
        \(\rank{\bm{AB}} = \rank{\bm{B}}\)
    \end{itemize}

    \minor{\item[25.] (Important) Suppose \tbm{A} and \tbm{B} are \(n \times
    n\) matrices, and \tbm{AB} = \tbm{I}. Prove from \(\rank{ AB } \leq
    \rank{ A }\) that the rank of \tbm{A} is n. So \tbm{A} is invertible and \tbm{B}
    must be its two-sided inverse. Therefore \(\bm{BA} = \bm{I}\)}
    \begin{itemize}
      \item \tbm{A} must have same size of \tbm{B}, given they are both \(n
        \times n\).
      \item If \tbm{A} was rank deficient, but \tbm{B} was full rank, then
      \(\rank{AB} \leq \rank{\bm{A}}\) would be invalid, forcing
      \(\rank{\bm{A}} = n\).
    \end{itemize}

    \minor{\item[70.] Explain why \tbm{A} and \(-\bm{A}\) always have the same
    reduced echelon form \tbm{R}.}
    \begin{itemize}
      \item Signed solutions are arbitrary; \tbm{-A} would have permutations that
        flip the sign and yield the same, reversible, solution.
    \end{itemize}
  \end{enumerate}
\end{itemize}

\section{2.3 Linear Independence, Basis, and Dimension}
\begin{itemize}
  \item []

  \subsection{Problems 9, 13, 28, 36}
  \begin{enumerate}
    \minor{\item[9.] Suppose \(\bm{v}_1, \bm{v}_2, \bm{v}_3, \bm{v}_4\) are
      vectors in \(\R^3\).
      \begin{enumerate}
        \item There four vectors are dependent because \aset{\(\R^3\) can only
          have 3 linearly independent vectors}.
        \item The two vectors \(\bm{v}_1\) and \(\bm{v}_2\) will be dependent
          if \aset{they are multiples of each other.}
        \item The vectors \(\bm{v}_1\) and \((0,0,0)\) are dependent because
          \aset{\(\bm{v}_1(0, 0, 0) = \nil\)}
      \end{enumerate}}

    \minor{\item[13.] Find the dimensions of:
      \begin{enumerate}
        \item the column space of \tbm{A}:
        \item the column space of \tbm{U}:
        \item the row space of \tbm{A}:
        \item the row space of \tbm{U}.
    \end{enumerate}
    \[%%%%%%%%%%
    \bm{A} =
    \begin{bmatrix}
      1 & 1 & 0 \\
      1 & 3 & 1 \\
      3 & 1 & -1
    \end{bmatrix} \quad ~\text{and}~ \quad
    \bm{U} =
    \begin{bmatrix}
      1 & 1 & 0  \\
      0 & 2 & 1  \\
      0 & 0 & 0
    \end{bmatrix}
    \]%%%%%%%%%%
    \linkpy{vectorSpaces}
    \begin{itemize}
      \item Which two of the spaces are the same?
        \begin{itemize}\color{foreground}
          \item The row space of \tbm{A} and \tbm{U} are the same. Why? Taking
            \tbm{A} to rref shows it's also rank 2, with a linearly dependent
            row (0, 0, 0), just like \tbm{U}.
        \end{itemize}
    \end{itemize}}

    \minor{\item[28.] True or false (give a good reason)?
    \begin{enumerate}
      \item If the columns of a matrix are dependent, so are the rows.
        \begin{itemize}\color{foreground}
          \item \false{false}; as shown in \ulink{s:2.2 Solving Ax = 0 and
            Ax = b}{problem 2.2.12}, the max rank is the minimum of either the rows
            or columns. You could have linearly dependent columns, if \(n >
            m\), but no linearly dependent rows.
        \end{itemize}

      \item The column space of a \(2 \times 2\) matrix is the same as its row space.
        \begin{itemize}\color{foreground}
          \item \false{false}; e.g., \(\begin{bsmallmatrix} 4 & 2 \\ 0 & 0 \end{bsmallmatrix}\)
        \end{itemize}

      \item The column space of a \(2 \times 2\) matrix has the same dimension as its row space.
        \begin{itemize}\color{foreground}
          \item \true{true}; \(\rank{\bm{A}} = C(\bm{A}) = C(\bm{A}^T)\)
        \end{itemize}

      \item The columns of a matrix are a basis for the column space.
        \begin{itemize}\color{foreground}
          \item \false{false}; one of the columns could be linearly dependent
            with another; only linearly independent columns forms a basis for
            the column space.
        \end{itemize}

    \end{enumerate}
  }

  \minor{\item[36.] If \tbm{A} is a \(64 \times 17\) matrix of rank 11, how
    many independent vectors satisfy \(\bm{Ax} = \nil\)? How many independent
    vectors satisfy \(\bm{A}^T \bm{y} = \nil\)?
  }
  \begin{itemize}
    \item \(17 - 11 = \aset{6}, \quad 64 - 11 = \aset{53}\), respectively.
  \end{itemize}


  \end{enumerate}
\end{itemize}

\section{2.4 The Four Fundamental Subspaces}
\begin{itemize}
  \item[]

  \subsection{Problems 6, 14, 15, 27}
  \begin{enumerate}
    \minor{\item[6.] Suppose \tbm{A} is an \(m \times n\) matrix of rank \(r\).
    Under what conditions on those numbers does
    \begin{enumerate}
      \item A have a two-sided inverse: \(\bm{A}\bm{A}^{-1} = \bm{A}^{-1}\bm{A} = \bm{I}\)?
      \item \(\bm{Ax} = \bm{b}\) have infinitely many solutions for every \tbm{b}?
    \end{enumerate}
    }

    \minor{\item[14.] 14. Find a left-inverse and/or a right-inverse (when they
      exist) for
      \[%%%%%%%%%%
      \bm{A} =
      \begin{bmatrix}
        1 & 1 & 0 \\
        0 & 1 & 1
      \end{bmatrix} \quad\text{and}\quad
      \bm{M} =
      \begin{bmatrix}
        1 & 0  \\
        1 & 1  \\
        0 & 1
      \end{bmatrix} \qand
      \bm{T} =
      \begin{bmatrix}
        a & b \\
        0 & a
      \end{bmatrix}
      \]%%%%%%%%%%
    }

    \minor{\item[15.] 15. If the columns of A are linearly independent, then
    the rank is \aset{<?>}, the nullspace is \aset{<?>}, the row space is
    \aset{<?>}, and there exists a \aset{<?>}-inverse.}

    \minor{\item[27.] (Important) \tbm{A} is an \(m \times n\) matrix of rank
    \(r\). Suppose there are right-hand sides \tbm{b} for which \(\bm{Ax} =
    \bm{b}\) has no solution.
    \begin{enumerate}
      \item What inequalities must be true between \(m, n, ~\text{and}~r\)?
      \item How do you know that \(\bm{A}^T \bm{y} = \nil\) has a nonzero
        solution?
    \end{enumerate}}

  \end{enumerate}
\end{itemize}

\section{2.6 Linear Transformations}
\begin{itemize}
  \item []

  \subsection{Problems 6, 14, 34, 39}
  \begin{enumerate}
    \minor{\item[6.] What 3 by 3 matrices represent the transformation that
      \begin{enumerate}
        \item project every vector onto the x-y plane?
        \item reflect every vector through the x-y plane?
        \item rotate the x-y plane through 90°, leaving the z-axis alone?
        \item rotate the x-y plane, then x-z, then y-z, through 90°?
        \item carry out the same three rotations, but each one through 180°?
      \end{enumerate}
    }

    \minor{\item[14.] Prove that \(T^2\) is a linear transformation if \(T\) is
    linear (from \(\R^3 \to \R^3) \)}

    \minor{\item[34.] The transformation T that transposes every matrix is
      definitely linear. Which of these extra properties are true?
      \begin{enumerate}
        \item \(T^2\) = identity transformation.
        \item The kernel of \(T\) is the zero matrix.
        \item Every matrix is in the range of \(T\).
        \item \(T(M) = -M\) is impossible.
      \end{enumerate}
    }

    \minor{\item[39.] If you keep the same basis vectors but put them in a
      different order, the change-of-basis matrix \tbm{M} is a \aset{<?>}
      matrix. If you keep the basis vectors in order but change their lengths,
      \tbm{M} is a \aset{<?>} matrix.
    }
  \end{enumerate}

\end{itemize}
