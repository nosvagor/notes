\chapter{3 Orthogonality}

\section{3.1 Orthogonal Vectors and Subspaces}
\begin{itemize}
  \item []
  \subsection{Problems 6, 46, 47}
  \begin{enumerate}
    \minor{\item[6.] Find all vectors in \(R^3\) that are orthogonal to \((1,1,1)\)
      and \((1,-1,0)\). Produce an orthonormal basis from these vectors
      (mutually orthogonal unit vectors).}
      \linkpy{orthogonality}
      \begin{itemize}
        \item Simply taking the cross product between two vectors in \(\R^3\)
          yields a new vector that is normal to the plane containing them.
        \item \((1,1,1) \times (1,-1,0) = \aset{(1,1,-2)}\)
      \end{itemize}

    \minor{\item[46.] Find \(\bm{A}^T \bm{A}\) if the columns of \tbm{A}
      are unit vectors, all mutually perpendicular.
    }
    \[%%%%%%%%%%
    \begin{bmatrix}
      \norm{\bm{a_1}}^2 & a_\perp & \cdots & a_\perp \\
      a_\perp & \norm{\bm{a}_2}^2 & a_\perp & \vdots \\
      \vdots & a_\perp & \ddots & a_\perp  \\
      a_\perp & \cdots & a_\perp &\norm{\bm{a}_n}^2
    \end{bmatrix} =
    \begin{bmatrix}
      1 & 0 & \cdots & 0 \\
      0 & 1 & 0 & \vdots \\
      \vdots & 0 & \ddots & 0  \\
      0 & \cdots & 0 & 1
    \end{bmatrix} = \bm{I}
    \]%%%%%%%%%%

    \minor{\item[47.] Construct a \(3 \times 3\) matrix \tbm{A} with no zero
      entries whose columns are mutually perpendicular. Compute \(\bm{A}^T
      \bm{A}\). Why is it a diagonal matrix?
    }

    \linkpy{orthogonality}
    \begin{itemize}
      \item Note: I tried to set up with random matrices each time, sometimes it
        fails and gives the zero matrix.
      \item The way I set up only yields a diagonal matrix with
      \(\bm{A}\bm{A}^T \), \(\bm{A}^T \bm{A} \) yields a symmetric matrix. Why?
    \end{itemize}

  \end{enumerate}
\end{itemize}

\section{3.2 Cosines and Projections onto Lines}
\begin{itemize}
  \item []

  \subsection{Projection Proof (class problem)}
  \begin{itemize}
    \item If I recall the problem correctly, we were requested to \aset{prove what
      \(\proj{v}{w}\) is equal to}.

    \item In my notes I have that a orthogonal projection occurs when the dot
      product between \tbm{v} and distance \tbm{w} from \tbm{v} is equal to
      zero. This follows from the definition of the inner product, i.e.,
      \[%%%%%%%%%%
      \lambda = \bm{v}^T \bm{w} = \norm{v} \norm{w} \cos \theta
      \]%%%%%%%%%%
    \item If \(\theta = 90^\circ\), then the vectors are perpendicular, i.e.,
      orthogonal. What we are missing is \aset{the distance} from \tbm{w} to
      \tbm{v}; the distance that yields an inner product of zero with a
      normalized \tbm{v} \aset{\textit{is the projection}}.
    \item This means we need a scaled version of \tbm{v}, let's call it
      \(\bm{v}\beta\), at which such inner product is equal to zero. At this
      point, the difference between \tbm{w} and \(\bm{v}\beta\) is exactly what
      we need in order to solve for a \(\beta\) that maintains an inner
      product of zero with the original vector \tbm{v}, i.e.,
      \begin{align*}
        \bm{v}^T (\bm{w}-\bm{v}\beta) &= 0 \\
        \bm{v}^T\bm{w}-\bm{v}^T \bm{v}\beta &= 0 \\
        \bm{v}^T \bm{v}\beta &= \bm{v}^T\bm{w} \\
        \beta &= \frac{\bm{v}^T\bm{w}}{\bm{v}^T \bm{v}}\\\\
        \then
        \proj{v}{w} = \bm{v}\beta &= \bm{v}\frac{\bm{v}^T\bm{w}}{\bm{v}^T \bm{v}}
      \end{align*}
    \item I've internalized this as a mapping of \tbm{w} onto
      \tbm{v} over a magnitude (the norm) of \tbm{v}.
      \begin{itemize}
        \item The mapping is important because it tells us the shortest
          distance from \tbm{w} onto \tbm{v}, i.e., when they are orthogonal.
        \item The magnitude is important, because it is the basis at which
          \tbm{w} is parallel to \tbm{v}, which when added mapping distance,
          yields \tbm{w}.
      \end{itemize}
  \end{itemize}

  \newpage
  \subsection{Problems 10, 13, 15}
  \begin{itemize}
    \minor{\item[10.] Is the projection matrix P invertible? Why or why
    not?
    }

    \minor{\item[13.] Prove that the trace of \(P = \frac{\bm{aa}^T}{\bm{a}^T\bm{a}}\) always equals 1.
    }

    \minor{\item[15.] Show that the length of \tbm{Ax} equals the length of
      \(\bm{A}^T \bm{x}\) if \(\bm{AA}^T = \bm{A}^T \bm{A}\).
    }
  \end{itemize}

\end{itemize}

\section{3.3 Projections and Least Squares}
\begin{itemize}
  \item []

\end{itemize}

\section{3.4 Orthogonal Bases and Gram-Schmidt}
\begin{itemize}
  \item []

\end{itemize}

\section{3.5 The Fast Fourier Transform}
\begin{itemize}
  \item []

\end{itemize}
