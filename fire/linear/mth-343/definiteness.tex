\chapter{6 Positive Definite Matrices}

\section{6.1 Minima, Maxima, and Saddle Points}
\begin{itemize}
  \item []

  \subsection{Problems 6, 13, 17}
  \begin{enumerate}
    \minor{\item[6.] Suppose that positive coefficients \(a\) and \(c\)
      dominate \(b\) in the sense that \(a+c > 2b\). Find an example that has
      \(ac < b^2\), so that the matrix is not positive definite.
    }

    \minor{\item[13.] Under what conditions on \(a,b,c\) is \(ax^2 + 2bxy + cy^2 > x^2 + y^2 \quad\forall x,y\)?
    }

    \minor{\item[17.] (Important) If \tbm{A} has independent columns, then
      \(\bm{A}^T \bm{A}\) is square, symmetric, and invertible.

      Rewrite \(\bm{x}^T \bm{A}^T \bm{Ax} \) to show why it is positive except
      when \(x=0\). Then \(\bm{A}^T \bm{A}\) is positive definite.
    }

  \end{enumerate}


\end{itemize}

\section{6.2 Tests for Positive Definiteness}
\begin{itemize}
  \item []

  \subsection{Problems 5, 12, 15}
  \begin{enumerate}
    \minor{\item[5.] If \tbm{A} and \tbm{B} are positive definite, then
      \(\bm{A} + \bm{B} \) is positive definite. Pivots and eigenvalues are not
      convenient for \(\bm{A}+\bm{B}\). Much better to prove that \(\bm{x}^T
      (\bm{A}+\bm{B})\bm{x}>\nil\).
    }

    \minor{\item[12.] In three dimensions, \(\lambda_1y^2_1 + \lambda_1y^2_2 +
      \lambda_1y^2_3 = 1\) represents an ellipsoid when all \(\lambda_i > 0\).
      Describe all the different kinds of surfaces that appear in the positive
      semi-definite case when one or more of the eigenvalues is zero.
    }

    \minor{\item[15.] Suppose \tbm{A} is symmetric positive definite and
      \tbm{Q} is an orthogonal matrix. True of false:
      \begin{enumerate}
        \item \(\bm{Q}^T \bm{AQ}\) is a diagonal matrix.
        \begin{itemize}\color{foreground}
          \item
        \end{itemize}

        \item \(\bm{Q}^T \bm{AQ}\) is symmetric positive definite.
        \begin{itemize}\color{foreground}
          \item
        \end{itemize}

        \item \(\bm{Q}^T \bm{AQ}\) has the same eigenvalues as \tbm{A}.
        \begin{itemize}\color{foreground}
          \item
        \end{itemize}

        \item \(e^{-\bm{A}} \) is a symmetric positive definite.
        \begin{itemize}\color{foreground}
          \item
        \end{itemize}

      \end{enumerate}
    }
  \end{enumerate}
\end{itemize}

\section{6.3 Singular Value Decomposition}
\begin{itemize}
  \item []


\end{itemize}

\section{6.4 Minimum Principles}
\begin{itemize}
  \item []


\end{itemize}

\section{6.5 The Finite Element Method}
\begin{itemize}
  \item []


\end{itemize}
