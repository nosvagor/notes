\chapter{6 Positive Definite Matrices}

\section{6.1 Minima, Maxima, and Saddle Points}
\begin{itemize}
  \item []

  \subsection{Problems 6, 13, 17}
  \begin{enumerate}
    \minor{\item[6.] Suppose that positive coefficients \(a\) and \(c\)
      dominate \(b\) in the sense that \(a+c > 2b\). Find an example that has
      \(ac < b^2\), so that the matrix is not positive definite.
    }
    \begin{itemize}\color{foreground}
      \item If \(a \lor c = 0\), while the other coefficient remains \(> 2b\),
        then the matrix would be positive semi-definite, guaranteeing \(ac < b^2\).
    \end{itemize}

    \minor{\item[13.] Under what conditions on \(a,b,c\) is \(ax^2 + 2bxy +
      cy^2 > x^2 + y^2 \quad\forall x,y\)?
    }
    \begin{align*}
      &\then ax^2 - x^2 + 2 bxy + cy^2 - y ^2 > 0 \\
      &\then (a-1)x^2 + 2bxy + (c-1)y^2 > 0 \\
      &\then \begin{bmatrix} x && y \end{bmatrix}
      \begin{bmatrix}
        a-1 & b \\
        b & c-1
      \end{bmatrix}
      \begin{bmatrix} x \\ y \end{bmatrix} > 0 \\\\
      &\then \aset{a > 1} \land \aset{(a-1)(c-1) > b^2}\quad \forall x,y
    \end{align*}

    \minor{\item[17.] (Important) If \tbm{A} has independent columns, then
      \(\bm{A}^T \bm{A}\) is square, symmetric, and invertible.

      Rewrite \(\bm{x}^T \bm{A}^T \bm{Ax} \) to show why it is positive except
      when \(\bm{x}=\nil\). Then \(\bm{A}^T \bm{A}\) is positive definite.
    }
    \begin{align*}
      \bm{x}^T \bm{A}^T \bm{Ax} =  (\bm{Ax})^T(\bm{Ax}) = \norm{\bm{Ax}}^2 \\\\
      \norm{\bm{Ax}}^2 > \nil \iff \norm{\bm{Ax}} > \nil \iff \bm{Ax} \neq \nil \\\\
      \then \bm{x} \neq \nil \given ~\text{\tbm{A} has independent columns}~
    \end{align*}

    \minor{\item[21.] What is a test on \(F(x,y)\) to have a saddle at \((0,0)\)?
    }
    \begin{itemize}\color{foreground}
      \item Saddle points occur when the matrix is indefinite, i.e., there is
        at least two eigenvalues with opposite signs and a stationary point
        (not a minima or maxima).
      \item Thus, for \(F(x,y) = ax^2 + 2bxy + cy^2 \) to be indefinite and have a stationary point at (0,0), there
        must be \[ac < b^2 \given a,c \neq 0 \then \lambda < 0 \land \lambda > 0\]
    \end{itemize}
  \end{enumerate}


\end{itemize}

\section{6.2 Tests for Positive Definiteness}
\begin{itemize}
  \item []

  \subsection{Problems 5, 12, 15}
  \begin{enumerate}
    \minor{\item[5.] If \tbm{A} and \tbm{B} are positive definite, then
      \(\bm{A} + \bm{B} \) is positive definite. Pivots and eigenvalues are not
      convenient for \(\bm{A}+\bm{B}\). Much better to prove that \(\bm{x}^T
      (\bm{A}+\bm{B})\bm{x}>\nil\).
    }
    \begin{align*}
      &: \bm{x}^T \bm{A} \bm{x} > 0, \quad \bm{x}^T \bm{Bx} > 0
      && \given \bm{x}^T \bm{kx} > 0 \quad\forall \bm{x} > \nil \in \R \\
      &\then \bm{x}^T \bm{Ax} + \bm{x}^T \bm{Bx} > 0 \\
      &\then \bm{x}^T (\bm{A}+\bm{B}) \bm{x} > 0
    \end{align*}

    \minor{\item[12.] In three dimensions, \(\lambda_1y^2_1 + \lambda_1y^2_2 +
      \lambda_1y^2_3 = 1\) represents an ellipsoid when all \(\lambda_i > 0\).
      Describe all the different kinds of surfaces that appear in the positive
      semi-definite case when one or more of the eigenvalues is zero.
    }
    \begin{itemize}
      \item \(\times1 ~ \lambda = 0 \to \) the ellipsoid becomes an ellipse.
      \item \(\times 2 ~ \lambda = 0 \to \) the ellipsoid projects down to a
        parabola.
      \item \(\times 3 ~ \lambda = 0 \to \) a point at the origin.
    \end{itemize}


    \minor{\item[15.] Suppose \tbm{A} is symmetric positive definite and
      \tbm{Q} is an orthogonal matrix. True or false:
      \begin{enumerate}
        \item \(\bm{Q}^T \bm{AQ}\) is a diagonal matrix.
        \begin{itemize}\color{foreground}
          \item \false{false}; \tbm{A} is diagonalizable, given it's symmetric
            positive definite. \tbm{Q} needs to contain the eigenvectors of
            \tbm{A}, however.
        \end{itemize}

        \item \(\bm{Q}^T \bm{AQ}\) is symmetric positive definite.
        \begin{itemize}\color{foreground}
          \item \true{true}; \(\bm{Q}^T = \bm{Q}^{-1}\), and \(\bm{Q}^{-1}
            \bm{AQ}\) has the same eigenvalues of \tbm{A}, thus \(\bm{Q}^T
            \bm{AQ} \) is similar to \tbm{A} which implies it also symmetric
            positive definite.
        \end{itemize}

        \item \(\bm{Q}^T \bm{AQ}\) has the same eigenvalues as \tbm{A}.
        \begin{itemize}\color{foreground}
          \item \true{true}; see (b).
        \end{itemize}

        \item \(e^{-\bm{A}} \) is a symmetric positive definite.
        \begin{itemize}\color{foreground}
          \item \true{true}; again, \tbm{A} is symmetric positive definite,
            thus eigenvalues of \(e^{-\bm{A}} = e^{-\lambda_{ii} } \), which
            are always positive.
        \end{itemize}

      \end{enumerate}
    }
  \end{enumerate}
\end{itemize}
