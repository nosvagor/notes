\chapter{4 Determinants}

\section{4.2 Properties of the Determinant}
\begin{itemize}
  \item[]

  \subsection{Problems 8, 14}
  \begin{enumerate}
    \minor{\item[8.] Show how rule 6 (det = 0 if a row is zero) comes
      directly from rules 2 and 3.
    }
    \begin{itemize}
      \item Rule 2 states there is nothing special about the first
    row (only the sign changes). Rule 3 (the determinant depends
    linearly on the first row) implies that:
    \[%%%%%%%%%%
    \begin{bmatrix}
      ta & tb \\
      c & d
    \end{bmatrix} =
    t
    \begin{bmatrix}
      a  & b \\
      c & d
    \end{bmatrix}
    \]%%%%%%%%%%
    Thus, a common factor of row 1 could be zero \((t=0)\), meaning the
    determinant itself must be zero.
    \end{itemize}


    \minor{\item[14.] 14. True or false, with reason if true and counterexample
      if false:
      \begin{enumerate}
        \item If \tbm{A} and \tbm{B} are identical except that \(b_{11} =
          2a_{11}\) then \(\det{\bm{B}} = 2 \det{A}\)
          \begin{itemize}\color{foreground}
            \item \false{false}; the entire row must be scaled by 2, not just
              \(a_{11}\), e.g.,
              \[%%%%%%%%%%
              \bm{A} =
              \begin{bmatrix}
                1 & 1 \\
                1 & 0
              \end{bmatrix}, \quad
              \bm{B} =
              \begin{bmatrix}
                2 & 1 \\
                1 & 0
              \end{bmatrix}
              \then
              \det{\bm{A}} = -1 = \det{\bm{B}}
              \]%%%%%%%%%%
              Thus, \(\det{\bm{A}} \neq 2\det{\bm{B}}  \)
          \end{itemize}
        \item The determinant is the product of the pivots.
          \begin{itemize}\color{foreground}
            \item \true{true}; if it is an upper/lower diagonal matrix.
              \prn{I am assuming the pivots defined as elements in row-reduced form of a
              matrix}.
          \end{itemize}

        \item If \tbm{A} is invertible and \tbm{B} is singular, then \(\bm{A} +
          \bm{B}\) is invertible.
          \begin{itemize}\color{foreground}
            \item \false{false}; e.g.,
              \[%%%%%%%%%%
                \bm{A} =
                \begin{bmatrix}
                  6 & 9 \\
                  1 & 1
                \end{bmatrix}, \quad
                \bm{B} =
                \begin{bmatrix}
                  0 & 0 \\
                  -1 & -1
                \end{bmatrix},\quad
                \bm{A+B} =
                \begin{bmatrix}
                  6 & 9 \\
                  0 & 0
                \end{bmatrix}
              \]%%%%%%%%%%
            Thus, \(\det{\bm{A+B}} \) is singular despite \tbm{A} being
            invertible.
          \end{itemize}

        \item If \tbm{A} is invertible and \tbm{B} is singular, then \tbm{AB}
          is singular.
          \begin{itemize}\color{foreground}
            \item \true{true}; given \(\det{\bm{AB}} = \det{\bm{A}}
              \det{\bm{B}} \), then it must follow that \(\det{\bm{AB}= 0} \)
              if one of the matrices are singular.
          \end{itemize}

        \item The determinant of \(\bm{AB} - \bm{BA} \) is zero.
          \begin{itemize}\color{foreground}
            \item \false{false}; are- and post-multiplication of matrices can yield
              different results, thus \(\det{\bm{AB - BA}}\) is often more than
              just a sign flip.
          \end{itemize}

      \end{enumerate}
    }
  \end{enumerate}

\end{itemize}

\section{4.3 Formulas for the Determinant}
\begin{itemize}
  \item []

  \subsection{Problems 3, 9, 18}
  \begin{enumerate}
    \minor{\item[3.] True or false?
      \begin{enumerate}
        \item The determinant of \(\bm{S}^{-1}\bm{AS}\) equals the determinant
          of \tbm{A}.
          \begin{itemize}\color{foreground}
            \item \true{true}; \(\det{\bm{A^{-1}}} = \det{\bm{A}}^{-1}  \), thus
              \begin{align*}
                \aset{\det{\bm{S^{-1}AS}}} = \det{\bm{S^{-1}}} \det{\bm{A}}
                \det{\bm{S}} \\
                \then \det{\bm{S^{-1}}}\det{\bm{S}} \det{\bm{A}} = \aset{\det{\bm{A}}}
              \end{align*}
          \end{itemize}
        \item If \(\det{\bm{A}} = 0\) then at least one of those cofactors must
          be zero.
          \begin{itemize}\color{foreground}
            \item \false{false}; \(\det{\bm{A}} = 0 \iff\)the matrix is
              singular. All cofactors could be nonzero, but there could still
              be linearly dependent columns/rows, which would yield a singular
              matrix.
          \end{itemize}

        \item A matrix whose entries are 0s and 1s has determinant 1, 0, or -1.
          \begin{itemize}\color{foreground}
            \item \false{false}; a \textit{row-reduced matrix} that contains all 1s or
              0s would yield a determinant of 1, 0, or -1. Most matrices will
              not contain only 1s or 0s after row-reduction.
          \end{itemize}

      \end{enumerate}
    }

    \minor{\item[9.] How many multiplications to find an \(n \times n\)
      determinant from
      \begin{enumerate}
        \item The big formula (6)?
          \begin{itemize}\color{foreground}
            \item The big formula is the sum of all possible permutations,
              i.e., \aset{\(n \cdot n!\)}
          \end{itemize}
        \item The cofactor formula (10), building from the count for \(n-1\)?
          \begin{itemize}\color{foreground}
            \item The cofactor formula is just an algorithmic way to find all
              possible permutations by expanding all possible cofactors one
              element at a time until all are filled. This yields the same complexity as the big formula, i.e., \aset{\(n\cdot n!\)}
          \end{itemize}
        \item The product of the pivots formula (including the elimination
          steps)?
          \begin{itemize}\color{foreground}
            \item Each row requires \(n(n-1)\) steps to reduce, followed by
              multiplication along the diagonal, i.e., \aset{\(\sum_{i=2}^{n}
              i(i-1) + n\)}
          \end{itemize}

      \end{enumerate}
    }

    \minor{\item[18.] Place the smallest number of zeros in a \(4 \times 4\)
      matrix that will guarantee \(\det{\bm{A}}=0 \).
      \begin{itemize}\color{foreground}
        \item The smallest number of zeros to \textit{guarantee} \(\det{\bm{A}}
          =0 \) is \(n\), (a full row). I.e., \aset{}; all must be in same
          row/column.
      \end{itemize}

      Place as many zeros as possible while still allowing \(\det{\bm{A}}\neq 0\)
      \begin{itemize}\color{foreground}
        \item The most is \(n^2 - n\), i.e., \aset{12}; all diagonal elements
          must be non-zero.
      \end{itemize}
    }
  \end{enumerate}

\end{itemize}

\section{4.4 Applications of Determinants}
\begin{itemize}
  \item []

  \subsection{Problems 6, 10, 19, 22}
  \begin{enumerate}
    \minor{\item[6.] Explain in terms of volumes why \(\det{\bm{A}}3 =
      3^{n}\det{\bm{A}} \) for an \(n \times n\) matrix.
    }
    \begin{itemize}\color{foreground}
      \item Scaling a determinant implies stretching/shrinking every side side
        of the \(n\)-dimensional object. The determinant represents the
        ``volume'' of this object. Every column represents orthogonal
        dimension that needs to be scaled.

        Thus, \(\det{\bm{A}}3 \) scales each column, implying that \(3^{n} \det{\bm{A}} \)
        represents the scaling of the volume of the object.
    \end{itemize}

    \minor{\item[10.] If \tbm{P} is an odd permutation, explain why
      \(\bm{P}^2\) is even but \(\bm{P}^{-1} \) is odd.
    }
    \begin{itemize}\color{foreground}
      \item \tbm{P} is odd when \(\det{\bm{P}} = -1 \), and even if it is 1.
        \begin{itemize}
          \item Thus, \(P^2= -1^2\), i.e., even.
        \end{itemize}

      \item Since \(\det{\bm{P^{-1}}} = \det{\bm{P}}^{-1},\) then \(\bm{P}^{-1} = \dfrac{1}{-1} \), i.e., odd.
    \end{itemize}


    \minor{\item[19.] If all the cofactors are zero, how do you know that
      \tbm{A} has no inverse? If none of the cofactors are zero, is \tbm{A}
      sure to be invertible?
    }
    \begin{itemize}\color{foreground}
      \item Yes, if all the cofactors are zero, then it implies there is a
        row/column of zeros, yielding a singular matrix.
      \item Not always, only knowing the cofactors will not tell you if there
        are linearly dependent dimensions. Only upon summation of the cofactor
        expansion will you be able to tell for sure. E.g.,
        \[%%%%%%%%%%
        \bm{A} =
        \begin{bmatrix}
          2 & 3 \\
          6 & 9
        \end{bmatrix}
        \]%%%%%%%%%%
    \end{itemize}

    \minor{\item[22.] From the formula \(\bm{AC}^T = \det{\bm{A}} \bm{I}\) show
      that \(\det{\bm{C}}  = \det{\bm{A}}^{n-1}\)
    }
    \begin{align*}
      \then \det{\bm{AC}^T } &= \det{\det{\bm{A}} \bm{I}} \\
      \det{\bm{A}} \det{\bm{C^T }} &= \det{\bm{A}}^n \det{\bm{I}}
      && \text{Product Rule, Rule 3} \\
      \det{\bm{C}}  &=\det{\bm{A}}^n(\det{\bm{A}}^{-1} )
      && \text{Rearrange, Rule 10} \\
      \then \det{\bm{C}} &= \det{\bm{A}}^{n-1} && \text{Simplify} \\
    \end{align*}

  \end{enumerate}

\end{itemize}
