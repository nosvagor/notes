\chapter{5 Eigenvalues and Eigenvectors}

\section{5.1 Introduction to Eigenvalues and Eigenvectors}
\begin{itemize}
  \item[]

  \subsection{Problems 39, 40}
  \begin{enumerate}
    \minor{\item[39.]  Challenge problem: Is there a real 2 by 2 matrix (other
      than \(I\)) with \(\bm{A}^3 = I\)? Its eigenvalues must satisfy \(\lambda^3 =
      I\). They can be \(e^{ 2\pi\i /3}\) and \( e^{-2\pi\i i/3}\).

      What trace and determinant would this give? Construct \tbm{A}.
    }
    \begin{align*}
      \det{\bm{A}} = e^{\frac{2\pi\i}{3}} \cdot e^{-\frac{2\pi\i}{3}} = e^0 = 1 \\
      \tr{\bm{A}} = e^{\frac{2\pi\i}{3}} + e^{-\frac{2\pi\i}{3}} = -1 \\
      \text{A rotation matrix satisfies the condition, thus:}~ \\
      \bm{A} =
      \begin{bmatrix}
        \cos \frac{2\pi}{3}  & -\sin \frac{2\pi}{3}  \\
        \sin \frac{2\pi}{3} & \cos \frac{2\pi}{3}
      \end{bmatrix}
    \end{align*}


    \minor{\item[40.] There are six 3 by 3 permutation matrices \tbm{P}.

      \fg{Note: \ulink{ss:Problems 14, 18}{Section 1.5, problem 14} lists the 6
        permutation matrices; all of which can represented as the product of
        elementary matrices.}

      \begin{enumerate}
        \item What numbers can be the determinants of \tbm{P}?

        \begin{itemize}\color{foreground}
          \item Multiplication by an elementary matrix at most produces a sign flip,
          thus, the determinants are \(\pm 1\).
        \end{itemize}

        \item What numbers can be pivots?
        \begin{itemize}\color{foreground}
          \item All pivots must be equal to 1 (only nonzero number available).
        \end{itemize}

        \item What numbers can be the trace of \tbm{P}?
        \begin{itemize}\color{foreground}
          \item Ranges from 0--3, depends on how many 1's are present in the diagonal.
        \end{itemize}

        \item What numbers can be eigenvalues of \tbm{P}?
        \begin{itemize}\color{foreground}
          \item \(\lambda^3 - 1 = 0\), thus:
            \begin{itemize}
              \item \(\lambda_1 = 1\)
              \item \(\displaystyle \lambda_2 = -\frac{1}{2} - \frac{i\sqrt{3} }{2}\)
              \item \(\displaystyle \lambda_3 = -\frac{1}{2} + \frac{i\sqrt{3}}{2}\)
            \end{itemize}
        \end{itemize}

      \end{enumerate}
    }
  \end{enumerate}
\end{itemize}

\section{5.2 Diagonalization of a Matrix}
\begin{itemize}
  \item []

  \subsection{Problems 11, 14, 39}
  \begin{enumerate}
    \minor{\item[11.] If the eigenvalues of \tbm{A} are 1, 1, 2, which of the
      following are certain to be true? Give a reason if true or a
      counterexample if false:
      \begin{enumerate}
        \item \tbm{A} is invertible.
        \begin{itemize}\color{foreground}
          \item \true{true}; none of the eigenvalues are zero, meaning the
            determinant is nonzero, which means \tbm{A} is non-degenerate (invertible).
        \end{itemize}

        \item \tbm{A} is diagonalizable.
        \begin{itemize}\color{foreground}
          \item \false{false}; depends on number of linearly independent
            eigenvectors.
          \[%%%%%%%%%%
          \bm{A} =
          \begin{bmatrix}
            1 & 1 & 1\\
            0 & 1 & 1 \\
            0 & 0 & 2
          \end{bmatrix}
          \]%%%%%%%%%%
        \end{itemize}

        \item \tbm{A} is not diagonalizable.
        \begin{itemize}\color{foreground}
          \item \false{false}; the following is certainly a diagonal matrix.
          \[%%%%%%%%%%
          \bm{A} =
          \begin{bmatrix}
            1 & 0 & 0\\
            0 & 1 & 0 \\
            0 & 0 & 2
          \end{bmatrix}
          \]%%%%%%%%%%
        \end{itemize}

      \end{enumerate}
    }

    \minor{\item[14.] Suppose the eigenvector matrix \tbm{S} has \(\bm{S}^T =
      \bm{S}^{-1}\). Show that \(\bm{A} = \bm{S\Lambda S}^{-1}\) is symmetric
      and has orthogonal eigenvectors.
    }
    \begin{align*}
      \bm{A} &= \bm{S\Lambda S}^{-1}  \\
      \bm{A}^T  &= (\bm{S\Lambda S}^{-1})^T   \\
      \bm{A}^T  &= (\bm{S}^{-1})^T \Lambda^T \bm{S}^T     \\
      \bm{A}^T  &= \bm{S} \Lambda \bm{S}^T = \bm{S} \Lambda \bm{S}^{-1}  && \bm{S}^T = \bm{S}^{-1}\\
                &\then \bm{A} = \bm{A}^T && ~\text{\tbm{A} is symmetric}~ \\\\
      \bm{S}^T \bm{S} &= \bm{S}^{-1}\bm{S} = \bm{I} && ~\text{\tbm{S} is orthonormal}~
    \end{align*}


    \minor{\item[39.] Suppose \(\bm{Ax} = \lambda \bm{x}\). If \(\lambda = 0,\)
      then \(x\) is in the nullspace. If \(\lambda \neq 0\), then \(x\) is in
      the column space. Those spaces have dimensions \((n-r)+r = n\). So why
      doesn't every square matrix have \(n\) linearly independent eigenvectors?
    }
    \begin{itemize}
      \item Because the dimensions of the \textit{row space} and null space
        must equal \(n\), the \textit{column space} and null space can overlap.
      \item This means that not every independent vector is an independent
        eigenvector. Non-distinct eigenvalues could map to the same
        eigenvector, but still yield independent columns in the original
        matrix.
    \end{itemize}

  \end{enumerate}
\end{itemize}

\section{5.5 Complex Matrices}
\begin{itemize}
  \item []

  \subsection{Problems 2, 12, 50}
  \begin{enumerate}
    \minor{\item[2.] What can you say about
      \begin{enumerate}
        \item the sum of a complex number and its conjugate?
        \begin{itemize}\color{foreground}
          \item is twice the real component; the imaginary component cancels
            out.
        \end{itemize}

        \item the conjugate of a number on the unit circle?
        \begin{itemize}\color{foreground}
          \item is on the unit circle, but with an inverted handed angle
          (reflected about the real axis).
        \end{itemize}

        \item the product of two numbers on the unit circle?
        \begin{itemize}\color{foreground}
          \item is on the unit circle, with real axis equal to the sum of
            angels of the factors of the product.
        \end{itemize}

        \item the sum of two numbers on the unit circle?
        \begin{itemize}\color{foreground}
          \item is not on the unit circle, but lies on a circle that is less on or
            inside another circle with the radius of 2.
        \end{itemize}

      \end{enumerate}
    }

    \minor{\item[12.] Give a reason if true or a counterexample if false:
      \begin{enumerate}
        \item If \tbm{A} is Hermitian, then \(\bm{A} + \i \bm{I}\) is invertible.
        \begin{itemize}\color{foreground}
          \item \true{true}; \(\bm{A} + \i\bm{I}\) is guaranteed to have all
            non-zero complex entries along the diagonal, making it nonsingular.
        \end{itemize}

        \item If \tbm{Q} is orthogonal, then \(\bm{Q} +\frac{1}{2} \bm{I}\) is
          invertible.
        \begin{itemize}\color{foreground}
          \item \true{true}; \(\bm{Q}\bm{Q}^{-1} = \bm{Q}^{-1}\bm{Q} = \bm{I}\) remains true.
        \end{itemize}

        \item If \tbm{A} is real, then \(\bm{A} + \i \bm{I}\) is invertible.
        \begin{itemize}\color{foreground}
          \item \false{false}; counterexample:
            \[%%%%%%%%%%
            \bm{A} =
            \begin{bmatrix}
              0 & 1 \\
              -1 & 0
            \end{bmatrix} \then
            \bm{A} + \i \bm{I} =
            \begin{bmatrix}
              \i & 1 \\
              -1 & \i
            \end{bmatrix}
            \]%%%%%%%%%%
            if \(\i = 0\), then this matrix would remain singular.
        \end{itemize}

      \end{enumerate}
    }

    \minor{\item[50.] A matrix with the orthonormal eigenvectors has the form
      \(\bm{A} = \bm{U\Lambda U}^{-1} = \bm{U\Lambda U}^{\t}\). Prove that
      \(\bm{AA}^{\t} = \bm{A}^{\t} \bm{A}\). These are exactly the normal
      matrices.
    }
    \begin{align*}
      \bm{AA}^{\t} &= \bm{U\Lambda U}^{\t} (\bm{U\Lambda U}^{\t})^{\t}  \\
                   &= \bm{U\Lambda U}^{\t} \bm{U\Lambda^{\t}  U}^{\t}  \\
                   &= \bm{U\Lambda} \bm{\Lambda^{\t}  U}^{\t}
      = \aset{\bm{U\Lambda}^{\t}  \bm{\Lambda  U}^{\t}}
      && \bm{\Lambda\Lambda}^{\t} = \bm{\Lambda^{\t} \Lambda}\\\\
      \bm{A^{\t} A} &= (\bm{U\Lambda U}^{\t})^{\t}  \bm{U\Lambda U}^{\t}  \\
                    &= \bm{U \Lambda^{\t}  U}^{\t} \bm{U\Lambda  U}^{\t}  \\
                    &= \aset{\bm{U\Lambda}^{\t}  \bm{\Lambda  U}^{\t}}
                    &&\then \bm{AA}^{\t} = \bm{A^{\t} A}
    \end{align*}

  \end{enumerate}
\end{itemize}

\section{5.6 Similarity Transformations}
\begin{itemize}
  \item []

  \subsection{Problems 17, 21, 42}
  \begin{enumerate}
    \minor{\item[17.] Prove that every unitary matrix \tbm{A} is
      diagonalizable, in two steps:
      \begin{enumerate}
        \item If \tbm{A} is unitary, and \tbm{U} is too, then so is \(\bm{T} =
          \bm{U}^{-1}\bm{AU}.\)
        \fg{\begin{align*}
            \bm{TT}^{\t} &= \bm{U^{-1}A U} (\bm{UA U}^{-1})^{\t} \\
                         &= \bm{U^{-1}A U} \bm{U^{\t} A^{\t}  U}^{\t} \\
                         &= \bm{U^{-1}A }  \bm{A^{\t}  U}^{\t} \\
                         &= \bm{U^{-1} } \bm{U}^{\t}
            && \text{Given \tbm{A} is unitary}  \\
                         &= \bm{I}
        \end{align*}
        \(\bm{TT}^{\t}  = \bm{I} \), thus \tbm{T} is unitary.}

        \item An upper triangular \(\bm{T}\) that is unitary must be diagonal.
          Thus, \(\bm{T}=\bm{A}\)
      \end{enumerate}

      Any unitary matrix \tbm{A} (distinct eigenvalues or not) has a complete
      set of orthonormal eigenvectors. All eigenvalues satisfy \(|\lambda | =
      1\).
      \begin{itemize}\color{foreground}
        \item Unitary matrices are the analogous to orthonormal matrices, thus
          each element along the diagonal must be nonzero and an absolute value
          equal to 1, and all lower triangular elements must be zero in order
          for \tbm{A} to be a series of unit vectors.

          Hence, \tbm{A} must be upper triangular matrix in order to be
          considered unitary.
      \end{itemize}
    }

    \minor{\item[21.] Prove that a matrix with orthonormal eigenvectors must be
      normal, as claimed: if \(\bm{U}^{-1} \bm{NU} = \bm{A}\) or \(\bm{N} =
      \bm{U\Lambda U}^{\t}\), then \(\bm{NN}^{\t} = \bm{N}^{\t}\bm{N}.\)
    }
    \begin{itemize}\color{foreground}
      \item See \ulink{ss:Problems 2, 12, 50}{question 50 from 5.5}. If \(\bm{N} =
        \bm{U\Lambda U}^{\t}, \) then \(\bm{NN}^{\t} = \bm{N}^{\t} \bm{N} \)
        --- these are the normal matrices.
    \end{itemize}

    \minor{\item[41.] True or false, with a good reason:
      \begin{enumerate}
        \item An invertible matrix can't be similar to a singular matrix.
          \begin{itemize}\color{foreground}
            \item \true{true}; the product of the eigenvalues of invertible
              matrices is always nonzero, while a singular matrix will have at
              least 0 eigenvalue---thus, they aren't similar.
          \end{itemize}

        \item A symmetric matrix can't be similar to a nonsymmetric matrix.
          \begin{itemize}\color{foreground}
            \item \false{false}; \tbm{B} could be a product of permutations,
              which means \tbm{A} could be symmetric, but \tbm{B} would not be;
              \tbm{A} and \tbm{B} would still be similar.
          \end{itemize}

        \item \tbm{A} can't be similar to \tbm{-A} unless \(\bm{A}= 0\).
          \begin{itemize}\color{foreground}
            \item \false{false}; ~
              \(%%%%%%%%%%
              \bm{A} = \begin{bmatrix}
                1 & 0  \\
                0 & -1
              \end{bmatrix}, \quad
              -\bm{A} =
              \begin{bmatrix}
                -1 & 0 \\
                 0 & 1
              \end{bmatrix}
              \)%%%%%%%%%%
              ~--- these are similar.
          \end{itemize}

        \item \(\bm{A} - \bm{I} \) can be similar to \(\bm{A}+\bm{I}\).
          \begin{itemize}\color{foreground}
            \item \true{true}; similar matrices must have the same eigenvalues
              --- if \tbm{A} was in upper triangular form, then subtracting vs.\@ adding 1 from all the eigenvalues will
              certainly yield different eigenvalues.
          \end{itemize}

      \end{enumerate}
    }

  \end{enumerate}
\end{itemize}
