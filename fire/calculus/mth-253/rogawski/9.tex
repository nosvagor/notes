\section{9 Rogawski: Review}
\begin{itemize}
  \item[]

  \subsection{Chapter 9 Toolbox}
  \begin{itemize}
    \item \dd{Separable first-order}: a differential equation in the form
      \[%%%%%%%%%%
      \frac{dy}{dx} = f (x) g (y)
      \]%%%%%%%%%%
      \begin{itemize}
        \item \dd{General solution}: when \( \frac{dy}{dt} = ky \), then \(\aset{y
          (t) = De^{kt}}\)
          \begin{align*}
            y^{-1}dy &= kdt  \\
            \int y^{-1} dy &= \int kdt  \\
            \ln|y| &= kt + C \\
            |y| &= e^{kt}  \\
            y &= De^{kt}
          \end{align*}
        \item Exponential decay: \(k < 0\); half-life: \((\ln 0.5)k^{-1}\)
        \item Exponential growth: \(k > 0\); doubling: \((\ln 2)k^{-1}\)
      \end{itemize}

    \item \dd{First-order linear constant coefficient}: when a quantity \(y\)
      whose rate of change is proportional to the difference \(y-b\), i.e.,
      \[%%%%%%%%%%
        \frac{dy}{dt} = k(y-b)
      \]%%%%%%%%%%
    \begin{itemize}
      \item \dd{General solution}: using separation of variables,
        \[%%%%%%%%%%
           \aset{ y (t) = b + Ce^{kt}
           \quad\leftrightarrow\quad
           \frac{d}{dt} (y-b) = k(y-b)}
        \]%%%%%%%%%
      \item \dd{Newton's law of Cooling}: where \(k\) is the cooling constant
        \prn{dependent on object} and \(T_0\) is the ambient temperature.
        \[%%%%%%%%%%
          \frac{dy}{dt} = -k(y-T_0) \then y (t) = T_0  + C^{-kt}
        \]%%%%%%%%%%

      \item \dd{Newton's Second Law of Motion}: \(F = ma = mv' = -mg - kv\),
        i.e.,
        \[%%%%%%%%%%
          \frac{dv}{dt} = -\frac{k}{m} \left( v + \frac{mg}{k} \right)
          \then
          v (t) = -\frac{mg}{k} + C e^{-\frac{k}{m}t}
        \]%%%%%%%%%%

      \item \dd{Annuity/Compound interest}: modeling balance in annuity by the
        differential equation
        \[%%%%%%%%%%
          \frac{dP}{dt} = rP - N = r(P - \frac{N}{r})
          \then
          P (t) = \frac{N}{r} + C^{rt}
        \]%%%%%%%%%%
    \end{itemize}

    \newpage
  \item \dd{Slope filed}: when a first-order differential equation
    \(\frac{dy}{dt} = F(t,y)\) is obtained by drawing small segments of slope
    \(F(t,y)\) at points \(t,y\).
    \begin{itemize}
      \item Test points particular points, often two easy tests are enough to
        match an equation to  graph via elimination of potential options.
    \end{itemize}

  \item \dd{Euler's Method}: an approximate solution to \(\frac{dy}{dt} =
    F(\MM{t},\YY{y})\) when given an initial condition \(y(t_0) = \YY{y_0}\) and time step \(\aseg{h}\).
    \begin{itemize}
      \item Setting \(t_k = t_0 + kh\) yields \(y_1,y_2,\ldots,y_{n}\) through
        recursive application of
        \[%%%%%%%%%%
          \aset{y_k = y_{k-1} + hF(t_{k-1}, y_{k-1})}
        \]%%%%%%%%%%
      \item I.e.,
        \begin{align*}
          \aset{y_1} &= \YY{y_{0}} + \aseg{h}F(\MM{t_0},\YY{y_{0}}) \\
                    &\aset{\searrow} \\
          \RR{y_{2}} &= \aset{y_{1}} + \aseg{h}F(\MM{t_{0+\aset{1} \PP{h}}},\aset{y_1}) \\
                    &\RR{\searrow} \\
          \BB{y_{3}} &= \RR{y_{2}} + \aseg{h}F(\MM{t_{0+\RR{2} \PP{h}}},\RR{y_{2}}) \\
                    &\BB{\searrow} \\
                    &~\vdots
        \end{align*}
        where each \(y_k\) is an approximate of \(y(t_n)\)
    \end{itemize}

  \item \dd{Logistic differential equation}: where \(y(t)\) is the population
    at time \(t\) and \(A\) denotes the carrying capacity, yielding a
    representation of room for growth \(A-y (t) \).
    \begin{itemize}
      \item The assumption is that the \(\frac{dy}{dt} \) is proportional to
        the amount of \(y (t) \) present and amount of \(A - y (t) \) of room
        for growth, i.e.,
        \[%%%%%%%%%%
          \frac{dy}{dt} = Ky(A-y), \qquad K = ~\text{proportionality constant}~
        \]%%%%%%%%%
      \item Which an be written as
        \[%%%%%%%%%%
          \aset{\frac{dy}{dt} = ky(1-\frac{y}{A})}, \qquad k = KA
        \]%%%%%%%%%%
      \item \dd{General non-equilibrium solution}: when \(k > 0 \land A > 0\):
        \[%%%%%%%%%%
          \aset{y = \frac{A}{1-\frac{e^{-kt}}{B}}
          \quad\leftrightarrow\quad
        \frac{y}{y-A} = Be^{kt}}
        \]%%%%%%%%%
    \end{itemize}

  \item Two equilibrium constant solutions:
    \begin{itemize}
      \item \( y= 0\); unstable equilibrium.
      \item \(y=A\); a stable equilibrium.
    \end{itemize}

  \item If the initial value \(y_0 = y(0) \) satisfies \(y_0 > 0\), then
    \(\lim_{t \to \infty} y (t) = A\)
  \end{itemize}
  \newpage

  \item \dd{First-Order Linear Equations}: method of solving all first-order
    linear differential equations, separable or not, as long as the equation
    can be put in the form:
    \[%%%%%%%%%%
      \aset{\frac{dy}{dx} + P(x)y = Q(x)}
    \]%%%%%%%%%%
    \begin{itemize}
      \item \dd{Integrating factor}:
        \[%%%%%%%%%%
          \aset{\alpha (x) = e^{\int P(x) dx}}
        \]%%%%%%%%%%
      \item \dd{General solution}:
        \[%%%%%%%%%%
          \aset{y = \alpha (x)^{-1} \left( \int \alpha (x) Q (x) dx + C \right)}
        \]%%%%%%%%%%
      \item Approach to the problems:
        \begin{enumerate}
          \item Arrange equation in first-order linear form.
          \item Find the Integrating factor.
          \item Solve general solution.
          \item Solve initial value by finding \(C\) in solved general
            solution, if given \(y(t)\).
        \end{enumerate}
    \end{itemize}

  \subsection{9.4.9 Spread of Rumor}
  \begin{itemize}
    \item One model for the spread of a rumor is that the rate of change of the
      percent of the population that has heard the rumor is proportional to the
      product of the percent  of the population that has heard the rumor and
      the percent that has not heard the rumor. Suppose a small town has a population of 1,000 people. At 9 AM, 60
      people had heard a rumor. By noon, half of the town had heard it. Set up
      an initial value problem to model this situation.
      \begin{align*}
        \frac{y}{y-A} &= Be^{kt},\qquad y(0) = 60, A = 1000\\
        \frac{60}{60-1000} &= Be^{kt}\\
        B &= -\frac{3}{47} \\
        y = \frac{1000}{1-\frac{e^{-kt}}{-\frac{3}{47}}} &= \frac{3000}{3+47e^{-kt}}\\\\
        500 &= \frac{3000}{3+47e^{-3k}}, \qquad y(3) = 500 \\
        1500 + 23500e^{-3k} &= 3000  \\
        e^{-3k} &= \frac{1500}{23500} \\
        -3k &= \ln\left( \frac{3}{47} \right) \\
        k &= -\frac{1}{3}\ln \left( \frac{3}{47} \right)
        \approx 0.917\\
      \end{align*}
  \end{itemize}

\end{itemize}
