\chapter{Power Series: 10.6}

\section{Power Series Notes}
\begin{itemize}
  \item \dd{Power series}: a infinite series in the form:
    \[%%%%%%%%%%
    F(x) = \sum_{n=0}^{\infty} a_n (x - c)^n
    \]%%%%%%%%%%
    Where the constant \(c\) is the \textit{center} of the power series F(x).

  \item \dd{Radius of convergence \(R\)}: the range of values of the variable \(x\)
    whereby the power series \(F(x)\) converges.
    \begin{itemize}
      \item Every power series converges at \(x = c\), as \((x-c)^0 = 1\),
        though the series may diverge for other values of \(x\).

      \item \(F(x)\) converges for \(\left| x - c \right| < R \) and diverges
        for \(\left| x - c \right| > R \)

      \item \(F(x)\) may converge of diverge at endpoints \(c-R\) and \(c+R\)

      \item \dd{Interval of convergence}: the open interval \((c-R, c+R)\) and
        possibly one of both of the endpoints, each must be tested.

      \item In most cases, the \ulink{ss:Tests for Non-Positive Series}{ratio
        test} can be used to find R.

    \end{itemize}

\end{itemize}

\section{Power Series Problems}
\begin{itemize}
  \item
\end{itemize}

