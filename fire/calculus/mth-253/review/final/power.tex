\chapter{Power Series: 10.6}

\section{Power Series Notes}
\begin{itemize}
  \item \dd{Power series}: a infinite series in the form:
    \[%%%%%%%%%%
    F(x) = \sum_{n=0}^{\infty} a_n (x - c)^n
    \]%%%%%%%%%%
    Where the constant \(c\) is the \textit{center} of the power series F(x).

  \item \dd{Radius of convergence \(R\)}: the range of values of the variable \(x\)
    whereby the power series \(F(x)\) converges.
    \begin{itemize}
      \item Every power series converges at \(x = c\), as \((x-c)^0 = 1\),
        though the series may diverge for other values of \(x\).

      \item \(F(x)\) converges for \(\left| x - c \right| < R \) and diverges
        for \(\left| x - c \right| > R \)

      \item \(F(x)\) may converge of diverge at endpoints \(c-R\) and \(c+R\)

      \item \dd{Interval of convergence}: the open interval \((c-R, c+R)\) and
        possibly one of both of the endpoints, each must be tested.
        \begin{itemize}
          \item In most cases, the \ulink{ss:Tests for Non-Positive
            Series}{ratio test} can be used to find R.

          \item If \(R > 0\), then \(F\) is differentiable over the interval of
            convergence; the derivative and antiderivative can be obtained
            using the following:
            \[%%%%%%%%%%
            F'(x) = \sum_{n=1}^{\infty} na_n(x-c)^{n-1} \qquad \quad
            F(x)dx = C + \sum_{n=0}^{\infty} \frac{a_n}{n+1}(x-c)^{n+1}
            \]%%%%%%%%%%
        \end{itemize}

      \item \dd{Useful Power Series}: the following power series
        (more examples: \dlink{s:Taylor Series Notes}{Taylor series}) can be used to drive
        expansions of other related functions via substitution, integration, or
        differentiation:
        \[%%%%%%%%%%
          \frac{1}{1-x} = \sum_{n=0}^{\infty} x^n \quad \given |x| < 1
          \qquad \qquad
          e^x = \sum_{n=0}^{\infty} \frac{x^n}{n!}
        \]%%%%%%%%%%

    \end{itemize}

\end{itemize}

\section{Power Series Problems}

\subsection{10.6 Preliminary Questions}
\begin{enumerate}
  \item Suppose that \(\sum a_nx^n\) converges for \(x=5\). Must it also
    converge for \(x=4\)? What about \(x=-3\)?
    \begin{align*}
      R = 5, \quad c = 0 \then ~\text{series converges for}~|x| < 5 \\
      \then -5 < x < 5
    \end{align*}
    Both -3 and 4 are inside the interval, thus it must \aset{converge for both}.

  \item Suppose that \(\sum a_n(x-6)^n\) converges for \(x=10\). At which of
    the following points must it also converge?
    \begin{align*}
      R = 10, c = 6 \then |x-6| < 4 \\
      \then 2 < x < 10
    \end{align*}
    \begin{multicols}{2}
    \begin{enumerate}
      \item \(x=8\) \aset{converges}

      \item \(x=11\) uncertain

      \item \(x=3\) \aset{converges}

      \item \(x=0\) uncertain
    \end{enumerate}
    \end{multicols}

  \item What is the radius of converges of \(F(3x)\) if \(F(x)\) is a power
    series with \(R=12\)?
    \[%%%%%%%%%%
    R = \frac{12}{3} = \aset{4}
    \]%%%%%%%%%%

  \item The power series \(\displaystyle F(x) = \sum_{n=1}^{\infty} nx^n\) has
    a radius of converge \(R = 1\).

    What is the power series expansion of \(F'(x)\) and what is its radius of
    convergence?
    \begin{align*}
      F'(x) &= \aset{\sum_{n=1}^{\infty} n^2x^{n-1}}  && \given F'(x) = \sum_{n=1}^{\infty} na_n(x-c)^{n-1}\\
      R &= \aset{1}
    \end{align*}

\end{enumerate}

\subsection{10.6 Exercises}
\begin{enumerate}
  \item

\end{enumerate}
