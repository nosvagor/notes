\chapter{Parametric Equations: 11.1}

\section{Parametric Equations Notes}
\begin{itemize}
  \item \dd{Parametric equation}: defines a group of quantities as functions of
    one or more independent variables called parameters, commonly expressed as
    coordinates of points that make up a geometric object.
    \begin{itemize}
      \item \dd{Parametrization}: the representation of a geometrical curve
        \(\mathcal{C}\) with parameter \(t\), i.e.,
        \[%%%%%%%%%%
          c(t) = (x(t), y(t))
        \]%%%%%%%%%%
        \begin{itemize}
          \item Note: parametrizations are not unique; the path \(c(t)\) may
            traverse all or part of \(\mathcal{C}\) more than once.
        \end{itemize}

      \item \dd{Parametrization of a line}: a line through point \(P = (a,b)\)
        with slope \(m \):
        \[%%%%%%%%%%
          x = a + t, \quad y = b + mt \qquad \given - \infty < t < \infty
        \]%%%%%%%%%%

      \item \dd{Parametrization of a circle} with radius \(R\) and center \((a, b)\):
        \[%%%%%%%%%%
          c(t) = (a + R \cos \theta, b + R \sin \theta)
        \]%%%%%%%%%%

      \item \dd{Parametrization of an ellipse}:
        \[%%%%%%%%%%
          \left( \frac{x}{a} \right) ^2 + \left( \frac{y}{b} \right)^2 = 1 \qquad  \to \qquad
          c(\theta) = (a \cos \theta, b \sin \theta)
        \]%%%%%%%%%%

      \item \dd{Parametrization of a cycloid}: generated by a circle of radius \(R\),
        \[%%%%%%%%%%)
          c(\theta) = \left( R(t-\sin \theta), R(1-\cos \theta) \right)
        \]%%%%%%%%%%

      \item \dd{Graph of \(\bm{y = f(x)}\)}: \[c(t) = (t, f(t))\]
    \end{itemize}

  \item \dd{Slope of tangent lie at \(\bm{c(t)}\)}:
    \[%%%%%%%%%%
      \frac{dy}{dx}  = \frac{dy}{dt} \cdot \frac{dt}{dx} = \frac{y'(t)}{x'(t)}
      \qquad \given x'(t) \neq 0
    \]%%%%%%%%%%

  \item \dd{Area under a parametric curve}: valid when the curve \(y = h(x)\)
    is traced \aset{once} by the parametric curve \(c(t) = (x(t), y(t))\).
    \begin{align*}
      y &= h(x) \to y(t), \qquad  dx \to x'(t) dt \\
      &\then A = \int_{t_0}^{t_1}  y(t)x'(t) dt
    \end{align*}


\end{itemize}

\section{Parametric Problems}
\subsection{11.1 Exercises}
Find parametric equations for the given curve.
\begin{enumerate}[itemsep=14em]
  \item Line through \((3,1)\) and \((-5, 4)\).
    \begin{align*}
      &P(3,1), \quad Q(-5, 4) \\
      &\then m = \frac{y_Q - y_P}{x_Q - x_P}  = \frac{4-1}{-5 -3} = -\frac{3}{8} \\
      &\then y = m (x-x_P) + y_P = -\frac{3}{8}(x-3) + 1 = -\frac{3}{8}x + \frac{17}{8}\\
      &\begin{cases}
        x = t \\
        y = -\frac{3}{8}t + \frac{17}{8}
      \end{cases}\\
      &\then \aset{\boxed{c(t) = \left( t, -\frac{3}{8}t + \frac{17}{8} \right)}}
    \end{align*}

  \vspace{-12em}
  \item Circle of radius \(4\) with center \((3, 9)\).
    \begin{align*}
      c(t) &= (a + R \cos \theta , b + R \sin \theta) \quad
      \given \text{ \ulink{s:Parametric Equations Notes}{Parametrization of a circle} }\\
      \\
      \then c(t) &= (3 + 4 \cos \theta, 9 + 4 \sin \theta) \in \left[ 0, 2\pi \right)
    \end{align*}

  \vspace{-9em}
  \item The following ellipse its center translated center to \((7, 4)\)
    \[%%%%%%%%%%
      \left( \frac{x}{5} \right)^2 + \left( \frac{y}{12} \right)^2 = 1
    \]%%%%%%%%%%
    \begin{align*}
      c(\theta) &= (a \cos \theta , b \sin \theta)
                && \text{ \ulink{s:Parametric Equations Notes}{Parametrization of a ellipse} }\\
      \\
      c(\theta) &= (5 \cos \theta, 12 \sin \theta)
                && \given \left( \frac{x}{5} \right)^2 + \left( \frac{y}{12} \right)^2 = 1 \\
      c(\theta) &= \aset{\boxed{(7 + 5 \cos \theta, 4 + 12 \sin \theta)}}
                && \given \left( \frac{x-7}{5} \right)^2 + \left( \frac{y-4}{12} \right)^2 = 1
    \end{align*}

\end{enumerate}
