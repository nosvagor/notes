\chapter{Final Review Questions}

Note: these questions were taken form a provided review sheet;
they focus on sections 10.6--11.4. Some questions already exist on the
quizzes, but will be duplicated here.
\vspace{-2pt}
\hrule
\vspace{10pt}

\begin{enumerate}
  \item Find the interval of convergence of the following power series.

  \begin{enumerate}[itemsep=24em]
    \item \(\displaystyle \sum_{n=1}^{\infty} \frac{5^n}{n}x^n \)

    \item \(\displaystyle \frac{(x-2)^n}{n^2+1} \)
  \end{enumerate}

%%%%%%%%%%%%%
\newpage %%%%%%%%%%%%%%%%%%%%%%%%%%%%%%%%%%%%%%%%%%%%%%%%%%%%%%%%%%%%%%%%%%%%%%
%%%%%%%%%%%%%

  \item Find the Taylor series of the following functions \(f(x)\) centered at
    the given value of \(c\) using the definition.
    \begin{enumerate}[itemsep=24em]
      \item \(f(x) = e^x, \quad c = 2\)

      \item \(f(x) = \sqrt{x}, \quad c = 1\)
    \end{enumerate}

%%%%%%%%%%%%%
\newpage %%%%%%%%%%%%%%%%%%%%%%%%%%%%%%%%%%%%%%%%%%%%%%%%%%%%%%%%%%%%%%%%%%%%%%
%%%%%%%%%%%%%

  \item Find the Maclaurin series of the following functions using substitution
    and/or multiplication.

    \begin{enumerate}[itemsep=24em]
      \item \(f(x) = x \cos (2x)\)

      \item \(f(x) = \dfrac{x^3}{1+x}\)
    \end{enumerate}

%%%%%%%%%%%%%
\newpage %%%%%%%%%%%%%%%%%%%%%%%%%%%%%%%%%%%%%%%%%%%%%%%%%%%%%%%%%%%%%%%%%%%%%%
%%%%%%%%%%%%%

  \item Express the following integral as a power series, first by finding the
    Maclaurin series of the integrand, then integrating this series
    term-by-term:
    \[%%%%%%%%%%
    \int_{0}^{1} e^{-x^2}  dx
    \]%%%%%%%%%%

%%%%%%%%%%%%%
\newpage %%%%%%%%%%%%%%%%%%%%%%%%%%%%%%%%%%%%%%%%%%%%%%%%%%%%%%%%%%%%%%%%%%%%%%
%%%%%%%%%%%%%

  \item Find the parametric equations for the following curves.
    \begin{enumerate}[itemsep=16em]
      \item The line through \((3,6)\) and \((-2, 0)\).

      \item The circle of radius \(5\) centered at \((1,7)\).

      \item The ellipse
        \[%%%%%%%%%%
          \left( \frac{x-1}{2} \right)^2 + \frac{y+1}{3}^2 = 1
        \]%%%%%%%%%%
    \end{enumerate}

%%%%%%%%%%%%%
\newpage %%%%%%%%%%%%%%%%%%%%%%%%%%%%%%%%%%%%%%%%%%%%%%%%%%%%%%%%%%%%%%%%%%%%%%
%%%%%%%%%%%%%

  \item Find the equation of the tangent line to the curve
    \[%%%%%%%%%%
    x = \sin (2t) + \cos (t), \quad y = \cos (2t) - \sin (t), \qquad \given t
    = \pi
    \]%%%%%%%%%%

%%%%%%%%%%%%%
\newpage %%%%%%%%%%%%%%%%%%%%%%%%%%%%%%%%%%%%%%%%%%%%%%%%%%%%%%%%%%%%%%%%%%%%%%
%%%%%%%%%%%%%

\item Find the arc length of the curve
  \[%%%%%%%%%%
  x = \frac{2}{3}t^2, \quad y = t^2 - 2, \qquad \given 0 \leq t \leq 2
  \]%%%%%%%%%%


%%%%%%%%%%%%%
\newpage %%%%%%%%%%%%%%%%%%%%%%%%%%%%%%%%%%%%%%%%%%%%%%%%%%%%%%%%%%%%%%%%%%%%%%
%%%%%%%%%%%%%

  \item Find the surface area obtained by rotating the following around the
    \(x\)-axis;
    \[%%%%%%%%%%
    x = e^t - t, \quad y = 4e^{\frac{t}{2}}, \qquad \given 0 \leq t \leq 1
    \]%%%%%%%%%%


%%%%%%%%%%%%%
\newpage %%%%%%%%%%%%%%%%%%%%%%%%%%%%%%%%%%%%%%%%%%%%%%%%%%%%%%%%%%%%%%%%%%%%%%
%%%%%%%%%%%%%

  \item Match each equation in rectangular coordinates with its equation in
    polar coordinates.

  \begin{multicols}{2}
    \begin{enumerate}
      \item \(x^2 + y^2 = 4\)
      \item \(x^2 + (y-1)^2 = 1\)
      \item \(x^2 - y^2 = 4\)
      \item \(x + y = 4\)
    \end{enumerate}

    \begin{enumerate}[label=(\roman*)]
      \item \(r^2 (1- 2 \sin ^2 \theta) = 4\)
      \item \(r (\cos \theta + \sin \theta ) = 4\)
      \item \(r = \sin \theta\)
      \item \(r = 2 \)
    \end{enumerate}
  \end{multicols}

  \vspace{8em}

  \item Find the area enclosed by one loop of the curve
    \[%%%%%%%%%%
    r^2 \cos 2\theta
    \]%%%%%%%%%%

\end{enumerate}
