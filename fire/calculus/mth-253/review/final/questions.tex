\chapter{Final Review Questions}

Note: these questions were taken form a provided review sheet;
they focus on sections 10.6--11.4. Some questions already exist on the
quizzes, but will be duplicated here.
\vspace{-2pt}
\hrule
\vspace{10pt}

\begin{enumerate}
  \item Find the interval of convergence of the following power series.

  \begin{enumerate}[itemsep=24em]
    \item \(\displaystyle \sum_{n=1}^{\infty} \frac{5^n}{n}x^n \)
      \begin{align*}
      C &= \lim_{n \to \infty} \left| a_n  \right|^{\frac{1}{n}}
           && \text{Apply the \ulink{ss:Tests for Non-Positive Series}{root test}} \\
           &= \lim_{n \to \infty}  \left|
           \frac{5^nx^n}{n}
          \right|^{\frac{1}{n}}  \\
           &= 5|x| \lim_{n \to \infty}  \frac{1}{\sqrt{n}} = 5|x| \\
           \\
           &\then |x| < \frac{1}{5}
           && \text{converges for } C <1 \\
           \\
           f\left(- \frac{1}{5}\right)
           &= \sum_{n=1}^{\infty}  \frac{5^n}{n}\left(-\frac{1}{5}\right)^n \\
           &= \sum_{n=1}^{\infty} (-1)^n \frac{1}{n}, \quad a_n = \frac{1}{n}
           && \text{Apply the \ulink{ss:Tests for Non-Positive Series}{alternating series test}} \\
           \lim_{n \to \infty} a_n &= 0 \oldland
           |a_n|~\text{decreases monotonically}
                                   &&\then \text{converges} \\
                                   \\
           f\left(\frac{1}{5}\right)
           &= \sum_{n=1}^{\infty}  \frac{5^n}{n}\left(\frac{1}{5}\right)^n
           = \sum_{n=1}^{\infty} \frac{1}{n}
           && \then \text{diverges by \ulink{ss: Test for Positive Series}{\(p\)-series}} \\
           \\
           &\then \text{Interval of convergence: } \aset{\left[-\frac{1}{5}, \frac{1}{5}\right)}
    \end{align*}

    \item \(\displaystyle \sum_{n=1}^{\infty} \frac{(x-2)^n}{n^2+1} \)
      \begin{align*}
      \rho &= \lim_{n \to \infty} \left| \frac{a_{n+1}}{a_n}  \right|
           && \text{Apply the \ulink{ss:Tests for Non-Positive Series}{ratio test}} \\
           &= \lim_{n \to \infty} \left|
           \frac{(x-2)^{n+1}}{(n+1)^2 + 1} \cdot
           \frac{n^2+1}{(x-2)^n}
           \right| \\
           &= \lim_{n \to \infty}
           \frac{(|x-2|)^n \cdot |x-2|}{n^2 + 2n + 2} \cdot
           \frac{n^2+1}{(|x-2|)^n} \\
           &= |x-2| \lim_{n \to \infty}
           \frac{n^2 + 1}{n^2+2n+2} \\
           &= |x-2| \lim_{n \to \infty}
           \frac{2}{2}
           && \text{By \hopital's Rule} \\
           \\
           &\then |x-2| < 1
           && \text{converges for } \rho <1 \\
           &\then 1 < x < 3 \\
           \\
           f(1) &= \sum_{n=1}^{\infty} \frac{(-1)^n}{n^2 + 1}
           \quad a_n = \frac{1}{n^2 + 1}
           && \text{Apply the \ulink{ss:Tests for Non-Positive Series}{alternating series test}} \\
           &\lim_{n \to \infty} a_n = 0
           \then \text{converges}
           && |a_n|~\text{decreases monotonically}\\
           \\
           f(3) &= \sum_{n=1}^{\infty} \frac{1}{n^2 + 1}\\
                & b_n ~\text{converges} \to a_n~\text{converges}~
                && \text{Apply the \ulink{ss:Tests for Positive Series}{direct comparison test}} \\
           b_n &= \frac{1}{n^2} \then b_n ~\text{converges}
                && \text{By \ulink{ss:Tests for Positive Series}{p-series}} \\
                \\
                &\then \text{Interval of convergence: }
                \aset{\left[ 1, 3 \right] }
      \end{align*}
  \end{enumerate}

%%%%%%%%%%%%%
\newpage %%%%%%%%%%%%%%%%%%%%%%%%%%%%%%%%%%%%%%%%%%%%%%%%%%%%%%%%%%%%%%%%%%%%%%
%%%%%%%%%%%%%

  \item Find the Taylor series of the following functions \(f(x)\) centered at
    the given value of \(c\) using the definition.
    \begin{enumerate}[itemsep=24em]
      \item \(f(x) = e^x, \quad c = 2\)
        \begin{align*}
          e^x &= \sum_{n=0}^{\infty} \frac{x^n}{n!}, \quad \given \forall n \geq 0
              && \text{Use relevant \ulink{ss:Taylor Series}{Maclaurin series}} \\
              & && f^{(n)}(2) = e^2 \given f^{(n)}(x) = e^x\\
                     &\then \aset{\boxed{\sum_{n=0}^{\infty} \frac{e^2(x-2)^n}{n!}}}
                     && \text{Center series, } c= 2 \\
        \end{align*}

      \vspace{-16em}
      \item \(f(x) = \sqrt{x}, \quad c = 1\)
        \begin{align*}
          (1 + x)^\alpha  &= \sum_{n=0}^{\infty}
          \begin{pmatrix} \alpha \\ n \end{pmatrix} x^n \quad \given |x| < 1
          && \text{Use relevant \ulink{ss:Taylor Series}{Maclaurin series}} \\
          &\then
          \aset{\boxed{\sum_{n \geq 0}^{\infty}
          \begin{pmatrix} \frac{1}{2} \\ n \end{pmatrix} (x-1)^n}}
          && \text{note:}~(1+(x-1))^{\frac{1}{2}} = \sqrt{n} \\
        \end{align*}
    \end{enumerate}

%%%%%%%%%%%%%
\newpage %%%%%%%%%%%%%%%%%%%%%%%%%%%%%%%%%%%%%%%%%%%%%%%%%%%%%%%%%%%%%%%%%%%%%%
%%%%%%%%%%%%%

  \item Find the Maclaurin series of the following functions using substitution
    and/or multiplication.

    \begin{enumerate}[itemsep=24em]
      \item \(f(x) = x \cos (2x)\)
        \begin{align*}
          \cos x &= \sum_{n=0}^{\infty} \frac{(-1)^nx^{2n} }{(2n)!} \quad \given \forall x
                 && \text{Use relevant \ulink{ss:Taylor Series}{Maclaurin series}} \\
          \then \cos(x^2) &= \sum_{n=0}^{\infty} \frac{(-1)^n(2x)^{2n} }{(2n)!} \\
                          &= \sum_{n=0}^{\infty} \frac{(-1)^n 4^n x^{2n} }{(2n)!} \\
                          &= \sum_{n=0}^{\infty} \frac{(-4)^n x^{2n} }{(2n)!}
                          && \text{Substitution of}~x^2 \\
          x \cdot \cos (2x) &= x \cdot \sum_{n=0}^{\infty} \frac{(-4)^nx^{2n} }{(2n)!} \\
                            &= \sum_{n=0}^{\infty} \frac{(-4)^n x^{2n+1} }{(2n)!}
                            && \text{Multiply by } x\\
                            \\
          \then f(x) &= \aset{\boxed{\sum_{n=0}^{\infty} \frac{(-4)^nx^{2n+1} }{(2n)!}}}
          \qquad \given \forall x
        \end{align*}

      \vspace{-22em}
      \item \(f(x) = \dfrac{x^3}{1+x}\)
        \begin{align*}
          \frac{1}{1+x} &= \sum_{n=0}^{\infty} (-1)^n x^n \quad \given |x| < 1
                 && \text{Use relevant \ulink{ss:Taylor Series}{Maclaurin series}} \\
                 \\
          \then \frac{x^3}{1+x} &= \sum_{n=1}^{\infty} (-1)^n (x^3)^n \\
                 &= \aset{\boxed{\sum_{n=1}^{\infty} (-1)^n x^{n+3}}} \quad \given |x| < 1
        \end{align*}
    \end{enumerate}

%%%%%%%%%%%%%
\newpage %%%%%%%%%%%%%%%%%%%%%%%%%%%%%%%%%%%%%%%%%%%%%%%%%%%%%%%%%%%%%%%%%%%%%%
%%%%%%%%%%%%%

  \item Express the following integral as a power series, first by finding the
    Maclaurin series of the integrand, then integrating this series
    term-by-term:
    \[%%%%%%%%%%
    \int_{0}^{1} e^{-x^2}  dx
    \]%%%%%%%%%%
    \begin{align*}
      e^x &= \sum_{n=0}^{\infty} \frac{x^n}{n!} \quad \given \forall x
      && \text{Use relevant \ulink{ss:Taylor Series}{Maclaurin series}} \\
      \\
      \then f(x) = e^{-x^2} &=  \sum_{n=0}^{\infty} \frac{(-x^2)^n}{n!}  \\
                 &= \sum_{n=0}^{\infty} (-1)^n \frac{x^{2n}}{n!}
                 && \given \forall x\\
      \\
      \then \int_{0}^{1} e^{-x^2}  dx
       &= \int_{0}^{1} \sum_{n=0}^{\infty} (-1)^n \frac{x^{2n}}{n!} dx \\
       &= \sum_{n=0}^{\infty} \frac{(-1)^n}{n!} \int_{0}^{1} x^{2n} dx \\
       &= \sum_{n=0}^{\infty} \frac{(-1)^n}{n!} \cdot \frac{x^{2n+1}}{2n+1} \bigg|_{0}^{1} \\
       &= \sum_{n=0}^{\infty} \frac{(-1)^n}{(2n+1)n!}
       - 0 \\
       &= \aset{\boxed{\sum_{n=0}^{\infty} \frac{(-1)^n}{(2n+1)n!}}}
    \end{align*}

%%%%%%%%%%%%%
\newpage %%%%%%%%%%%%%%%%%%%%%%%%%%%%%%%%%%%%%%%%%%%%%%%%%%%%%%%%%%%%%%%%%%%%%%
%%%%%%%%%%%%%

  \item Find the parametric equations for the following curves.
    \begin{enumerate}[itemsep=16em]
      \item The line through \((3,6)\) and \((-2, 0)\).
        \begin{align*}
          &P(-2,0), \quad Q(3, 6) \\
          &\then m = \frac{y_Q - y_P}{x_Q - x_P}  = \frac{6-0}{3+2} = \frac{6}{5} \\
          &\then y = m (x-x_P) + y_P = \frac{6}{5}(x+2) + 0 = \frac{6}{5}x + \frac{12}{5}\\
          &\begin{cases}
            x = t \\
            y = \frac{6}{5}x + \frac{12}{5}
          \end{cases}\\
          &\then \aset{\boxed{c(t) = \left( t, \frac{6}{5}t + \frac{12}{5}\right)}}
        \end{align*}

      \vspace{-13em}
      \item The circle of radius \(5\) centered at \((1,7)\).
        \begin{align*}
          c(t) &= (a + R \cos \theta , b + R \sin \theta) \quad
          \given \text{ \ulink{s:Parametric Equations Notes}{Parametrization of a circle} }\\
          \\
          \then c(t) &= (1 + 5 \cos \theta, 7 + 5 \sin \theta) \in \left[ 0, 2\pi \right)
        \end{align*}

      \vspace{-10em}
      \item The ellipse
        \[%%%%%%%%%%
          \left( \frac{x-1}{2} \right)^2 + \left(\frac{y+1}{3}\right)^2 = 1
        \]%%%%%%%%%%
        \begin{align*}
          c(\theta) &= (a \cos \theta , b \sin \theta)
                    && \text{ \ulink{s:Parametric Equations Notes}{Parametrization of a ellipse} }\\
                    \\
          c(\theta) &= (2 \cos \theta, 3 \sin \theta)
                    && \given \left( \frac{x}{2} \right)^2 + \left( \frac{y}{3} \right)^2 = 1 \\
          c(\theta) &= \aset{\boxed{(1 + 2 \cos \theta, -1 + 3 \sin \theta)}}
                    && \given \left( \frac{x-1}{2} \right)^2 + \left( \frac{y+1}{3} \right)^2 = 1
        \end{align*}
    \end{enumerate}

%%%%%%%%%%%%%
\newpage %%%%%%%%%%%%%%%%%%%%%%%%%%%%%%%%%%%%%%%%%%%%%%%%%%%%%%%%%%%%%%%%%%%%%%
%%%%%%%%%%%%%

  \item Find the equation of the tangent line to the curve
    \[%%%%%%%%%%
    x = \sin (2t) + \cos (t), \quad y = \cos (2t) - \sin (t), \qquad \given t
    = \pi
    \]%%%%%%%%%%
    \begin{align*}
      \frac{dy}{dx} &= \frac{y'(t)}{x'(t)} \quad \given x'(t) \neq 0
     &&\text{\ulink{s:Parametric Equations Notes}{Slope of a tangent line}} \\
     \\
     &\then \frac{dy}{dx} = \frac{-2\sin(2t) - \cos t}{2\cos(2t) - \sin t} \\
     &\then m = \frac{dy}{dx} \bigg|_{t=\pi} = \frac{0 - (-1)}{2 - 0} = \frac{1}{2}\\
     \\
     &\then \aset{\boxed{y - 1 = \frac{1}{2}(x+1)}}
      && c(\pi) = (-1, 1)\\
    \end{align*}

  \vspace{3em}
  \item Find the arc length of the curve
    \[%%%%%%%%%%
      x = \frac{2}{3}t^2, \quad y = t^2 - 2, \qquad \given 0 \leq t \leq 2
    \]%%%%%%%%%%


%%%%%%%%%%%%%
\newpage %%%%%%%%%%%%%%%%%%%%%%%%%%%%%%%%%%%%%%%%%%%%%%%%%%%%%%%%%%%%%%%%%%%%%%
%%%%%%%%%%%%%

  \item Find the surface area obtained by rotating the following around the
    \(x\)-axis;
    \[%%%%%%%%%%
    x = e^t - t, \quad y = 4e^{\frac{t}{2}}, \qquad \given 0 \leq t \leq 1
    \]%%%%%%%%%%


%%%%%%%%%%%%%
\newpage %%%%%%%%%%%%%%%%%%%%%%%%%%%%%%%%%%%%%%%%%%%%%%%%%%%%%%%%%%%%%%%%%%%%%%
%%%%%%%%%%%%%

  \item Match each equation in rectangular coordinates with its equation in
    polar coordinates.

  \begin{multicols}{2}
    \begin{enumerate}
      \item \(x^2 + y^2 = 4\)
      \item \(x^2 + (y-1)^2 = 1\)
      \item \(x^2 - y^2 = 4\)
      \item \(x + y = 4\)
    \end{enumerate}

    \begin{enumerate}[label=(\roman*)]
      \item \(r^2 (1- 2 \sin ^2 \theta) = 4\)
      \item \(r (\cos \theta + \sin \theta ) = 4\)
      \item \(r = \sin \theta\)
      \item \(r = 2 \)
    \end{enumerate}
  \end{multicols}

  \vspace{8em}

  \item Find the area enclosed by one loop of the curve
    \[%%%%%%%%%%
    r^2 = \cos 2\theta
    \]%%%%%%%%%%

\end{enumerate}
