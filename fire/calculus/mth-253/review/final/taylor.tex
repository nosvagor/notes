\chapter{Power/Taylor Series: 10.6--10.8}

\section{Power/Taylor Series Notes}

\subsection{Power Series}
\begin{itemize}
  \item \dd{Power series}: a infinite series in the form:
    \[%%%%%%%%%%
    F(x) = \sum_{n=0}^{\infty} a_n (x - c)^n
    \]%%%%%%%%%%
    Where the constant \(c\) is the \textit{center} of the power series F(x).

  \item \dd{Radius of convergence \(R\)}: the range of values of the variable \(x\)
    whereby the power series \(F(x)\) converges.
    \begin{itemize}
      \item Every power series converges at \(x = c\), as \((x-c)^0 = 1\),
        though the series may diverge for other values of \(x\).

      \item \(F(x)\) converges for \(\left| x - c \right| < R \) and diverges
        for \(\left| x - c \right| > R \)

      \item \(F(x)\) may converge of diverge at endpoints \(c-R\) and \(c+R\)

    \end{itemize}

  \item \dd{Interval of convergence}: the open interval \((c-R, c+R)\) and
    possibly one of both of the endpoints, each must be tested.
    \begin{itemize}
      \item In most cases, the \ulink{s:Tests for Non-Positive
        Series}{ratio test} can be used to find R.

      \item If \(R > 0\), then \(F\) is differentiable over the interval of
        convergence; the derivative and antiderivative can be obtained
        using the following:
        \[%%%%%%%%%%
          F'(x) = \sum_{n=1}^{\infty} na_n(x-c)^{n-1} \qquad \quad
          F(x)dx = C + \sum_{n=0}^{\infty} \frac{a_n}{n+1}(x-c)^{n+1}
        \]%%%%%%%%%%
    \end{itemize}


\end{itemize}

%%%%%%%%%%%%%
\newpage %%%%%%%%%%%%%%%%%%%%%%%%%%%%%%%%%%%%%%%%%%%%%%%%%%%%%%%%%%%%%%%%%%%%%%
%%%%%%%%%%%%%

\subsection{Taylor Series}
\begin{itemize}
  \item \dd{Taylor series}: the power series of a infinitely differentiable
    function \(f(x)\) centered at \(c\),
    \[%%%%%%%%%%
      \sum_{n=0}^{\infty} \frac{f^{(n)}(c) }{n!} (x-c)^n
    \]%%%%%%%%%%
  \begin{itemize}
    \item \dd{\(\bm{n}^{th} \) Taylor polynomial}: a polynomial of degree \(n\)
      that is formed partial sum formed by the first \(n+1\) terms of a Taylor
      series, i.e.,
      \[%%%%%%%%%%
      f(c) + f'(c)(x-c) + \frac{f''(c)}{2!}(x-c)^2 + \frac{f'''(c)}{3!}(x-c)^3
      + \cdots +  \frac{f^n (c)}{n!}(x-c)^n
      \]%%%%%%%%%%
    \item \dd{Maclaurin series}: when \(c = 0\), i.e.,
      \[%%%%%%%%%%
        \sum_{n=0}^{\infty} \frac{f^{(n)}0 }{n!} (x)
      \]%%%%%%%%%%
  \end{itemize}

  \item \dd{Useful Maclaurin Series}: useful Taylor series centered at 0 that
    can be used to derive other series via differentiation, integration,
    multiplication, or substitution.
    \begin{align*}
      e^x &= \sum_{n=0}^{\infty} \frac{x^n}{n!} && \given \forall x\\
      \sin x &= \sum_{n=0}^{\infty} \frac{(-1)^n x^{2n+1} }{(2n+1)!}  && \given \forall x \\
      \cos x &= \sum_{n=0}^{\infty} \frac{(-1)^n x^{2n} }{(2n)!} && \given  \forall x\\
      \frac{1}{1-x} &= \sum_{n=0}^{\infty} x^n && \given |x| < 1 \\
      \frac{1}{1+x} &= \sum_{n=0}^{\infty} (-1)^n x^n && \given |x| < 1 \\
      \ln (1+x) &= \sum_{n=1}^{\infty}  \frac{(-1)^{n-1}x^n }{n} && \given  |x| < 1 \oldland x = 1\\
      \tan ^{-1} x &= \sum_{n=0}^{\infty}  \frac{(-1)^n x^{2n+1} }{2n+1} && \given |x| \leq 1 \\
      (1 + x)^a x &= \sum_{n=0}^{\infty} \begin{pmatrix} \alpha \\ n
      \end{pmatrix} x^n  && \given |x| < 1 \\
                  &\text{where}~ \begin{pmatrix} \alpha \\ n \end{pmatrix} = \sum_{n=0}^{\infty} \prod_{k=1}^{n} \frac{\alpha - k + 1}{k}
      \end{align*}

\end{itemize}

\section{Power/Taylor Series Problems}

\subsection{10.6 Exercises}
Find the interval of convergence.

\begin{enumerate}[itemsep=24em]
  \item \(\displaystyle \sum_{n=0}^{\infty}  (-1)^n \frac{n}{4^n}x^{2n} \)
  \item \(\displaystyle \sum_{n=8}^{\infty} n^7 x^n \)

%%%%%%%%%%%%%
\newpage %%%%%%%%%%%%%%%%%%%%%%%%%%%%%%%%%%%%%%%%%%%%%%%%%%%%%%%%%%%%%%%%%%%%%%
%%%%%%%%%%%%%

  \item \(\displaystyle \sum_{n=2}^{\infty} \frac{x^n}{\ln n} \)
  \item \(\displaystyle \sum_{n=1}^{\infty} \frac{(-5)^n(x-3)^n}{n^2} \)

\end{enumerate}

%%%%%%%%%%%%%
\newpage %%%%%%%%%%%%%%%%%%%%%%%%%%%%%%%%%%%%%%%%%%%%%%%%%%%%%%%%%%%%%%%%%%%%%%
%%%%%%%%%%%%%

Use \(~\displaystyle \frac{1}{1-x} = \sum_{n=0}^{\infty} x^n ~\given |x| < 1 \)
to expand the function in a power series with \(c = 0\), and determine interval
of convergence.

\begin{enumerate}[itemsep=24em, resume]

  \item  \(\displaystyle f(x) = \frac{1}{4+3x} \)

  \item  \(\displaystyle f(x) = \frac{1}{1-x^4} \)
\end{enumerate}
%%%%%%%%%%%%%
\newpage %%%%%%%%%%%%%%%%%%%%%%%%%%%%%%%%%%%%%%%%%%%%%%%%%%%%%%%%%%%%%%%%%%%%%%
%%%%%%%%%%%%%

\subsection{10.8 Exercises}

Find the Maclaurin series and find the interval on which the expression is
valid.
\begin{enumerate}[itemsep=24em]
  \item \(\displaystyle  f(x) = \sin(2x) \)
  \item \(\displaystyle  f(x) = x^2e^{x^{2} } \)
\end{enumerate}

%%%%%%%%%%%%%
\newpage %%%%%%%%%%%%%%%%%%%%%%%%%%%%%%%%%%%%%%%%%%%%%%%%%%%%%%%%%%%%%%%%%%%%%%
%%%%%%%%%%%%%

Find the Taylor series centered at c and the interval on which the expansion is
valid.
\begin{enumerate}[itemsep=18em, resume]
  \item \(\displaystyle  f(x) = e^{3x}, \quad c = -1 \)
  \item \(\displaystyle  f(x) = \sin(x), \quad c = \frac{\pi}{2} \)
  \item \(\displaystyle  f(x) = \frac{1}{1-x^2}, \quad c = 3\)
\end{enumerate}


