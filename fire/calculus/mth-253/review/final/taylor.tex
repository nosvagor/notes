\chapter{Power/Taylor Series: 10.6--10.8}

\section{Power/Taylor Series Notes}

\vspace{-16pt}
\subsection{Power Series}
\begin{itemize}
  \item \dd{Power series}: a infinite series in the form:
    \[%%%%%%%%%%
    F(x) = \sum_{n=0}^{\infty} a_n (x - c)^n
    \]%%%%%%%%%%
    Where the constant \(c\) is the \textit{center} of the power series F(x).

  \item \dd{Radius of convergence \(R\)}: the range of values of the variable \(x\)
    whereby the power series \(F(x)\) converges.
    \begin{itemize}
      \item Every power series converges at \(x = c\), as \((x-c)^0 = 1\),
        though the series may diverge for other values of \(x\).

      \item \(F(x)\) converges for \(\left| x - c \right| < R \) and diverges
        for \(\left| x - c \right| > R \)

      \item \(F(x)\) may converge of diverge at endpoints \(c-R\) and \(c+R\)

      \item \dd{Interval of convergence}: the open interval \((c-R, c+R)\) and
        possibly one of both of the endpoints, each must be tested.
        \begin{itemize}
          \item In most cases, the \ulink{ss:Tests for Non-Positive
            Series}{ratio test} can be used to find R.

          \item If \(R > 0\), then \(F\) is differentiable over the interval of
            convergence; the derivative and antiderivative can be obtained
            using the following:
            \[%%%%%%%%%%
            F'(x) = \sum_{n=1}^{\infty} na_n(x-c)^{n-1} \qquad \quad
            F(x)dx = C + \sum_{n=0}^{\infty} \frac{a_n}{n+1}(x-c)^{n+1}
            \]%%%%%%%%%%
        \end{itemize}

      \item \dd{Useful Power Series}: the following power series
        (more examples: \dlink{ss:Taylor Series}{Taylor series}) can be used to drive
        expansions of other related functions via substitution, integration, or
        differentiation:
        \[%%%%%%%%%%
          \frac{1}{1-x} = \sum_{n=0}^{\infty} x^n \quad \given |x| < 1
          \qquad \qquad
          e^x = \sum_{n=0}^{\infty} \frac{x^n}{n!}
        \]%%%%%%%%%%

    \end{itemize}
\end{itemize}


%%%%%%%%%%%%%
\newpage %%%%%%%%%%%%%%%%%%%%%%%%%%%%%%%%%%%%%%%%%%%%%%%%%%%%%%%%%%%%%%%%%%%%%%
%%%%%%%%%%%%%

\subsection{Taylor Series}
\begin{itemize}
  \item

\end{itemize}

\section{Power/Taylor Series Problems}

\subsection{10.6 Exercises}
\begin{enumerate}[itemsep=3em]
  \item

\end{enumerate}

\subsection{10.8 Exercises}
\begin{enumerate}[itemsep=3em]
  \item
\end{enumerate}
