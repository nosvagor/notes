\chapter{Power/Taylor Series: 10.6--10.8}

\section{Power/Taylor Series Notes}

\subsection{Power Series}
\begin{itemize}
  \item \dd{Power series}: a infinite series in the form:
    \[%%%%%%%%%%
    F(x) = \sum_{n=0}^{\infty} a_n (x - c)^n
    \]%%%%%%%%%%
    Where the constant \(c\) is the \textit{center} of the power series F(x).

  \item \dd{Radius of convergence \(R\)}: the range of values of the variable \(x\)
    whereby the power series \(F(x)\) converges.
    \begin{itemize}
      \item Every power series converges at \(x = c\), as \((x-c)^0 = 1\),
        though the series may diverge for other values of \(x\).

      \item \(F(x)\) converges for \(\left| x - c \right| < R \) and diverges
        for \(\left| x - c \right| > R \)

      \item \(F(x)\) may converge of diverge at endpoints \(c-R\) and \(c+R\)

    \end{itemize}

  \item \dd{Interval of convergence}: the open interval \((c-R, c+R)\) and
    possibly one of both of the endpoints, each must be tested.
    \begin{itemize}
      \item In most cases, the \ulink{ss:Tests for Non-Positive
        Series}{ratio test} can be used to find R.

      \item If \(R > 0\), then \(F\) is differentiable over the interval of
        convergence; the derivative and antiderivative can be obtained
        using the following:
        \[%%%%%%%%%%
          F'(x) = \sum_{n=1}^{\infty} na_n(x-c)^{n-1} \qquad \quad
          F(x)dx = C + \sum_{n=0}^{\infty} \frac{a_n}{n+1}(x-c)^{n+1}
        \]%%%%%%%%%%
    \end{itemize}


\end{itemize}

%%%%%%%%%%%%%
\newpage %%%%%%%%%%%%%%%%%%%%%%%%%%%%%%%%%%%%%%%%%%%%%%%%%%%%%%%%%%%%%%%%%%%%%%
%%%%%%%%%%%%%

\subsection{Taylor Series}
\begin{itemize}
  \item \dd{Taylor series}: the power series of a infinitely differentiable
    function \(f(x)\) centered at \(c\),
    \[%%%%%%%%%%
      \sum_{n=0}^{\infty} \frac{f^{(n)}(c) }{n!} (x-c)^n
    \]%%%%%%%%%%
  \begin{itemize}
    \item \dd{\(\bm{n}^{th} \) Taylor polynomial}: a polynomial of degree \(n\)
      that is formed partial sum formed by the first \(n+1\) terms of a Taylor
      series, i.e.,
      \[%%%%%%%%%%
      f(c) + f'(c)(x-c) + \frac{f''(c)}{2!}(x-c)^2 + \frac{f'''(c)}{3!}(x-c)^3
      + \cdots +  \frac{f^n (c)}{n!}(x-c)^n
      \]%%%%%%%%%%
    \item \dd{Maclaurin series}: when \(c = 0\), i.e.,
      \[%%%%%%%%%%
        \sum_{n=0}^{\infty} \frac{f^{(n)}0 }{n!} (x)
      \]%%%%%%%%%%
  \end{itemize}

  \item \dd{Useful Maclaurin Series}: useful Taylor series centered at 0 that
    can be used to derive other series via differentiation, integration,
    multiplication, or substitution.
    \begin{align*}
      e^x &= \sum_{n=0}^{\infty} \frac{x^n}{n!} && \given \forall x\\
      \sin x &= \sum_{n=0}^{\infty} \frac{(-1)^n x^{2n+1} }{(2n+1)!}  && \given \forall x \\
      \cos x &= \sum_{n=0}^{\infty} \frac{(-1)^n x^{2n} }{(2n)!} && \given  \forall x\\
      \frac{1}{1-x} &= \sum_{n=0}^{\infty} x^n && \given |x| < 1 \\
      \frac{1}{1+x} &= \sum_{n=0}^{\infty} (-1)^n x^n && \given |x| < 1 \\
      \ln (1+x) &= \sum_{n=1}^{\infty}  \frac{(-1)^{n-1}x^n }{n} && \given  |x| < 1 \oldland x = 1\\
      \tan ^{-1} x &= \sum_{n=0}^{\infty}  \frac{(-1)^n x^{2n+1} }{2n+1} && \given |x| \leq 1 \\
      (1 + x)^a x &= \sum_{n=0}^{\infty} \begin{pmatrix} \alpha \\ n
      \end{pmatrix} x^n  && \given |x| < 1 \\
                  &\text{where}~ \begin{pmatrix} \alpha \\ n \end{pmatrix} = \sum_{n=0}^{\infty} \prod_{k=1}^{n} \frac{\alpha - k + 1}{k}
      \end{align*}

\end{itemize}

\section{Power/Taylor Series Problems}

\subsection{10.6 Exercises}
Find the interval of convergence.

\begin{enumerate}[itemsep=24em]
  \item \(\displaystyle \sum_{n=0}^{\infty}  (-1)^n \frac{n}{4^n}x^{2n} \)
    \begin{align*}
      \rho &= \lim_{n \to \infty} \left| \frac{a_{n+1}}{a_n}  \right|
           && \text{Apply the \ulink{ss:Tests for Non-Positive Series}{ratio test}} \\
      &= \lim_{n \to \infty}
      \left|  \frac{(-1)^{n+1}(n+1)(x^{2(n+1)})}{4^{n+1}}
      \cdot
        \frac{4^n}{(-1)^{n}n(x^{2n})}
      \right| \\
      &= \lim_{n \to \infty}
      \frac{(n+1)(|x|^{2n} \cdot |x|^2)}{4^n \cdot 4}
      \cdot
      \frac{4^n}{n\cdot |x|^{2n}} \\
      &= \lim_{n \to \infty}
      \frac{(n+1)|x|^2}{4n} \cdot \frac{n^{-1}}{n^{-1}} \\
      &= \lim_{n \to \infty}
      \frac{(1+n^{-1})|x|^2}{4} = \frac{|x|^2}{4} \\\\
      &\then \frac{|x|^2}{4} < 1
      \then |x| < 2
      && \text{converges for}~\rho < 1
    \end{align*}
    Both endpoints tend toward \(\infty \) (diverge), thus the
    interval of convergence is \aset{\((-2,2)\)}.

  \vspace{-22em}
  \item \(\displaystyle \sum_{n=8}^{\infty} n^7 x^n \)
    \begin{align*}
      \rho &= \lim_{n \to \infty} \left| \frac{a_{n+1}}{a_n}  \right|
           && \text{Apply the \ulink{ss:Tests for Non-Positive Series}{ratio test}} \\
      &\then \lim_{n \to \infty}  \left| \frac{(n+1)^7 x^{n+1}}{n^7x^n} \right| \\
      &= \lim_{n \to \infty}  \frac{(n+1)^7 \cdot |x|^n \cdot |x|}{n^7|x|^n} \\
      &= \lim_{n \to \infty}  \frac{(n+1)^7|x|}{n^7}\\
      &= |x| \lim_{n \to \infty}  \left(\frac{n+1}{n}\right)^{7}  = |x| \\
      \\
      &\then |x| < 1
      && \text{converges for}~\rho < 1
    \end{align*}
    Both endpoints tend toward \(\infty\) (diverge), thus the
    interval of convergence is \aset{\((-1, 1)\)}.


%%%%%%%%%%%%%
\newpage %%%%%%%%%%%%%%%%%%%%%%%%%%%%%%%%%%%%%%%%%%%%%%%%%%%%%%%%%%%%%%%%%%%%%%
%%%%%%%%%%%%%

  \item \(\displaystyle \sum_{n=2}^{\infty} \frac{x^n}{\ln n} \)
    \begin{align*}
      \rho &= \lim_{n \to \infty} \left| \frac{a_{n+1}}{a_n}  \right|
           && \text{Apply the \ulink{ss:Tests for Non-Positive Series}{ratio test}} \\
      &\then \lim_{n \to \infty}
      \left| \frac{x^{n+1}}{\ln(n+1)} \cdot \frac{\ln n}{x^n}  \right| \\
      &= \lim_{n \to \infty}
     \frac{|x| \ln n}{\ln(n+1)} \\
      &= |x| \lim_{n \to \infty}
      \frac{\ln n}{\ln(n+1)}  = \frac{\infty}{\infty}\\
      &= |x| \lim_{n \to \infty} \frac{n^{-1}}{(n+1)^{-1}}
      && \text{By \hopital's Rule} \\
      &= |x| \lim_{n \to \infty} \frac{n+1}{n} \\
      &= |x| \lim_{n \to \infty} 1+n^{-1} = |x| \\
      \\
      &\then |x| < 1
      && \text{converges for}~\rho < 1 \\
      \\
      f(1) &= \sum_{n=1}^{\infty} \frac{1}{\ln n} \\
      &\then 0 \leq b_n \leq a_n
           && \text{Apply the \ulink{ss:Tests for Positive Series}{direct comparison test}} \\
      &\then 0 \leq \frac{1}{n} \leq \frac{1}{\ln n} \\
      &\then a_n ~\text{diverges}
      && \frac{1}{n} \to \text{diverges as } p \leq 1 \\
      \\
      f(-1) &= \sum_{n=1}^{\infty} \frac{(-1)^n}{\ln n} \\
      &\sum_{n=1}^{\infty} (-1)^n a_n, \quad a_n = \frac{1}{\ln n}
      && \text{Apply the \ulink{ss:Tests for Non-Positive Series}{alternating series test}} \\
      &\then \lim_{n \to \infty} \frac{1}{\ln n} = 0
      && \text{Note:}~|a_n| ~\text{decreases monotonically}\\
    \end{align*}
     \(f(-1)\) converges and \(f(1)\) diverges, thus the interval of convergence
    is \aset{\([-1,1)\)}

%%%%%%%%%%%%%
\newpage %%%%%%%%%%%%%%%%%%%%%%%%%%%%%%%%%%%%%%%%%%%%%%%%%%%%%%%%%%%%%%%%%%%%%%
%%%%%%%%%%%%%

  \item \(\displaystyle \sum_{n=1}^{\infty} \frac{(-5)^n(x-3)^n}{n^2} \)
    \begin{align*}
      \rho &= \lim_{n \to \infty} \left| \frac{a_{n+1}}{a_n}  \right|
           && \text{Apply the \ulink{ss:Tests for Non-Positive Series}{ratio test}} \\
      &\then \lim_{n \to \infty} \left|
      \frac{(-5)^{n+1} (x-3)^{n+1}}{(n+1)^2} \cdot
      \frac{n^2}{(-5)^n(x-3)^n}
      \right| \\
      &= \lim_{n \to \infty}
      \frac{5^n\cdot 5 \cdot (|x-3|)^{n} \cdot |x-3|}{n^2} \cdot
      \frac{n^2}{5^n(|x-3|)^n} \\
      &=\lim_{n \to \infty} 5|x-3| \\
      &\then |x-3| < \frac{1}{5} \\
      &\then -\frac{14}{5} < x < \frac{16}{5}
      && \text{converges for}~\rho < 1 \\
      \\
      &f\left(\frac{-14}{5}\right) = \frac{(-5)^n\left(-\frac{1}{5}\right)^n}{n^2} = \frac{1}{n^2}
      \\
      &f\left(\frac{16}{5}\right) = \frac{(-5)^n\left(\frac{1}{5}\right)^n}{n^2} = \frac{(-1)^n}{n^2}
    \end{align*}

    \(\lim_{n \to \infty}  a_n\) (of both points) \(= 0\) and the \(|a_n|\) of both
    endpoints decrease monotonically; \(R = \dfrac{1}{5}, c = 3\), thus the
    interval of convergence is \aset{\(\left[\displaystyle -\frac{14}{5}, \frac{16}{5}\right]\)}

\end{enumerate}

%%%%%%%%%%%%%
\newpage %%%%%%%%%%%%%%%%%%%%%%%%%%%%%%%%%%%%%%%%%%%%%%%%%%%%%%%%%%%%%%%%%%%%%%
%%%%%%%%%%%%%

Use the following equation to expand the function in a power series with \(c =
0\) \[~\displaystyle \frac{1}{1-x} = \sum_{n=0}^{\infty} x^n ~\given |x| < 1 \]
and determine the interval of convergence.

\begin{enumerate}[itemsep=24em, resume]
  \item  \(\displaystyle f(x) = \frac{1}{4+3x} \)
    \begin{align*}
      \frac{1}{4+3x} &= \frac{\frac{1}{4}}{1 - (-\frac{3x}{4})} \\
      &= \frac{1}{4} \sum_{n=0}^{\infty} \left( -\frac{3x}{4} \right)^n
      = \sum_{n=0}^{\infty} (-1)^n \frac{3^nx^n}{4^{n+1} }
      && \text{Expansion} \\
      &\then \left| \frac{3x}{4} \right| < 1
      && \sum_{n=1}^{\infty} \given |x| < 1 \\
      &\then -\frac{4}{3} < x < \frac{4}{3}
      && \text{Interval of convergence}
    \end{align*}
    Thus, \(\displaystyle \frac{1}{4+3x} =
    \aset{\sum_{n=0}^{\infty} (-1)^n \frac{3^nx^n}{4^{n+1}}} \)
    with an interval of convergence of
   \aset{\(\displaystyle \left( -\frac{4}{3}, \frac{4}{3} \right) \)}

  \vspace{-16em}
  \item  \(\displaystyle f(x) = \frac{1}{1-x^4} \)
    \begin{align*}
      \frac{1}{1-x^4} &= \sum_{n=0}^{\infty} \left( x^4 \right)^n
      = \sum_{n=0}^{\infty} x^{4n}
      && \text{Expansion} \\
      &\then |x^4| < 1
      && \sum_{n=1}^{\infty} \given |x| < 1 \\
      &\then -1 < x < 1
      && \text{Interval of convergence}
    \end{align*}
    Thus, \(\displaystyle \frac{1}{1-x^4} =
    \aset{\sum_{n=0}^{\infty} x^{4n}}  \)
    with an interval of convergence of
   \aset{\(\displaystyle \left(-1, 1\right) \)}
\end{enumerate}

%%%%%%%%%%%%%
\newpage %%%%%%%%%%%%%%%%%%%%%%%%%%%%%%%%%%%%%%%%%%%%%%%%%%%%%%%%%%%%%%%%%%%%%%
%%%%%%%%%%%%%

\subsection{10.8 Exercises}

Find the Maclaurin series and find the interval on which the expression is
valid.
\begin{enumerate}[itemsep=24em]
  \item \(\displaystyle  f(x) = \sin(2x) \)
    \begin{align*}
      \sin x &= \sum_{n=0}^{\infty} \frac{(-1)^nx^{2n+1} }{(2n+1)!}
      && \text{Use relevant \ulink{ss:Taylor Series}{Maclaurin series}} \\
      &\then \sin 2 x = (-1)^n \frac{(2x)^{2n+1} }{(2n+1)!}
    \end{align*}
    \(\sin(x)\) converges \(\forall x\), thus \(\sin(2x)\) also converges \(\forall x\).
    Therefore:
    \[%%%%%%%%%%
      f(x) = \aset{(-1)^n \frac{(2x)^{2n+1} }{(2n+1)!}}
      \qquad \given \forall x \in \R
    \]%%%%%%%%%%

  \vspace{-15em}
  \item \(\displaystyle  f(x) = x^2e^{x^{2} } \)
    \begin{align*}
      e^x &= \sum_{n=0}^{\infty} \frac{x^n}{n!}
      && \text{Use relevant \ulink{ss:Taylor Series}{Maclaurin series}} \\
      &\then e^{x^2} = \sum_{n=0}^{\infty} \frac{(x^2)^n}{n!} \\
      &\then x^2e^{x^2} = \sum_{n=0}^{\infty} \frac{x^2 \cdot (x^2)^n}{n!}
      = \sum_{n=0}^{\infty} \frac{x^{2n+2} }{n!}
    \end{align*}
    \(e^x\) converges \(\forall x\), thus \(e^{x^2} \) also converges \(\forall x\).
    Therefore:
    \[%%%%%%%%%%
      f(x) = \aset{\sum_{n=0}^{\infty} \frac{x^{2n+2}}{n!}}
      \qquad \given \forall x \in \R
    \]%%%%%%%%%%
\end{enumerate}

%%%%%%%%%%%%%
\newpage %%%%%%%%%%%%%%%%%%%%%%%%%%%%%%%%%%%%%%%%%%%%%%%%%%%%%%%%%%%%%%%%%%%%%%
%%%%%%%%%%%%%

Find the Taylor series centered at c and the interval on which the expansion is
valid.
\begin{enumerate}[itemsep=18em, resume]
  \item \(\displaystyle  f(x) = e^{3x}, \quad c = -1 \)
    \begin{align*}
      e^x &= \sum_{n=0}^{\infty} \frac{x^n}{n!}
      && \text{Use relevant \ulink{ss:Taylor Series}{Maclaurin series}} \\
      \then
      f(x) &= e^{3(x-1)} = e^{-3} \cdot e^{3(x+1)}
      && \text{Center series \((x-c\))} \\
      &= e^{-3} \sum_{n=0}^{\infty} \frac{\left(3(x+1)\right)^n}{n!} \\
      &= e^{-3} \sum_{n=0}^{\infty} \frac{3^n(x+1)^n}{n!}
    \end{align*}
    \(e^x\) converges \(\forall x\), thus \(e^{3x} \) also converges \(\forall x\).
    Therefore:
    \[%%%%%%%%%%
      f(x) = \aset{e^{-3} \sum_{n=0}^{\infty} \frac{3^n(x+1)^n}{n!}}
      \qquad \given \text{convergence interval:}~\aset{(-\infty, \infty)}
    \]%%%%%%%%%%

  \vspace{-12em}
  \item \(\displaystyle  f(x) = \sin(x), \quad c = \frac{\pi}{2} \)
    \begin{align*}
      \sin x &= \sum_{n=0}^{\infty} \frac{(-1)^nx^{2n+1} }{(2n+1)!}
      && \text{Use relevant \ulink{ss:Taylor Series}{Maclaurin series}} \\
      \then f(x) &= \sin\left(x-\frac{\pi}{2}\right) \\
      &= \sum_{n=0}^{\infty} (-1)^n \frac{\left( x - \frac{\pi}{2}\right)^{2n+1} }{(2n+1)!}
      && \text{Center series} (x-c)
    \end{align*}
    \(\sin(x)\) converges \(\forall x\), therefore:
    \[%%%%%%%%%%
      f(x)
      = \aset{\sum_{n=0}^{\infty} (-1)^n \frac{\left( x-\frac{\pi}{2} \right) ^{2n+1} }{(2n+1)!}}
      \qquad \given \text{convergence interval:}~\aset{(-\infty, \infty)}
    \]%%%%%%%%%%
\end{enumerate}


