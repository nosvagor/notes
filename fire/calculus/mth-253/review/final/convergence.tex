\chapter{Convergence: 10.3--10.5}
\section{Convergence Notes}
\begin{itemize}

  \item Let \(\displaystyle \sum_{n=1}^{\infty} a_n\) be given and note for
    which series convergence is known, i.e.:

    \vspace{10pt}

    \begin{multicols}{2}
      \dd{Geometric}: let \(c \neq 0\), if \(|r| < 1\), then
      \[%%%%%%%%%%
        \sum_{n=0}^{\infty} cr^n = \frac{c}{1-r}
      \]%%%%%%%%%%
      \(|r| > 1 \then \text{diverges}\)

      \dd{\(\bm{p}\)-Series}: converges if \(p > 1\).
      \[%%%%%%%%%%
        \sum_{n=0}^{\infty} \frac{1}{n^p}
      \]%%%%%%%%%%
      \(p < 1 \then \text{diverges}\)
    \end{multicols}

  \item \dd{The \(\bm{n}^{\bm{th}}\) Term Divergence Test}: a relatively easy
    test that can be used to quickly determine if a test diverges if the
    \(\lim_{n \to \infty} a_n \neq 0 \). If \(\lim_{n \to \infty} a_n = 0\),
    then the test is inconclusive and other tests must be applied.

\end{itemize}


\subsection{Tests for Positive Series}
\begin{itemize}
  \item \dd{Direct Comparison Test}: use if dropping terms from the
    denominator or numerator gives a series \(b_n\) wherein convergence
    is easily found, then compare to the original series \(a_n\) as
    follows:
    \[%%%%%%%%%%
      \sum_{n=1}^{\infty} b_n ~\text{converges}~ \then
      \sum_{n=1}^{\infty} a_n ~\text{converges} \quad \given 0 \leq a_n
      \leq b_n
    \]%%%%%%%%%%

    \[%%%%%%%%%%
      \sum_{n=1}^{\infty} b_n ~\text{diverges}~ \then
      \sum_{n=1}^{\infty} a_n ~\text{diverges} \quad \given 0 \leq
      b_n \leq a_n
    \]%%%%%%%%%%

  \item \dd{Limit Comparison Test}: use when the direct comparison test
    isn't convenient or when comparing two series. One can to take the
    dominant term in the numerator and denominator from \(a_n\) to form
    a new positive sequence \(b_n\) if needed.

    Assuming the following limit \(L = \lim_{n \to \infty}
    \frac{a_n}{b_n}\) exists, then:
    \begin{align*}
      L > 0 \then \sum_{n=1}^{\infty} a_n  ~\text{converges}~
      &\iff
      \sum_{n=1}^{\infty} b_n ~\text{converges}\\
      L = 0 ~\text{and}~ \sum_{n=1}^{\infty} b_n ~\text{converges}~
      &\then
      \sum_{n=1}^{\infty} a_n ~\text{converges}\\
      L = \infty ~\text{and}~ \sum_{n=1}^{\infty} a_n ~\text{converges}~
      &\then \sum_{n=1}^{\infty} b_n ~\text{converges}
    \end{align*}

    %%%%%%%%%%%%%
    \newpage %%%%%%%%%%%%%%%%%%%%%%%%%%%%%%%%%%%%%%%%%%%%%%%%%%%%%%%%%%%%%%%%%%%%%%
    %%%%%%%%%%%%%

  \item \dd{Ratio Test}: often used in the presence of a factorial
    \dd{\((\bm{n!})\)} or when the are constants raised to the power of
    \(n\) \dd{\((\bm{c^n})\)}.

    Assuming the following limit \(\rho = \lim_{n \to \infty} \left|
    \frac{a_n + 1}{a_n} \right| \) exists, then
    \begin{align*}
      \rho &< 1 \then \sum a_n ~\text{converges absolutely} \\
      \rho &> 1 \then \sum a_n ~\text{diverges} \\
      \rho &= 1 \then ~\text{test is inconclusive}~
    \end{align*}

  \item \dd{Root Test}: used when there is a term in the form of
    \dd{\(\bm{f(n)^{\bm{g(n)}}}\)}.

    Assuming the following limit \(C = \lim_{n \to \infty}
    |a_n|^{\frac{1}{n}} \) exists, then
    \begin{align*}
      C &< 1 \then \sum a_n ~\text{converges absolutely} \\
      C &> 1 \then \sum a_n ~\text{diverges} \\
      C &= 1 \then ~\text{test is inconclusive}~
    \end{align*}

  \item \dd{Integral Test}: if the other tests fail and \(a_n = f(n)\) is
    a decreasing function, then one can use the improper integral
    \(\int_1^\infty f(x)dx\) to test for convergence.

    Let \(a_n = f(n)\) be a positive, decreasing, and continuous function
    \(\forall x \geq 1\), then:
    \begin{align*}
      \int_1^\infty f(x)dx ~\text{converges} &\then \sum_{n=1}^{\infty}
      a_n ~\text{converges}\\
      \int_1^\infty f(x)dx ~\text{diverges} &\then \sum_{n=1}^{\infty}
      a_n ~\text{diverges}
    \end{align*}
\end{itemize}

\subsection{Tests for Non-Positive Series}
\vspace{-8pt}
  \begin{itemize}
    \item \dd{Alternating Series Test}: used for series in the form
      \(%%%%%%%%%%
      \displaystyle  \sum_{n=0}^{\infty} (-1)^n a_n
      \)%%%%%%%%%%

      Converges if \(|a_n|\) decreases monotonically \(\left(|a_n+1| \leq |a_n|\right)\)
      and if \(\lim_{n \to \infty} a_n = 0\)

      \vspace{16pt}

    \item \dd{Absolute Convergence}: used if the series \(\sum a_n\) is not
      alternating; simply test if \(\sum |a_n| \) converges using the test
      for positive series.

      \vspace{16pt}

      \(\sum a_n\) converges \dd{conditionally} if \(\sum a_n\) converges, but \(\sum
      |a_n|\) diverges.

      \(\sum a_n\) converges \dd{absolutely} if \(\sum |a_n|\) converges.


  \end{itemize}


%%%%%%%%%%%%%
\newpage %%%%%%%%%%%%%%%%%%%%%%%%%%%%%%%%%%%%%%%%%%%%%%%%%%%%%%%%%%%%%%%%%%%%%%
%%%%%%%%%%%%%

\section{Convergence Problems}

\subsection{10.5 Exercises}
Determine convergence or divergence using any method.
\vspace{1em}
\begin{enumerate}[itemsep=12em]
  \item \(\displaystyle \sum_{n=1}^{\infty} \frac{2^n + 4^n }{7^n}\)

    \begin{align*}
      &\then
      \sum_{n=1}^{\infty} \frac{2^n}{7^n} +
      \sum_{n=1}^{\infty} \frac{4^n }{7^n}
      && \text{Separate into two \ulink{s:Convergence Notes}{geometric series}} \\
      &\then
      r = \frac{2}{7} < 1, \quad r = \frac{4}{7} < 1
    \end{align*}

    Both geometric series converge, thus the original series \aset{converges}.

  \item \(\displaystyle \sum_{n=1}^{\infty} \frac{n^3}{n!} \)
    \begin{align*}
      \then \rho &= \lim_{n \to \infty} \left| \frac{(n+1)^3}{(n+1)!} \cdot
      \frac{n!}{n^3} \right|
      && \text{Apply the \ulink{ss:Tests for Non-Positive Series}{ratio test}} \\
       &= \lim_{n \to \infty} \frac{n^3 + 3n^2 + 3n + 1}{(n+1)n!} \cdot
      \frac{n!}{n^3} \\
       &= \lim_{n \to \infty} \frac{n^3 + 3n^2 + 3n + 1}{n^4+n^3} \\
       &= \lim_{n \to \infty} \frac{n^3 + 3n^2 + 3n + 1}{n^4+n^3} \cdot \frac{n^{-4}}{n^{-4}}\\
       &= \lim_{n \to \infty} \frac{n^{-1}  + 3n^{-2} + 3n^{-3}  + n^{-4} }{1+n^{-1} } = 0
    \end{align*}

    \(\rho = 0 < 1\), thus the series \aset{converges}.

  \item \(\displaystyle \sum_{n=1}^{\infty} \frac{n}{2n+1} \)
    \begin{align*}
      &\then \lim_{n \to \infty} \frac{n}{2n+1}
      && \text{Apply the \ulink{ss:Tests for Positive Series}{\(n^{th} \) term test}} \\
      &\then \lim_{n \to \infty} \frac{n}{2n+1} = \frac{1}{2}
      && \text{by \hopital's Rule}
    \end{align*}

    \(\lim_{n \to \infty} a_n \neq 0\), thus the series \aseg{diverges}.

  \item \(\displaystyle \sum_{n=1}^{\infty} 2^\frac{1}{n} \)
    \begin{align*}
      &\then \lim_{n \to \infty} 2^{\frac{1}{n}} = 2^0 = 1
      && \text{Apply the \ulink{ss:Tests for Positive Series}{\(n^{th} \) term test}}
    \end{align*}

    \(\lim_{n \to \infty} a_n \neq 0\), thus the series \aseg{diverges}.

  \vspace{-16pt}
  \item \(\displaystyle \sum_{n=1}^{\infty} \frac{\sin n}{n^2} \)
    \begin{align*}
      0 \leq \sin  n &\leq 1 && \given \forall n \geq 1 \\
      0 \leq \frac{\sin n}{n^2} &\leq \frac{1}{n^2}
      && \text{Apply the \ulink{ss:Tests for Positive Series}{direct comparison test}} \\
      b_n = \frac{1}{n^2} &\to ~\text{converges}
      && \text{by \ulink{ss:Tests for Positive Series}{\(p\)-series}}
    \end{align*}

    The larger (\(b_n\)) series converges, thus the smaller (\(a_n\)) \aset{converges}.

  \vspace{-16pt}
  \item \(\displaystyle \sum_{n=1}^{\infty} \frac{n!}{(2n)!} \)
    \begin{align*}
      \then \rho &= \lim_{n \to \infty} \left|
     \frac{(n+1)!}{(2n+2)!} \cdot \frac{(2n)!}{n!} \right|
     && \text{Apply the \ulink{ss:Tests for Non-Positive Series}{ratio test}} \\
     &=\lim_{n \to \infty}
     \frac{(n+1)n!}{(2n+2)(2n+1)2n!} \cdot \frac{(2n)!}{n!} \\
     &=\lim_{n \to \infty}
     \frac{n+1}{(2n+2)(2n+1)} =
     \frac{n+1}{4n^2+6n+2} \\
     &=\lim_{n \to \infty} \frac{1}{8n + 6} = 0
     && \text{By \hopital's Rule} \\
    \end{align*}

  \(\rho = 0 < 1\), thus the series \aset{converges}.

  \vspace{-16pt}
  \item \(\displaystyle \sum_{n=1}^{\infty} \frac{1}{n + \sqrt{n} } \)
    \begin{align*}
      0 &\leq n \leq n +\sqrt{n}  && \given \forall n \geq 1 \\
      0 &\leq \frac{1}{n+\sqrt{n} } \leq \frac{1}{n}
      && \text{Apply the \ulink{ss:Tests for Positive Series}{direct comparison test}} \\
      b_n &= \frac{1}{n} \to ~\text{diverges}
    \end{align*}

  The smaller (\(b_n\)) series diverges, thus the larger \(a_n\) original
  series \aseg{diverges}.

  \item \(\displaystyle \sum_{n=2}^{\infty} \frac{1}{n(\ln n)^3} \)
  \begin{align*}
    &f~\text{is positive, decreasing, and continuous for}~x\geq 2
    && \text{Apply the \ulink{ss:Tests for Non-Positive Series}{integral test}} \\
    &\then \int_{2}^{\infty} f(x)dx = \lim_{R \to \infty} \int_{2}^{R}
    \frac{1}{x(\ln x)^3} dx &
    &\ln x = u, \quad xdu = dx
  \end{align*}
  \begin{align*}
    \then \lim_{R \to \infty} \int_{2}^{R} \frac{1}{x(u)^3} xdu &=
    \int_{2}^{R} \frac{1}{u}^3 du \\
     &= -\frac{1}{2(u)^2} \\
     &= -\frac{1}{2\ln ^2(x)} + C ~\bigg|_{2}^{\infty} \\
    \then 0 - \left(-\frac{1}{2\ln ^2(2)}\right) = \frac{1}{2\ln ^2(2)}
  \end{align*}

  The improper integral converges, thus the original series \aset{converges}.

  \item \(\displaystyle \sum_{n=1}^{\infty} \frac{n^3}{5^n} \)
    \begin{align*}
      \then \rho &= \lim_{n \to \infty} \left| \frac{(n+1)^3}{5^{n+1} }
      \cdot \frac{5^n}{n^3} \right|
     && \text{Apply the \ulink{ss:Tests for Non-Positive Series}{ratio test}} \\
      &= \lim_{n \to \infty} \frac{n^3+1}{5^n + 5^1}
      \cdot \frac{5^n}{n^3} = \frac{1}{5}
    \end{align*}

  \(\rho = \dfrac{1}{5} < 1\), thus the series \aset{converges}.


  \item \(\displaystyle \sum_{n=2}^{\infty} \frac{1}{\sqrt{n^3 - n^2}} \)
    \begin{align*}
      L &= \lim_{n \to \infty} \frac{a_n}{b_n}, \quad b_n = \frac{1}{\sqrt{n^3} }
      && \text{Apply the \ulink{ss:Tests for Positive Series}{limit comparison test}} \\
      \then L &= \lim_{n \to \infty} \frac{1}{\sqrt{n^3 - n^2}}
      \cdot \frac{\sqrt{n^3}}{1} \\
      &= \lim_{n \to \infty} \sqrt{\frac{n^3}{n^3(1-n ^{-1})}} \\
      &= \sqrt{\frac{1}{1(1-0)}} = 1
    \end{align*}
    \(L > 0\), thus \(a_n\) converges if \(b_n\) converges.

    \(b_n\) converges by the \(p\)-series test, as \(\frac{3}{2} > 1\), thus
    \(a_n\) \aset{converges}.


  \item \(\displaystyle \sum_{n=1}^{\infty} \frac{n^2 + 4n}{3n^4 + 9} \)
    \begin{align*}
      L &= \lim_{n \to \infty} \frac{a_n}{b_n}, \quad b_n = \frac{1}{n^2}
      && \text{Apply the \ulink{ss:Tests for Positive Series}{limit comparison test}} \\
      &= \lim_{n \to \infty} \frac{n^2 + 4n}{3n^4 + 9} \cdot n^2 \\
      &= \lim_{n \to \infty} \frac{n^4 + 4n^3}{3n^4 + 9} \cdot \frac{n^{-4}}{n^{-4}} \\
      &= \lim_{n \to \infty} \frac{1 + 4n^{-1}}{3 + 9n^{-4}} = \frac{1}{3}
    \end{align*}

    \(L > 0\), thus \(a_n\) converges if \(b_n\) converges.

    \(b_n\) converges by the \(p\)-series test, as \(2 > 1\), thus
    \(a_n\) \aset{converges}.

  \item \(\displaystyle \sum_{n=1}^{\infty} (0.8)^{-n} n^{-0.8}  \)
    \begin{align*}
      \rho &= \lim_{n \to \infty} \left| \frac{a_{n+1}}{a_n}  \right|
     && \text{Apply the \ulink{ss:Tests for Non-Positive Series}{ratio test}} \\
     &= \lim_{n \to \infty}
     \left|  \frac{(0.8)^{-(n+1)} (n+1)^{-0.8}}{(0.8)^{-n} n^{-0.8}} \right| \\
     &= \lim_{n \to \infty}  \frac{(0.8)^{n} n^{0.8}}{(0.8)^{n+1} (n+1)^{0.8}} \\
     &= \lim_{n \to \infty}  \frac{1}{0.8} = 1.25
    \end{align*}

    \(\rho = 1.25 > 1 \), thus \(a_n\) \aseg{diverges}.

  \item \(\displaystyle \sum_{n=1}^{\infty} 4^{-2n+1}  \)
    \begin{align*}
      \rho &= \lim_{n \to \infty} \left| \frac{a_{n+1}}{a_n}  \right|
           && \text{Apply the \ulink{ss:Tests for Non-Positive Series}{ratio test}} \\
           &= \lim_{n \to \infty} \frac{4^{-2(n+1) + 1}}{4^{-2n+1} } \\
           &= \lim_{n \to \infty} \frac{4^{-2n-1}}{4^{-2n+1} } \\
           &= \lim_{n \to \infty} \frac{4^{-2n} 4^{-1}}{4^{-2n}4} = \frac{1}{16}
    \end{align*}

  \(\rho = \frac{1}{16} < 1 \), thus \(a_n\) \aset{converges}.

  \item \(\displaystyle \sum_{n=1}^{\infty} \frac{(-1)^{n-1}}{\sqrt{n} }  \)
    \begin{align*}
      &\sum_{n=1}^{\infty} \left| a_n \right|
           && \text{Apply the \ulink{ss:Tests for Non-Positive Series}{Absolute convergence test}} \\
           &\then \sum_{n=1}^{\infty} \left|  \frac{(-1)^{n-1}}{\sqrt{n} }\right|
           = \sum_{n=1}^{\infty} \frac{1}{n^{\frac{1}{2}} }
    \end{align*}
    \(|a_n|\) diverges by the \(p\)-series, as \(\frac{1}{2} < 1\), meaning
    \(a_n\) \aset{converges conditionally} since \(|a_n|\) decreases monotonically and
    \(\lim_{n \to \infty} a_n = 0\)

  \item \(\displaystyle \sum_{n=1}^{\infty} \sin \frac{1}{n^2}  \)
    \begin{align*}
     L &= \lim_{n \to \infty} \frac{a_n}{b_n}, \quad b_n = \frac{1}{n^2}
       && \text{Apply the \ulink{ss:Tests for Positive Series}{limit comparison test}} \\
       &\then \lim_{n \to \infty} \frac{\sin(n^{-2})}{n^{-2}} =  \frac{0}{0} \\
       &= \lim_{n \to \infty} \frac{\cos (n^{-2})(-2n^{-3})}{-2n^{-3}}
       && \text{by \hopital's Rule} \\
       &= \lim_{n \to \infty} \cos (n^{-2}) = 1
    \end{align*}

    \(L > 0\), thus \(a_n\) converges if \(b_n\) converges.

    \(b_n\) converges by the \(p\)-series test, as \(2 > 1\), thus
    \(a_n\) \aset{converges}.


  \item \(\displaystyle \sum_{n=1}^{\infty} (-1)^n \cos n^{-1} \)

  \item \(\displaystyle \sum_{n=1}^{\infty} \frac{(-2)^n}{\sqrt{n} } \)

  \item \(\displaystyle \sum_{n=1}^{\infty} \left( \frac{n}{n+12} \right)^n \)

  \item \(\displaystyle \sum_{n=1}^{\infty} (-1)^n \cos n^{-1} \)


\end{enumerate}

