\section{Convergence: 10.3--10.5}
\begin{itemize}
  \item[]

  \subsection{Convergence Notes}
  \begin{itemize}
    \item \dd{\large Fundamentals}
    \begin{itemize}

     \item Let \(\displaystyle \sum_{n=1}^{\infty} a_n\) be given and note for
      which series for which convergence is known, i.e., the geometric series
      and \(p\)-series:
      \begin{multicols}{2}
        \dd{Geometric}: let \(c \neq 0\), if \(|r| < 1\), then
        \[%%%%%%%%%%
          \sum_{n=0}^{\infty} cr^n = \frac{c}{1-r}
        \]%%%%%%%%%%

        \dd{\(\bm{p}\)-Series}: converges if \(p > 1\).
        \[%%%%%%%%%%
          \sum_{n=1}^{\infty} \frac{1}{n^p}
        \]%%%%%%%%%%
      \end{multicols}

      \item \dd{The \(\bm{n}^{\bm{th}}\) Term Divergence Test}: a relatively easy
      test that can be used to quickly determine if a test diverges if the
      \(\lim_{n \to \infty} a_n \neq 0 \). If \(\lim_{n \to \infty} a_n = 0\),
      then the test is inconclusive and other tests must be applied.
      \end{itemize}

    \vspace{16pt}

    \item \dd{\large Tests for Positive Series}
      \begin{itemize}
        \item \dd{Direct Comparison Test}: use if dropping terms from the
          denominator or numerator gives a series \(b_n\) wherein convergence
          is easily found, then compare to the original series \(a_n\) as
          follows:
          \[%%%%%%%%%%
          \sum_{n=1}^{\infty} b_n ~\text{converges}~ \then
          \sum_{n=1}^{\infty} a_n ~\text{converges} \quad \given 0 \leq a_n
          \leq b_n
          \]%%%%%%%%%%

          \[%%%%%%%%%%
          \sum_{n=1}^{\infty} b_n ~\text{diverges}~ \then
          \sum_{n=1}^{\infty} a_n ~\text{diverges} \quad \given 0 \leq
          b_n \leq a_n
          \]%%%%%%%%%%

        \item \dd{Limit Comparison Test}: use when the direct comparison test
          isn't convenient or when comparing two series. One can to take the
          dominant term in the numerator and denominator from \(a_n\) to form
          a new positive sequence \(b_n\) if needed.

          Assuming the following limit \(L = \lim_{n \to \infty}
          \frac{a_n}{b_n}\) exists, then:
          \begin{align*}
            L &> 0 \then \sum_{n=1}^{\infty} a_n  ~\text{converges}~ \iff
            \sum_{n=1}^{\infty} b_n ~\text{converges}\\
            L &= 0 ~\text{and}~ \sum_{n=1}^{\infty} b_n ~\text{converges}~ \then
            \sum_{n=1}^{\infty} a_n ~\text{converges}\\
            L &= \infty ~\text{and}~ \sum_{n=1}^{\infty} a_n ~\text{converges}~
            \then \sum_{n=1}^{\infty} b_n ~\text{converges}
          \end{align*}

          %%%%%%%%%%%%%
          \newpage %%%%%%%%%%%%%%%%%%%%%%%%%%%%%%%%%%%%%%%%%%%%%%%%%%%%%%%%%%%%%%%%%%%%%%
          %%%%%%%%%%%%%

        \item \dd{Ratio Test}: often used in the presence of a factorial
          \dd{\((\bm{n!})\)} or when the are constants raised to the power of
          \(n\) \dd{\((\bm{c^n})\)}.

          Assuming the following limit \(\rho = \lim_{n \to \infty} \left|
          \frac{a_n + 1}{a_n} \right| \) exists, then
          \begin{align*}
            \rho &< 1 \then \sum a_n ~\text{converges absolutely} \\
            \rho &> 1 \then \sum a_n ~\text{diverges} \\
            \rho &= 1 \then ~\text{test is inconclusive}~
          \end{align*}

        \item \dd{Root Test}: used when there is a term in the form of
          \dd{\(\bm{f(n)^{\bm{g(n)}}}\)}.

          Assuming the following limit \(C = \lim_{n \to \infty}
          |a_n|^{\frac{1}{n}} \) exists, then
          \begin{align*}
            C &< 1 \then \sum a_n ~\text{converges absolutely} \\
            C &> 1 \then \sum a_n ~\text{diverges} \\
            C &= 1 \then ~\text{test is inconclusive}~
          \end{align*}

        \item \dd{Integral Test}: if the other tests fail and \(a_n = f(n)\) is
          a decreasing function, then one can use the improper integral
          \(\int_1^\infty f(x)dx\) to test for convergence.

          Let \(a_n = f(n)\) be a positive, decreasing, and continuous function
          \(\forall x \geq 1\), then:
          \begin{align*}
            \int_1^\infty f(x)dx ~\text{converges} &\then \sum_{n=1}^{\infty}
            a_n ~\text{converges}\\
            \int_1^\infty f(x)dx ~\text{diverges} &\then \sum_{n=1}^{\infty}
            a_n ~\text{diverges}
          \end{align*}
      \end{itemize}

    \vspace{16pt}

    \item \dd{\large Tests for Non-Positive Series}
      \begin{itemize}
        \item \dd{Alternating Series Test}: used for series in the form
          \(%%%%%%%%%%
          \displaystyle  \sum_{n=0}^{\infty} (-1)^n a_n
          \)%%%%%%%%%%

          Converges if \(|a_n|\) decreases monotonically \(\left(|a_n+1| \leq |a_n|\right)\)
          and if \(\lim_{n \to \infty} a_n = 0\)

          \vspace{16pt}

        \item \dd{Absolute Convergence}: used if the series \(\sum a_n\) is not
          alternating; simply test if \(\sum |a_n| \) converges using the test
          for positive series.

      \end{itemize}


  \end{itemize}

  %%%%%%%%%%%%%
  \newpage %%%%%%%%%%%%%%%%%%%%%%%%%%%%%%%%%%%%%%%%%%%%%%%%%%%%%%%%%%%%%%%%%%%%%%
  %%%%%%%%%%%%%

  \subsection{Convergence Problems}
  \begin{itemize}
    \item
  \end{itemize}

\end{itemize}
