\chapter{Quiz Questions}

\section{Quiz 3}
\begin{enumerate}
  \item Indicate whether the following statements are \textbf{True} or
    \textbf{False}, with justification.

  \begin{enumerate}[itemsep=12em]
    \item The series \(\displaystyle \sum_{n=1}^{\infty} (-1)^n \cos
      \left(\frac{1}{n}\right) \) converges.

      \false{\textbf{False:}}
      \begin{align*}
        \lim_{n \to \infty} a_n \stackrel{?}{=} 0, \quad a_n &= \cos \left(\frac{1}{n}\right)
        && \text{Apply the \ulink{ss:Tests for Positive Series}{\(n^{th} \) term test}} \\
        \then \lim_{n \to \infty} \cos\left( \frac{1}{n} \right) &= 1
      \end{align*}
      \(\lim_{n \to \infty} a_n \neq 0\), thus the series \aseg{diverges}.

    \item If the radius of converges of the power series \(\displaystyle
      \sum_{n=0}^{\infty} a_nx^n \) is \(R=5\), then the series must converge
      for \(x = -3\) and \(x = -4\).

      \true{\textbf{True:}}
      \begin{align*}
        c = 0, \quad R = 5 \then ~&\text{converges}~\forall x \in (-5, 5) \\
        &\text{By the \ulink{ss:Power Series}{Interval of convergence}}
      \end{align*}
      \(x = -3 \oldland x = 4 \in (-5, 5)\), thus the series \aset{must converge} at
      these values.
  \end{enumerate}

%%%%%%%%%%%%%
\newpage %%%%%%%%%%%%%%%%%%%%%%%%%%%%%%%%%%%%%%%%%%%%%%%%%%%%%%%%%%%%%%%%%%%%%%
%%%%%%%%%%%%%
  \item Determine whether the following series converge absolutely/conditionally, or diverge.
    \begin{enumerate}[itemsep=6em]
      \item \(\displaystyle \sum_{n=1}^{\infty} \frac{(-1)^{n+1}n }{2n+5}\)
        \begin{align*}
          \sum_{n=1}^{\infty} (-1)^n a_n, \quad a_n &= \frac{n}{2n+5}
          && \text{Apply the \ulink{ss:Tests for Non-Positive Series}{alternating series test}} \\
          \then \lim_{n \to \infty} \frac{n}{2n+5} &= \frac{\infty}{\infty} \\
          \then \lim_{n \to \infty} \frac{n}{2n+5} &= \frac{1}{2}
          && \text{By \hopital's Rule}
        \end{align*}
        \(\lim_{n \to \infty} a_n \neq 0 \), thus the series \aseg{diverges}

      \item \(\displaystyle \sum_{n=1}^{\infty} \frac{(-1)^n}{2\sqrt{n} - 1} \)
        \begin{align*}
          &\sum_{n=1}^{\infty} (-1)^n a_n, \quad a_n = \frac{1}{2\sqrt{n} - 1}
          && \text{Apply the \ulink{ss:Tests for Non-Positive Series}{alternating series test}} \\
          &\then \lim_{n \to \infty} \frac{1}{2\sqrt{n}-1} = 0 \\
          &\then a_n ~\text{converges}
          && ~\text{Note:}~|a_n| ~\text{decreases monotonically} \\\\
          &\sum_{n=1}^{\infty} \left|  \frac{(-1)^n}{2\sqrt{n} - 1} \right|
          \stackrel{?}{=} ~\text{converges}
          && \text{Apply the \ulink{ss:Tests for Non-Positive Series}{absolute convergence test}} \\
          &\then \lim_{n \to \infty}  \frac{1}{2\sqrt{n} - 1}
          && \lim_{n \to \infty} a_n = 0 \to n^{th}~\text{term inconclusive\ldots} \\
          L &= \lim_{n \to \infty} \frac{a_n}{b_n}, \quad b_n = \frac{1}{\sqrt{n} }
          && \text{Apply the \ulink{ss:Tests for Positive Series}{limit comparison test}} \\
          &= \lim_{n \to \infty}  \frac{1}{2\sqrt{n} - 1}
          \cdot \sqrt{n}\\
          &= \lim_{n \to \infty}  \frac{\sqrt{n} }{2\sqrt{n} - 1} \cdot \frac{n^{-\frac{1}{2}}}{n^{-\frac{1}{2}}} \\
          &= \lim_{n \to \infty}  \frac{1}{2 - n^{-\frac{1}{2}} } = \frac{1}{2}
        \end{align*}
        \(L > 0\), and \(b_n\) diverges by the \(p\)-series, implying the \(|a_n|\)
        diverges. Thus, the original series \aset{converges conditionally}.
    \end{enumerate}

%%%%%%%%%%%%%
\newpage %%%%%%%%%%%%%%%%%%%%%%%%%%%%%%%%%%%%%%%%%%%%%%%%%%%%%%%%%%%%%%%%%%%%%%
%%%%%%%%%%%%%

  \item Find a power series expansion with the center \(c = 0 \) for
     \[%%%%%%%%%%
     f(x) = \frac{1}{1+x^3}
     \]%%%%%%%%%%
     and find the interval of convergence. Hint: use \(\displaystyle \frac{1}{1-x} = \sum_{n=0}^{\infty} x^n \quad \given |x| < 1 \)
     \begin{align*}
       \then \frac{1}{1+x^3} &= \frac{1}{1 - (-x^3)} \\
       &= \sum_{n=1}^{\infty} (-x^3)^n = \sum_{n=1}^{\infty} (-1)^nx^{3n}
       && \text{Apply hint} \\
       \then \frac{1}{1+x^3} &= \sum_{n=1}^{\infty} (-1)^n x^{3n} \qquad \given |x| < 1
     \end{align*}
     Thus, the interval of convergence is all values in the interval
     \aset{\((-1, 1)\)}.

  \vspace{4em}

  \item Find the radius of convergence of the power series given by
    \[%%%%%%%%%%
    \sum_{n=1}^{\infty} \frac{(-1)^n x^{2n+1} }{2^n n}
    \]%%%%%%%%%%
    \begin{align*}
      \rho &= \lim_{n \to \infty} \left| \frac{a_{n+1}}{a_n}  \right|
           && \text{Apply the \ulink{ss:Tests for Non-Positive Series}{ratio test}} \\
           &\then \lim_{n \to \infty}
           \left| \frac{(-1)^{n+1} x^{2(n+1)+1}}{2^{n+1}(n+1)}
           \cdot \frac{2^n n}{(-1)^n x^{2n+1} }
           \right| \\
           &= \lim_{n \to \infty} \frac{|x|^{2n+3}}{2^{2n}(n+1)}
           \cdot \frac{2^n n}{|x|^{2n+1} } \\
           &= \lim_{n \to \infty} \frac{|x|^{2n+1} \cdot |x|^2}{2^n \cdot 2(n+1)}
           \cdot \frac{2^n n}{|x|^{2n+1} } \\
           &= |x|^2 \lim_{n \to \infty} \frac{n}{2n+2} = \frac{\infty}{\infty}\\
           &= |x|^2 \lim_{n \to \infty} \frac{1}{2}
           && \text{By \hopital's Rule} \\
           &\then \frac{|x|^2}{2} < 1
           && \text{converges when}~\rho < 1 \\
           &= |x| < \sqrt{2}
    \end{align*}
    Thus, the interval of convergence is \aset{\((-\sqrt{2}, \sqrt{2})\)} with \(R = \sqrt{2}, ~\text{and}~c = 0\)
    \prn{endpoints not required to be tested for this problem}.

\end{enumerate}

\section{Quiz 4}
\begin{enumerate}
  \item Indicate whether the following statements are \textbf{True} or
    \textbf{False}, with justification.

  \begin{enumerate}[itemsep=22em]
    \item The curve with parametric representations \(c(t) = (4+3\cos t, 5 + 3
      \sin t)\) is a circle with radius \(R= 3\) centered art the origin.

    \item The parametric representation given by \(c(t) = (\sin t, t)\) can be
      represented by function of the form \(y=f(x)\).
  \end{enumerate}

%%%%%%%%%%%%%
\newpage %%%%%%%%%%%%%%%%%%%%%%%%%%%%%%%%%%%%%%%%%%%%%%%%%%%%%%%%%%%%%%%%%%%%%%
%%%%%%%%%%%%%
  \item Determine whether the following series converge of diverge, with
    justification.
    \begin{enumerate}[itemsep=22em]
      \item \(\displaystyle \sum_{n=1}^{\infty} \frac{n^3}{n!} \)

      \item \(\displaystyle \sum_{n=0}^{\infty} \left( \frac{n}{3n+1} \right)^n  \)
    \end{enumerate}

%%%%%%%%%%%%%
\newpage %%%%%%%%%%%%%%%%%%%%%%%%%%%%%%%%%%%%%%%%%%%%%%%%%%%%%%%%%%%%%%%%%%%%%%
%%%%%%%%%%%%%

  \item Find the Maclaurin series of (using substitution and/or multiplication)
    \[f(x) = x \cos (x^2)\]

%%%%%%%%%%%%%
\newpage %%%%%%%%%%%%%%%%%%%%%%%%%%%%%%%%%%%%%%%%%%%%%%%%%%%%%%%%%%%%%%%%%%%%%%
%%%%%%%%%%%%%

  \item Express the following integral as a infinite series, first by finding the
    Maclaurin series of the integrand, then integrating this series.
    \[%%%%%%%%%%
    \int_{0}^{1} e^{-x^2}  dx
    \]%%%%%%%%%%

%%%%%%%%%%%%%
\newpage %%%%%%%%%%%%%%%%%%%%%%%%%%%%%%%%%%%%%%%%%%%%%%%%%%%%%%%%%%%%%%%%%%%%%%
%%%%%%%%%%%%%

  \item Consider the curve with parametric representation \[c(t) = (\sin 2t +
    \cos t, \cos 2t - \sin t)\]
    Find an equation of the tangent line at \(t=\pi\)

\end{enumerate}
