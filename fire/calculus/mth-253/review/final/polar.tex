\chapter{Arc Length, Polar Coordinates: 11.2--11.4}
\section{11.2--11.4 Notes}

\subsection{Arc Length and Speed}
\begin{itemize}
  \item \dd{Arc Length of \(\mathcal{C}\)}: valid if \( c(t) = (x(t), y(t))\)
    directly traverses \(\mathcal{C}\) for \(a \leq t \leq b\), then
    \[%%%%%%%%%%
      s = \int_{a}^{b} \sqrt{x'(t)^2 + y'(t)^2} dt
    \]%%%%%%%%%%
    \begin{itemize}
      \item Can be interpreted as the \dd{distance traveled} along the
        path from \(t = a \to b\)
      \item \dd{Displacement}: less than or equal to the distance traveled;
        simply the distance from starting point \(c(a)\) to endpoint \(c(b)\).
      \item Distance traveled as as \dd{function of \(t\)}, starting at \(t_0\):
        \[%%%%%%%%%%
          s(t) = \int_{t_0}^{t_1}  \sqrt{x'(u)^2 + y'(u)^2} du
        \]%%%%%%%%%%
    \end{itemize}

  \item \dd{Speed} at time \(t\):
    \[%%%%%%%%%%
      \frac{ds}{dt} = \sqrt{ x'(t)^2 +y'(t)^2}
    \]%%%%%%%%%%

  \item \dd{Surface area}: obtained via rotation of the parametric equation
    about the \(x\)-axis for \(a \leq t \leq b\), given \(y(t) \geq 0, x(t)\)
    is increasing, and \(x'(t) \oldland y'(t)\) are continuous:
    \[%%%%%%%%%%
      S = 2\pi \int_{a}^{b}  y(t) \sqrt{x'(t)^2 + y'(t)^2}  dt
    \]%%%%%%%%%%
\end{itemize}

%%%%%%%%%%%%%
\newpage %%%%%%%%%%%%%%%%%%%%%%%%%%%%%%%%%%%%%%%%%%%%%%%%%%%%%%%%%%%%%%%%%%%%%%
%%%%%%%%%%%%%

\subsection{Polar Coordinates}
\begin{itemize}
  \item \dd{Polar coordinate system}: a two-dimensional coordinate system
    wherein each point is determined by the distance and angle from a reference
    point and direction.
    \begin{itemize}
      \item \dd{Radial coordinate, \(r\)}: the distance from reference point.
      \item \dd{Angular coordinate, \(\theta\)}: the angle from reference direction.
      \item A point \(P\) has polar coordinates \((r, \theta)\) with the angle
        measured in the counterclockwise direction by convention.
    \end{itemize}

  \item \dd{Conversion between polar and rectangular coordinates}:
    \begin{align*}
      x &= r \cos \theta \qquad
      y = r \sin \theta \qquad
      r = \sqrt{x^2 + y^2} \qquad
      \tan \theta = \frac{y}{x} ~ \given  x\neq 0
    \end{align*}

  \item \dd{If \(\bm{r > 0}\) then}: \((r,\theta)\) must lie in quadrant I or IV;
    \[%%%%%%%%%%
    \theta =
      \begin{cases}
        \tan ^{-1} \frac{y}{x} & \given x > 0 \\
        \tan ^{-1} \frac{y}{x} + \pi & \given x < 0 \\
        \oldpm \frac{\pi}{2} &\given x = 0
      \end{cases}
    \]%%%%%%%%%%

  \item \dd{Non-uniqueness}: Multiple representations can represent the same point, i.e.,
    \[%%%%%%%%%%
      (r, \theta) \equiv (r, \theta + 2 n \pi) \equiv (-r, \theta + (2n +1)\pi) \qquad \given n \in \mathbb{Z}
    \]%%%%%%%%%%

  \item \dd{Polar Equations}:
    \begin{table}[h]
      \centering
      \begin{tabular}{lc}
        \toprule
         \dd{Curve} & \dd{Polar Equation} \\
        \midrule
         Circle of radius \(R\), center at origin
                    & \(r = R\) \\\\
         Line through origin slope \(m = \tan\theta_0\)
                    & \(\theta = \theta_0\)  \\\\
         Line, where \(P_0 = (d, \alpha) \) is closest to the origin
                    & \( r = d \sec(\theta - \alpha)\)\\ \\
        Circle radius \(a\), center at \((a,0)\)
                    & \(r = 2a \cos \theta\) \\
         \((x-a)^2 + y^2 = a^2\) \\\\
        Circle radius \(a\), center at \((0,a)\)
                    & \(r = 2a \sin \theta\) \\
         \(x^2 + (y-a)^2 = a^2\) \\
        \bottomrule
       \end{tabular}
     \end{table}
 \end{itemize}

%%%%%%%%%%%%%
\newpage %%%%%%%%%%%%%%%%%%%%%%%%%%%%%%%%%%%%%%%%%%%%%%%%%%%%%%%%%%%%%%%%%%%%%%
%%%%%%%%%%%%%

\subsection{Area and Arc Length in Polar Coordinates}
\begin{itemize}
  \item \dd{Area in Polar Coordinates}: given that \(f\) is continuous, then
    the sector is bounded by:
    \begin{itemize}
      \item \dd{Polar curve, \(\bm{r}\)}: \(r = f(\theta)\)
      \item \dd{Two rays, \(\bm{\alpha, \beta}\)}: where each ray is an angle
        \(\theta\) with \(\alpha < \beta, \quad \beta = \theta - \alpha\)
      \item Thus, the area is equal to the integral between \(\alpha\) and \(\beta\), i.e.
        \[%%%%%%%%%%
         A = \frac{1}{2} \int_{\alpha}^{\beta} f(\theta)^2 d\theta
        \]%%%%%%%%%%
    \end{itemize}

  \item \dd{Arc length of polar curve}: given \(\alpha \leq \theta \leq \beta\):
    \[%%%%%%%%%%
      s = \int_{\alpha}^{\beta} \sqrt{f(\theta)^2 + f'(\theta)^2}  d\theta
    \]%%%%%%%%%%

\end{itemize}

\section{Polar Coordinate Problems}

\subsection{11.2 Exercises}
\begin{enumerate}[itemsep=3em]
  \item
\end{enumerate}

%%%%%%%%%%%%%
\newpage %%%%%%%%%%%%%%%%%%%%%%%%%%%%%%%%%%%%%%%%%%%%%%%%%%%%%%%%%%%%%%%%%%%%%%
%%%%%%%%%%%%%

\subsection{11.3 Exercises}
\begin{enumerate}[itemsep=3em]
  \item
\end{enumerate}

%%%%%%%%%%%%%
\newpage %%%%%%%%%%%%%%%%%%%%%%%%%%%%%%%%%%%%%%%%%%%%%%%%%%%%%%%%%%%%%%%%%%%%%%
%%%%%%%%%%%%%

\subsection{11.4 Exercises}
\begin{enumerate}[itemsep=3em]
  \item
\end{enumerate}
