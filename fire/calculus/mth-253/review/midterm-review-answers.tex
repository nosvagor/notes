\documentclass{nosvagor-notes}
\usepackage{nosvagor-math}
\usepackage{nosvagor-commands}
\usetikzlibrary{calc}
\newcommand*{\TickSize}{2pt}%

\colorlet{title-color}{red}
\newcommand{\theTitle}{%
  \href{https://github.com/nosvagor/notes/tree/main/fire/calculus}%
  {Calculus III Midterm Review (\RR{Solutions})}%
}

\begin{document}

\begin{enumerate}
  \item True or false? (with justification)
  \begin{enumerate}
    \item If \({a_n}\) is bounded, then it converges.

      \false{False}; sequences can oscillate between values, yielding a bounded sequence that does not converge to a consistent value. E.g., \(a_n = (-1)^{n}\)
      \vspace{90pt}

    \item If \({a_n}\) converges, then is must be bounded.

      \true{True}; by Convergent Sequences theorem, i.e., if
      \({a_n}\) converges, then \({a_n} \) is bounded.
      \vspace{90pt}

    \item If \({a_n}\) is not bounded, then it diverges.

      \true{True};  inverse of (b); sequences must be bounded in order to
      converge, if they aren't then they diverge.
      \vspace{90pt}

    \item If \({a_n}\) diverges, then it is not bounded.

      \false{False};  inverse of (a); bounded sequences can diverge. Another example is \(a_n = \cos \pi n\)
      \vspace{30pt}
  \end{enumerate}

  \newpage

 \item A hot anvil with cooling constant \(k = \SI{0.02}{\per\second}\) is
   submerged in a large pool of water whose temperature is \SI{10}{\celsius}.
   Let \(y(t)\) be the anvil’s temperature \(t\) seconds later.
   \begin{enumerate}
     \item What is the differential equation satisfied by \(y(t)?\)
       \[%%%%%%%%%%
         \text{Newton's Law of Cooling:}~ \aset{y' = -k(y - T_0)}
       \]%%%%%%%%%%
       \[%%%%%%%%%%
        y(t) = ~\text{temp. Of object}, \quad T_0 = ~\text{ambient temp.}, \quad k = ~\text{cooling constant}.
       \]%%%%%%%%%%

       \[%%%%%%%%%%
         k = 0.02, \qquad T_0 = 10 \then \boxed{y' = -0.02(y - 10)}
       \]%%%%%%%%%%
       \vspace{90pt}

     \item Find a formula for \(y(t)\), assuming the object initial temperature
       is \SI{100}{\celsius}.
       \[%%%%%%%%%%
        \text{General Solution of}~y' = -k(y - b) \to \aset{y = b + Ce^{-kt}}
       \]%%%%%%%%%%
       \begin{align*}
         b = 10, \quad k = -0.02, &\then y = 10 + Ce^{-0.02 t} \\
         P(0) = 100, &\then 100 = 10 + Ce^{0} \\
                     &\then C = 90 \\
                     &\then \boxed{y = 10 + 90e^{-0.02t}}
       \end{align*}
       \vspace{60pt}
   \end{enumerate}

  \item As an object cools, its rate of cooing slows. Explain how this follows
    from Newton's Law of Cooling.

    \vspace{16pt}
    \begin{itemize}
      \item The difference in temperature between a cooling object and the ambient
        temperature is decreasing. Hence, the rate of cooling, which is
        proportional to this difference, is also decreasing in magnitude.

    \end{itemize}
    \newpage

  \item In Yellowstone park there were approximately 500 bison in 1970 and
    3,000 bison in 1990.
    \begin{enumerate}
      \item Using the model that the rate of change of the population is
        proportional to the population itself, set up (do not solve) an initial
        value problem to model this situation.
        \begin{align*}
          ~\text{Given: }~
          P(t) &= ~\text{population of bison at time}~ t \\
          1970 &\to t = 0 \\
          P' &\propto P \\
             &\then \boxed{P' = kP \quad
             P(0) = 500 \land P(20) = 3,000}
        \end{align*}
        \vspace{30pt}

      \item Now find the particular solution satisfying your initial value
        problem.
        \begin{align*}
          ~\text{Given:}~ P'&=kP, \quad y = Ce^{kt} ~(\text{general solution to}~y'=ky)\\
                   &\then P = Ce^{kt} \\
          P(0) = 500 &\then 500 = Ce^{0} \\
                     &\then C = 500, \quad P = 500e^{kt} \\\\
          P(3,000) = 500 e^{20k} &\then k = (\ln \frac{3000/500}{20}) \approx 0.0896 \\
                     &\then \boxed{P = 500e^{0.0896t}}
        \end{align*}
        \vspace{30pt}
    \end{enumerate}

  \item Which of the following are first-order linear equations?
    \begin{multicols}{2}
    \begin{enumerate}
      \item \(y'+ x^2y = 1\) \true{true}
      \item \(y' + xy^2 = 1\) \false{false}
      \item \(x^5y' + y = e^x\) \true{true}
      \item \(x^5y' + y = e^y\) \false{false}
    \end{enumerate}
    \end{multicols}
    \vspace{40pt}

  \item For what function \(P\) is the integrating factor \(\alpha (x) \) equal
    to \(x?\) What about \(e^x\)?
    \begin{align*}
      \Aboxed{P(x) &= x^{-1}} \to \alpha (x) = x \\\\
      \Aboxed{P(x) &= 1} \to \alpha (x) = e^x
    \end{align*}

  \newpage

  \item Find the limit of the sequence \(a_n = \dfrac{n+1}{3n+2}\). Be sure to
    justify your answer.

    \begin{align*}
      ~\text{Using the Limit Laws:} \\
      \lim_{n \to \infty}\frac{n+1}{3n+1} =
      \lim_{x \to \infty} \frac{x+1}{3x+1} &=
      \lim_{x \to \infty} \frac{x+1}{3x+1} \cdot \frac{x^{-1}}{x^{-1}} \\
      &= \lim_{x \to \infty} \frac{1+x^{-1}}{3+x^{-1}} \\
      &= \lim_{x \to \infty} \frac{1}{3} + \lim_{x \to \infty} \frac{x^{-1}}{x^{-1}} \\
      &= \frac{1}{3} + \frac{0}{0} = \boxed{\frac{1}{3}}
    \end{align*}

    \begin{align*}
      ~\text{Using \hopital Rule:}\\
      f(x) &= n+1, \quad g(x) = 3n+2 \\\\
      \lim_{x \to \infty} \frac{f(x)}{g(x)} &= \frac{\infty}{\infty} \\
      &\then \frac{f'(x)}{g'(x)} = \boxed{\frac{1}{3}}
    \end{align*}

    \vspace{30pt}

  \item Which of the following sequences converge to zero?
    \[%%%%%%%%%%
      \frac{n^2}{n^2+1} \qquad\qquad 2^n \qquad\qquad \boxed{\left( -\frac{1}{2} \right) ^n}
    \]%%%%%%%%%%
    \vspace{10pt}

  \item Which of the following sequences is defined recursively?
    \[%%%%%%%%%%
      a_n = \sqrt{4+n} \qquad \qquad \boxed{b_n = \sqrt{4+b_{n-1}}}
    \]%%%%%%%%%%
    \vspace{10pt}

  \item Let \(a_n\) be the \(n^{th}\) decimal approximation to \(\sqrt{2} \).
    I.e., \(a_1 = 1, a_2 = 1.4, a_3 = 1.41\) and so on. What is \(\lim_{n \to
    \infty} a_n\)?

    \[\boxed{\lim_{n \to \infty} a_n = \sqrt{2} }\]
  \newpage

  \item Biologists stocked a lake with 400 fish and estimated the carrying
    capacity to be 10,000. The number of fish tripled in the first year.
    Assuming the size of the fish population satisfies the logistic equation,
    find an expression for the size of the population after \(t\) years.
    \vspace{10pt}

    \begin{itemize}
      \item The \aset{logistic equation} and it general non-equilibrium solution
    \((k>0 \land A > 0)\)
    \[%%%%%%%%%%
      \frac{dy}{dt} = ky(1-\frac{y}{A}) \to y = \frac{A}{1-e^{-kt}/B} \equiv \frac{y}{y-A} = Be^{kt}
    \]%%%%%%%%%%
    \begin{align*}
      ~\text{Given:}\\
      y(0) &= 400, \quad y(1) = 1200, \quad A = 10,000 \\\\
      \frac{400}{400-10,000} &= Be^0 \\
      B &= -\frac{1}{24} \\
        &\then y = \frac{10,000}{1-e^{-kt}/-\frac{1}{24}} =
        \frac{10,000}{1+24e^{-kt}}\\\\
      1,200 &= \frac{10,000}{1+24e^{-k}} \\
      1+24e^{-k} &= \frac{10,000}{1,200} \\
    e^{-k} &= \frac{22}{3} \cdot \frac{1}{24} = \frac{2\cdot 11}{3\cdot 2 \cdot 12} = \frac{11}{36}\\
    k &= -\ln \left(\frac{11}{36}\right) \approx 1.186 \\\\
      &\then \boxed{y = \frac{10,000}{1+24e^{-1.186 t}}}
    \end{align*}
    \end{itemize}


  \newpage

  \item Find the general term, \(a_n\), for the sequence given below. Assume that
  we start our sequence at \(n=1\).
  \[%%%%%%%%%%
    1,\frac{4}{2}, \frac{9}{4}, \frac{16}{8}, \frac{25}{16}, \frac{36}{32}\ldots
  \]%%%%%%%%%%
  \begin{align*}
    &\then
    \frac{1^2}{2^{1-1}},
    \frac{2^2}{2^{2-1}},
    \frac{3^2}{2^{3-1}},
    \frac{4^2}{2^{4-1}},
    \frac{5^2}{2^{5-1}},
    \frac{6^2}{2^{6-1}}, \cdots, \frac{n^2}{2^{n-1}} \\
    &\then \boxed{ a_n = \frac{n^2}{2^{n-1}}}
  \end{align*}
  \vspace{20pt}

  \item Determine whether each of the following series converge or diverge by
    using either the Comparison Test or the Limit Comparison Test to justify
    your answer.
    \begin{enumerate}
      \item \(\displaystyle\sum_{n=1}^{\infty} \frac{1}{e^n + n^2}\)
        \begin{align*}
          e^n + n^2 \geq n^2 \\
          \frac{1}{e^n + n^2} \leq \frac{1}{n^2} \\\\
          a_n = \frac{1}{e^n + n^2}, \quad b_n = \frac{1}{n^2}
        \end{align*}
      Using direct compassion test: if \(b_n\) converges, then \(a_n\)
      converges. We know \(\frac{1}{n^2}\) converges by the \(p\)-series, since
      \(p > 1\). \aset{Thus, \(a_n\) converges}.
      \vspace{30pt}
      \item \(\displaystyle\sum_{n=1}^{\infty} \frac{1}{e^n - n^2}\)
      \begin{align*}
        ~\text{Let}~ a_n = \frac{1}{e^n - n^2}, \quad b_n &= \frac{1}{e^n} \\\\
        \then \lim_{n \to \infty} \frac{a_n}{b_n} &= \frac{e^n}{e^n-n^2}\\
           &= \lim_{n \to \infty} \frac{e^n}{e^n-n^2} \cdot \frac{e^{-n}}{e^{-n}} \\
           &= \frac{1}{1-n^2 e^{-n}} = 1
      \end{align*}
      Thus, since \(L > 0\), and \(\sum b_n\) converges, then \aset{\(\sum
      a_n\) converges} by the Limit Comparison Test.
    \end{enumerate}

  \newpage

  \item True or false? If \(k > 0\), then all solutions of \(y' = -k(y-b)\)
  approach the same limit as \(t \to \infty\).

  \true{True}; \(k > 0\) will yield a general solution in the form of \(y = b +
  Ce^{kt}\). Thus, all solutions will approach the same limit as \(t \to \infty\).
  \vspace{30pt}

  \item Write a solution to \(y' = 4(y-5)\) that tends to \(-\infty\) as \(t\to
  \infty\).

  \[\boxed{y(t) = 5 - Ce^{4t}, \quad c > 0}\]
  \vspace{30pt}

  \item Does \(y'=-4(y-5)\) have a solution that tends to \(\infty\) and \(t \to
  \infty\)?

  \false{No}; any solution eventually tend to \(-\infty\) with \(k = -4\) as \(t \to \infty\).
  \vspace{30pt}

  \item Find the general solution \(y' = xe^{-\sin x} - y \cos x\).
  \begin{align*}
    y' + P(x)y &= Q(x)
      && \text{First-order linear DEQ} \\
    \alpha (x) &= e^{\int P(x)dx}
      && \text{Integrating factor} \\
    y &= \alpha (x)^{-1} \left( \int \alpha (x) Q (x) dx + C \right)
      && \text{General solution} \\\\
    y' + \cos (x) y &= xe^{- \sin x}
      && \text{Rewrite equation} \\
    \alpha (x) &= e^{\int \cos x dx } = e^{\sin x}
      && \text{Find integrating factor} \\\\
    y &= e^{-\sin  x} \left( \int e^{\sin x} xe^{-\sin x} dx + C \right)
      && \text{Plug in}~\alpha (x)  \\
    \Aboxed{y &=  e^{-\sin x}\left( \frac{x^2}{2} + C \right)}
      && \text{Simplify, integrate} \\
  \end{align*}

  \newpage

  \item True or false?
  \begin{enumerate}
    \item \(t \frac{dy}{dt} = 3\sqrt{1+y} \) is a separable differential
      equation.

    \true{True};
    \begin{align*}
      t \frac{dy}{dt} &= 3\sqrt{1+y} \\
    \then \frac{dy}{dt} &= \frac{3\sqrt{1+y}}{t} = 3t^{-1} \cdot \sqrt{1+y}
    \end{align*}
    \item \(yy' + x + y = 0\) is a first-order linear differential equation.

    \false{False}; \(yy'\) makes this equation nonlinear. Note: it is still
      first order.
  \end{enumerate}
  \vspace{80pt}

  \item Determine the order of the following differential equations:
    \begin{multicols}{2}
      \begin{enumerate}
        \item \(x^5y'=1 \to\) \aset{1st}
        \item \((y')^3+x=1 \to \) \aset{1st}
        \item \(y''' + x^4y' = 2 \to\) \aset{3rd}
        \item \(\sin (y'') + x = y \to \) \aset{2nd}
      \end{enumerate}
    \end{multicols}
    \vspace{80pt}

  \item Which of the following differential equations are directly integrable?
  \begin{multicols}{2}
    \begin{enumerate}
      \item \(y'=x+y\) \false{false}
      \item \(x \dfrac{dy}{dx} = 3\) \true{true}
      \item \(\dfrac{dP}{dt} = 4P + 1\) \false{false}
      \item \(\dfrac{dw}{dt} = \dfrac{2t}{1+4t}\) \true{true}
      \item \(\dfrac{dx}{dt} = t^2 e^{-3t}\) \true{true}
      \item \(t^2 \dfrac{dx}{dt} = x - 1\) \false{false}
    \end{enumerate}
  \end{multicols}

  Note: directly integrable differential equation is in the form \(\frac{dy}{dx} = f(x)\).

  \newpage

  \item Which of the following differential equations are separable?
  \begin{multicols}{2}
    \begin{enumerate}
      \item \(\dfrac{dy}{dx} = x - 2y\) \false{false}
      \item \(xy' + 8ye^x = 0\) \true{true}
      \item \(y' = x^2y^2\) \true{true}
      \item \(y' = 1-y^2\) \true{true}
      \item \(t \dfrac{dy}{dt} = 3 \sqrt{1+y} \) \true{true}
      \item \(\dfrac{dP}{dt} = \dfrac{P+t}{t}\) \false{false}
    \end{enumerate}
  \end{multicols}
  \vspace{256pt}

  \item Which of the following equations are first-order?
    \begin{multicols}{2}
      \begin{enumerate}
        \item \(y' = x^2 \) \true{true}
        \item \(y'' = y^2\) \false{false}
        \item \((y')^3 + yy' = \sin x\) \true{true}
        \item \(x^2y' - e^xy = \sin y\) \true{true}
        \item \(y'' + 3y' = \dfrac{y}{x}\) \false{false}
        \item \(yy' + x + y = 0\) \true{true}
      \end{enumerate}
    \end{multicols}

  \newpage

  \item Water is draining from a cylindrical tank with cross sectional area
  \(4m^2\) and height \(5m\). Torricelli’s  Law says that the rate of change of the
  height of the water in such a cylindrical tank is proportional to the square
  root of the height of the water in the tank. Suppose that the tank starts
  full of water and after 30 minutes the height of the water has decreased to
  \(4m\). Set up an initial value problem to model this situation.

  \vspace{30pt}
  Let \(h(t)\) be the height of the water (meters) at time \(t\) (minutes), Then
  \begin{align*}
    \frac{dy}{dt} &= k\sqrt{y}, \quad y(0) = 5, \quad y(30)= 4 \\
    &\then \int \frac{dy}{\sqrt{y} } = k\int dt \\
    2\sqrt{y} &= kt + C \\
    &\then 2\sqrt{5} = C \\\\
    \then 2\sqrt{30} &= k(4) + 2\sqrt{5} \\
    4k &= 2\sqrt{30}-2\sqrt{5} \\
    k &= \frac{\sqrt{30} - \sqrt{5}}{2} \approx 1.621 \\\\
    \then 2\sqrt{y} &= 1.621t + 2\sqrt{5} \\
       \sqrt{y} &= 0.8105t + \sqrt{5} \\
       \Aboxed{y &= (0.8105t)^2 + 5}
  \end{align*}

  \newpage

  \item Find the limit of each of the following sequences. Justify your answer
    using limit laws, the squeeze theorem, and/or \hopital's Rule.
    \begin{enumerate}
      \item \(\displaystyle a_n = \frac{2(-1)^{n+1}}{2n-1}\)
      \begin{align*}
        \forall n \geq 1,
        \qquad \BB{-\frac{2}{2n-1}} \leq
        \frac{2(-1)^{n+1}}{2n-1} \leq
        \RR{\frac{2}{2n-1}} \\\\
        \lim_{n \to \infty} \BB{-\frac{2}{2n-1}} =
        \lim_{n \to \infty} \RR{\frac{2}{2n-1}} &= 0 \\\\
        \then \lim_{n \to \infty} \frac{2(-1)^{n+1}}{2n-1} = 0, \quad \text{By the squeeze theorem} \\
      \end{align*}

    \vspace{90pt}
      \item \(\displaystyle a_n = \frac{n^2}{2^{n-1}}\)
        \begin{align*}
          \lim_{n \to \infty} \frac{n^2}{2^{n-1}} &= \frac{\infty}{\infty} \\
          &\then \lim_{n \to \infty} \frac{2n}{\ln(2)2^{n-1}}
          && \text{by \hopital's Rule} \\
          &\then \lim_{n \to \infty} \frac{2}{\ln(2)^2 2^{n-1}}
          && \text{by \hopital's Rule} \\
          &\then \boxed{a_n = 0}
        \end{align*}
    \end{enumerate}

  \newpage

  \item Does the series \(\displaystyle \sum_{n=1}^{\infty} \cos \left(
    \frac{1}{n} \right) \) converge or diverge? How do you know?

    \FF{Diverge};
    \[%%%%%%%%%%
      \lim_{n \to \infty} \cos n^{-1} = \cos \left(\lim_{n \to \infty} n^{-1}\right) = \cos 0 = 1
    \]%%%%%%%%%%
    By the \(n^{th}\) Term Divergence test, i.e., if \(\lim_{n \to \infty} a_n
    \neq 0\), then \(\sum_{n=1}^{\infty} a_n\) diverges.

    Note: a series may still diverge even if \(a_n\) tends to zero.
    \vspace{80pt}

  \item Find the sum of each of the following geometric series or state that it
    diverges. Be sure to explain how you arrived at your solution.
    \begin{enumerate}
      \item \(\dfrac{1}{6} + \dfrac{1}{12} + \dfrac{1}{24} + \dfrac{1}{48} + \dfrac{1}{96} + \cdots\)

      \begin{align*}
        \frac{1}{6}\left(1+\frac{1}{2}+\frac{1}{4}+\frac{1}{8}\right)
        && \text{Factor out ratio between terms} \\
        \then c = \frac{1}{6}, \quad r = \frac{1}{2}
        && |r| < 1, \text{thus the series converges} \\
        \sum_{n=1}^{\infty} cr^n = \frac{c}{1-r}
        && \text{Sum of a Geometric Series} \\
        \then \frac{\frac{1}{6}}{1-\frac{1}{2}} = \boxed{\frac{1}{3}}
      \end{align*}
      \vspace{60pt}

      \item \(-2 + \dfrac{2}{5} - \dfrac{2}{25} + \dfrac{2}{125} - \dfrac{2}{625} + \cdots \)
      \begin{align*}
        -2\left(1-\frac{1}{5}+\frac{1}{25}-\frac{1}{125}+\frac{1}{625}\right)
        && \text{Factor out ratio between terms} \\
        \then c = 2, \quad r = -\frac{1}{5}
        && |r| < 1, \text{thus the series converges} \\
        \then \frac{2}{1-\frac{-1}{5}} = \boxed{-\frac{5}{3}}
      \end{align*}
    \end{enumerate}
  \newpage

  \item What role do partial sums play in defining the sum of an infinite
    series?

    \vspace{16pt}
    \begin{itemize}
      \item The sum of an infinite series is defined as the limit of the
        sequence of partial sums. If the limit of this sequence does not exist,
        the series is said to diverge.
    \end{itemize}

    \vspace{30pt}

  \item Indicate whether of not the reasoning in the following statements are
    correct:
    \begin{enumerate}
      \item \(\displaystyle \sum_{n=1}^{\infty} \frac{1}{n^2} = 0\) because
        \(\frac{1}{n^2}\) tends to zero.

      \vspace{16pt}
      \false{False}; the infinite sum still tends towards infinity. The series
      does converge to zero, however.

      \vspace{30pt}
      \item \(\displaystyle \sum_{n=1}^{\infty} \frac{1}{\sqrt{n}}\) because
        \(\displaystyle \lim_{n \to \infty}\frac{1}{\sqrt{n} } = 0\)

      \vspace{16pt}
      \false{False}; again, same reasoning, the infinite sum will still tend
      toward infinity. This series does not converge, however.

    \end{enumerate}
    \vspace{60pt}

  \item Find an \(N\) such that \(S_N > 25\) for the series \(\displaystyle
    \sum_{n=1}^{\infty} 2\).

    \[%%%%%%%%%%
    \frac{25}{2} = 12.5, ~\text{thus,}~ N = 13
    \]%%%%%%%%%%
    \vspace{60pt}

  \item Does there exist an \(N\) such that \(S_N > 25\) for the series
    \(\displaystyle \sum_{n=1}^{\infty} 2^{-n}\)? Explain.

    \vspace{16pt}
    \false{No}; the series converges to 1 and is increasing, thus \(S_N \leq
    1\) for all \(N\).

  \newpage

  \item For the series \(\displaystyle \sum_{n=1}^{\infty} a_n\), if the
    partial sums \(S_N\) are increasing, then (choose the correct conclusion):
    \begin{enumerate}
      \item \(\{a_n\}\) is an increasing sequence. \false{false}
      \item \(\{a_n\}\) is a positive sequence. \true{true}
    \end{enumerate}
  \vspace{60pt}

  \item Which test would you use to determine whether the following series converge?
  \begin{enumerate}
    \item \(\displaystyle \sum_{n=1}^{\infty} n^{-3.2}\)
      \vspace{16pt}

      \(p\)-Series \(\left( \frac{1}{n^{3.2}} \right) \), or the Integral Test.

      \vspace{60pt}

    \item \(\displaystyle \sum_{n=1}^{\infty} \frac{1}{2^n + \sqrt{n} }\)

      \vspace{16pt}
      Direct Comparison Test; as it is easy to choose \(a_n\) and \(b_n\)
  \end{enumerate}

\end{enumerate}
\end{document}
