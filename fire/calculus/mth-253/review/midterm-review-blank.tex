\documentclass{nosvagor-notes}
\usepackage{nosvagor-math}
\usepackage{nosvagor-commands}
\usetikzlibrary{calc}
\newcommand*{\TickSize}{2pt}%

\colorlet{title-color}{black!85!white}
\newcommand{\theTitle}{%
  \href{https://github.com/nosvagor/notes/tree/main/fire/calculus}%
  {Calculus III Midterm Review}%
}

\pagecolor{white}
\color{black}

\begin{document}

This review contains questions from quiz 1 \& 2, in class midterm review questions,
and preliminary questions from sections 9.1--9.5 (excluding 9.3) \& 10.1--10.3.

Questions are listed in no particular.

\begin{enumerate}
  \item True or false? (with justification)
  \begin{enumerate}
    \item If \({a_n}\) is bounded, then it converges.
      \vspace{130pt}
    \item If \({a_n}\) converges, then is must be bounded.
      \vspace{130pt}
    \item If \({a_n}\) is not bounded, then it diverges.
      \vspace{130pt}
    \item If \({a_n}\) diverges, then it is not bounded.
      \vspace{130pt}
  \end{enumerate}

  \newpage

 \item A hot anvil with cooling constant \(k = \SI{0.02}{\per\second}\) is
   submerged in a large pool of water whose temperature is \SI{10}{\celsius}.
   Let \(y(t)\) be the anvil’s temperature \(t\) seconds later.
   \begin{enumerate}
     \item What is the differential equation satisfied by \(y(t)?\)
       \vspace{140pt}

     \item Find a formula for \(y(t)\), assuming the object initial temperature
       is \SI{100}{\celsius}.
       \vspace{180pt}
   \end{enumerate}

  \item As an object cools, its rate of cooing slows. Explain how this follows
    from Newton's Law of Cooling.

    \newpage

  \item In Yellowstone park there were approximately 500 bison in 1970 and
    3,000 bison in 1990.
    \begin{enumerate}
      \item Using the model that the rate of change of the population is
        proportional to the population itself, set up (do not solve) an initial
        value problem to model this situation.
        \vspace{140pt}

      \item Now find the particular solution satisfying your initial value problem.
        \vspace{140pt}
    \end{enumerate}

  \item Which of the following are first-order linear equations?
    \begin{multicols}{2}
    \begin{enumerate}
      \item \(y' + x^2y = 1\)
      \item \(y' + xy^2 = 1\)
      \item \(x^5y' + y = e^x\)
      \item \(x^5y' + y = e^y\)
    \end{enumerate}
    \end{multicols}
    \vspace{80pt}

  \item For what function \(P\) is the integrating factor \(\alpha (x) \) equal
    to \(x?\) What about \(e^x\)?
  \newpage

  \newpage

  \item Find the limit of the sequence \(a_n = \dfrac{n+1}{3n+2}\). Be sure to justify your answer.
  \vspace{256pt}

  \item Which of the following sequences converge to zero?
    \[%%%%%%%%%%
      \frac{n^2}{n^2+1} \qquad\qquad 2^n \qquad\qquad \left( -\frac{1}{2} \right) ^n
    \]%%%%%%%%%%
    \vspace{40pt}

  \item Which of the following sequences is defined recursively?
    \[%%%%%%%%%%
    a_n = \sqrt{4+n} \qquad \qquad b_n = \sqrt{4+b_{n-1}}
    \]%%%%%%%%%%
    \vspace{40pt}

  \item Let \(a_n\) be the \(n^{th}\) decimal approximation to \(\sqrt{2} \).
    I.e., \(a_1 = 1, a_2 = 1.4, a_3 = 1.41\) and so on. What is \(\lim_{n \to
    \infty} a_n\)?
  \newpage

  \item Biologists stocked a lake with 400 fish and estimated the carrying
    capacity to be 10,000. The number of fish tripled in the first year.
    Assuming the size of the fish population satisfies the logistic equation,
    find an expression for the size of the population after \(t\) years.

  \newpage

  \item Find the general term, \(a_n\), for the sequence given below. Assume that
  we start our sequence at \(n=1\).
  \[%%%%%%%%%%
    1,\frac{4}{2}, \frac{9}{4}, \frac{16}{8}, \frac{25}{16}, \frac{36}{32}\ldots
  \]%%%%%%%%%%
  \vspace{180pt}

  \item Determine whether each of the following series converge or diverge by
    using either the Comparison Test or the Limit Comparison Test to justify
    your answer.
    \begin{enumerate}
      \item \(\displaystyle\sum_{n=1}^{\infty} \frac{1}{e^n + n^2}\)
      \vspace{180pt}
      \item \(\displaystyle\sum_{n=1}^{\infty} \frac{1}{e^n - n^2}\)
    \end{enumerate}

  \newpage

  \item True or false? If \(k > 0\), then all solutions of \(y' = -k(y-b)\)
  approach the same limit as \(t \to \infty\).
  \vspace{60pt}

  \item Write a solution to \(y' = 4(y-5)\) that tends to \(-\infty\) as \(t\to
  \infty\).
  \vspace{90pt}

  \item Does \(y'=-4(y-5)\) have a solution that tends to \(\infty\) and \(t \to
  \infty\)?
  \vspace{90pt}

  \item Find the general solution \(y' = xe^{-\sin x} - y \cos x\).


  \newpage

  \item True or false?
  \begin{enumerate}
    \item \(t \frac{dy}{dt} = 3\sqrt{1+y} \) is a separable differential
      equation.
    \vspace{80pt}
    \item \(yy' + x + y = 0\) is a first-order linear differential equation.
    \vspace{80pt}
  \end{enumerate}

  \item Determine the order of the following differential equations:
    \begin{multicols}{2}
      \begin{enumerate}
        \item \(x^5y'=1\)
        \item \((y')^3+x=1\)
        \item \(y''' + x^4y' = 2\)
        \item \(\sin (y'') + x = y\)
      \end{enumerate}
    \end{multicols}
    \vspace{80pt}

  \item Which of the following differential equations are directly integrable?
  \begin{multicols}{2}
    \begin{enumerate}
      \item \(y'=x+y\)
      \item \(x \frac{dy}{dx} = 3\)
      \item \(\frac{dP}{dt} = 4P + 1\)
      \item \(\frac{dw}{dt} = \frac{2t}{1+4t}\)
      \item \(\frac{dx}{dt} = t^2 e^{-3t}\)
      \item \(t^2 \frac{dx}{dt} = x - 1\)
    \end{enumerate}
  \end{multicols}

  \newpage

  \item Which of the following differential equations are separable?
  \begin{multicols}{2}
    \begin{enumerate}
      \item \(\frac{dy}{dx} = x - 2y\)
      \item \(xy' + 8ye^x = 0\)
      \item \(y' = x^2y^2\)
      \item \(y' = 1-y^2\)
      \item \(t \frac{dy}{dt} = 3 \sqrt{1+y} \)
      \item \(\frac{dP}{dt} = \frac{P+t}{t}\)
    \end{enumerate}
  \end{multicols}
  \vspace{256pt}

  \item Which of the following equations are first-order?
    \begin{multicols}{2}
      \begin{enumerate}
        \item \(y' = x^2 \)
        \item \(y'' = y^2\)
        \item \((y')^3 + yy' = \sin x\)
        \item \(x^2y' - e^xy = \sin y\)
        \item \(y'' + 3y' = \frac{y}{x}\)
        \item \(yy' + x + y = 0\)
      \end{enumerate}
    \end{multicols}

  \newpage

  \item Water is draining from a cylindrical tank with cross sectional area
  \(4m^2\) and height \(5m\). Torricelli’s  Law says that the rate of change of the
  height of the water in such a cylindrical tank is proportional to the square
  root of the height of the water in the tank. Suppose that the tank starts
  full of water and after 30 minutes the height of the water has decreased to
  \(4m\). Set up an initial value problem to model this situation.

  \newpage

  \item Find the limit of each of the following sequences. Justify your answer
    using limit laws, the squeeze theorem, and/or \hopital's Rule.
    \begin{enumerate}
      \item \(\displaystyle a_n = \frac{2(-1)^{n+1}}{2n-1}\)
      \vspace{300pt}

      \item \(\displaystyle a_n = \frac{n^2}{2^{n-1}}\)
    \end{enumerate}

  \newpage

  \item Does the series \(\displaystyle \sum_{n=1}^{\infty} \cos \left(
    \frac{1}{n} \right) \) converge or diverge? How do you know?
    \vspace{160pt}

  \item Find the sum of each of the following geometric series or state that it
    diverges. Be sure to explain how you arrived at your solution.
    \begin{enumerate}
      \item \(\dfrac{1}{6} + \dfrac{1}{12} + \dfrac{1}{24} + \dfrac{1}{48} + \dfrac{1}{96} + \cdots\)
      \vspace{200pt}

      \item \(-2 + \dfrac{2}{5} - \dfrac{2}{25} + \dfrac{2}{125} - \dfrac{2}{625} + \cdots \)
    \end{enumerate}
  \newpage

  \item What role do partial sums play in defining the sum of an infinite
    series?
    \vspace{90pt}

  \item Indicate whether of not the reasoning in the following statements are
    correct:
    \begin{enumerate}
      \item \(\displaystyle \sum_{n=1}^{\infty} \frac{1}{n^2} = 0\) because
        \(\frac{1}{n^2}\) tends to zero.
      \vspace{110pt}

      \item \(\displaystyle \sum_{n=1}^{\infty} \frac{1}{\sqrt{n}}\) because
        \(\displaystyle \lim_{n \to \infty}\frac{1}{\sqrt{n} } = 0\)
      \vspace{110pt}
    \end{enumerate}

  \item Find an \(N\) such that \(S_N > 25\) for the series \(\displaystyle
    \sum_{n=1}^{\infty} 2\).
    \vspace{110pt}

  \item Does there exist an \(N\) such that \(S_N > 25\) for the series
    \(\displaystyle \sum_{n=1}^{\infty} 2^{-n}\)? Explain.
    \vspace{90pt}

  \newpage

  \item For the series \(\displaystyle \sum_{n=1}^{\infty} a_n\), if the
    partial sums \(S_N\) are increasing, then (choose the correct conclusion):
    \begin{enumerate}
      \item \({a_n}\) is an increasing sequence.
      \item \({a_n}\) is a positive sequence.
    \end{enumerate}
  \vspace{60pt}

  \item Which test would you use to determine whether the following series converge?
  \begin{enumerate}
    \item \(\displaystyle \sum_{n=1}^{\infty} n^{-3.2}\)
  \vspace{110pt}

    \item \(\displaystyle \sum_{n=1}^{\infty} \frac{1}{2^n + \sqrt{n} }\)
  \vspace{110pt}
  \end{enumerate}

\end{enumerate}
\end{document}
