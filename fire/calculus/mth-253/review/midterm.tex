\documentclass{nosvagor-notes}
\usepackage{nosvagor-math}
\usepackage{nosvagor-commands}
\usetikzlibrary{calc}
\newcommand*{\TickSize}{2pt}%

\colorlet{title-color}{black!85!white}
\newcommand{\theTitle}{%
  \href{https://github.com/nosvagor/notes/tree/main/fire/calculus}%
  {Calculus III Midterm Review}%
}

\pagecolor{white}
\color{black}

\begin{document}

This review document contains questions from quiz 1 \& 2, as well as
preliminary questions from sections 9.1--9.5 \& 10.1. Questions are listed in
no particular order.

\begin{enumerate}
  \item True or false? (with justification)
  \begin{enumerate}
    \item If \({a_n}\) is bounded, then it converges.
      \vspace{140pt}
    \item If \({a_n}\) converges, then is must be bounded.
      \vspace{140pt}
    \item If \({a_n}\) is not bounded, then it diverges.
      \vspace{140pt}
    \item If \({a_n}\) diverges, then it is not bounded.
      \vspace{140pt}
  \end{enumerate}

  \newpage

 \item A hot anvil with cooling constant \(k = \SI{0.02}{\per\second}\) is
   submerged in a large pool of water whose temperature is \SI{10}{\celsius}.
   Let \(y(t)\) be the anvil’s temperature \(t\) seconds later.
   \begin{enumerate}
     \item What is the differential equation satisfied by \(y(t)?\)
       \vspace{140pt}

     \item Find a formula for \(y(t)\), assuming the object initial temperature
       is \SI{100}{\celsius}.
       \vspace{180pt}
   \end{enumerate}

  \item As an object cools, its rate of cooing slows. Explain how this follows
    from Newton's Law of Cooling.

    \newpage

  \item In Yellowstone park there were approximately 500 bison in 1970 and
    3,000 bison in 1990.
    \begin{enumerate}
      \item Using the model that the rate of change of the population is
        proportional to the population itself, set up (do not solve) an initial
        value problem to model this situation.
        \vspace{140pt}

      \item Now find the particular solution satisfying your initial value problem.
        \vspace{140pt}
    \end{enumerate}

  \item Which of the following are first-order linear equations?
    \begin{multicols}{2}
    \begin{enumerate}
      \item \(y' + x^2y = 1\)
      \item \(y' + xy^2 = 1\)
      \item \(x^5y' + y = e^x\)
      \item \(x^5y' + y = e^y\)
    \end{enumerate}
    \end{multicols}
    \vspace{80pt}

  \item For what function \(P\) is the integrating factor \(\alpha (x) \) equal
    to \(x?\) What about \(e^x\)?
  \newpage

\item Match the differential equation that corresponds to the slope fields
  below. Then give a brief justification explaining how you arrived at your
  choice.

  \[%%%%%%%%%%
  \frac{dy}{dx} = -\frac{x}{y} \qquad \frac{dy}{dx} = y - x \qquad
  \frac{dy}{dx} = 2 - y \qquad \frac{dy}{dx} = 0.25y(4-y)
  \]%%%%%%%%%%

  \begin{tikzpicture}[declare function={f(\x,\y)=2-\y;}]
    \def\xmax{3} \def\xmin{-3}
    \def\ymax{3} \def\ymin{-3}
    \def\nx{12}
    \def\ny{12}

    \pgfmathsetmacro{\hx}{(\xmax-\xmin)/\nx}
    \pgfmathsetmacro{\hy}{(\ymax-\ymin)/\ny}
    \foreach \i in {0,...,\nx}
    \foreach \j in {0,...,\ny}{
      \pgfmathsetmacro{\yprime}{f({\xmin+\i*\hx},{\ymin+\j*\hy})}
      \draw[blue+1,shift={({\xmin+\i*\hx},{\ymin+\j*\hy})}]
        (0,0)--($(0,0)!4mm!(.1,.1*\yprime)$);
    }
    \draw [->] (\xmin,0) -- (\xmax,0) node[below right] {$x$};
    \draw [->] (0,\ymin) -- (0,\ymax) node[above left] {$y$};

    \foreach \x in {\xmin,...,\xmax} {%
      \draw ($(\x,0) + (0,-\TickSize)$) -- ($(\x,0) + (0,\TickSize)$)
        node [below] {\scriptsize{$\x$}};
    }

    \foreach \y in {1,...,\ymax} {%
      \draw ($(0,\y) + (-\TickSize,0)$) -- ($(0,\y) + (\TickSize,0)$)
        node [left] {\scriptsize{$\y$}};
    }

    \foreach \y in {\ymin,...,-1} {%
      \draw ($(0,\y) + (-\TickSize,0)$) -- ($(0,\y) + (\TickSize,0)$)
        node [left] {\scriptsize{$\y$}};
    }
  \end{tikzpicture}

  \begin{tikzpicture}[declare function={f(\x,\y)=\y - \x;}]
    \def\xmax{3} \def\xmin{-3}
    \def\ymax{3} \def\ymin{-3}
    \def\nx{14}
    \def\ny{14}

    \pgfmathsetmacro{\hx}{(\xmax-\xmin)/\nx}
    \pgfmathsetmacro{\hy}{(\ymax-\ymin)/\ny}
    \foreach \i in {0,...,\nx}
    \foreach \j in {0,...,\ny}{
      \pgfmathsetmacro{\yprime}{f({\xmin+\i*\hx},{\ymin+\j*\hy})}
      \draw[blue+1,shift={({\xmin+\i*\hx},{\ymin+\j*\hy})}]
        (0,0)--($(0,0)!4mm!(.1,.1*\yprime)$);
    }
    \draw [->] (\xmin,0) -- (\xmax,0) node[below right] {$x$};
    \draw [->] (0,\ymin) -- (0,\ymax) node[above left] {$y$};

    \foreach \x in {\xmin,...,\xmax} {%
      \draw ($(\x,0) + (0,-\TickSize)$) -- ($(\x,0) + (0,\TickSize)$)
        node [below] {\scriptsize{$\x$}};
    }

    \foreach \y in {1,...,\ymax} {%
      \draw ($(0,\y) + (-\TickSize,0)$) -- ($(0,\y) + (\TickSize,0)$)
        node [left] {\scriptsize{$\y$}};
    }

    \foreach \y in {\ymin,...,-1} {%
      \draw ($(0,\y) + (-\TickSize,0)$) -- ($(0,\y) + (\TickSize,0)$)
        node [left] {\scriptsize{$\y$}};
    }
  \end{tikzpicture}

  \begin{tikzpicture}[declare function={f(\x,\y)=0.25*\y * (4 - \y);}]
    \def\xmax{3} \def\xmin{-3}
    \def\ymax{5} \def\ymin{-2}
    \def\nx{14}
    \def\ny{14}

    \pgfmathsetmacro{\hx}{(\xmax-\xmin)/\nx}
    \pgfmathsetmacro{\hy}{(\ymax-\ymin)/\ny}
    \foreach \i in {0,...,\nx}
    \foreach \j in {0,...,\ny}{
      \pgfmathsetmacro{\yprime}{f({\xmin+\i*\hx},{\ymin+\j*\hy})}
      \draw[blue+1,shift={({\xmin+\i*\hx},{\ymin+\j*\hy})}]
        (0,0)--($(0,0)!4mm!(.1,.1*\yprime)$);
    }
    \draw [->] (\xmin,0) -- (\xmax,0) node[below right] {$x$};
    \draw [->] (0,\ymin) -- (0,\ymax) node[above left] {$y$};

    \foreach \x in {\xmin,...,\xmax} {%
      \draw ($(\x,0) + (0,-\TickSize)$) -- ($(\x,0) + (0,\TickSize)$)
        node [below] {\scriptsize{$\x$}};
    }

    \foreach \y in {1,...,\ymax} {%
      \draw ($(0,\y) + (-\TickSize,0)$) -- ($(0,\y) + (\TickSize,0)$)
        node [left] {\scriptsize{$\y$}};
    }

    \foreach \y in {\ymin,...,-1} {%
      \draw ($(0,\y) + (-\TickSize,0)$) -- ($(0,\y) + (\TickSize,0)$)
        node [left] {\scriptsize{$\y$}};
    }
  \end{tikzpicture}

  \newpage

  \item Find the limit of the sequence \(a_n = \dfrac{n+1}{3n+2}\). Be sure to justify your answer.
  \vspace{256pt}

  \item What is \(a_4\) for the sequence \(a_n = n^2 - n\)?
    \vspace{40pt}

  \item Which of the following sequences converge to zero?
    \[%%%%%%%%%%
      \frac{n^2}{n^2+1} \qquad\qquad 2^n \qquad\qquad \left( -\frac{1}{2} \right) ^n
    \]%%%%%%%%%%
    \vspace{40pt}

  \item Which of the following sequences is defined recursively?
    \[%%%%%%%%%%
    a_n = \sqrt{4+n} \qquad \qquad b_n = \sqrt{4+b_{n-1}}
    \]%%%%%%%%%%
    \vspace{40pt}

  \item Let \(a_n\) be the \(n^{th}\) decimal approximation to \(\sqrt{2} \).
    I.e., \(a_1 = 1, a_2 = 1.4, a_3 = 1.41\) and so on. What is \(\lim_{n \to
    \infty} a_n\)?
  \newpage

  \item Biologists stocked a lake with 400 fish and estimated the carrying
    capacity to be 10,000. The number of fish tripled in the first year.
    Assuming the size of the fish population satisfies the logistic equation,
    find an expression for the size of the population after \(t\) years.

  \newpage

  \item Find the general term, \(a_n\), for the sequence given below. Assume that
  we start our sequence at \(n=1\).
  \[%%%%%%%%%%
    1,\frac{4}{2}, \frac{9}{4}, \frac{16}{8}, \frac{25}{16}, \frac{36}{32}\ldots
  \]%%%%%%%%%%
  \vspace{140pt}

  \item Use Euler’s method with \(h= 0 .2\) to estimate \(y(0.4)\), where \(y(x)\) is the
    solution to the following initial value problem:
    \[%%%%%%%%%%
      y' = 2xy^2, y(0) = 1
    \]%%%%%%%%%%
    \vspace{140pt}

 \item What is the slope of the segment in the slope field for \(\frac{dy}{dt}
   = ty+1 \) at point (2,3)?
   \vspace{60pt}

  \item What is the equation of the isocline of slope \(c=1\) for
    \(\frac{dy}{dt} = y^2 -t\)?
    \vspace{60pt}

  \item Let \(y(t)\) be the solution to \(\frac{dy}{dt} = F(t,y)\) with
    \(y(1)=3\). How many iterations of Euler's Method are required to
    approximate \(y(3)\) if the time step is \(h=0.1\)?

  \newpage

  \item True of false? If \(k > 0\), then all solutions of \(y' = -k(y-b)\)
  approach the same limit as \(t \to \infty\).
  \vspace{60pt}

  \item Write a solution to \(y' = 4(y-5)\) that tends to \(-\infty\) as \(t\to
  \infty\).
  \vspace{90pt}

  \item Does \(y'=-4(y-5)\) have a solution that tends to \(\infty\) and \(t \to
  \infty\)?
  \vspace{90pt}

  \item Find the general solution \(y' = xe^{-\sin x} - y \cos x\).


  \newpage

  \item True or false?
  \begin{enumerate}
    \item \(t \frac{dy}{dt} = 3\sqrt{1+y} \) is a separable differential
      equation.
    \vspace{80pt}
    \item \(yy' + x + y = 0\) is a first-order linear differential equation.
    \vspace{80pt}
  \end{enumerate}

  \item Determine the order of the following differential equations:
    \begin{multicols}{2}
      \begin{enumerate}
        \item \(x^5y'=1\)
        \item \((y')^3+x=1\)
        \item \(y''' + x^4y' = 2\)
        \item \(\sin (y'') + x = y\)
      \end{enumerate}
    \end{multicols}
    \vspace{80pt}

  \item Which of the following differential equations are directly integrable?
  \begin{multicols}{2}
    \begin{enumerate}
      \item \(y'=x+y\)
      \item \(x \frac{dy}{dx} = 3\)
      \item \(\frac{dP}{dt} = 4P + 1\)
      \item \(\frac{dw}{dt} = \frac{2t}{1+4t}\)
      \item \(\frac{dx}{dt} = t^2 e^{-3t}\)
      \item \(t^2 \frac{dx}{dt} = x - 1\)
    \end{enumerate}
  \end{multicols}

  \newpage

  \item Which of the following differential equations are separable?
  \begin{multicols}{2}
    \begin{enumerate}
      \item \(\frac{dy}{dx} = x - 2y\)
      \item \(xy' + 8ye^x = 0\)
      \item \(y' = x^2y^2\)
      \item \(y' = 1-y^2\)
      \item \(t \frac{dy}{dt} = 3 \sqrt{1+y} \)
      \item \(\frac{dP}{dt} = \frac{P+t}{t}\)
    \end{enumerate}
  \end{multicols}
  \vspace{256pt}

  \item Which of the following equations are first-order?
    \begin{multicols}{2}
      \begin{enumerate}
        \item \(y' = x^2 \)
        \item \(y'' = y^2\)
        \item \((y')^3 + yy' = \sin x\)
        \item \(x^2y' - e^xy = \sin y\)
        \item \(y'' + 3y' = \frac{y}{x}\)
        \item \(yy' + x + y = 0\)
      \end{enumerate}
    \end{multicols}

\end{enumerate}
\end{document}
