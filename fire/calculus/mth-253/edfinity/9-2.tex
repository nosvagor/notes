% chktex-file 1
% chktex-file 3
% chktex-file 21
% chktex-file 36
% chktex-file 27

\section{9.2 Edfinity: Models Involving y'=k(y-b)}
\begin{itemize}
  \item []

    \subsection{9.2.2}
    \begin{itemize}
      \item Find the general solution of \( y'=5\left( y-16 \right)  \).
        \begin{align*}
          \aset{y(t) = b + Ce^{kt}} && \aset{y'=k(y-b)}\\ \\
          y(t) &= 16 + Ce^{5t} \\
          30 &= 16 + C \\
          C &= 14 \\
          y(t) &= 16 + 14e^{5t} \\
          1 &= 16 + C \\
          C &= -15 \\
          y(t) &= 16 + -15e^{5t}
        \end{align*}
    \end{itemize}

    \subsection{9.2.3}
    \begin{itemize}
      \item A 62 kg skydiver jumps out of an airplane. What is her terminal
        velocity in miles per hour, assuming that \( k = 10 \frac{kg}{s} \) for
        free fall?
        \begin{align*}
          -\frac{gm}{k} = -\frac{9.8(62)}{10} = -60.76 \tfrac{m}{s} = 199.343
          \tfrac{ft}{s} = -134.916~\text{mph}
        \end{align*}
    \end{itemize}

    \subsection{9.2.4}
    \begin{itemize}
      \item A continuous annuity with withdrawal rate \( N = \$\SI{600}{y}
        \) and interest rate \( r = 5 \)\% is funded by an initial deposit \(
        P_0 \)
      \item When will the annuity run out of funds if \( P_0 = \$10,000 \)?
        \begin{align*}
          P(t) = Nr^{-1} + Ce^{rt} = 600(0.05)^{-1} + Ce^{0.05t} &= 12,000 +
          Ce^{0.05t} \\
          10,000 &= 12,000 + C \\
          C &= -2,000 \\
          t = 0.05^{-1} \ln \frac{12,000}{2,000} &= 35.83 \approx38~\text{years}
        \end{align*}
      \item Which initial deposit \( P_0 \) yields a constant balance?
        \begin{align*}
          P(t) &= 12,000 + Ce^{0.05^t}, \quad C = 0 \\
          P_0 &= 12,000
        \end{align*}
    \end{itemize}

    \subsection{9.2.5}
    \begin{itemize}
      \item A cup of coffee, cooling off in a room temperature
        \SI{20}{\celsius}, has cooling constant \( k = 0.085~\text{min}^{-1}\).

      \item How fast is the coffee cooling when its temperature is \( T =
        \SI{70}{\celsius} \)?
        \begin{align*}
          \aset{k(T-T_0)} \\
          0.085(70-20) &= \SI{4.25}{\celsius\per\minute} \\
        \end{align*}

      \item Use the Linear Approximation to estimate the change in temperature
        over the next 4 seconds when \( T= \SI{70}{\celsius} \)
        \begin{align*}
          \SI{4.25}{\celsius\per\minute}(4\text{s})\SI{60}{\s\per\minute} =
          \SI{0.283}{\celsius}
        \end{align*}

      \item The coffee is served at a temperature of \SI{86}{\celsius}. How
        long should you wait before drinking it if the optimal temperature is
        \SI{65}{\celsius}?
        \begin{align*}
          65 &= 20 + 66e^{-0.085t} \\
          t &= -\left( 0.085 \right)^{-1} \ln \left( \frac{45}{66} \right) \\
          t &\approx 4.5 ~\text{min}
        \end{align*}
    \end{itemize}
\end{itemize}
