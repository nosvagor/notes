\section{7.7 Edfinity: Improper Integrals}
\begin{itemize}
  \item \dd{Improper integral}: defined as the limit of definite integrals:
  \[%%%%%%%%%%
  \int_{a}^{\infty} f (x) dx = \lim_{R \to \infty}  \int_{a}^{R}  f (x) dx
  \]%%%%%%%%%%
  The improper integral converges if this limit exists, and it diverges
  otherwise.
  \item \dd{\textit{p}-integral}: an improper integral of \(x^{-p}\) for \(a > 0\).
  \begin{align*}
    \RR{p > 1} &: \quad \int_{a}^{\infty} x^{-p} dx \quad \text{\TT{converges}} \qand \int_{0}^{a} x^{-p}dx \quad \text{\FF{diverges}}\\
    \BB{p < 1} &: \quad \int_{a}^{\infty} x^{-p} dx \quad \text{\FF{diverges}}~ \qand \int_{0}^{a} x^{-p}dx \quad \text{\TT{converges}}\\
    p = 1 &: \quad \int_{a}^{\infty} x^{-p} dx  \qand \int_{0}^{a} x^{-p}dx \quad \text{\FF{diverge}}
  \end{align*}
  \item \dd{Integral comparison test}: given \aset{\(f\)} and \aset{\(g\)} are
    continuous functions, and \(f (x) \geq g (x) \geq 0 \quad\forall x \geq a \then\)
    \begin{align*}
      \text{If}~\int_{a}^{\infty} \mathlarger{\mathlarger{f (x)}} dx ~\text{\TT{converges}}~&\then
      \int_{a}^{\infty} \mathsmaller{\mathsmaller{g (x)}} dx ~\text{\TT{converges}}\\
      \text{If}~\int_{a}^{\infty} \mathsmaller{\mathsmaller{g (x)}} dx ~\text{\FF{diverges}}~&\then
      \int_{a}^{\infty} \mathlarger{\mathlarger{f (x)}} dx ~\text{\FF{diverges}}
    \end{align*}
    \begin{itemize}
      \item The compassion test provides no information if the integral of the
        larger functions diverges, or if the integral of the smaller function
        converges.
      \item The test is also valid for improper integrals of functions with in
        infinite discontinuities at an endpoint of the integral.
    \end{itemize}

  \subsection{7.7.1}
  \begin{itemize}
    \item Compute the value of the following improper integral.
      \[%%%%%%%%%%
      \int_{-\infty}^{\infty} x^7 e^{-x^8} dx
      \]%%%%%%%%%%
      \(-(e^{-x^8}x^7) = e^{x^8}-x^7\)and the interval is \((-\infty,
      \infty)\) is symmetric about 0, then the integral over the infinite
      domain is zero.
  \end{itemize}

  \subsection{7.7.2}
  \begin{itemize}
    \item Compute the value of the following improper integral.
      \[%%%%%%%%%%
      \int_{-\infty}^{-1} e^{8t}dt
      \]%%%%%%%%%%
      \begin{align*}
        \int_{-\infty}^{-1} e^{8t}dt &= \lim_{n \to -\infty} \int_{n}^{-1} e^{8t}dt
        = \lim_{n \to -\infty} \frac{e^{8t}}{8}\bigg|_n^{-1} \\
        &=\lim_{n \to -\infty} \left( \frac{e^{-8}}{8} - \frac{e^{8n}}{8} \right)
        = \frac{e^{-8}}{8}
      \end{align*}
  \end{itemize}

  \subsection{7.7.3}
  \begin{itemize}
    \item Compute the value of the following improper integral.
      \[%%%%%%%%%%
        \int_{-\infty}^{\infty} (2-v^5)dv
      \]%%%%%%%%%%
    \begin{align*}
      &= \int 2dv - \int v^5 dv \\
      &= 2v - \frac{v^6}{6} + C \\
      \lim_{v \to \infty} &= 2v - \frac{v^6}{6} + C  = -\infty = ~\text{\FF{diverges}}~\\
    \end{align*}
  \end{itemize}

  \subsection{7.7.4}
  \begin{itemize}
    \item Compute the value of the following improper integral.
      \[%%%%%%%%%%
        \int_{1}^{\infty} (2x+1)^{-2}dx
      \]%%%%%%%%%%
      \begin{align*}
        &\then \lim_{n \to \infty} \int_{1}^{\infty}  (2x+1)^{-2}dx \\
        &\then \lim_{n \to \infty}  -2(2x+1)^{-1} \bigg|_{1}^{\infty} \\
        &= 0 - -\frac{1}{6}  = \aset{\frac{1}{6}}
      \end{align*}
  \end{itemize}

\end{itemize}
