\section{10.6 Edfinity: Power Series}
\begin{itemize}
  \item \dd{Power series}: infinite series of the form
    \[%%%%%%%%%%
    F (x) = \sum_{n=0}^{\infty} a_n (x - c)^n
    \]%%%%%%%%%%
  \item \dd{Radius of convergence, \(R\)}:
    \begin{itemize}
      \item \(F(x)\) converges absolutely for \(\left|x -c \right| < R\) and
        diverges for \(\left|x - c\right| > R\)
      \item \(F (x) \) may converge of diverge at the endpoints \(c-R\) and \(c+R\)
      \item \(R=0 \iff F (x) \) converges for \(x = c \land R = \infty \iff F
        (x)\) converges \(\forall x\)
    \end{itemize}
  \item \dd{Interval of convergence}: the open interval \((c-R, c+R)\) of
    \(F(x) \), possibly including one or both endpoints.

  \item The \aset{Ratio Test} can be used to find the \(R\). Endpoints must be
    checked separately.

  \item If \(R > 0\), then \(F\) is differential and has antiderivatives on the
    interval of convergence. Obtained via differentiating/antidifferentiating
    the respective power series for \(F\):
    \[%%%%%%%%%%
    F' (x) = \sum_{n=1}^{\infty} n a_n (x - c)^{n-1}, \qquad
    \int F (x) dx = A + \sum_{n=0}^{\infty} \frac{a_n}{n+1}(x-c)^{n+1}
    \]%%%%%%%%%%
    \begin{itemize}
      \item \(A\) is any constant.
      \item Both power series have the same \(R\).
    \end{itemize}

  \item The expansion \(\displaystyle \frac{1}{1-x} = \sum_{n=0}^{\infty} x^n
    \) is valid for \(|x| < 1\). Used to derive expansions of other relations
    functions.


\end{itemize}
