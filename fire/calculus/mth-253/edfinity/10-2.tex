\chapter{10 Infinite Series}

\section{10.2 Edfinity: Summing Infinite Series}
\begin{itemize}
  \item Partial sum:
    \[%%%%%%%%%%
      \aset{\frac{c(1-r^{N+1})}{1-r}}
    \]%%%%%%%%%%
  \item Geometric series, assuming \(|r| < 1, c \neq 0\)
    \[%%%%%%%%%%
    \aset{\sum_{n=0}^{\infty} cr^n = \frac{c}{1-r}}
    \]%%%%%%%%%%

  \subsection{ 10.2.5}
  \begin{itemize}
  \item[5.] Use the formula for the sum of geometric series to find the sum or
    state that the series diverges.
    \begin{align*}
      \sum_{n=2}^{\infty} e^{3-2n} &= \sum_{n=2}^{\infty} e^3 e^{-2n},\qquad
      c = e^{-1}, r = e^{-2} \\
      &= e^{-1} \frac{1}{1-e^{-2}} = e^{-1} \frac{e^2}{e^2-1} \\
      &= \frac{e}{e^2-1}
    \end{align*}
  \end{itemize}

  \subsection{ 10.2.6}
  \begin{itemize}
    \item[6.] Use the formula for the sum of geometric series to find the sum or
    state that the series diverges.
    \[%%%%%%%%%%
    \frac{81}{64} + \frac{9}{8} + 1 + \frac{8}{9} + \frac{64}{81} + \frac{512}{729} + \cdots
    \]%%%%%%%%%%

    \begin{align*}
      c &= \frac{81}{64}, r = \frac{8}{9} \\
      \sum_{n=0}^{\infty} \frac{81}{49}\left( \frac{8}{9} \right)^n &=
      \frac{81}{64}\left( \frac{1}{1-\frac{8}{9}} \right) =
      \frac{81(9)}{64(9-8)}
    \end{align*}
  \end{itemize}

  \newpage
  \subsection{ 10.2.7}
  \begin{itemize}
    \item Calculate \(S_3, S_4\) and \(S_5\), then find the sum for the
      telescoping series
      \[%%%%%%%%%%
        S = \sum_{n=4}^{\infty} \left( \frac{1}{n+1} - \frac{1}{n+2} \right)
      \]%%%%%%%%%%
      where \(S_k\) is the partial sum using the first \(k\) values of the
      series.
  \end{itemize}
  \begin{align*}
    S_3 &= \left(\frac{1}{5} - \frac{1}{6}\right) + \left( \frac{1}{6} - \frac{1}{7} \right) + \left( \frac{1}{7} - \frac{1}{8} \right)
    = \frac{1}{5}-\frac{1}{8} = \frac{3}{40} \\
    S_4 &= S_3 + \left(\frac{1}{8} - \frac{1}{9}\right) = \frac{1}{5} - \frac{1}{9} = \frac{4}{45} \\
    S_5 &= \frac{1}{5} - \frac{1}{10} = \frac{1}{10} \\
    S &= \lim_{n \to \infty} S_N = \lim_{n \to \infty} \left(\frac{1}{5} - \frac{1}{5+N} \right) = \frac{1}{5}
  \end{align*}

  \subsection{ 10.2.8}
  \begin{itemize}
    \item Write \(S = \sum_{n=9}^{\infty} \frac{1}{n(n-1)}\) as a telescoping
      series and find its sum.
      \begin{align*}
        S_n &= \frac{1}{n(n-1)} = \frac{1}{n-1} - \frac{1}{n}\\
        S_n &= \frac{1}{8}-\frac{1}{N} \\
        S &= \frac{1}{8}
      \end{align*}
  \end{itemize}

  \subsection{ 10.2.9}
  \begin{itemize}
    \item A ball dropped from a height of 15 feet begins to bounce. Each time
      it strikes the round, it returns \(\frac{4}{5}\) of its previous height.
      What is the total distance traveled by the ball if it bounces infinitely
      many times?
      \begin{align*}
        c &= 15 + 30\left(\frac{4}{5}\right), r = \frac{4}{5} \\
        \sum_{n=0}^{\infty} 15 + 30 \left( \frac{4}{5} \right)
        \left(\frac{1}{1-\frac{4}{5}}\right) &= 15+30\left(\frac{4}{5}\right)\frac{1}{5} \\
        &=15+30(4) ~\text{ft}~
      \end{align*}
  \end{itemize}
\end{itemize}
