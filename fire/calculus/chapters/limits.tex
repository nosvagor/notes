\chapter{Limits and Continuity}

% chktex-file 36

\section{Limits}
\src{
  \link{https://en.wikipedia.org/wiki/Limit_(mathematics)}{Limit} |
  \thomas{2.2--2.4}
}

\begin{itemize}
  \item \dd{Limit \(\lim_{x \to c}\)}: the value of a function \prn{or
    sequence} as the input \prn{or index} approaches some value
    \prn{note: an informal definition}.
    \begin{itemize}
      \item Limits are used to define \dlink{s:Continuity}{continuity},
        \dlink{c:Derivatives}{derivatives}, and \dlink{c:Integrals}{integrals}.
    \end{itemize}

  \subsection{Limits of a Functions and Sequences}
  \src{
    \link{https://en.wikipedia.org/wiki/Limit_of_a_function}{Limit of a function} |
    \link{https://en.wikipedia.org/wiki/Limit_of_a_sequence}{Limit of a sequence} |
    \linky{https://youtu.be/kfF40MiS7zA?t=281}{Essence of Calculus, E7}
  }
  \begin{itemize}
    \item \dd{Limit of a function}: a fundamental concept in calculus and
      analysis concerning the behavior of a function near a particular input
      \(c\), i.e.,
      \[
      \lim_{x \to c}f(x)=L
      \]
   \begin{itemize}
     \item Reads as ``\(f\) of \(x\) tends to \(L\) as \(x\) tends to \(c\)''
   \end{itemize}

 \item \dd{ \( \epsilon, \delta \) Limit of a function}: a formalized
   definition, wherein \(f(x)\) is defined on an open interval
   \(\1{\mathcal{I}}\), except possibly at \(c\) itself, leading to the
   informal definition, \2{if and only if}
   \[
    f : \R \to \R, c,L \in \R \Rightarrow \lim_{x \to c} f(x) = L
  \]\vspace{-24pt}
   \[
    \2{\Updownarrow}
   \]
   \[
     \forall \MM{\dvarepsilon > 0} \left(
       \exists \YY{\delta > 0} : \forall x \in \1{\mathcal{I}} \left(
        0 < |x-c| < \YY{\delta} \Rightarrow |f(x)-L| < \MM{\dvarepsilon}
        \right)
      \right)
   \]
  \item functions \FF{do not have} a limit when the function:
    \begin{itemize}
      \item has a \FF{unit step}, i.e., it ``jumps'' at a point;
      \item is \FF{not bounded}, i.e., it tends towards infinity;
      \item or it \FF{oscillates}, i.e., does not stay close to any single number.
    \end{itemize}

  \item \dd{Limit of a sequence}: the value that the terms of a sequence
    (\(x_\PP{n}\))\(_{\PP{n} \in \N}\) ``tends to'' \prn{and not to any other} as \PP{\(n\)}
    approaches infinity \prn{or some other point}, i.e.,
    \[
      \lim_{\PP{n} \to \YY{\infty}} x_\PP{n} = x
    \]
    \item \dd{\( \varepsilon\) Limit of a sequence}: for every measure of
      closeness \MM{\( \varepsilon\)}, the sequence's \( x_\PP{n}\) term eventually
      converges to the limit, i.e.,
      \[
        \forall \MM{ \dvarepsilon > 0} \left(
          \exists \PP{N} \in \N \left(
            \forall\PP{n} \in \N \left(
              \PP{n} \ge \PP{N} \Rightarrow |x_\PP{n} - x| < \MM{ \dvarepsilon}
            \right)
          \right)
        \right)
      \]
      \item

  \end{itemize}


  \subsection{Properties of Limits}
  \begin{itemize}
    \item
  \end{itemize}

  \subsection{One-Sided Limit}
  \begin{itemize}
    \item
  \end{itemize}
\end{itemize}

\section{Continuity}
\begin{itemize}
  \item []

  \subsection{Continuous Functions}
  \begin{itemize}
    \item
  \end{itemize}

  \subsection{Intermediate Value Theorem}
  \begin{itemize}
    \item
  \end{itemize}

\end{itemize}

\section{Limits Involving Infinity}
\begin{itemize}
  \item []

  \subsection{Limits at Infinity and Infinite Limits}
  \begin{itemize}
    \item
  \end{itemize}

  \subsection{Asymptotes of functions}
  \begin{itemize}
    \item
  \end{itemize}

\end{itemize}
