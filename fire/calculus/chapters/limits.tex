\chapter{Limits and Continuity}

% chktex-file 1
% chktex-file 3
% chktex-file 21
% chktex-file 36
% chktex-file 27

\section{Limits}
\src{
  \link{https://en.wikipedia.org/wiki/Limit_(mathematics)}{Limit} |
  \thomas{2.2--2.4}
}

\begin{itemize}
  \item \dd{Limit \(\lim_{x \to c}\)}: the value of a function \prn{or
    sequence} as the input \prn{or index} approaches some value
    \prn{note: an informal definition}.
    \begin{itemize}
      \item Limits are used to define \dlink{s:Continuity}{continuity},
        \dlink{c:Derivatives}{derivatives}, and \dlink{c:Integrals}{integrals}.
    \end{itemize}

  \subsection{Limits of a Functions and Sequences}
  \src{
    \link{https://en.wikipedia.org/wiki/Limit_of_a_function}{Limit of a function} |
    \link{https://en.wikipedia.org/wiki/Limit_of_a_sequence}{Limit of a sequence} |
    \linky{https://youtu.be/kfF40MiS7zA?t=281}{Essence of Calculus, Ea}
  }
  \begin{itemize}
    \item \dd{Limit of a function}: a fundamental concept in calculus and
      analysis concerning the behavior of a function near a particular input
      \(c\), i.e.,
      \[
      \lim_{x \to c}f(x)=L
      \]
   \begin{itemize}
     \item Reads as ``\(f\) of \(x\) tends to \(L\) as \(x\) tends to \(c\)''
   \end{itemize}

 \item \dd{ \( \epsilon, \delta \) Limit of a function}: a formalized
   definition, wherein \(f(x)\) is defined on an open interval
   \(\1{\mathcal{I}}\), except possibly at \(c\) itself, leading to the
   informal definition, \2{if and only if}
   \[
      f : \R \to \R, c,L \in \R \Rightarrow \lim_{x \to c} f(x) = L
    \]
    \vspace{-22pt}
   \[
    \2{\Updownarrow}
   \]
   \[
     \forall \MM{\dvarepsilon > 0} \left(
       \exists \YY{\delta > 0} : \forall x \in \1{\mathcal{I}} \left(
        0 < |x-c| < \YY{\delta} \Rightarrow |f(x)-L| < \MM{\dvarepsilon}
        \right)
      \right)
   \]
 \item Functions \FF{do not have a limit} when the function:
    \begin{itemize}
      \item has a \FF{unit step}, i.e., it ``jumps'' at a point;
      \item is \FF{not bounded}, i.e., it tends towards infinity;
      \item or it \FF{oscillate}, i.e., it does not stay close to any single number.
    \end{itemize}

  \item \dd{Limit of a sequence}: the value that the terms of a sequence
    (\(x_\PP{n}\))\(_{\PP{n} \in \N}\) ``tends to'' \prn{and not to any other} as \PP{\(n\)}
    approaches infinity \prn{or some other point}, i.e.,
    \[
      \lim_{\PP{n} \to \YY{\infty}} x_\PP{n} = x
    \]
    \item \dd{\( \varepsilon\) Limit of a sequence}: for every measure of
      closeness \MM{\( \varepsilon\)}, the sequence's \( x_\PP{n}\) term eventually
      converges to the limit, i.e.,
      \[
        \forall \MM{ \dvarepsilon > 0} \left(
          \exists \PP{N} \in \N \left(
            \forall\PP{n} \in \N \left(
              \PP{n} \ge \PP{N} \then |x_\PP{n} - x| < \MM{ \dvarepsilon}
            \right)
          \right)
        \right)
      \]
    \begin{itemize}
      \item \dd{Convergent}: when a limit of a sequence \TT{exists}.
      \item \dd{Divergent}: a sequence that \FF{does not} converge.
    \end{itemize}

  \end{itemize}


  \subsection{Properties of Limits}
  \src{
    \link{https://en.wikipedia.org/wiki/List_of_limits}{List of limits} |
    \link{https://en.wikipedia.org/wiki/Squeeze_theorem}{Squeeze theorem}
  }
  \begin{itemize}
    \item \dd{Operations on a single known limit}: if \( \lim_{x \to c} f(x) =
      L\), then:
      \begin{itemize}
        \item \( \lim_{x \to c}\left[ f(x) \pm \PP{ \alpha } \right] = L \pm
          \PP{ \alpha } \)
        \item \( \lim_{x \to c} \PP{ \alpha } f(x) = \PP{
          \alpha }  L \)
        \item \( \lim_{x \to c} f(x)^{\PP{-1}} = L^{\PP{-1}}, L \neq 0 \)
        \item \( \lim_{x \to c} f(x)^{\PP{n}} = L^{\PP{n}}, \PP{n} \in \N \)
        \item \( \lim_{x \to c} f(x)^{\PP{n^{-1}}} = L^{\PP{n^{-1}}},
          \text{if}~\PP{n} \in \N_\RR{e} \then L~\RR{>}~0\)
      \end{itemize}

    \item \dd{Operations on two known limits}: if \( \lim_{x \to c} \) and
      \( \lim_{x \to c} g(x) = L_2 \), then:
      \begin{itemize}
        \item \( \lim_{x \to c} \left[ f(x) \pm g(x) \right] = L_1 \pm L_2 \)
        \item \( \lim_{x \to c} \left[ f(x)g(x) \right] = L_1 L_2 \)
        \item \( \lim_{x \to c} f(x)g(x)^{-1} = L_1 L_2^{-1} \)
      \end{itemize}

    \item \dd{Squeeze theorem}: used to \aset{confirm the limit} of a
      \aset{difficult to compute function} via comparison with two other
      functions whose limits are easily known or computed.
      \begin{itemize}
        \item Let \( \I  \) be an interval having the point \( c \) as a limit point.
        \item Let \( g, f, \) and \( h \) be functions defined on \( \I \),
          except possibly at \( c \) itself.
        \item Suppose that \( \forall x \in \I \land x \neq \then \BB{g(x)} \le
          \aset{f(x)} \le \RR{h(x)} \)
        \item and \( \lim_{x \to c} \BB{g(x)} = \lim_{x \to c} \RR{h(x)} = L \)
        \item then, \( \lim_{x \to c} \aset{f(x)} = L \)
        \item Essentially, the hard to compute limit of the ``middle function''
          can be found by finding the limit of two other ``easier'' functions
          that that ``squeeze'' the middle function at a point of interest.
      \end{itemize}

  \end{itemize}

  \subsection{One-Sided Limit}
  \src{
    \link{https://en.wikipedia.org/wiki/One-sided_limit}{One-Sided Limit}
  }
  \begin{itemize}
    \item \dd{One-sided limit}: one of two limits of \( f(x) \) as \( x \)
      approaches a specified point from either the \BB{left} or from the right
      \RR{right}.
      \begin{multicols}{2}
        \begin{itemize}
          \item From the \BB{left}: \( \lim_{x \to \BB{c^-}} = L \)
          \item From the \RR{right}: \( \lim_{x \to \RR{c^+}} = L \)
        \end{itemize}
      \end{multicols}

    \item If the left and right limits exist and are equal, then
    \[
      \lim_{x \to c} f(x) = L \ifandif \lim_{x \to \BB{c^-}} f(x) = L \land
        \lim_{ x \to \RR{c^+}} f(x) = L
    \]
    \item Limits can still exist, even if the function is defined at a
      different point, as long as both one-sided limits approach the same value
      near the given input.

  \end{itemize}
\end{itemize}

\section{Continuity}
\begin{itemize}
  \item []

  \subsection{Continuous Functions}
  \begin{itemize}
    \item
  \end{itemize}

  \subsection{Intermediate Value Theorem}
  \begin{itemize}
    \item
  \end{itemize}

\end{itemize}

\section{Limits Involving Infinity}
\begin{itemize}
  \item []

  \subsection{Limits at Infinity and Infinite Limits}
  \begin{itemize}
    \item
  \end{itemize}

  \subsection{Asymptotes of functions}
  \begin{itemize}
    \item
  \end{itemize}

\end{itemize}
