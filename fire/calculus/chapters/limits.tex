\chapter{Limits and Continuity}

% chktex-file 1
% chktex-file 3
% chktex-file 21
% chktex-file 36
% chktex-file 27

\section{Limits}
\src{
  \link{https://en.wikipedia.org/wiki/Limit_(mathematics)}{Limit} |
  \thomas{2.2--2.4}
}

\begin{itemize}
  \item \dd{Limit \(\lim_{x \to c}\)}: the value of a function \prn{or
    sequence} as the input \prn{or index} approaches some value
    \prn{note: an informal definition}.
    \begin{itemize}
      \item Limits are used to define \dlink{s:Continuity}{continuity},
        \dlink{c:Derivatives}{derivatives}, and \dlink{c:Integrals}{integrals}.
    \end{itemize}

  \subsection{Limits of a Functions and Sequences}
  \src{
    \link{https://en.wikipedia.org/wiki/Limit_of_a_function}{Limit of a function} |
    \link{https://en.wikipedia.org/wiki/Limit_of_a_sequence}{Limit of a sequence} |
    \linky{https://youtu.be/kfF40MiS7zA?t=281}{Essence of Calculus, Ea}
  }
  \begin{itemize}
    \item \dd{Limit of a function}: a fundamental concept in calculus and
      analysis concerning the behavior of a function near a particular input
      \(c\), i.e.,
      \[
      \lim_{x \to c}f(x)=L
      \]
   \begin{itemize}
     \item Reads as ``\(f\) of \(x\) tends to \(L\) as \(x\) tends to \(c\)''
   \end{itemize}

 \item \dd{ \( \epsilon, \delta \) Limit of a function}: a formalized
   definition, wherein \(f(x)\) is defined on an open interval
   \(\1{\mathcal{I}}\), except possibly at \(c\) itself, leading to the
   informal definition, \2{if and only if}
   \[
      f : \R \to \R, c,L \in \R \Rightarrow \lim_{x \to c} f(x) = L
    \]
    \vspace{-22pt}
   \[
    \2{\Updownarrow}
   \]
   \[
     \forall \MM{\dvarepsilon > 0} \left(
       \exists \YY{\delta > 0} : \forall x \in \1{\mathcal{I}} \left(
        0 < |x-c| < \YY{\delta} \Rightarrow |f(x)-L| < \MM{\dvarepsilon}
        \right)
      \right)
   \]
 \item Functions \FF{do not have a limit} when the function:
    \begin{itemize}
      \item has a \FF{unit step}, i.e., it ``jumps'' at a point;
      \item is \FF{not bounded}, i.e., it tends towards infinity;
      \item or it \FF{oscillate}, i.e., it does not stay close to any single number.
    \end{itemize}

  \item \dd{Limit of a sequence}: the value that the terms of a sequence
    (\(x_\PP{n}\))\(_{\PP{n} \in \N}\) ``tends to'' \prn{and not to any other} as \PP{\(n\)}
    approaches infinity \prn{or some other point}, i.e.,
    \[
      \lim_{\PP{n} \to \YY{\infty}} x_\PP{n} = x
    \]
    \item \dd{\( \varepsilon\) Limit of a sequence}: for every measure of
      closeness \MM{\( \varepsilon\)}, the sequence's \( x_\PP{n}\) term eventually
      converges to the limit, i.e.,
      \[
        \forall \MM{ \dvarepsilon > 0} \left(
          \exists \PP{N} \in \N \left(
            \forall\PP{n} \in \N \left(
              \PP{n} \ge \PP{N} \then |x_\PP{n} - x| < \MM{ \dvarepsilon}
            \right)
          \right)
        \right)
      \]
    \begin{itemize}
      \item \dd{Convergent}: when a limit of a sequence \TT{exists}.
      \item \dd{Divergent}: a sequence that \FF{does not} converge.
    \end{itemize}

  \end{itemize}


  \subsection{Properties of Limits}
  \src{
    \link{https://en.wikipedia.org/wiki/List_of_limits}{List of limits} |
    \link{https://en.wikipedia.org/wiki/Squeeze_theorem}{Squeeze theorem}
  }
  \begin{itemize}
    \item \dd{Operations on a single known limit}: if \( \lim_{x \to c} f(x) =
      L\), then:
      \begin{itemize}
        \item \( \lim_{x \to c}\left[ f(x) \pm \PP{ \alpha } \right] = L \pm
          \PP{ \alpha } \)
        \item \( \lim_{x \to c} \PP{ \alpha } f(x) = \PP{
          \alpha }  L \)
        \item \( \lim_{x \to c} f(x)^{\PP{-1}} = L^{\PP{-1}}, L \neq 0 \)
        \item \( \lim_{x \to c} f(x)^{\PP{n}} = L^{\PP{n}}, \PP{n} \in \N \)
        \item \( \lim_{x \to c} f(x)^{\PP{n^{-1}}} = L^{\PP{n^{-1}}},
          \text{if}~\PP{n} \in \N_\RR{e} \then L~\RR{>}~0\)
      \end{itemize}

    \item \dd{Operations on two known limits}: if \( \lim_{x \to c} \) and
      \( \lim_{x \to c} g(x) = L_2 \), then:
      \begin{itemize}
        \item \( \lim_{x \to c} \left[ f(x) \pm g(x) \right] = L_1 \pm L_2 \)
        \item \( \lim_{x \to c} \left[ f(x)g(x) \right] = L_1 L_2 \)
        \item \( \lim_{x \to c} f(x)g(x)^{-1} = L_1 L_2^{-1} \)
      \end{itemize}

    \item \dd{Squeeze theorem}: used to \aset{confirm the limit} of a
      \aset{difficult to compute function} via comparison with two other
      functions whose limits are easily known or computed.
      \begin{itemize}
        \item Let \( \I  \) be an interval having the point \( c \) as a limit point.
        \item Let \( g, f, \) and \( h \) be functions defined on \( \I \),
          except possibly at \( c \) itself.
        \item Suppose that \( \forall x \in \I \land x \neq \then \BB{g(x)} \le
          \aset{f(x)} \le \RR{h(x)} \)
        \item and \( \lim_{x \to c} \BB{g(x)} = \lim_{x \to c} \RR{h(x)} = L \)
        \item then, \( \lim_{x \to c} \aset{f(x)} = L \)
        \item Essentially, the hard to compute limit of the ``middle function''
          can be found by finding the limit of two other ``easier'' functions
          that that ``squeeze'' the middle function at a point of interest.
      \end{itemize}

  \end{itemize}

  \subsection{One-Sided Limit}
  \src{
    \link{https://en.wikipedia.org/wiki/One-sided_limit}{One-Sided Limit}
  }
  \begin{itemize}
    \item \dd{One-sided limit}: one of two limits of \( f(x) \) as \( x \)
      approaches a specified point from either the \BB{left} or from the right
      \RR{right}.
      \begin{multicols}{2}
        \begin{itemize}
          \item From the \BB{left}: \( \lim_{x \to \BB{c^-}} = L \)
          \item From the \RR{right}: \( \lim_{x \to \RR{c^+}} = L \)
        \end{itemize}
      \end{multicols}

    \item If the left and right limits exist and are equal, then
    \[
      \lim_{x \to c} f(x) = L \iff \lim_{x \to \BB{c^-}} f(x) = L \land
        \lim_{ x \to \RR{c^+}} f(x) = L
    \]
    \item Limits can still exist, even if the function is defined at a
      different point, as long as both one-sided limits approach the same value
      near the given input.

  \end{itemize}
\end{itemize}

\section{Continuity}
\src{
  \thomas{2.5}
}
\begin{itemize}
  \item Continuity of functions is one of the core concepts of topology,
    however, there are definitions in terms of limits that prove useful; the
    following is only a primer.

  \subsection{Continuous Functions}
  \src{
    \link{https://en.wikipedia.org/wiki/Continuous_function}{Continuous function} |
    \link{https://en.wikipedia.org/wiki/Classification_of_discontinuities}{Discontinuities}
  }
  \begin{itemize}
    \item \dd{Continuous function}: a function that does not have any abrupt
      changes in value.
    \begin{itemize}
      \item I.e., a function is continuous if and only if arbitrarily small
        changes in its output can be assured by restricting to sufficiently
        small changes in its input.
    \end{itemize}

  \item \dd{Discontinuous}: when a function is not continuous at a point in its
    domain, leading to a discontinuity; there are three classifications:
    \begin{itemize}
      \item \dd{Removable}: when both \ulink{ss:One-Sided Limit}{one-sided
        limits} exist, are finite, and are equal, but the actual value of \(
        f(x) \) is not equal to the limit and instead equal to some other
        value.
        \begin{itemize}
          \item The discontinuity can be removed to regain continuity.
          \item Sometimes the term \textit{removable discontinuity} is mistaken
            for a \textit{removable singularity}, or a ``whole'' in the function
            \prn{the point is not defined elsewhere}.
        \end{itemize}
      \item \dd{Jump}: when a single limit does not exist because the one-sided
        limits exist and are finite, but not equal.
        \begin{itemize}
          \item Points can be defined at the discontinuity, but the function
            can not be made continuous.
        \end{itemize}

      \item \dd{Essential}: when at least one of the two one-sided limits do
        not exist; can be the result of oscillating or unbounded functions.

    \end{itemize}

  \end{itemize}

  \subsection{Intermediate Value Theorem}
  \src{
    \link{https://en.wikipedia.org/wiki/Intermediate_value_theorem}{Intermediate value theorem}
  }
  \begin{itemize}
    \item \dd{Intermediate value theorem}: if \( f \) is a continuous function
      whose domain contains the interval \( \left[ a, b \right] \), then it
      \aset{takes on any given value between} \( f(a) \) and \( f(b) \) at some
      point within the intervals.
    \item Relevant deductions, i.e., important corollaries:
      \begin{itemize}
        \item \dd{Bolzano's theorem}: if a continuous function has values of
          opposite sign inside an interval, then it \aset{has a root} in that
          interval.
        \item The image of a continuous function over an interval is itself an
          interval.
      \end{itemize}

    \item Thus, the image set \( f(\I) \) \prn{which has no gaps} is also an
      interval, and it contains:
      \[
        \left[ \min{f(a),f(b)}, \max{f(a),f(b)} \right]
      \]
  \end{itemize}

\end{itemize}

\section{Limits Involving Infinity}
\src{
  \thomas{2.6}
}
\begin{itemize}
  \item Let \( S \subset \R, x \in S \) and \( f : S \mapsto \R \), then limits
    of these functions can approach arbitrarily large \prn{\( \pm \)} values,
    providing a connection to asymptotes, and thus, analysis.

  \subsection{Limits at Infinity and Infinite Limits}
  \src{
    \link{https://en.wikipedia.org/wiki/Limit_of_a_function\#Limits_involving_infinity}{Limits involving infinity}
  }
  \begin{itemize}
    \item \dd{Limits at infinity}: limits defined as \( f(x) \pm \) infinity
      are defined much like normal limits:
      \[
        \lim_{x \to \BB{-\infty}} f(x) = L \qquad \lim_{x \to \RR{\infty}} f(x) = L
      \]
      \begin{itemize}
        \item Formally, for all measures of closeness \MM{\(\varepsilon \)} there
          exists a point \( c \) such that \( |f(x) - L| < \MM{ \varepsilon} \)
          whenever \( x \BB{<} c \lor x \RR{>} c \) \prn{respectively}, i.e.,
          \[
            \forall \MM{\dvarepsilon > 0} \left(
              \exists c \left(
              \forall x \por{\BB{<}}{\RR{>}} c : | f(x) - L
                | < \MM{\dvarepsilon}
              \right)
            \right)
          \]
        \item Basic rules for rational functions \( f(x) = p(x)q(x)^{-1} \),
          where \( p \) and \( q \) are polynomials, where the degree of each
          is denoted as \( \por{p}{q}^\circ \), and where the leading coefficients
          are denoted as \( P, Q \), then:
          \begin{itemize}
            \item \( p ^\circ \gt q ^\circ \then \pm L \), depending on the sign
              of the leading coefficient.
            \item \( p ^\circ = q^\circ \then L = PQ^{-1} \)
            \item \( p^\circ \lt q^\circ \then L = 0 \)
          \end{itemize}

      \end{itemize}

    \item \dd{Infinite limits}: the usual limit does not exist for a limit that
      grows out of bounds, however, limits with infinite values can be
      introduced:
      \[
        \lim_{x \to c} f(x) = \infty, \quad \text{i.e.,} \quad
        \forall \MM{n > 0}\left(
          \exists \YY{ \delta > 0} : f(x) > \MM{n} \iff 0 < |x-c| < \YY{ \delta}
        \right)
      \]
  \end{itemize}

  \subsection{Asymptotes of functions}
  \src{
   \link{https://en.wikipedia.org/wiki/Asymptote}{Asymptotes}
  }
  \begin{itemize}
    \item \dd{Asymptote}: a tangent line of a curve at a point at infinity; the
      distance between the curve and the line approaches zero as a coordinate
      tends to infinity.
    \item There are three kinds of asymptotes: \textit{horizontal, vertical}
      and \textit{oblique}; the nature of the asymptote is dependent on a
      function's relation to infinity.
      \begin{itemize}
        \item  \dd{Horizontal asymptote}: a result of \aset{limits at
          infinity}, i.e., when \( x \to \pm \infty \).
        \item \dd{Vertical asymptote}: a result of \aset{infinite limits},
          i.e., when \( x \to \pm c = \pm \infty \)
        \item \dd{Oblique asymptote}: when a linear asymptote is not parallel
          to either axis; \( f(x) \) is asymptotic to the straight line \( y =
          mx + n~ \prn{m\neq 0} \) if:
          \[
            \lim_{x \to \pm \infty} \left[ f(x) - (mx+n) \right] = 0
          \]
      \end{itemize}

  \end{itemize}

\end{itemize}
