\chapter{Derivatives}

% chktex-file 1
% chktex-file 3
% chktex-file 21
% chktex-file 36
% chktex-file 27

\section{Derivative Fundamentals}
\src{
  \link{https://en.wikipedia.org/wiki/Derivative}{Derivative} | \thomas{3.2, 3.4}
}
\begin{itemize}
  \item \dd{Derivative}: the measure of \aset{sensitivity to change} of the
    function \YY{value} with respect to some change in its \MM{in argument}.
    \begin{itemize}
      \item Often descried as the \aset{instantaneous rate of change} of a
        single variable function, since it is the slope of a tangent line at a
        particular point, when it exists.
        \begin{itemize}
          \item \dd{Tangent line}: the line through a pair of points on a curve
            \prn{secant line}, except the points are \aset{infinitely close},
            hence, it is the rate of change at that ``instant''.
        \end{itemize}

    \end{itemize}


  \subsection{Derivative Notation}
  \begin{itemize}
    \item Formally, a derivative of the function \( f(x) \), with respect to the
      variable  \( x \), is the function \( f' \) whose value at \( x \) is
      \prn{provided the limit exists}:
      \[
        f'(x) = \lim_{h \to 0} \frac{f(x+h) - f(x)}{h}
      \]
      \begin{itemize}
        \item Let \( z = x + h \), then \( h = z - x \land h \to 0 \ifandif z
            \to x \); this leads to an equivalent definition of the derivative
            \prn{sometimes more convenient}:
            \[
              f'(x) = \lim_{z \to x} \frac{f(z) - f(x)}{z - x}
            \]
      \end{itemize}

    \item \dd{Notation}: there are many ways to denote the derivative;
      different notation can be useful in various contexts, some common
      notations \prn{for \( y = f(x) \)}:
      \[
        f'(x) = y' = \dot{y} = \frac{dy}{dx} = \frac{d}{dx} f(x) = D(f)(x) =
        D_x f(x)
      \]
    \item \dd{Differentiation}: the process of finding a derivative; if \( f'
      \) exists at a particular point, then \( f \) is said to be
      differentiable at that point.
      \begin{itemize}
        \item If \( f' \) exists at every point on an interval, then \( f \) is
          differentiable on that interval.
        \item \( f' \) is differentiable on a closed interval \( \left[ a,b
          \right] \) if both \ulink{ss:One-Sided Limit}{one-sided limits} of
          the function \prn{\(\mathsmaller{ h \to \left\{ \RR{0^+:a},~\BB{0^-:b}
          \right\}  }\)} exist at the end points, and it is differentiable on the
          interior.
        \item Not all continuous functions have a derivative, but \TT{functions
          with a derivative are continuous}; functions with any of the
          following \FF{do not have derivatives}:
          \begin{itemize}
            \item \FF{corners} \prn{one-sided derivatives differ at a point},
            \item \FF{cusps} \prn{slope approaches alternating \( \pm \infty \)
              on both sides of a point},
            \item \FF{discontinuities}, or \FF{vertical tangent lines}.
          \end{itemize}
      \end{itemize}
  \end{itemize}
\end{itemize}

\section{Differentiation Rules}
\src{
  \link{https://en.wikipedia.org/wiki/Differentiation_rules}{Differentiation rules} |
  \thomas{3.3, 3.5, 3.6}
}
\begin{itemize}
   \item Derivatives can be found by computing the limit, but there are several
     methods that use combinations of simpler functions to make computation
     easier.

  \subsection{Linear, Product, Chain, Inverse}
  \src{
    \link{https://en.wikipedia.org/wiki/Product_rule}{Product} |
    \link{https://en.wikipedia.org/wiki/Chain_rule}{Chain} |
    \link{https://en.wikipedia.org/wiki/Inverse_functions_and_differentiation}{Inverse}
  }
  \begin{itemize}
    \item \dd{Linear}: differentiation of linear functions consists of the
      constant and sum rules, given the following:
    \[
      \forall(f \land g) \land \forall(\PP{a} \land \PP{b} \in \R) \then
      \frac{d(\PP{a}f + \PP{b}g)}{dx} = \PP{a}\frac{df}{dx} +
      \PP{b}\frac{dg}{dx}
    \]
    \vspace{-18pt}
    \begin{multicols}{3}
      \dd{Constant} \\ \( \frac{d}{dx} (c) = 0\) \\
      \dd{Constant factor} \\ \( (\PP{a}f)' = \PP{a}f' \) \\
      \dd{Sum / Difference} \\ \( (f + g)' = f' + g' \) \\
    \end{multicols}

    \item \dd{Product rule}: used for the product of two functions;
      \dlink{ss:General Leibniz Rule}{can be generalized}
      \[
        \frac{d(\BB{f}\RR{g})}{dx} = \RR{g} \frac{d\BB{f}}{dx} +
        \BB{f}\frac{d\RR{g}}{dx}
      \]
    \item \dd{Chain rule}: used for the composition of two functions \( f(g(x))
      \); if \YY{\(z\)} depends on \YY{\( y \)}, which is dependent on
      \MM{\(x\)}, then \YY{\( z \)} depends on \MM{\( x \)} as well, i.e.,
      \[
        \frac{d \YY{z}}{d \MM{x}} = \frac{d \YY{z}}{d \YY{y}} \cdot \frac{d
        \YY{y}}{dx}
      \]
      \begin{itemize}

        \item The following is used to indicate points of evaluation:
        \[
          \left.\frac{d \YY{z}}{d \MM{x}}\right|_\MM{x} = \left. \frac{d
          \YY{z}}{d \YY{y}}\right|_{ \YY{y}\left(\MM{x}  \right) } \cdot
          \left. \frac{d \YY{y}}{dx}\right|_\MM{x}
        \]
        \item \dd{Outside-Inside rule}: take the derivative of the ``outside''
          function, leave the ``inside'' alone, and multiply it by the
          derivative of the ``inside.''
        \item This method must be recursively ``chained'' when there are
          further compositions in the inside function, hence the name.

      \end{itemize}

    \item \dd{Inverse function rule}: can be applied if the function \( f \)
      has an inverse function \( g \), i.e., ``undoes'' the effect of \( f \).
      \[
      \pand{g(f(\MM{x})) = \MM{x}}{f(g(\YY{y})) = \YY{y}} \then \frac{d
      \MM{x}}{d \YY{y}} = \left( \frac{d \YY{y}}{d \MM{x}} \right)^{-1}
      \]
      \begin{itemize}
        \item Application of the chain rule on \( f ^{-1}(\YY{y}) = \MM{x} \)
          in terms of \MM{\( x \)} clearly shows the result, if the derivatives
          exist and are reciprocal,
          \[
          \frac{d\MM{x}}{d\YY{y}} \cdot \frac{d\YY{y}}{d\MM{x}} =
          \frac{d\MM{x}}{d\MM{x}} = 1
          \]
      \end{itemize}

  \end{itemize}

  \subsection{Power, Polynomial, Reciprocal, Quotient}
  \src{
    \link{https://en.wikipedia.org/wiki/Power_rule}{Power} |
    \link{https://en.wikipedia.org/wiki/Reciprocal_rule}{Reciprocal} |
    \link{https://en.wikipedia.org/wiki/Quotient_rule}{Quotient}
  }
  \begin{itemize}
    \item \dd{Power rule}: used to differentiate functions in the form of \(
      f(x) = x^r \); can be applied to polynomials since differentiation is
      linear.
      \begin{itemize}
        \item Let \( f : \R \mapsto \R \) be a function satisfying
        \[
          f(x) = x^r, \quad \forall x \left( r \in \R \right) \then
          \frac{d}{dx} = r x^{r-1}
        \]
      \end{itemize}

    \item \dd{Reciprocal rule}: yields the derivative of the reciprocal
      \prn{multiplicative inverse}  of a function \( f \) in terms of the
      derivative of \( f \).
      \begin{itemize}
        \item Can be used to show that the power rule holds for negative
          exponents.
        \item The product and reciprocal rules can be used to
          deduce the quotient rule.
        \item Let  \( f \) be differentiable at \( x \) and \( f(x) \neq 0 \),
          then \( g(x) = f(x)^{-1} \) is also differentiable and
          \[
            \frac{d(f^{-1})}{dx} = -f^{-2}\frac{df}{dx} \quad \text{i.e.,}
            \quad g' = -\frac{f'}{f^2}
          \]
      \end{itemize}

    \item \dd{Quotient rule}: used to find the derivative of a function that is
      a ratio of two differentiable functions.
     \begin{itemize}
       \item Let \( f \) and \( g \) be differentiable and \( g(x) \neq 0 \),
         then
         \[
           \left( \frac{f}{g}
           \right) ' = \frac{f'g - fg'}{g^2}
         \]
     \end{itemize}

  \end{itemize}

  \subsection{Trigonometric Differentiation}
  \src{
    \link{https://en.wikipedia.org/wiki/Differentiation_of_trigonometric_functions}{Trigonometric functions}
  }
  \begin{itemize}
    \item All derivatives of circular trigonometric functions can be found from
      those of \( \sin(x) \) and \( \cos(x) \) by means of the quotient rule.
      \begin{align*}
        \sin(x) \quad &\to \quad  \cos(x) & \arcsin(x) \quad &\to \quad \left( \sqrt{1-x^2} \right)^{-1}   \\
        \cos(x) \quad &\to \quad  -\sin(x) & \arccos(x) \quad &\to \quad -\left( \sqrt{1-x^2}  \right)^{-1} \\
        \tan(x) \quad &\to \quad  \sec^2(x) & \arctan(x) \quad &\to \quad \left( x^2 + 1 \right) ^{-1} \\
        \cot(x) \quad &\to \quad  -\csc^2(x) & \arccot(x) \quad &\to \quad -\left( x^2 + 1 \right) ^{-1}\\
        \sec(x) \quad &\to \quad  \sec(x)\tan(x) & \arcsec(x) \quad &\to \quad \left( |x|\sqrt{x^2 -1}  \right) ^{-1}\\
        \csc(x) \quad &\to \quad -\csc(x)\cot(x) & \arccsc(x) \quad &\to \quad -\left( |x|\sqrt{x^2 -1}  \right) ^{-1}
      \end{align*}

    \item Inverse trigonometric functions are found using \dlink{ss:Implicit
      Differentiation}{implicit differentiation}.

  \end{itemize}

\end{itemize}


\section{Differentiation Concepts}
\src{
  \thomas{3.7, 3.8}
}
\begin{itemize}
  \item []

  \subsection{Implicit Differentiation}
  \src{
    \link{https://en.wikipedia.org/wiki/Implicit_function\#Implicit_differentiation}{Implicit differentiation}
  }
  \begin{itemize}
    \item
  \end{itemize}

  \subsection{Logarithmic Differentiation}
  \src{
    \link{https://en.wikipedia.org/wiki/Logarithmic_differentiation}{Logarithmic differentiation}
  }
  \begin{itemize}
    \item
  \end{itemize}

  \subsection{Higher Order Derivatives}
  \src{
    \link{https://en.wikipedia.org/wiki/Second_derivative}{Second derivative}
  }
  \begin{itemize}
    \item
  \end{itemize}

  \subsection{Related Rates}
  \src{
    \link{https://en.wikipedia.org/wiki/Related_rates}{Related rates}
  }
  \begin{itemize}
    \item
  \end{itemize}

\end{itemize}
